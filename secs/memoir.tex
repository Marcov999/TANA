\begin{defn}
  Sia $z=x+iy$ con $x>0$. Si dice \textit{funzione $\vartheta$ di Jacobi} la seguente serie totalmente convergente:
  $$\vartheta(z)=\sum_{n \in \mathbb{Z}} e^{-\pi n^2z}.$$
\end{defn}

\begin{lm}
  Per $x=\mathfrak{Re}\,z>0$ si ha
  $$\vartheta(z)=\frac{1}{\sqrt{z}}\vartheta\left(\frac{1}{z}\right).$$
\end{lm}

\begin{proof}
  Possiamo dimostrare la formula per $z=x>0$, che valga in tutto il semipiano $\mathfrak{Re}\,z>0$ segue per prolungamento analitico. Sia $f(\xi)=e^{-\pi\xi^2}$ e sia $f_x(\xi)=f(\sqrt{x}\xi)=e^{-\pi x\xi^2}$. Allora
  \begin{gather*}
    \hat{f}_x(\xi)=\int_{\mathbb{R}} f(\sqrt{x}t)e^{-2\pi i\xi t}\diff t \overset{s=t\sqrt{x}}{=} \frac{1}{\sqrt{x}}\int_{\mathbb{R}} f(s)e^{-2\pi i\frac{\xi}{\sqrt{x}}s}\diff s= \\
    =\frac{1}{\sqrt{x}}\hat{f}\left(\frac{\xi}{\sqrt{x}}\right)=\frac{1}{\sqrt{x}}f\left(\frac{\xi}{\sqrt{x}}\right)=\frac{1}{\sqrt{x}}f_{\frac{1}{x}}(\xi).
  \end{gather*}
  Applicando la formula di Poisson abbiamo
  \begin{gather*}
    \sum_{n \in \mathbb{Z}} f_x(n)=\sum_{n \in \mathbb{Z}} \hat{f}_x(n)=\sum_{n \in \mathbb{Z}} \frac{1}{\sqrt{x}}f_{\frac{1}{x}}(n),
  \end{gather*}
  quindi
  $$\vartheta(x)=\sum_{n \in \mathbb{Z}} e^{-\pi n^2x}=\frac{1}{\sqrt{x}}\sum_{n \in \mathbb{Z}}e^{-\frac{\pi}{x}n^2}=\frac{1}{\sqrt{x}}\vartheta\left(\frac{1}{x}\right).$$
\end{proof}

Vogliamo usare quest'identità della funzione di Jacobi per dimostrare il teorema di Riemann sulla funzione $\zeta$. D'ora in avanti, per motivi storici la variabile sarà $s=\sigma+it$ con $\sigma, t \in \mathbb{R}$.

\begin{thm}
  (Riemann, 1860) La funzione $\zeta(s)$ è meromorfa in $\mathbb{C}$ con un polo semplice in $s=1$ con residuo $1$. Inoltre, posta
  $$\xi(s)=\frac{s(s-1)}{2}\pi^{-s/2}\Gamma\left(\frac{s}{2}\right)\zeta(s),$$
  allora $\xi$ è intera e fornisce il prolungamento analitico di $\zeta$. Si ha
  \begin{equation} \label{xifunctional}
    \xi(s)=\xi(1-s).
  \end{equation}
\end{thm}

\begin{proof}
  Se $\sigma>0$, allora
  \begin{gather*}
    \Gamma\left(\frac{s}{2}\right)=\int_0^{+\infty} e^{-x}x^{\frac{s}{2}}\frac{\diff x}{x} \implies \\
    \implies \frac{\pi^{-s/2}}{n^s}\Gamma\left(\frac{s}{2}\right)=\int_0^{+\infty} e^{-x}\left(\frac{x}{\pi n^2}\right)^{s/2}\frac{\diff x}{x}.
  \end{gather*}
  Se $\sigma>1$, allora
  \begin{gather*}
    \pi^{-s/2}\Gamma\left(\frac{s}{2}\right)\zeta(s)=\sum_{n=1}^{+\infty} \int_0^{+\infty} e^{-x}\left(\frac{x}{\pi n^2}\right)^{s/2}\frac{\diff x}{x} \overset{y=\frac{x}{\pi n^2}}{=} \\
    =\sum_{n=1}^{+\infty} \int_0^{+\infty} e^{-\pi n^2 y}y^{s/2}\frac{\diff y}{y}=\int_0^{+\infty} \sum_{n=1}^{+\infty} e^{-\pi n^2y}y^{s/2}\frac{\diff y}{y}= \\
    =\frac{1}{2}\int_0^{+\infty} \big(\vartheta(y)-1\big)y^{s/2}\frac{\diff y}{y}=\frac{1}{2}\left(\int_0^1+\int_1^{+\infty}\right)\big(\vartheta(y)-1\big)y^{s/2}\frac{\diff y}{y}.
  \end{gather*}
  Abbiamo che
  \begin{gather*}
    \frac{1}{2} \int_0^1 \big(\vartheta(y)-1\big)y^{s/2}\frac{\diff y}{y} \overset{x=\frac{1}{y}}{=} \frac{1}{2}\int_1^{+\infty} \Bigg(\vartheta\left(\frac{1}{x}\right)-1\Bigg)x^{-s/2}\frac{\diff x}{x}= \\
    =\frac{1}{2}\int_1^{+\infty} \big(\vartheta(x)-1\big)\sqrt{x}\cdot x^{-s/2}\frac{\diff x}{x}+\frac{1}{2}\int_1^{+\infty} x^{\frac{1-s}{2}}\frac{\diff x}{x}-\frac{1}{2}\int_1^{+\infty} x^{-\frac{s}{2}}\frac{\diff x}{x}= \\
    =\frac{1}{2}\int_1^{+\infty} \big(\vartheta(x)-1\big)\cdot x^{\frac{1-s}{2}}\frac{\diff x}{x}+\frac{1}{s-1}-\frac{1}{s}=\\
    =\frac{1}{2}\int_1^{+\infty} \big(\vartheta(x)-1\big)\cdot x^{\frac{1-s}{2}}\frac{\diff x}{x}+\frac{1}{s(s-1)}.
  \end{gather*}
  Si ha dunque
  \begin{gather*}
    \pi^{-s/2}\Gamma\left(\frac{s}{2}\right)\zeta(s)=\frac{1}{s(s-1)}+\frac{1}{2}\int_1^{+\infty} \big(\vartheta(x)-1\big)(x^{\frac{s}{2}}+x^{\frac{1-s}{2}})\frac{\diff x}{x} \implies \\
    \implies \xi(s)=\frac{s(s-1)}{2}\pi^{-s/2}\Gamma\left(\frac{s}{2}\right)\zeta(s)=\\
    =\frac{1}{2}+\frac{s(s-1)}{4}\int_1^{+\infty} \big(\vartheta(x)-1\big)(x^{\frac{s}{2}}+x^{\frac{1-s}{2}})\frac{\diff x}{x}.
  \end{gather*}
  Poiché
  $$\frac{1}{2}\big(\vartheta(x)-1\big)=\sum_{n=1}^{+\infty} e^{-\pi n^2x} \le \sum_{n=1}^{+\infty} e^{-\pi nx}=\frac{1}{e^{\pi x}-1} \ll e^{-\pi x},$$
  l'ultimo integrale nella formula per $\xi(s)$ converge uniformemente in ogni striscia $a \le \sigma \le b$; allora $\xi$ definita da quell'integrale è una funzione intera e la relazione con $\Gamma$ e $\zeta$ data nell'enunciato è valida per $\sigma>1$. Che $\xi(1-s)=\xi(s)$ è ovvio. Poiché $\Gamma$ non ha zeri e lo zero semplice di $s/2$ in $0$ è cancellato dal polo di $\Gamma$, questo definisce l'estensione analitica di $\zeta$ a $\mathbb{C}\setminus\{1\}$. Sappiamo già che sui reali $\displaystyle \lim_{s \longrightarrow +\infty} \zeta(s)=1$. Mostriamo che è un polo semplice di residuo $1$. Si ha
  \begin{gather*}
    \zeta(s)=\xi(s)\frac{\pi^{s/2}}{\frac{s}{2}\Gamma\left(\frac{s}{2}\right)}\frac{1}{s-1} \text{ e} \\
    \underset{s=1}{\text{Res}}\,\zeta=\lim_{s \longrightarrow 1} \big(\zeta(s)(s-1)\big)=\xi(1)\frac{\pi^{1/2}}{\frac{1}{2}\Gamma\left(\frac{1}{2}\right)}=\frac{1}{2}\cdot\frac{\sqrt{\pi}}{\frac{1}{2}\sqrt{\pi}}=1.
  \end{gather*}
\end{proof}

\begin{oss}
  $\xi$ non ha zeri fuori dalla striscia $0 \le \sigma \le 1$.
\end{oss}

\begin{cor}
  $\zeta(-2k)=0$ per ogni $k \ge 1$.
\end{cor}

\begin{proof}
  Guardare i poli di $\Gamma$.
\end{proof}

\begin{oss}
  All'interno della striscia  $0 \le \sigma \le 1$ abbiamo che $\zeta$ e $\xi$ si annullano insieme.
\end{oss}

\begin{oss}
  Se $\rho$ è uno zero, dalla \eqref{xifunctional} anche $1-\rho$ è uno zero. Poiché $\xi$ è reale sui reali, se $\rho$ è uno $0$ anche $\bar{\rho}$ è uno zero.
\end{oss}

\begin{oss}
  Definendo $\Xi(s)=\xi(is+1/2)$ si ha
  $$\Xi(-s)=\xi(1/2-is)=\xi(1/2+is)=\Xi(s).$$
  Se $s=x \in \mathbb{R}$, allora $\overline{\xi(1/2+ix)}=\xi(1/2-ix)=\xi(1/2+ix)$, quindi $\Xi(x)$ è reale.
\end{oss}

\begin{oss}
  $\zeta(s)\not=0$ per $\sigma>1$. Infatti
  \begin{gather*}
    \left|\zeta(s)\prod_{p \le N} \left(1-\frac{1}{p^s}\right)\right|=\left|\prod_{p>N} \left(1-\frac{1}{p^s}\right)^{-1}\right|= \\
    =\left|1+\sum_{p \mid n \implies p>N} \frac{1}{n^s}\right| \ge 1-\sum_{n>N} \frac{1}{n^{\sigma}}=1+O\left(\frac{1}{N^{\sigma-1}}\right).
  \end{gather*}
\end{oss}

\begin{oss}
  Prendiamo $0 \le \sigma \le 1$ e $t=0$. Si ha
  $$\xi(\sigma)=\frac{1}{2}+\frac{\sigma(\sigma-1)}{4}\int_1^{+\infty}\big(\vartheta(x)-1\big)(x^{\frac{\sigma}{2}}+x^{\frac{1-\sigma}{2}})\frac{\diff x}{x}.$$
  Ricordiamo che $\dfrac{\vartheta(x)-1}{2} \le \dfrac{1}{e^{\pi x}-1}$ e per $x \ge 1$ vale che $x \ge \sqrt{x}$, dunque $e^{\pi x}-1 \ge \pi x \ge 2\sqrt{x}$, perciò $\vartheta(x)-1 \le \dfrac{1}{\sqrt{x}}$. Inoltre
  $$\int_1^{+\infty} (x^{\frac{\sigma-1}{2}}+x^{\frac{-\sigma}{2}})\frac{\diff x}{x}=\frac{2}{\sigma(\sigma-1)}.$$
  Mettendo assieme, troviamo che
  $$\xi(\sigma) \ge \frac{1}{2}-\frac{\sigma(1-\sigma)}{4}\cdot\frac{2}{\sigma(1-\sigma)}=0.$$
\end{oss}

\begin{oss}
  $\xi(0)=1/2$ e $\displaystyle \lim_{s \longrightarrow 1} \frac{s}{2}\Gamma\left(\frac{s}{2}\right)=1$, dunque dalla formula che lega $\zeta$ e $\xi$ si ha $\zeta(0)=-1/2$.
\end{oss}

\begin{exc}
  Dimostrare che $\zeta(-1)=-1/12$.
\end{exc}
