\begin{defn}
  Sia $z=x+iy$ con $x>0$. Si dice \textit{funzione $\vartheta$ di Jacobi} la seguente serie totalmente convergente:
  $$\vartheta(z)=\sum_{n \in \mathbb{Z}} e^{-\pi n^2z}.$$
\end{defn}

\begin{lm}
  Per $x=\mathfrak{Re}\,z>0$ si ha
  $$\vartheta(z)=\frac{1}{\sqrt{z}}\vartheta\left(\frac{1}{z}\right).$$
\end{lm}

\begin{proof}
  Possiamo dimostrare la formula per $z=x>0$, che valga in tutto il semipiano $\mathfrak{Re}\,z>0$ segue per prolungamento analitico. Sia $f(\xi)=e^{-\pi\xi^2}$ e sia $f_x(\xi)=f(\sqrt{x}\xi)=e^{-\pi x\xi^2}$. Allora
  \begin{gather*}
    \hat{f}_x(\xi)=\int_{\mathbb{R}} f(\sqrt{x}t)e^{-2\pi i\xi t}\diff t \overset{s=t\sqrt{x}}{=} \frac{1}{\sqrt{x}}\int_{\mathbb{R}} f(s)e^{-2\pi i\frac{\xi}{\sqrt{x}}s}\diff s= \\
    =\frac{1}{\sqrt{x}}\hat{f}\left(\frac{\xi}{\sqrt{x}}\right)=\frac{1}{\sqrt{x}}f\left(\frac{\xi}{\sqrt{x}}\right)=\frac{1}{\sqrt{x}}f_{\frac{1}{x}}(\xi).
  \end{gather*}
  Applicando la formula di Poisson abbiamo
  \begin{gather*}
    \sum_{n \in \mathbb{Z}} f_x(n)=\sum_{n \in \mathbb{Z}} \hat{f}_x(n)=\sum_{n \in \mathbb{Z}} \frac{1}{\sqrt{x}}f_{\frac{1}{x}}(n),
  \end{gather*}
  quindi
  $$\vartheta(x)=\sum_{n \in \mathbb{Z}} e^{-\pi n^2x}=\frac{1}{\sqrt{x}}\sum_{n \in \mathbb{Z}}e^{-\frac{\pi}{x}n^2}=\frac{1}{\sqrt{x}}\vartheta\left(\frac{1}{x}\right).$$
\end{proof}

Vogliamo usare quest'identità della funzione di Jacobi per dimostrare il teorema di Riemann sulla funzione $\zeta$. D'ora in avanti, per motivi storici la variabile sarà $s=\sigma+it$ con $\sigma, t \in \mathbb{R}$.

\begin{thm}
  (Riemann, 1860) La funzione $\zeta(s)$ è meromorfa in $\mathbb{C}$ con un polo semplice in $s=1$ con residuo $1$. Inoltre, posta
  $$\xi(s)=\frac{s(s-1)}{2}\pi^{-s/2}\Gamma\left(\frac{s}{2}\right)\zeta(s),$$
  allora $\xi$ è intera e fornisce il prolungamento analitico di $\zeta$. Si ha
  \begin{equation} \label{xifunctional}
    \xi(s)=\xi(1-s).
  \end{equation}
\end{thm}

\begin{proof}
  Se $\sigma>0$, allora
  \begin{gather*}
    \Gamma\left(\frac{s}{2}\right)=\int_0^{+\infty} e^{-x}x^{\frac{s}{2}}\frac{\diff x}{x} \implies \\
    \implies \frac{\pi^{-s/2}}{n^s}\Gamma\left(\frac{s}{2}\right)=\int_0^{+\infty} e^{-x}\left(\frac{x}{\pi n^2}\right)^{s/2}\frac{\diff x}{x}.
  \end{gather*}
  Se $\sigma>1$, allora
  \begin{gather*}
    \pi^{-s/2}\Gamma\left(\frac{s}{2}\right)\zeta(s)=\sum_{n=1}^{+\infty} \int_0^{+\infty} e^{-x}\left(\frac{x}{\pi n^2}\right)^{s/2}\frac{\diff x}{x} \overset{y=\frac{x}{\pi n^2}}{=} \\
    =\sum_{n=1}^{+\infty} \int_0^{+\infty} e^{-\pi n^2 y}y^{s/2}\frac{\diff y}{y}=\int_0^{+\infty} \sum_{n=1}^{+\infty} e^{-\pi n^2y}y^{s/2}\frac{\diff y}{y}= \\
    =\frac{1}{2}\int_0^{+\infty} \big(\vartheta(y)-1\big)y^{s/2}\frac{\diff y}{y}=\frac{1}{2}\left(\int_0^1+\int_1^{+\infty}\right)\big(\vartheta(y)-1\big)y^{s/2}\frac{\diff y}{y}.
  \end{gather*}
  Abbiamo che
  \begin{gather*}
    \frac{1}{2} \int_0^1 \big(\vartheta(y)-1\big)y^{s/2}\frac{\diff y}{y} \overset{x=\frac{1}{y}}{=} \frac{1}{2}\int_1^{+\infty} \Bigg(\vartheta\left(\frac{1}{x}\right)-1\Bigg)x^{-s/2}\frac{\diff x}{x}= \\
    =\frac{1}{2}\int_1^{+\infty} \big(\vartheta(x)-1\big)\sqrt{x}\cdot x^{-s/2}\frac{\diff x}{x}+\frac{1}{2}\int_1^{+\infty} x^{\frac{1-s}{2}}\frac{\diff x}{x}-\frac{1}{2}\int_1^{+\infty} x^{-\frac{s}{2}}\frac{\diff x}{x}= \\
    =\frac{1}{2}\int_1^{+\infty} \big(\vartheta(x)-1\big)\cdot x^{\frac{1-s}{2}}\frac{\diff x}{x}+\frac{1}{s-1}-\frac{1}{s}=\\
    =\frac{1}{2}\int_1^{+\infty} \big(\vartheta(x)-1\big)\cdot x^{\frac{1-s}{2}}\frac{\diff x}{x}+\frac{1}{s(s-1)}.
  \end{gather*}
  Si ha dunque
  \begin{gather*}
    \pi^{-s/2}\Gamma\left(\frac{s}{2}\right)\zeta(s)=\frac{1}{s(s-1)}+\frac{1}{2}\int_1^{+\infty} \big(\vartheta(x)-1\big)(x^{\frac{s}{2}}+x^{\frac{1-s}{2}})\frac{\diff x}{x} \implies \\
    \implies \xi(s)=\frac{s(s-1)}{2}\pi^{-s/2}\Gamma\left(\frac{s}{2}\right)\zeta(s)=\\
    =\frac{1}{2}+\frac{s(s-1)}{4}\int_1^{+\infty} \big(\vartheta(x)-1\big)(x^{\frac{s}{2}}+x^{\frac{1-s}{2}})\frac{\diff x}{x}.
  \end{gather*}
  Poiché
  $$\frac{1}{2}\big(\vartheta(x)-1\big)=\sum_{n=1}^{+\infty} e^{-\pi n^2x} \le \sum_{n=1}^{+\infty} e^{-\pi nx}=\frac{1}{e^{\pi x}-1} \ll e^{-\pi x},$$
  l'ultimo integrale nella formula per $\xi(s)$ converge uniformemente in ogni striscia $a \le \sigma \le b$; allora $\xi$ definita da quell'integrale è una funzione intera e la relazione con $\Gamma$ e $\zeta$ data nell'enunciato è valida per $\sigma>1$. Che $\xi(1-s)=\xi(s)$ è ovvio. Poiché $\Gamma$ non ha zeri e lo zero semplice di $s/2$ in $0$ è cancellato dal polo di $\Gamma$, questo definisce l'estensione analitica di $\zeta$ a $\mathbb{C}\setminus\{1\}$. Sappiamo già che sui reali $\displaystyle \lim_{s \longrightarrow +\infty} \zeta(s)=1$. Mostriamo che è un polo semplice di residuo $1$. Si ha
  \begin{gather*}
    \zeta(s)=\xi(s)\frac{\pi^{s/2}}{\frac{s}{2}\Gamma\left(\frac{s}{2}\right)}\frac{1}{s-1} \text{ e} \\
    \underset{s=1}{\text{Res}}\,\zeta=\lim_{s \longrightarrow 1} \big(\zeta(s)(s-1)\big)=\xi(1)\frac{\pi^{1/2}}{\frac{1}{2}\Gamma\left(\frac{1}{2}\right)}=\frac{1}{2}\cdot\frac{\sqrt{\pi}}{\frac{1}{2}\sqrt{\pi}}=1.
  \end{gather*}
\end{proof}

\begin{oss}
  $\xi$ non ha zeri fuori dalla striscia $0 \le \sigma \le 1$.
\end{oss}

\begin{cor}
  $\zeta(-2k)=0$ per ogni $k \ge 1$.
\end{cor}

\begin{proof}
  Guardare i poli di $\Gamma$.
\end{proof}

\begin{oss}
  All'interno della striscia  $0 \le \sigma \le 1$ abbiamo che $\zeta$ e $\xi$ si annullano insieme.
\end{oss}

\begin{oss}
  Se $\rho$ è uno zero, dalla \eqref{xifunctional} anche $1-\rho$ è uno zero. Poiché $\xi$ è reale sui reali, se $\rho$ è uno $0$ anche $\bar{\rho}$ è uno zero.
\end{oss}

\begin{oss}
  Definendo $\Xi(s)=\xi(is+1/2)$ si ha
  $$\Xi(-s)=\xi(1/2-is)=\xi(1/2+is)=\Xi(s).$$
  Se $s=x \in \mathbb{R}$, allora $\overline{\xi(1/2+ix)}=\xi(1/2-ix)=\xi(1/2+ix)$, quindi $\Xi(x)$ è reale.
\end{oss}

\begin{oss}
  $\zeta(s)\not=0$ per $\sigma>1$. Infatti
  \begin{gather*}
    \left|\zeta(s)\prod_{p \le N} \left(1-\frac{1}{p^s}\right)\right|=\left|\prod_{p>N} \left(1-\frac{1}{p^s}\right)^{-1}\right|= \\
    =\left|1+\sum_{p \mid n \implies p>N} \frac{1}{n^s}\right| \ge 1-\sum_{n>N} \frac{1}{n^{\sigma}}=1+O\left(\frac{1}{N^{\sigma-1}}\right).
  \end{gather*}
\end{oss}

\begin{oss}
  Prendiamo $0 \le \sigma \le 1$ e $t=0$. Si ha
  $$\xi(\sigma)=\frac{1}{2}+\frac{\sigma(\sigma-1)}{4}\int_1^{+\infty}\big(\vartheta(x)-1\big)(x^{\frac{\sigma}{2}}+x^{\frac{1-\sigma}{2}})\frac{\diff x}{x}.$$
  Ricordiamo che $\dfrac{\vartheta(x)-1}{2} \le \dfrac{1}{e^{\pi x}-1}$ e per $x \ge 1$ vale che $x \ge \sqrt{x}$, dunque $e^{\pi x}-1 \ge \pi x \ge 2\sqrt{x}$, perciò $\vartheta(x)-1 \le \dfrac{1}{\sqrt{x}}$. Inoltre
  $$\int_1^{+\infty} (x^{\frac{\sigma-1}{2}}+x^{\frac{-\sigma}{2}})\frac{\diff x}{x}=\frac{2}{\sigma(\sigma-1)}.$$
  Mettendo assieme, troviamo che
  $$\xi(\sigma) \ge \frac{1}{2}-\frac{\sigma(1-\sigma)}{4}\cdot\frac{2}{\sigma(1-\sigma)}=0.$$
\end{oss}

\begin{oss}
  $\xi(0)=1/2$ e $\displaystyle \lim_{s \longrightarrow 1} \frac{s}{2}\Gamma\left(\frac{s}{2}\right)=1$, dunque dalla formula che lega $\zeta$ e $\xi$ si ha $\zeta(0)=-1/2$.
\end{oss}

\begin{exc}
  Dimostrare che $\zeta(-1)=-1/12$.
\end{exc}

Vogliamo ora capire qual è l'ordine di $\xi$. Abbiamo $\xi(s)=(s-1)\Gamma\left(\frac{s}{2}+1\right)\pi^{-\frac{s}{2}}\zeta(s)$. Analizziamo i vari fattori.

Con $s=2n$ si ha $\Gamma\left(\dfrac{s}{2}+1\right)=\Gamma(n+1)=n!\sim e^{\left(n+\frac{1}{2}\right)\log{n}-n+(\dots)} \ll_{\epsilon} e^{n^{1+\epsilon}}$. Più in particolare, per il corollario \ref{gammaord} abbiamo che questo vale in generale per $\sigma>1$.
Nella stessa regione, $\zeta(s)$ e $\pi^{-\frac{s}{2}}$ sono limitate a infinito e $s-1$ è ``piccola''. Questo ci dice che dev'essere $ord(\xi) \ge 1$ e, per quanto visto su $\Gamma$, se valesse l'uguale l'ordine sarebbe un $\inf$ e non un $\min$.

Per simmetria, ci resta da verificare solo la regione $1/2 \le \sigma \le 1$. Vediamo che, se l'ordine fosse proprio $1$, per il teorema di Hadamard avremmo che $\xi$ ha un prodotto di Weierstrass della forma $\displaystyle \xi(s)=e^{as+b}\prod_p (\dots)$. Se non ci fossero zeri, il prodotto vuoto sarebbe $1$ e $\xi(s) \ll e^{a|s|}$, assurdo per via del contributo $e^{s\log{s}}$ dovuto a $\Gamma$. Per lo stesso motivo, gli zeri devono essere infiniti.

\begin{lm} \label{lls}
  Per $\sigma \ge \epsilon$ si ha $\zeta(s) \ll_{\epsilon} |s|$ (o $|t|$) uniformemente.
\end{lm}

\begin{proof}
  Sia $\sigma>1$. Per sommazione parziale,
  \begin{gather*}
    \sum_{n \le x} \frac{1}{n^s}=\frac{\lfloor x\rfloor}{x^s}+s\int_1^x \frac{\lfloor u\rfloor}{u^{s+1}}\diff u=\frac{\lfloor x\rfloor}x^s+s\int_1^x \frac{\diff u}{u^s}-s\int_1^x \frac{\{u\}}{u^{s+1}}\diff u= \\
    =\frac{\lfloor x\rfloor}{x^s}+\frac{s}{s-1}-s\int_1^x \frac{\{u\}}{u^{s+1}}\diff u \implies \zeta(s)=\frac{1}{s-1}+1-s\int_1^{+\infty} \frac{\{u\}}{u^{s+1}}\diff u;
  \end{gather*}
  questa è l'estensione di $\zeta$ a $\sigma>0$ (l'integrale converge uniformemente per $\sigma \ge \epsilon$).
  \begin{oss}
    Prendendo $s=1$, otteniamo
    \begin{gather*}
      \sum_{n \le x} \frac{1}{n}=1-\frac{\{x\}}{x}+\log{x}-\int_1^x \frac{\{u\}}{u^2}\diff u \implies \\
      \lim_{x \longrightarrow +\infty} \left(\sum_{n \le x} \frac{1}{n}-\log{x}\right)=1-\int_1^{+\infty} \frac{\{u\}}{u^2}\diff u=\gamma.
    \end{gather*}
    Ma passiamo oltre.
  \end{oss}

  Utilizzando i polinomio di Bernoulli, abbiamo
  \begin{gather*}
    \zeta(s)=\frac{1}{s-1}+1-\frac{s}{2}\int_1^{+\infty} \frac{\diff u}{u^{s+1}}-s\int_1^{+\infty} \frac{B_1(\{u\})}{u^{s+1}}\diff u= \\
    =\frac{1}{s-1}+1-\frac{1}{2}-s\int_1^{+\infty} \frac{B_1(\{u\})}{u^{s+1}}\diff u.
  \end{gather*}

  \begin{oss}
    Ricordando la relazione tra polinomi di Bernoulli successivi e le loro derivate, si può integrare per parti fino a ottenere il prolungamento di $\zeta$ in un semipiano che inizia arbitrariamente a sinistra. Ma, senza l'equazione funzione, questa costruzione è abbastanza inutile.
  \end{oss}

  Tornando a noi, per $\sigma \ge \epsilon$ e usando che $B_1(\{u\})$ è limitato, abbiamo
  $$|\zeta(s)| \ll 1+|s|\int_1^{+\infty} \frac{\diff u}{u^{1+\epsilon}} \ll \frac{|s|}{\epsilon} \ll_{\epsilon} |s|.$$
\end{proof}

Curiosità: la congettura di Lindelöf ipotizza che $\zeta\left(\dfrac{1}{2}+it\right) \ll_{\epsilon} t^{\epsilon}$ per ogni $\epsilon>0$.

\begin{prop}
  (formula di Riemann-Von Mangoldt) Contando con moltepicità, sia
  $$N(T)=\sharp\{\rho=\beta+i\gamma \mid \zeta(\rho)=0, 0 \le \beta \le 1, 0<\gamma<T\},$$
  allora per $T \longrightarrow +\infty$ si ha
  \begin{equation} \label{rievonmformula}
    N(T)=\frac{T}{2\pi}\log\left(\frac{T}{2\pi}\right)-\frac{T}{2\pi}+O(\log{T}).
  \end{equation}
\end{prop}

\begin{proof}
  Consideriamo il bordo rettangolare $R$, in verde nella figura.
  \begin{center}
    \definecolor{ttccqq}{rgb}{0.2,0.8,0}
    \definecolor{uququq}{rgb}{0.25,0.25,0.25}
    \begin{tikzpicture}[line cap=round,line join=round,>=triangle 45,x=1.0cm,y=1.0cm]
        \draw[->,color=black] (-2.09,0) -- (3.18,0);
        \foreach \x in {-2,-1,1,2,3}
        \draw[shift={(\x,0)},color=black] (0pt,2pt) -- (0pt,-2pt);
        \draw[->,color=black] (0,-0.31) -- (0,6.22);
        \foreach \y in {,1,2,3,4,5,6}
        \draw[shift={(0,\y)},color=black] (2pt,0pt) -- (-2pt,0pt);
        \clip(-2.09,-0.31) rectangle (3.18,6.22);
        \draw [line width=1pt,color=ttccqq] (-1,0)-- (2,0);
        \draw [line width=1pt,color=ttccqq] (2,0)-- (2,5.58);
        \draw [line width=1pt,color=ttccqq] (-1,5.58)-- (2,5.58);
        \draw [line width=1pt,color=ttccqq] (-1,5.58)-- (-1,0);
        \draw (1,5.58)-- (1,0);
        \draw [dash pattern=on 3pt off 3pt] (0.5,5.58)-- (0.5,0);
        \begin{scriptsize}
        \fill [color=black] (2,0) circle (1.5pt);
        \draw[color=black] (2,-0.2) node {$2$};
        \fill [color=black] (1,0) circle (1.5pt);
        \draw[color=black] (1,-0.2) node {$1$};
        \fill [color=black] (0.5,0) circle (1.5pt);
        \draw[color=black] (0.5,-0.2) node {$1/2$};
        \fill [color=black] (-1,0) circle (1.5pt);
        \draw[color=black] (-1,-0.2) node {$-1$};
        \fill [color=uququq] (0.5,5.58) circle (1.5pt);
        \draw[color=uququq] (0.5,5.8) node {$\frac{1}{2}+iT$};
        \fill [color=uququq] (2,5.58) circle (1.5pt);
        \draw[color=uququq] (2.1,5.8) node {$2+iT$};
        \fill [color=black] (0,0) circle (1.5pt);
        \draw[color=black] (-0.15,-0.2) node {$0$};
        \draw[color=black] (3, -0.2) node {$\sigma$};
        \draw[color=black] (-0.2, 6) node {$t$};
        \draw[color=ttccqq] (2.5, 2.9) node {$R$};
      \end{scriptsize}
    \end{tikzpicture}
  \end{center}

  Poiché $\xi$ è una funzione intera senza zeri fuori dalla striscia critica, all'interno della quale ha invece gli stessi zeri di $\zeta$ con la stessa moltepicità, abbiamo che
  $$N(T)=\frac{1}{2\pi i} \oint_R \frac{\xi'(s)}{\xi(s)}\diff s=\frac{1}{2\pi}\Delta_R \arg\big(\xi(s)\big).$$
  Adesso notiamo che, per quello che sappiamo di $\xi$ tra $0$ e $1$ e fuori dalla striscia critica e usando l'equazione funzionale, essa è sempre reale e mai nulla tra $-1$ e $2$, dunque l'argomento non cambia e possiamo trascurare quel segmento. Dato che $\xi$ è reale sui reali, si ha anche $\xi(s)=\xi(1-s)=\overline{\xi(1-\bar{s})}$; perciò la variazione da $\frac{1}{2}+iT$ a $-1$ è la stessa che tra $2$ e $\frac{1}{2}+iT$.
  Detto allora $L$ il sottotratto di $R$ formato dai due segmenti da $2$ a $2+iT$ e da $2+iT$ e $\frac{1}{2}+iT$, si ha $N(T)=\dfrac{1}{\pi}\Delta_L \arg\big(\xi(s)\big)$. Ricordiamo ora la definizione di $\xi$:
  $$\xi(s)=\frac{s(s-1)}{2}\pi^{-\frac{s}{2}}\Gamma\left(\frac{s}{2}\right)\zeta(s)=(s-1)\pi^{-\frac{s}{2}}\Gamma\left(\frac{s}{2}+1\right)\zeta(s);$$
  si ha dunque
  $$\Delta_L\arg\big(\xi(s)\big)=\Delta_L\arg(s-1)+\Delta_L\arg\left(\pi^{-\frac{s}{2}}\right)+\Delta_L\arg\Bigg(\Gamma\left(\frac{s}{2}+1\right)\Bigg)+\Delta_L\arg\big(\zeta(s)\big).$$
  Studiamo i singoli pezzi.
  \begin{gather*}
    \Delta_L\arg(s-1)=\arg\left(-\frac{1}{2}+iT\right)=\frac{\pi}{2}+O\left(\frac{1}{T}\right) \text{ e}\\
    \Delta_L\arg\left(\pi^{-\frac{s}{2}}\right)=\Delta_L\arg\left(e^{-\frac{s}{2}\log{\pi}}\right)=-\frac{T}{2}\log{\pi}.
  \end{gather*}
  Ricordiamo la formula di Stirling:
  $$\log\big(\Gamma(z)\big)=\left(z-\frac{1}{2}\right)\log{z}-z+\log{\sqrt{2\pi}}+O\left(\frac{1}{|z|}\right);$$
  abbiamo quindi che
  \begin{gather*}
    \Delta_L\arg\Bigg(\Gamma\left(\frac{s}{2}+1\right)\Bigg)=\mathfrak{Im}\,\log\Bigg(\Gamma\left(\frac{5}{4}+i\frac{T}{2}\right)\Bigg)= \\
    =\mathfrak{Im}\Bigg(\left(\frac{3}{4}+i\frac{T}{2}\right)\log\left(\frac{5}{4}+i\frac{T}{2}\right)-\frac{5}{4}-i\frac{T}{2}+O\left(\frac{1}{T}\right)\Bigg)= \\
    =\frac{3}{4}\Bigg(\frac{\pi}{2}+O\left(\frac{1}{T}\right)\Bigg)+\frac{T}{2}\log\sqrt{\frac{T^2}{4}+\frac{25}{16}}-\frac{T}{2}+O\left(\frac{1}{T}\right)=\\
    \frac{3}{8}\pi+\frac{T}{2}\log\left(\frac{T}{2}\right)+\frac{T}{2}\log\left(1+\frac{25}{4T^2}\right)-\frac{T}{2}+O\left(\frac{1}{T}\right)=\\
    \frac{3}{8}\pi+\frac{T}{2}\log\left(\frac{T}{2}\right)-\frac{T}{2}+O\left(\frac{1}{T}\right).
  \end{gather*}
  Mettendo assieme si ha $N(T)=\dfrac{T}{2\pi}\log\left(\dfrac{T}{2\pi}\right)+\dfrac{7}{8}-\dfrac{T}{2\pi}+S(T)+O\left(\dfrac{1}{T}\right)$, dove poniamo $S(T)=\dfrac{1}{\pi}\arg\Bigg(\zeta\left(\dfrac{1}{2}+iT\right)\Bigg)$. Si conclude con il lemma seguente.
\end{proof}

\begin{lm} \label{Slllog}
  Sia $S(T)=\dfrac{1}{\pi}\arg\Bigg(\zeta\left(\dfrac{1}{2}+iT\right)\Bigg)$; allora
  $$S(T) \ll \log{T}.$$
\end{lm}

\begin{proof}
  Scriviamo $$\arg\Bigg(\zeta\left(\frac{1}{2}+iT\right)\Bigg)=\arg\big(\zeta(2+iT)\big)+\Bigg[\arg\Bigg(\zeta\left(\frac{1}{2}+iT\right)\Bigg)-\arg\big(\zeta(2+iT)\big)\Bigg]$$
  e stimiamo i due addendi. Si ha
  \begin{gather*}
    \mathfrak{Re}\,\zeta(2+iT)=1+\sum_{n=2}^{+\infty} \mathfrak{Re}\frac{1}{n^{2+iT}} \ge 1-\sum_{n=2}^{+\infty} \frac{1}{n^2}=1-\left(\frac{\pi^2}{6}-1\right)>\frac{1}{3} \implies \\
    \implies |\arg\big(\zeta(2+iT)\big)| \le \pi/2.
  \end{gather*}
  Sia $m=\sharp\{\sigma_j \in [1/2,2] \mid \mathfrak{Re}\,\zeta(\sigma_j+iT)=0\}$.
  Per come sono definiti, tra un $\sigma_j$ e il successivo l'argomento di $\zeta$ cambia al più di $\pi$. Allora
  $$\left|\arg\Bigg(\zeta\left(\frac{1}{2}+iT\right)\Bigg)-\arg\big(\zeta(2+iT)\big)\right| \le (m+1)\pi.$$
  Dobbiamo stimare $m$. Definiamo $f(s)=\zeta(s+iT)+\zeta(s-iT)$, che è una funzione olomorfa per $s+iT\not=1$, quindi lo è in particolare per $T$ sufficientemente grande in modulo ($\not=0$), che è quello che ci interessa. Per motivi di coniugio ($\zeta$ è reale sui reali), abbiamo che $f(\sigma)=2\mathfrak{Re}\,\zeta(\sigma+iT)$. Perciò otteniamo che $m=\sharp\{\sigma_j \in [1/2,2] \mid f(\sigma_j)=0\}$; vale dunque che
  $$m \le M=\sharp\{s \in \mathbb{C} \mid |s-2| \le 3/2, f(s)=0\}.$$
  Per il corollario \ref{1.2.5} con $r=3/2$ e $R=7/4$, troviamo
  \begin{gather*}
    M \le \frac{1}{\log(R/r)}\log\left(\frac{\max_{|s-2| \le 7/4}|f(s)|}{f(2)}\right) \le \\
    \le \frac{1}{\log(7/6)}\log\left(\frac{\max_{|s-2| \le 7/4} 2|\zeta(s+iT)|}{2/3}\right) \le C_1\log{T}+C_2 \ll \log{T},
  \end{gather*}
  dove abbiamo stimato $f(2)$ usando quanto trovato per $\mathfrak{Re}\,\zeta(2+iT)$, mentre il massimo dentro al logaritmo è stimato usando che $|\zeta(\sigma+iT)| \ll T$ per il lemma \ref{lls} con $\sigma \ge 1/4$.
\end{proof}

\begin{ftt}
  Il termine che è $O(\log{T})$ in \eqref{rievonmformula} è comunemente chiamato $\mathcal{R}$. L'ipotesi di Lindelöf implica $\mathcal{R}=o(\log{T})$, mentre l'ipotesi di Riemann implica $\mathcal{R}=O\left(\dfrac{\log{T}}{\log{\log{T}}}\right)$.
\end{ftt}

Dalla dimostrazione della formula di Riemann-Von Mangoldt abbiamo che $N(T)-\dfrac{T}{2\pi}\log\left(\dfrac{T}{2\pi}\right)+\dfrac{T}{2\pi}=\dfrac{7}{8}+S(T)+O\left(\dfrac{1}{T}\right)$;
Littlewood ha dimostrato che $\displaystyle S_1(T)=\int_0^T S(t)\diff t \ll \log{T}$ (se il risultato del lemma \ref{Slllog} fosse ottimale, questo significherebbe che $S$ cambia spesso di segno, quindi ha molti zeri). Si ha dunque
$$\lim_{T \longrightarrow +\infty} \frac{1}{T} \int_0^T \left[N(t)-\frac{t}{2\pi}\left(\frac{t}{2\pi}\right)-\frac{t}{2\pi}\right)]\diff t=\frac{7}{8},$$
che ci dice che la quantità $\frac{7}{8}$ nella formula è relativamente importante, quindi non possiamo trascurarla con troppa leggerezza.

\begin{cor}
  uniformemente in $T$ e $H$ si ha che
  $$N(T+H)-N(T) \ll (H+1)\log(T+H).$$
  Inoltre, esiste $H_0$ t.c. per $H \ge H_0$ si ha
  $$N(T+H)-N(T) \gg H\log{T}.$$
\end{cor}

\begin{proof}
  \begin{gather*}
    N(T+H)-N(T)= \\
    =\frac{T+H}{2\pi}\log\left(\frac{T+H}{2\pi}\right)-\frac{T+H}{2\pi}-\Bigg(\frac{T}{2\pi}\log\left(\frac{T}{2\pi}\right)-\frac{T}{2\pi}\Bigg)+O\big(\log(T+H)\big)= \\
    =\int_{\frac{T}{2\pi}}^{\frac{T+H}{2\pi}} \log{t}\diff t+O\big(\log(T+H)\big).
  \end{gather*}
  Per il teorema del valor medio, esiste $0 \le \delta \le 1$ t.c.
  $$N(T+H)-N(T)=\frac{H}{2\pi}\log\left(\frac{T+\delta H}{2\pi}\right)+O\big(\log(T+H)\big);$$
  la prima parte della tesi segue immediatamente. Per la seconda, notiamo che
  $$\frac{H}{2\pi}\log\left(\frac{T+\delta H}{2\pi}\right)+O\big(\log(T+H)\big)>\frac{H}{2\pi}\log\left(\frac{T}{2\pi}\right)+O\big(\log(T+H)\big) \gg H\log{T},$$
  dove serve $H \ge H_0$ per evitare che domini il termine $O$-grande, che non potremmo stimare dal basso.
\end{proof}

\begin{cor}
  Siano $\rho_n=\beta_n+i\gamma_n$ gli zeri non banali di $\zeta$, ordinati per parte immaginaria crescente (consideriamo solo $\gamma_n>0$) e contati con moltepicità; allora si ha $\gamma_n \sim \dfrac{2\pi n}{\log n}$.

  (Littlewood: $\gamma_{n+1}-\gamma_n \ll \dfrac{1}{\log{\log{\log{\gamma_n}}}}$)
\end{cor}

\begin{proof}
  Si ha
  \begin{gather*}
    \frac{\gamma_n}{2\pi}\log{\gamma_n} \sim \frac{\gamma_n+1}{2\pi}\log\left(\frac{\gamma_n+1}{2\pi}\right) \sim N(\gamma_n+1) \ge n \ge \\
    \ge N(\gamma_n-1) \sim \frac{\gamma_n-1}{2\pi}\log\left(\frac{\gamma_n-1}{2\pi}\right) \sim \frac{\gamma_n}{2\pi}\log{\gamma_n} \implies \\
    \implies n \sim \frac{\gamma_n}{2\pi}\log{\gamma_n} \implies \log{n} \sim \log{\gamma_n}+\log\log{\gamma_n}-\log{2\pi} \sim \log{\gamma_n} \implies \\
    \implies \gamma_n \sim \frac{2\pi n}{\log{\gamma_n}} \sim \frac{2\pi n}{\log{n}}.
  \end{gather*}
\end{proof}

\begin{cor}
  La successione $\rho_n=\beta_n+i\gamma_n$ ha esponente di convergenza $1$.
\end{cor}

\begin{proof}
  Si ha
  \begin{gather*}
    \sum_{n=1}^{+\infty} \frac{1}{|\rho_n|^{1+\epsilon}} \le \sum_{n=1}^{+\infty} \frac{1}{|\gamma_n|^{1+\epsilon}} \le c_1 \sum_{n=1}^{+\infty} \frac{(\log{n})^{1+\epsilon}}{n^{1+\epsilon}} \le c_2 \sum_{n=1}^{+\infty} \frac{1}{n^{1+\epsilon/2}}<+\infty,
  \end{gather*}
  ma, poiché $|\beta_n| \le 1$, abbiamo che
  \begin{gather*}
    \sum_{n=1}^{+\infty}>\sum_{n=1}^{+\infty} \frac{1}{1+|\gamma_n|} \ge \sum_{n=1}^{+\infty} \frac{1}{1+\frac{c_3n}{\log{n}}}=\sum_{n=1}^{+\infty} \frac{\log{n}}{\log{n}+c_3n}=+\infty.
  \end{gather*}
\end{proof}
