Introduciamo ora un argomento importante: le funzioni intere di ordine finito.

Notazione: scriviamo che $f(z) \ll g(z)$ $(z \longrightarrow +\infty)$ $\iff$ $f(z)=O\big(g(z)\big)$. Inoltre, scriveremo $\ll_{\epsilon}$ per indicare che la costante dell'$O$-grande dipende da un parametro $\epsilon$.

\begin{defn}
  Data $F$ intera, si dice che ha \textit{ordine} $\alpha \ge 0$ se
  $$F(z) \ll_{\epsilon} e^{|z|^{\alpha+\epsilon}} \, (z \longrightarrow +\infty)$$
  per ogni $\epsilon>0$ e $\alpha$ è il minimo valore positivo per cui vale questa cosa. Equivalentemente, $\alpha=\inf\{A \ge 0 \mid F(z) \ll_A e^{|z|^A}\}$. Scriviamo $ord(F)=\alpha$.

  Se non vale $F(z) \ll_A e^{|z|^A}$ per nessun $A$ diciamo che $F$ ha ordine infinito.
\end{defn}

\begin{ex}
  \begin{enumerate}
    \item Sia $p$ un polinomio di grado $k$, allora $|p(z)| \le |a_k||z|^k\big(1+o(1)\big) \ll_{\epsilon} e^{|z|^{\epsilon}}$ per ogni $\epsilon>0$, dunque $p$ ha ordine $0$.
    \item $|e^{az+b}|=e^{\mathfrak{Re}(az+b)} \ll_{\epsilon} e^{|z|^{1+\epsilon}}$ per ogni $\epsilon>0$.
    \item Più in generale, dato $p_k$ un generico polinomio di grado $k$ abbiamo che $|e^{p_k(z)}|=e^{\mathfrak{Re}\big(p_k(z)\big)} \le e^{|p_k(z)|} \le e^{|a_k||z|^k\big(1+o(1)\big)} \ll_{\epsilon} e^{|z|^{k+\epsilon}}$ per ogni $\epsilon>0$.

    Inoltre, prendendo $z$ sulla retta $arg(z)=-arg(a_k)/k$, abbiamo che vale $|e^{p_k(z)}|=e^{\mathfrak{Re}\big(p_k(z)\big)}=e^{a_kz^k\big(1+o(1)\big)}=e^{|a_k||z|^k\big(1+o(1)\big)} \gg_{\epsilon} e^{|a_k||z|^k}$. Dunque l'ordine è $k$ e l'$\inf$ nella definizione non viene raggiunto.
  \end{enumerate}
\end{ex}

\begin{oss}
  Se $F_1$ e $F_2$ hanno ordine $\alpha_1$ e $\alpha_2$, allora $F_1+F_2$ e $F_1 \cdot F_2$ hanno ordine minore o uguale di $\max\{\alpha_1,\alpha_2\}$. La dimostrazione è lasciata come esercizio per il lettore.
\end{oss}

\begin{lm} \label{1.2.4}
  Sia $f$ olomorfa in $|z-z_0| \le R$ non costantemente nulla e sia $0<r<R$. Sia inoltre $N=\sharp\{z \in \mathbb{C} \mid |z-z_0| \le r, f(z)=0\}$ (ricordiamo che sono contati con molteplicità). Allora
  $$|f(z_0)| \le \left(\frac{r}{R}\right)^N\max_{|z-z_0|=R}|f(z)|.$$
\end{lm}

\begin{proof}
  Consideriamo senza perdita di generalità, a meno di una traslazione e di un'omotetia, $R=1,z_0=0$. Siano $z_n$ gli zeri di $f$ in $|z| \le r$ contati con molteplicità e sia $\displaystyle g(z)=f(z)\prod_{n=1}^N \frac{1-\bar{z}_nz}{z-z_n}$. Per $|z|=1$ scriviamo $z=e^{i\theta},\theta \in \mathbb{R}$.
  Allora $\displaystyle \left|\frac{1-\bar{z}_ne^{i\theta}}{e^{i\theta}-z_n}\right|=|e^{i\theta}|\left|\frac{\bar{z}_n-e^{-i\theta}}{z_n-e^{i\theta}}\right|=1,$ quindi, per il principio del massimo modulo per funzioni olomorfe (che d'ora in avanti useremo senza menzionarlo esplicitamente), $\displaystyle |g(z)| \le \max_{|z|=1} |f(z)|$ per $|z| \le 1$. Si ha dunque
  \begin{gather*}
    |f(w)|=|g(w)|\prod_{n=1}^N\left|\frac{w-z_n}{1-\bar{z}_nw}\right| \le \max_{|z|=1} |f(z)| \prod_{n=1}^N \left|\frac{w-z_n}{1-\bar{z}_nw}\right| \implies \\
    \implies |f(0)| \le \max_{|z|=1} |f(z)| \prod_{n=1}^N |z_n| \le r^N \max_{|z|=1} |f(z)|.
  \end{gather*}
\end{proof}

\begin{cor} \label{1.2.5}
  Siano $f, r, R, N$ come nel lemma \ref{1.2.4}. Se $f(z_0)\not=0$ allora
  $$N \le \frac{1}{\log(R/r)}\log\left(\frac{\max_{|z-z_0|=R}|f(z)|}{|f(z_0)|}\right).$$
\end{cor}

\begin{proof}
  Basta prendere la disugaglianza data dal lemma \ref{1.2.4}, portarla nella forma $\left(\frac{R}{r}\right)^N \le (\dots)$, prendere il logaritmo e dividere per $\log(R/r)$.
\end{proof}

\begin{thm} \label{1.2.6}
  Sia $F$ una funzione intera di ordine $\alpha<+\infty$ e consideriamo $N(r)=\sharp\{z \in \mathbb{C} \mid F(z)=0, |z| \le r\}$. Allora $N(r) \ll_{\epsilon} r^{\alpha+\epsilon}$ per ogni $\epsilon>0$.
\end{thm}

\begin{proof}
  Prendiamo $R=2r$, allora $\displaystyle \max_{|z|=R}|F(z)| \ll_{\epsilon} e^{(2r)^{\alpha+\epsilon}}$ per ogni $\epsilon>0$ $\implies$ $\displaystyle \log\left(\max_{|z|=R}|F(z)|\right) \ll_{\epsilon} r^{\alpha+\epsilon}$ per ogni $\epsilon>0$.
  Se $F(0)\not=0$, per il corollario \ref{1.2.5} abbiamo $N(r) \ll_{\epsilon} r^{\alpha+\epsilon}$ per ogni $\epsilon>0$. Se $F(0)=0$, consideriamo $\tilde{F}(z)=F(z)/z^m$ dove $m$ è la molteplicità di $0$ come zero. Per $|z| \le 1$, $\tilde{F} \ll 1$ per continuità.
  Per $|z|>1$, $\tilde{F}(z)=F(z)\frac{1}{z^n} \implies ord(\tilde{F}) \le \max\{\alpha,0\}=\alpha$. Allora si ripete la dimostrazione per $\tilde{F}$, poi si osserva che il numero di zeri di $F$ varia solo per la costante additiva $m$.
\end{proof}

\begin{defn}
  Sia $z_n\not=0$ una successione senza limiti finiti. Si dice \textit{esponente di convergenza} di $z_n$, se esiste, il numero $$\beta=\inf\{B>0 \mid \sum_{n=1}^{+\infty} \frac{1}{|z_n|^B}<+\infty\},$$
  ovvero $\displaystyle \sum_{n=1}^{+\infty} \frac{1}{|z_n|^{\beta+\epsilon}}<+\infty$ per ogni $\epsilon>0$ e $\beta$ è il minimo valore per cui è vero.
\end{defn}

\begin{ex}
  $z_n=\log{n}$ non ha esponente di convergenza finito.
\end{ex}

\begin{thm} \label{1.2.9}
  Sia $F$ una funzione intera di ordine $\alpha>0$, $F(0)\not=0$ t.c. la successione dei suoi zeri $z_n$ ha esponente di convergenza $\beta$. Allora $\beta \le \alpha$.
\end{thm}

\begin{proof}
  Se $ord(F)=\alpha<+\infty$, per il teorema \ref{1.2.6} si ha $N(r) \ll_{\epsilon} r^{\alpha+\epsilon}$ per ogni $\epsilon>0$.
  Prendendo $r_n=|z_n|$, ricordando che $z_n$ sono gli zeri di $F$ otteniamo $n \le N(r_n) \ll_{\epsilon} r_n^{\alpha+\epsilon}=|z_n|^{\alpha+\epsilon}$ per ogni $\epsilon>0$ (non è $n=N(r_n)$ perché potrebbe esserci più di uno zero sul cerchio $|z|=r_n$). Allora $|z_n| \gg_{\epsilon} n^{\frac{1}{\alpha+\epsilon}}$ e dunque
  $$\sum_{n=1}^{+\infty} |z_n|^{-(\alpha+2\epsilon)} \ll_{\epsilon} \sum_{n=1}^{+\infty} n^{-\frac{\alpha+2\epsilon}{\alpha+\epsilon}}<+\infty.$$
  Perciò $\beta \le \inf\{\alpha+2\epsilon \mid \epsilon>0\}=\alpha$.
\end{proof}

\begin{thm} \label{wfatt}
  Sia $F$ intera di ordine finito, $F(0)\not=0$. Allora la fattorizzazione di Weierstrass si scrive come
  \begin{equation} \label{wfattoformula}
    F(z)=e^{G(z)} \prod_{n=1}^{+\infty} \left(1-\frac{z}{z_n}\right)\exp\Bigg(\sum_{k=1}^p\frac{1}{k}\left(\frac{z}{z_n}\right)^k\Bigg),
  \end{equation}
  dove $p \ge 0$ è indipendente da $n$ e t.c. $\displaystyle \sum_n |z_n|^{-(p+1)}<+\infty$. Talvolta si trova scritta con la notazione $\displaystyle E(z,p)=(1-z)\exp\left(\sum_{k=1}^p\frac{1}{k}z^k\right)$ o simili.
\end{thm}

\begin{proof}
  Diamo solamente una traccia. Osserviamo che
  $$\sum_{n=1}^{+\infty} |m_n|\left|\frac{z}{z_n}\right|^{p_n+1}<+\infty \iff \sum_{n=1}^{+\infty} |z_n|^{-(p+1)}<+\infty\text{ per ogni }z \in \mathbb{C}.$$
  Scegliendo $p+1=\alpha+\epsilon$, per il teorema \ref{1.2.9} si ha $\displaystyle \sum_{n=1}^{+\infty} |z_n|^{-(p+1)}<+\infty$.
\end{proof}

\begin{oss}
  Sia $\beta$ l'esponente di convergenza di $z_n$, zeri di una funzione $F$ di ordine $\alpha$ ($\beta \le \alpha$). Il $p$ migliore è:
  \begin{enumerate}
    \item se $\beta$ non è intero, $p=\lfloor\beta\rfloor$;
    \item se $\beta$ è intero e nella definizione con l'$\inf$ è in realtà un minimo, allora $p=\beta-1$;
    \item se $\beta$ è intero ma nella definizione con l'$\inf$ questo non viene raggiunto, cioè $\beta$ non è un minimo, allora $p=\beta$.
  \end{enumerate}
  Abbiamo dunque $\beta-1 \le p \le \beta \le \alpha$. Scrivendo la fattorizzazione di Weierstrass come in \eqref{wfattoformula} usando il miglior $p$ possibile si ha la forma canonica.
\end{oss}

\begin{thm}
  (Borel-Carathéodory) Siano $0<r<R$ e sia $f$ olomorfa in $|z-z_0| \le R$. Allora
  \begin{equation*}
    \max_{|z-z_0|=r}|f(z)| \le \frac{2r}{R-r}\max_{|z-z_0|=R}\mathfrak{Re}\big(f(z)\big)+\frac{R+r}{R-r}|f(z_0)|.
  \end{equation*}
  Se inoltre $\displaystyle \max_{|z-z_0|=R}\mathfrak{Re}\big(f(z)\big) \ge 0$, allora
  \begin{equation*}
    \max_{|z-z_0|=r}|f^{(n)}(z)| \le \frac{n!2^{n+2}R}{(R-r)^{n+1}}\Big(\max_{|z-z_0|=R}\mathfrak{Re}\big(f(z)\big)+|f(z_0)|\Big).
  \end{equation*}
\end{thm}

\begin{proof}
  Supponiamo senza perdita di generalità $z_0=0$. Se $f$ è costante la tesi è banale, consideriamo dunque $f$ non costante. Facciamo il caso $f(0)=0$. Sia $\displaystyle A=\max_{|z|=R}\mathfrak{Re}\big(f(z)\big)$. $A>0$, infatti basta osservare che $|e^{f(z)}|=e^{\mathfrak{Re}\big(f(z)\big)}$ e applicare il principio del massimo modulo a $e^{f(z)}$ (se il massimo non fosse maggiore di $1$, sarebbe proprio $1$ perché $f(0)=0$ e per lo stesso motivo sarebbe costante). Sia $\varphi(z)=\frac{f(z)}{2A-f(z)}$.
  Osserviamo che $\mathfrak{Re}\big(2A-f(z)\big)=2A-\mathfrak{Re}\big(f(z)\big) \ge A>0 \implies 2A-f(z)\not=0$, quindi $\varphi$ è olomorfa in $|z| \le R$.
  Posto $f(z)=u+iv$, allora $|\varphi(z)|^2=\frac{u^2+v^2}{(2A-u)^2+v^2} \le 1$. Infatti, $u \le A \implies u \le 2A-u$ e ovviamente $-2A+u \le u$, dunque $u^2 \le (2A-u)^2$.
  Poiché $\varphi(0)=0$, $\varphi(z)/z$ è olomorfa in $|z| \le R$ e
  \begin{gather*}
    \left|\frac{\varphi(z)}{z}\right| \le \max_{|z|=R} \left|\frac{\varphi(z)}{z}\right| \le \frac{1}{R} \implies \max_{|z|=r} |\varphi(z)| \le \frac{r}{R} \implies \\
    \implies |f(z)|=\frac{2A|\varphi(z)|}{|1+\varphi(z)|} \le \frac{2Ar/R}{1-|\varphi(z)|} \le \frac{2Ar}{R-r} \text{ per } |z|=r.
  \end{gather*}
  Se $f(0)\not=0$, ripetiamo il ragionamento con $f(z)-f(0)$ ottenendo
  \begin{gather*}
    -|f(0)|+\max_{|z|=r}|f(z)| \le \max_{|z|=r} |f(z)-f(0)| \le \\
    \le \frac{2r}{R-r}\max_{|z|=R}\mathfrak{Re}\big(f(z)-f(0)\big) \le \frac{2r}{R-r}\max_{|z|=R}\mathfrak{Re}\big(f(z)\big)+\frac{2r}{R-r}|f(0)|.
  \end{gather*}
  Sommando $|f(0)|$ agli estremi di questa catena di disugaglianze otteniamo la prima parte della tesi.

  Osserviamo che se $A \ge 0$ si ha
  $$\max_{|z|=r} |f(z)| \le \frac{R+r}{R-r}\Big(\max_{|z|=R}\mathfrak{Re}\big(f(z)\big)+|f(z_0)|\Big).$$
  Prendiamo $|z| \le r$ e scriviamo
  $$f^{(n)}(z)=\frac{n!}{2\pi i} \oint_{|w-z|=\frac{R-r}{2}} \frac{f(w)}{(w-z)^{n+1}}\diff w.$$
  Dev'essere $|w|=|w-z+z| \le |w-z|+|z| \le \frac{R-r}{2}+r=\frac{R+r}{2}<R$. Allora
  \begin{gather*}
    |f(w)| \le \max_{|w|=\frac{R+r}{2}} |f(w)| \le \frac{R+\frac{R+r}{2}}{R-\frac{R+r}{2}}\Big(\max_{|w|=R}\mathfrak{Re}\big(f(w)\big)+|f(0)|\Big) \le \\
    \le \frac{4R}{R-r}\Big(\max_{|w|=R}\mathfrak{Re}\big(f(w)\big)+|f(0)|\Big) \implies \\
    \implies |f^{(n)}(z)| \le \frac{n!}{2\pi}\cdot\frac{4R}{R-r}\cdot\frac{2\pi}{\left(\frac{R-r}{2}\right)^n}\Big(\max_{|w|=R}\mathfrak{Re}\big(f(w)\big)+|f(0)|\Big).
  \end{gather*}
\end{proof}

\begin{thm} \label{hadamard}
  (Hadamard) Se $F$ è intera di ordine $\alpha$ finito e $F(0)\not=0$, allora la funzione intera $G$ del prodotto di Weierstrass \eqref{wprodformula} è un polinomio di grado minore o uguale a $\alpha$.
\end{thm}

\begin{proof}
  Sia $\nu=\lfloor\alpha\rfloor$, vogliamo $G^{(\nu+1)}\equiv 0$. Ricordiamo che
  $$\log{F(z)}=G(z)+\sum_n \log\left(1-\frac{z}{z_n}\right)+\sum_{k=1}^p \frac{1}{k}\left(\frac{z}{z_n}\right)^k.$$
  Derivando troviamo
  \begin{gather*}
    \frac{F'}{F}(z)=G'(z)-\sum_n \frac{1}{z_n-z}+D\Bigg(\sum_{k=1}^p \frac{1}{k}\left(\frac{z}{z_n}\right)^k\Bigg) \\
    D^{\nu}\left(\frac{F'}{F}(z)\right)=G^{(\nu+1)}(z)-\nu!\sum_n \frac{1}{(z_n-z)^{\nu+1}}+0.
  \end{gather*}
  dove lo $0$ in fondo è perché $p \le \alpha$. Fissiamo $R>0$ e definiamo la funzione $\displaystyle \varphi_R(z)=\frac{F(z)}{F(0)}\prod_{|z_n| \le R} \left(1-\frac{z}{z_n}\right)^{-1}$.
  Per $|z_n| \le R$ e $|z|=2R$ si ha
  \begin{gather*}
    \left|1-\frac{z}{z_n}\right| \ge \left|\frac{z}{z_n}\right|-1 \ge 2-1=1 \implies \\
    \implies |\varphi_R(z)| \le \frac{|F(z)|}{|F(0)|} \ll_{\epsilon} e^{(2R)^{\alpha+\epsilon}} \text{ per ogni } \epsilon>0.
  \end{gather*}
  Per il principio del massimo, è vero per ogni $|z| \le 2R$ e di conseguenza abbiamo che $\log{|\varphi_R(z)|} \le C(\epsilon)R^{\alpha+\epsilon}$. Sia $\psi_R(z)=\log{\varphi_R(z)}$; è olomorfa in $|z| \le R$ e $\psi_R(0)=0$. Si ha $\mathfrak{Re}\psi_R(z)=\log{|\varphi_R(z)|}$, dunque per il teorema di Borel-Carathéodory
  $$\max_{|z|=R/2}|\psi_R^{(\nu+1)}(z)| \ll_{\epsilon,\nu} \frac{R}{R^{\nu+2}}R^{\alpha+\epsilon}=R^{\alpha-\nu-1+\epsilon}.$$
  Sia $\epsilon>0$ t.c. $\alpha-\nu-1+\epsilon<0$. Allora
  \begin{gather*}
    \psi'_R(z)=\frac{\varphi'_R(z)}{\varphi_R(z)}+\sum_{|z_n| \le R} \frac{1}{z_n-z} \\
    \psi_R^{(\nu+1)}(z)=D^{\nu}\left(\frac{F'}{F}(z)\right)+\sum_{|z_n| \le R} \frac{\nu!}{(z_n-z)^{\nu+1}}=G^{(\nu+1)}(z)-\sum_{|z_n|>R} \frac{\nu!}{(z_n-z)^{\nu+1}}.
  \end{gather*}
  Supponiamo $|z|=R/2$, si ha
  $$\left|\sum_{|z_n|>R} \frac{\nu!}{(z_n-z)^{\nu+1}}\right| \le C_1(\nu) \sum_{|z_n|>R} \frac{1}{|z_n|^{\nu+1}}=o(1) \text{ per } R \longrightarrow +\infty.$$
  Si ha allora $|G^{(\nu+1)}(z)| \le C(\nu,\epsilon)R^{-\delta}+o(1)$ per $|z|=R/2$ e dunque anche per $|z| \le R/2$. Basta fissare $z$ e mandare $R \longrightarrow +\infty$.
\end{proof}

\begin{cor} \label{1.2.14}
  Sia $G$ come nel teorema di Hadamard. Se $q \le p$ si ha
  $$G(z)=\log{F(0)}+\sum_{k=1}^q D^{k-1}\left(\frac{F'}{F}(0)\right)_{z=0} \frac{z^k}{k!};$$
  se invece $p<q$ si ha
  $$G(z)=\log{F(0)}+\sum_{k=1}^q D^{k-1}\left(\frac{F'}{F}(0)\right)_{z=0}\frac{z^k}{k!}+\sum_{k=p+1}^q\left(\sum_n z_n^{-k}\right)\frac{z^k}{k}.$$
  Inoltre, per $k>\max\{p,q\}$ si ha
  $$\sum_n z_n^{-k}=-\frac{1}{(k-1)!}D^{k-1}\left(\frac{F'}{F}(z)\right)_{z=0}.$$
\end{cor}

\begin{proof}
  Per il corollario \ref{1.1.8} si ha
  $$G(z)=\log{F(0)}+\sum_{k=1}^{+\infty}\Bigg(\frac{1}{k!}D^{k-1}\left(\frac{F'}{F}(z)\right)_{z=0}+\frac{1}{k}\sum_{n:p_n<k}m_nz_n^{-k}\Bigg)z^k,$$
  con $p_n=p$ per ogni $n$. Abbiamo che $G$ è un polinomio di grado $q$. Se $q \le p$, si ha $k \le q \le p$, quindi non c'è il termine $\displaystyle \frac{1}{k}\sum_{n:p_n<k}m_nz_n^{-k}$, come voluto.

  Se $q>p$, abbiamo il termine $\displaystyle \frac{1}{k}\sum_n m_nz_n^{-k}$ per $p<k \le q$, cioè il termine $\displaystyle \sum_{k=p+1}^q\left(\sum_n z_n^{-k}\right)z^k$.

  Se $k>\max\{p,q\}$, non c'è il termine $z^k$ (nel quale la somma, poiché vale $k>p=p_n$, è fatta su tutti gli $n$). Allora si ha
  $$\frac{1}{k!}D^{k-1}\left(\frac{F'}{F}(z)\right)_{z=0}+\frac{1}{k}\sum_n z_n^{-k}$$
  che è ovviamente equivalente alla tesi.
\end{proof}

\begin{thm} \label{1.2.15}
  Sia $z_n\not=0$ una successione con esponente di convergenza $\beta$ finito e $p$ il minimo intero t.c. $\displaystyle \sum_{n=1}^{+\infty} |z_n|^{-(p+1)}<+\infty$. Sappiamo già che il prodotto di Weierstrass in forma canonica,
  $$\prod_{n=1}^{+\infty} \left(1-\frac{z}{z_n}\right)\exp\Bigg(\sum_{k=1}^p\frac{1}{k}\left(\frac{z}{z_n}\right)^k\Bigg),$$
  è uniformemente convergente, in particolare è una funzione intera. L'ordine di tale funzione, finito, è proprio $\beta$.
\end{thm}

\begin{proof}
  Sia $\displaystyle F(z)=\prod_nE\left(\frac{z}{z_n},p\right)$, Vogliamo mostrare che
  $$\log\left|\prod_nE\left(\frac{z}{z_n},p\right)\right|=\sum_n\log\left|E\left(\frac{z}{z_n},p\right)\right| \le C(\epsilon)|z|^{\beta+\epsilon} \text{ per ogni } \epsilon>0.$$
  Distinguiamo due casi.

  Caso $|z/z_n| \le 1/2$. Allora
  \begin{gather*}
    \log\left|E\left(\frac{z}{z_n},p\right)\right| \le \left|\log{E\left(\frac{z}{z_n},p\right)}\right|= \\
    =\left|\log\left(1-\frac{z}{z_n}\right)+\sum_{k=1}^p \frac{1}{k}\left(\frac{z}{z_n}\right)^k\right|=\left|\sum_{k=p+1}^{+\infty} \frac{1}{k} \left(\frac{z}{z_n}\right)^k\right| \le \\
    \le \sum_{k=p+1}^{+\infty} \left|\frac{z}{z_n}\right|^k \le \left|\frac{z}{z_n}\right|^{p+1}(1+1/2+1/4+\dots)=2\left|\frac{z}{z_n}\right|^{p+1} \implies \\
    \implies \sum_{|z_n| \ge 2|z|} \log\left|E\left(\frac{z}{z_n},p\right)\right| \le 2|z|^{p+1} \sum_{|z_n| \ge 2|z|} \frac{1}{|z_n|^{p+1}}.
  \end{gather*}
  Adesso, se $\beta=p+1$ l'ultima quantità è uguale a $\displaystyle 2|z|^{\beta}\sum_{|z_n| \ge 2|z|} \frac{1}{|z_n|^{p+1}} \le C|z|^{\beta}$. Se invece $\beta<p+1$ prendiamo $\epsilon<p+1-\beta$ per ottenere
  \begin{gather*}
    2|z|^{\beta+\epsilon}\sum_{|z_n| \ge 2|z|} \frac{1}{|z_n|^{p+1}}|z|^{p+1-\beta-\epsilon} \le \\
    \le \frac{|z|^{\beta+\epsilon}}{2^{p-\beta-\epsilon}}\sum_{|z_n| \ge 2|z|} \frac{1}{|z_n|^{\beta+\epsilon}} \le C(\epsilon)|z|^{\beta+\epsilon}.
  \end{gather*}

  Caso $|z/z_n|>1/2$. Allora
  \begin{gather*}
    \log\left|E\left(\frac{z}{z_n},p\right)\right|=\log\left|1-\frac{z}{z_n}\right|+\mathfrak{Re}\Bigg(\sum_{k=1}^p \frac{1}{k}\left(\frac{z}{z_n}\right)^k\Bigg) \le \\
    \le \log\left(1+\left|\frac{z}{z_n}\right|\right)+\sum_{k=1}^p \left|\frac{z}{z_n}\right|^k.
  \end{gather*}
  Adesso, se $p=0$ l'ultima quantità è $\le C(\epsilon)\left|\dfrac{z}{z_n}\right|^{\epsilon}$ per ogni $\epsilon>0$, dunque
  \begin{gather*}
    \sum_{|z_n|<2|z|}\log\left|E\left(\frac{z}{z_n},p\right)\right| \le C(\epsilon)|z|^{\epsilon}\sum_{|z_n|<2|z|}|z_n|^{-\epsilon}= \\
    =C(\epsilon)|z|^{\beta+\epsilon}\sum_{|z_n|<2|z|}|z_n|^{-\epsilon}|z|^{-\beta} \le \\
    C(\epsilon)2^{\beta}|z|^{\beta+\epsilon}\sum_{|z_n|<2|z|}\frac{1}{|z_n|^{\beta+\epsilon}} \le \tilde{C}(\epsilon)|z|^{\beta+\epsilon}.
  \end{gather*}
  Se invece $p \ge 1$, la quantità di prima è $\le C\left|\dfrac{z}{z_n}\right|^p$, dunque
  \begin{gather*}
    \sum_{|z_n|<2|z|}\log\left|E\left(\frac{z}{z_n},p\right)\right| \le C|z|^p \sum_{|z_n|<2|z|}|z_n|^{-p}= \\
    =C|z|^{\beta+\epsilon}\sum_{|z_n|<2|z|}|z_n|^{-p}|z|^{p-\beta-\epsilon} \le \\
    \le C2^{\beta+\epsilon-p}|z|^{\beta+\epsilon}\sum_{|z_n|<2|z|}\frac{1}{|z_n|^{\beta+\epsilon}} \le C(\epsilon)|z|^{\beta+\epsilon}.
  \end{gather*}
  È lasciato al lettore il divertentissimo esercizio di verificare che le varie costanti moltiplicative (i cui nomi non sono stati scelti con troppa cura dall'autore) non creano davvero problemi, cioè che sono tutte maggiorate da una costante che dipende solo da $\epsilon$ e non da $|z|$. Abbiamo adesso $\alpha \le \beta$ e l'altra disugaglianza già la sapevamo.
\end{proof}

\begin{cor}
  Se la funzione
  $$F(z)=e^{G(z)}\prod_{n=1}^{+\infty}\left(1-\frac{z}{z_n}\right)\exp\Bigg(\sum_{k=1}^p\frac{1}{k}\left(\frac{z}{z_n}\right)^k\Bigg)$$
  ha ordine $\alpha$, la successione $z_n\not=0$ ha esponente di convergenza $\beta$ e $G$ è un polinomio di grado $q$, allora $\alpha=\max\{q,\beta\}$.
\end{cor}

\begin{proof}
  $e^{G(z)}$ ha ordine $q$ e il prodotto, per il teorema \ref{1.2.15}, ha ordine $\beta$. Poiché l'ordine del prodotto di due funzioni è minore o uguale del massimo fra i due ordini, allora $\alpha \le \max\{q,\beta\}$. Ma abbiamo già visto che $\alpha \ge \beta$ e per il teorema di Hadamard \ref{hadamard} si ha anche $\alpha \ge q$.
\end{proof}
