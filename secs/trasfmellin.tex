\begin{defn}
  Sia $f:\mathbb{R}^+ \longrightarrow \mathbb{R}$ t.c. $\displaystyle \int_0^{+\infty} |f(x)|x^{\sigma}\frac{\diff x}{x}<+\infty$ per $\sigma \in \mathbb{R}$ fissato. Si dice \textit{trasformata di Mellin} di $f$ la seguente:
  $$\wideparen{f}(s)=\int_0^{+\infty} f(x)x^s\frac{\diff x}{x}\text{ con }s=\sigma+it.$$
\end{defn}

\begin{ex}
  Se $f(x)=e^{-x}$ e $\sigma>0$, allora $\wideparen{f}(s)=\Gamma(s)$.
\end{ex}

\begin{oss} \label{melfou}
  Sia $x=e^{-2\pi u} \implies \frac{\diff x}{x}=-2\pi\diff u$. Allora
  \begin{gather*}
    \wideparen{f}(\sigma+it)=2\pi\int_{-\infty}^{+\infty} f(e^{-2\pi u})e^{-2\pi u\sigma-2\pi itu}\diff u= \\
    =2\pi \int_{-\infty}^{+\infty} \phi_{\sigma}(u)e^{-2\pi itu}\diff u=2\pi\hat{\phi}_{\sigma}(t).
  \end{gather*}
\end{oss}

\begin{prop} \label{mellinv}
  Se $f$ è di classe $C^1$ e $\displaystyle \int_0^{+\infty} |f(x)|x^{\sigma}\frac{\diff x}{x}<+\infty$, si ha
  $$f(x)=\frac{1}{2\pi i} \int_{\sigma-i\infty}^{\sigma+i\infty} \wideparen{f}(s)x^{-s}\diff s.$$
\end{prop}

\begin{proof}
  Dall'osservazione \ref{melfou} abbiamo
  \begin{gather*}
    f(e^{-2\pi u})e^{-2\pi\sigma u}=\varphi_{\sigma}(u)=\widehat{\widehat{\varphi}}_{\sigma}(-u)=\frac{1}{2\pi}\int_{-\infty}^{+\infty} \wideparen{f}(\sigma+it)e^{2\pi iut}\diff t \implies \\
    \implies f(e^{-2\pi u})=\frac{1}{2\pi}\int_{-\infty}^{+\infty} \wideparen{f}(\sigma+it)e^{2\pi(\sigma+it)u}\diff t=\frac{1}{2\pi i}\int_{\sigma-i\infty}^{\sigma+i\infty} \wideparen{f}(s)e^{2\pi su}\diff s.
  \end{gather*}
  Basta porre $x=e^{-2\pi u}$.
\end{proof}

\begin{oss}
  Se $f$ è $C^1$ a tratti con limite destro e sinistro finiti nei punti di discontinuità, la proposizione \ref{mellinv} continua a valere purché si applichi a $\tilde{f}$, che coincide con $f$ a parte nei punti di discontinuità dove è uguale alla media dei due limiti, e interpretando l'integrale come limite. In tutti i casi si prende $\sigma>0$.
\end{oss}

\begin{oss}
  Data $g(s)$ olomorfa in $\sigma_1 < \sigma < \sigma_2$ che sia anche continua in $\sigma_1 \le \sigma \le \sigma_2$ e t.c. $g(\sigma+it) \overset{|t| \longrightarrow +\infty}{\longrightarrow} 0$ per ogni $\sigma_1 \le \sigma \le \sigma_2$, si può dimostrare che $\displaystyle f(x)=\frac{1}{2\pi i}\int_{\sigma-i\infty}^{\sigma+i\infty} g(s)x^{-s}\diff s$ non dipende da $\sigma$ e si ha $g(s)=\wideparen{f}(s)$.
\end{oss}

\begin{ex}
  Vediamo la trasformata di Mellin di alcune funzioni, poi applichiamo la formula di inversione scrivendo però $y=1/x$. Nel secondo e terzo esempio, l'integrale della trasformata si fa iterando per parti.
  \begin{enumerate}
    \item
    \begin{gather*}
      f(x)=\begin{cases}
        1 &\mbox{se } 0<x<1 \\
        1/2 &\mbox{se } x=1 \\
        0 &\mbox{se } x>1
    \end{cases} \implies \\
    \wideparen{f}(s)=\frac{1}{s}, \quad \frac{1}{2\pi i}\lim_{T \longrightarrow +\infty} \int_{\sigma-iT}^{\sigma+iT} \frac{y^s}{s}\diff s=\begin{cases}
      1 &\mbox{se } y>1 \\
      1/2 &\mbox{se } y=1 \\
      0 &\mbox{se } 0<y<1.
    \end{cases}
  \end{gather*}
  \item
  \begin{gather*}
    f(x)=\begin{cases}
      \dfrac{(1-x)^k}{k!} &\mbox{se } 0<x \le 1 \\
      0 &\mbox{se } x \ge 1
      \end{cases} \implies \\
      \wideparen{f}(s)=\frac{1}{s(s+1)\dots(s+k)}, \\ \frac{1}{2\pi i} \int_{\sigma-i\infty}^{\sigma+i\infty} \frac{y^s}{s(s+1)\dots(s+k)}\diff s=\begin{cases}
        \dfrac{1}{k!}\left(1-\dfrac{1}{y}\right)^k &\mbox{se } y \ge 1 \\
        0 &\mbox{se } 0<y \le 1.
      \end{cases}
    \end{gather*}
    \item
    \begin{gather*}
      f(x)=\begin{cases}
        \dfrac{(-\log{x})^k}{k!} &\mbox{se } 0<x \le 1 \\
        0 &\mbox{se } x \ge 1
      \end{cases} \implies \\
      \wideparen{f}(s)=\frac{1}{s^{k+1}}, \quad \frac{1}{2\pi i} \int_{\sigma-i\infty}^{\sigma+i\infty} \frac{y^s}{s^{k+1}}\diff s=\begin{cases}
        \dfrac{(\log{y})^k}{k!} &\mbox{se } y \ge 1 \\
        0 &\mbox{se } 0<y \le 1.
      \end{cases}
    \end{gather*}
  \end{enumerate}
\end{ex}

Nell'ultimo caso, prendendo $k=1$ e $y=x/n$, per $\sigma>1$ si ha
$$\sum_{n \le x} \Lambda(n)\log\left(\frac{x}{n}\right)=\frac{1}{2\pi i}\int_{\sigma-i\infty}^{\sigma+i\infty} \left(\sum_{n=1}^{+\infty} \frac{\Lambda(n)}{n^s}\right)\frac{x^s}{s^2}\diff s=\frac{1}{2\pi i}\int_{\sigma-i\infty}^{\sigma+i\infty} -\frac{\zeta'}{\zeta}(s)\frac{x^s}{s^2}\diff s.$$

\begin{oss} \label{psiuno}
  Sommando per parti otteniamo
  \begin{gather*}
    \sum_{n \le x} \Lambda(n)\log\left(\frac{x}{n}\right)=\sum_{n \le x} \Lambda(n) \cdot 0+\int_2^x \sum_{n \le u} \Lambda(u)\frac{\diff u}{u}= \\
    =\int_2^x \frac{\psi(u)}{u}\diff u=:\psi_1(u).
  \end{gather*}
\end{oss}

\begin{oss}
  Posto $\displaystyle \psi_l(x)=\int_2^x \frac{\psi_{l-1}(u)}{u}\diff u$ per $l \ge 2$, si ha
  $$\psi_k(x)=\frac{1}{k!}\sum_{n \le x} \Lambda(n)\Bigg(\log\left(\frac{x}{n}\right)\Bigg)^k.$$
\end{oss}

\begin{prop}
  Vale la seguente formula esplicita:
  $$\psi_1(x)=x-\sum_{\rho} \frac{x^{\rho}}{\rho^2}-\frac{\zeta'}{\zeta}(0)\log{x}-\left(\frac{\zeta'}{\zeta}\right)'(0)-\frac{1}{4}\sum_{n=1}^{+\infty} \frac{x^{-2n}}{n^2}.$$
\end{prop}

\begin{proof}
  Basta combinare l'osservazione \ref{psiuno} e la formula vista subito prima per poi applicare il teorema dei residui.
\end{proof}

\begin{cor}
  $\psi_1(x)=x+O\big(x\exp(-c\sqrt{\log{x}})\big)$.
\end{cor}

\begin{proof}
  Basta applicare de la Vallée-Poussin come si è fatto per $\psi_0$.
\end{proof}

Se valesse RH si otterrebbe $x+O(\sqrt{x})$, invece con QRH $x+O(x^{\theta})$.

Mostriamo adesso che $\psi_1(x)=x+o(x)$. Dalla formula esplicita si ha
$$\left|\sum_{\rho} \frac{x^{\rho}}{\rho^2}\right| \le \sum_{\rho} \frac{x^{\beta}}{|\rho|^2} \implies \frac{\psi_1(x)-x}{x} \ll \sum_{\rho} \frac{x^{\beta-1}}{|\rho|^2}.$$
Poiché $\beta<1$, si ha
$$\lim_{x \longrightarrow +\infty} \sum_{\rho} \frac{x^{\beta-1}}{|\rho|^2}=\sum_{\rho} \frac{1}{|\rho|^2} \lim_{x \longrightarrow +\infty} x^{\beta-1}=0.$$

\begin{oss} \label{xpiuox}
  Per $x \ge 2$ e $h \le x$ abbiamo
  \begin{gather*}
    \int_x^{x+h} \frac{\psi(u)}{u}\diff u \ge \frac{h}{x+h}\psi(x), \quad \int_{x-h}^x \frac{\psi(u)}{u}\diff u \le \frac{h}{x-h}\psi(x) \implies \\
    \implies (x-h)\frac{\psi_1(x)-\psi_1(x-h)}{h} \le \psi(x) \le (x+h)\frac{\psi_1(x+h)-\psi_1(x)}{h}.
  \end{gather*}
\end{oss}

\begin{prop}
  Si ha $\psi(x)=x+o(x)$.
\end{prop}

\begin{proof}
  \begin{gather*}
    \psi_1(y) \sim y \implies (x-h)\frac{\psi_1(x)-\psi_1(x-h)}{h} \sim (x-h)\frac{x-x+h}{h}\sim x-h, \\
    (x+h)\frac{\psi_1(x+h)-\psi_1(x)}{h} \sim x+h.
  \end{gather*}
  Basta prendere $h=o(x)$ (con $h=1$ costante si va sul sicuro) e dall'osservazione \ref{xpiuox} si ha
  $$x \sim f(x) \le \psi(x) \le g(x) \sim x \implies \psi(x) \sim x.$$
\end{proof}

Non dimostreremo il seguente risultato, che si trova nel capitolo 3 di \cite{T}.

\begin{thm}
  (Landau) Se si hanno $\phi(t), \theta(t)$ t.c. $\phi$ è monotona non decrescente e tende a $+\infty$, $\theta$ è monotona non crescente e $0 < \theta(t) \le 1$, inoltre $\dfrac{\phi(t)}{\theta(t)}=o(e^{\phi(t)})$ e per $1-\theta(t) \le \sigma \le 2$ si ha $\zeta(s) \ll e^{\phi(t)}$, allora esiste $c_0$ t.c.
  $$\beta<1-\frac{\theta(2t+1)}{c_0\phi(2t+1)}.$$
  In particolare, prendendo $\phi(t)=\log{t}$ e $\theta(t)=1/2$, si ottiene $\beta<1-\dfrac{1}{2c_0\log(2t+1)}$.
\end{thm}

Littlewood: $\theta(t)=\frac{(\log\log{t})^2}{\log{t}}, \phi(t)=A\log\log{t} \implies \beta<1-\frac{C\log\log{t}}{\log{t}}$.

Vinogradov: $\theta(t)=\frac{A}{(\log{t})^{2/3-2\epsilon}}, \phi(t)=(\log{t})^{\epsilon}, \zeta(s) \ll \exp\big((\log{t})^{\epsilon}\big) \implies$ \\
$\implies \beta<1-\frac{C}{(\log{t})^{2/3-\epsilon}}$.

Adesso altre cose che non vedremo nel dettaglio.

\begin{enumerate}
  \item $\displaystyle \sum_{N<n \le 2N} e^{2\pi i\alpha n} \ll \frac{1}{\|\alpha\|}$ con $0<\alpha<1$ e $\|\alpha\|=\min\big\{|\alpha-n|, n \in \mathbb{N}\cup\{0\}\big\}$;
  \item $\displaystyle \sum_{N<n \le 2N} \Lambda(n)e^{2\pi i\alpha n} \ll \left(\frac{N}{\sqrt{q}}+\sqrt{Nq}+N^{3/4}\right)\log^4{N}$ dove $\left|\alpha-\frac{a}{q}\right| \le \frac{1}{q^2}$.
  Con questo risultato si riesce a dimostrare il teorema di Vinogradov (1930): $2N+1=p_1+p_2+p_3$ (è la versione per i dispari della congettura di Goldbach);
  \item $\displaystyle \sum_{N<n \le 2N} n^{-it}$ (Vinogradov, 1958). Questo porta a delle stime per $\zeta(\sigma+it)$ quando $1-\sigma \ll \frac{1}{(\log{t})^{2/3-\epsilon}}$;
  \item c'è una stima per $\displaystyle \sum_{N<n \le 2N} \frac{\Lambda(n)}{n^{it}}$? Non davvero, però vale il seguente risultato dovuto a Turán: se $\displaystyle \sum_{N<n \le 2N} \frac{\Lambda(n)}{n^{it}} \ll \frac{N}{t^b}$ per $N \ge t^a$, allora $\beta<1-\frac{b^2}{a^3}$ (correggere i coefficienti di $a$ e $b$ se e quando lo ridice).
\end{enumerate}
