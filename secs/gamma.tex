Adesso, un caso particolare di un teorema che non dimostreremo; lo utilizzeremo per dimostrare una proposizione che potrebbe essere dimostrata anche con il lemma di sommazione di Abel.

\begin{thm}
  (formula di sommazione di Eulero-Maclaurin) Consideriamo $f:[a,b] \longrightarrow \mathbb{C}$ di classe $C^1$. Allora
  $$\sum_{a<k \le b}f(k)=\int_a^b f(x)\diff x-\bigg[B_1(\{x\})f(x)\bigg]_a^b+\int_a^b B_1(\{x\})f'(x)\diff x.$$
  Se $a=m, b=n$,
  $$\sum_{m<k \le n} f(k)=\int_m^nf(x)\diff x-B_1\cdot\big(f(n)-f(m)\big)+\int_m^n B_1(\{x\})f'(x)\diff x.$$
\end{thm}

\begin{cor} \label{EuMac-cor}
  $$\sum_{k=m}^n f(k)=\int_m^n f(x)\diff x+\frac{f(m)+f(n)}{2}+\int_m^n B_1(\{x\})f'(x)\diff x,$$
  dove ricordiamo che $B_1(\{x\})=\{x\}-1/2$.
\end{cor}

Per completezza, riportiamo anche il lemma di Abel.

\begin{lm}
  (formula di sommazione di Abel)
  $$\sum_{k=m}^n a_kf(k)=\left(\sum_{k=m}^n a_k\right)f(n)-\int_m^n\left(\sum_{m \le k \le \lfloor x \rfloor}a_k\right)f'(x)\diff x.$$
\end{lm}

\begin{prop}
  Esiste
  $$\lim_{n \longrightarrow +\infty} \sum_{k=1}^n \frac{1}{k}-\log{n}=\gamma,$$
  con $0<\gamma<1$.
\end{prop}

\begin{proof}
  Applichiamo il corollario \ref{EuMac-cor} con $m=1$ e $f(x)=1/x$, otteniamo
  \begin{gather*}
    \sum_{k=1}^n \frac{1}{k}=\int_1^n \frac{\diff x}{x}+\frac{1}{2}+\frac{1}{2n}-\int_1^n \frac{\{x\}-1/2}{x^2}\diff x= \\
    =\log{n}+\frac{1}{2}+\frac{1}{2n}-\int_1^{+\infty} \frac{\{x\}-1/2}{x^2}\diff x+\int_n^{+\infty}\frac{\{x\}-1/2}{x^2}\diff x= \\
    =\log{n}+\frac{1}{2}+\frac{1}{2n}-\int_1^{+\infty} \frac{\{x\}}{x^2}\diff x+\frac{1}{2}+O\left(\int_n^{+\infty}\frac{\diff x}{x^2}\right)= \\
    =\log{n}+1-\int_1^{+\infty} \frac{\{x\}}{x^2}\diff x+O(1/n)=:\log{n}+\gamma+O(1/n).
  \end{gather*}
  Poiché $\displaystyle 0<\int_1^{+\infty} \frac{\{x\}}{x^2}\diff x<1$, si ha $0<\gamma<1$ con $\displaystyle \gamma=\lim_{n \longrightarrow +\infty} \sum_{k=1}^n \frac{1}{k}-\log{n}$.
\end{proof}

\begin{oss}
  Se $f:[1,+\infty) \longrightarrow \mathbb{R}$ è di classe $C^1$, infinitesima e non crescente, allora esiste $C>0$ t.c. $\displaystyle \sum_{1 \le k \le x} f(k)=\int_1^x f(y)\diff y+C+O\big(f(x)\big)$. La dimostrazione è lasciata per esercizio.
\end{oss}

\begin{oss}
  Si può dimostrare che
  $$\sum_{k=1}^n \frac{1}{k}=\log{n}+\gamma+\frac{1}{2n}-\sum_{r=1}^q \frac{B_{2r}}{2r}\cdot\frac{1}{n^{2r}}+O\left(\frac{1}{n^{2q+2}}\right).$$
\end{oss}

\begin{defn}
  Sia $\dfrac{1}{z\Gamma(z)}$ la funzione intera $F$ di ordine $1$ con zeri tutti semplici nei punti $-1,-2,-3,dots$ e t.c. $F(0)=1$ e $F'(0)=\gamma$.
\end{defn}

Deve essere $q=\deg{G} \le 1$, $\beta=1$ e $\alpha=1$. $G(z)=\log\big(F(0)\big)+\frac{F'}{F}(0)=\gamma z$, quindi $q=1$. Vale anche $p=1$ e $g=1$. Si ha dunque
$$\frac{1}{z\Gamma(z)}=e^{\gamma z}\prod_{n=1}^{+\infty}\left(1+\frac{z}{n}\right)e^{-z/n}.$$

\begin{prop}
  (formula di Gauss) Per $z\not=-m$ con $m \in \mathbb{N}\cup\{0\}$ si ha
  \begin{equation}
    \Gamma(z)=\lim_{n \longrightarrow +\infty} \frac{n^zn!}{z(z+1)\cdots(z+n)}.
  \end{equation}
\end{prop}

\begin{proof}
  \begin{gather*}
    \frac{1}{\Gamma(z)}=ze^{\gamma z}\prod_{n=1}^{+\infty}\left(1+\frac{z}{n}\right)e^{-z/n}=\\
    =z\lim_{n \longrightarrow +\infty} e^{\left(\sum_{k=1}^n\frac{1}{k}-\log{n}\right)}\prod_{k=1}^n\left(1+\frac{z}{n}\right)e^{-z/k}=z\lim_{n \longrightarrow +\infty} \prod_{k=1}^n\left(1+\frac{z}{k}\right)n^{-z} \implies \\
    \implies \Gamma(z)=\lim_{n \longrightarrow +\infty} \frac{n^z\prod_{k=1}^n k}{z\prod_{k=1}^n (k+z)}=\lim_{n \longrightarrow +\infty} \frac{n^zn!}{z(z+1)\cdots(z+n)}.
  \end{gather*}
\end{proof}

\begin{prop}
  Per $k \in \mathbb{N}\cup\{0\}$ si ha $\underset{z=-k}{\normalfont{\text{Res}}}\Gamma(z)=\dfrac{(-1)^k}{k!}$.
\end{prop}

\begin{proof}
  Fissato $k$, consideriamo il limite a partire da $n \ge k$. Vogliamo calcolare $\big(\Gamma(z)(z+k)\big)_{z=-k}$. Si ha
  \begin{gather*}
    \left(\frac{n^zn!(z+k)}{z(z+1)\cdots(z+n)}\right)_{z=-k}=\frac{n^{-k}n!}{-k(-k+1)\cdots(-2)\cdot(-1)\cdot1\cdot2\cdots(n-k)}= \\
    =\frac{(-1)^k}{k!}\left(\frac{n!}{(n-k)!}n^k\right) \longrightarrow \frac{(-1)^k}{k!} \text{ per } n \longrightarrow +\infty.
  \end{gather*}
\end{proof}

\begin{prop} \label{eulgamma}
  (Eulero) Per $z \not=-m$ con $m \in \mathbb{N}\cup\{0\}$ si ha la seguente formula:
  \begin{equation}
    \Gamma(z)=\frac{1}{z}\prod_{n=1}^{+\infty}\left(1+\frac{1}{n}\right)^z\left(1+\frac{z}{n}\right)^{-1}.
  \end{equation}
\end{prop}

\begin{proof}
  Dalla dimostrazione della formula di Gauss abbiamo trovato $\displaystyle \frac{1}{\Gamma(z)}=z\lim_{n \longrightarrow +\infty}\prod_{k=1}^n\left(1+\frac{z}{k}\right)n^{-z}$. Da $\displaystyle n=\prod_{k=1}^{n-1} \frac{k+1}{k}$ otteniamo $\displaystyle n^{-z}=\prod_{k=1}^{n-1}\left(1+\frac{1}{k}\right)^{-z}$. Si ha dunque
  \begin{gather*}
    z\lim_{n \longrightarrow +\infty}\prod_{k=1}^n\left(1+\frac{z}{k}\right)n^{-z}=z\lim_{n \longrightarrow +\infty} \prod_{k=1}^{n-1}\left(1+\frac{1}{k}\right)^{-z}\prod_{k=1}^n\left(1+\frac{z}{k}\right)=\\
    =z\lim_{n \longrightarrow +\infty} \prod_{k=1}^n \left(1+\frac{1}{k}\right)^{-z}\left(1+\frac{z}{k}\right)\left(1+\frac{1}{n}\right)^z=z\prod_{n=1}^{+\infty} \left(1+\frac{1}{n}\right)^{-z}\left(1+\frac{z}{n}\right).
  \end{gather*}
  Perciò
  $$\Gamma(z)=\frac{1}{z}\prod_{n=1}^{+\infty} \left(1+\frac{1}{n}\right)^z\left(1+\frac{z}{n}\right)^{-1}.$$
\end{proof}

\begin{prop} \label{fatt}
  Si ha $\Gamma(z+1)=z\Gamma(z)$ per $z \in \mathbb{C}$.
\end{prop}

\begin{proof}
  Dalla proposizione \ref{eulgamma} abbiamo
  \begin{gather*}
    \Gamma(z+1)=\frac{1}{z+1}\prod_{n=1}^{+\infty} \left(1+\frac{1}{n}\right)^z\left(1+\frac{1}{n}\right)\left(1+\frac{z+1}{n}\right)^{-1}= \\
    =\frac{1}{z+1}\prod_{n=1}^{+\infty}\left(1+\frac{1}{n}\right)^z\left(\frac{n+1+z}{n}\cdot\frac{n}{n+1}\right)^{-1}=\\
    =\frac{1}{z+1}\prod_{n=1}^{+\infty}\left(1+\frac{1}{n}\right)^z\left(1+\frac{z}{n+1}\right)^{-1}=\\
    =\lim_{n \longrightarrow +\infty} \prod_{k=1}^n \left(1+\frac{1}{k}\right)^z\frac{1}{z+1}\prod_{k=1}^n\left(1+\frac{z}{k+1}\right)^{-1}=\\
    =\lim_{n \longrightarrow +\infty}\prod_{k=1}^n\left(1+\frac{1}{k}\right)^z\frac{1}{z+1}\prod_{k=2}^{n+1}\left(1+\frac{z}{k}\right)^{-1}=\\
    =\lim_{n \longrightarrow +\infty}\prod_{k=1}^n\left(1+\frac{1}{k}\right)^z\prod_{k=1}^{n+1}\left(1+\frac{z}{k}\right)^{-1}= \\
    =\lim_{n \longrightarrow +\infty}\prod_{k=1}^n\left(1+\frac{1}{k}\right)^z\prod_{k=1}^n\left(1+\frac{z}{k}\right)^{-1}\frac{n+1}{n+1+z}=\\
    =\lim_{n \longrightarrow +\infty}\prod_{k=1}^n\left(1+\frac{1}{k}\right)^z\prod_{k=1}^n\left(1+\frac{z}{k}\right)^{-1}=\prod_{n=1}^{+\infty}\left(1+\frac{1}{n}\right)^z\left(1+\frac{z}{n}\right)^{-1}=z\Gamma(z).
  \end{gather*}
\end{proof}

\begin{oss}
  Abbiamo
  $$\Gamma(n+1)=n\Gamma(n)=n(n-1)\Gamma(n-1)=\dots=n(n-1)\cdots2\cdot\Gamma(1)=n!.$$
\end{oss}

\begin{prop} \label{gammaze1-z}
  Per $z \in \mathbb{C}$ si ha la relazione
  \begin{equation}
    \Gamma(z)\Gamma(1-z)=\frac{\pi}{\sin(\pi z)}.
  \end{equation}
\end{prop}

\begin{proof}
  Per le proposizioni \ref{fatt} e \ref{eulgamma} si ha
  \begin{gather*}
    z\Gamma(z)\Gamma(1-z)=\Gamma(1+z)\Gamma(1-z)=\\
    =\frac{1}{1-z^2}\prod_{n=1}^{+\infty} \left(1+\frac{1}{n}\right)^{z+1}\left(1+\frac{1}{n}\right)^{1-z}\left(1+\frac{z+1}{n}\right)^{-1}\left(1+\frac{1-z}{n}\right)^{-1}=\\
    =\frac{1}{1-z^2}\prod_{n=1}^{+\infty}\left(1+\frac{1}{n}\right)^2\Bigg(\left(1+\frac{1}{n}\right)^2-\frac{z^2}{n^2}\Bigg)^{-1}=\frac{1}{1-z^2}\prod_{n=1}^{+\infty} \left(\frac{\left(1+\frac{1}{n}\right)^2-\frac{z^2}{n^2}}{\left(1+\frac{1}{n}\right)^2}\right)^{-1}=\\
    =\frac{1}{1-z^2}\prod_{n=1}^{+\infty} \left(1-\frac{z^2}{(n+1)^2}\right)^{-1}=\frac{1}{1-z^2}\prod_{n=2}^{+\infty}\left(1-\frac{z^2}{n^2}\right)^{-1}=\\
    =\prod_{n=1}^{+\infty}\left(1-\frac{z^2}{n^2}\right)^{-1}=\frac{\pi z}{\sin(\pi z)}.
  \end{gather*}
\end{proof}

Per $n \ge 1$, vale la seguente formula di moltiplicazione dovuta a Gauss:
$$\prod_{k=0}^{n-1}\Gamma\left(z+\frac{k}{n}\right)=(2\pi)^{\frac{n-1}{2}}n^{\frac{1}{2}-nz}\Gamma(nz).$$
Per i nostri scopi ci servirà solo un caso particolare, che è quello che dimostreremo.

\begin{prop}
  (formula di duplicazione di Legendre)
  \begin{equation}
    \Gamma(z)\Gamma(z+1/2)=\sqrt{\pi}2^{1-2z}\Gamma(2z).
  \end{equation}
\end{prop}

\begin{proof}
  Dalla formula di Gauss abbiamo
  \begin{gather*}
    \Gamma(z)=\lim_{m \longrightarrow +\infty} \frac{m^zm!}{z(z+1)\cdots(z+m)}= \\
    =\lim_{m \longrightarrow+\infty}\frac{m^z(m-1)!}{z(z+1)\dots(z+m-1)}\cdot\frac{m}{z+m}=\lim_{m \longrightarrow+\infty}\frac{m^z(m-1)!}{z(z+1)\dots(z+m-1)}.
  \end{gather*}
  Dunque
  \begin{gather*}
    \frac{2^{2z-1}\Gamma(z)\Gamma\left(z+\frac{1}{2}\right)}{\Gamma(2z)}=\\
    =\lim_{m \longrightarrow +\infty} \frac{2^{2z-1}}{\frac{(2m)^{2z}(2m-1)!}{2z(2z+1)\cdots(2z+2m-1)}}\cdot\frac{m^z(m-1)!}{z(z+1)\cdots(z+m-1)}\cdot\frac{m^{z+1/2}(m-1)!}{(z+1/2)(z+3/2)\cdots(z+m-1/2)}=\\
    =\lim_{m \longrightarrow +\infty} \frac{2^{2z-1}m^{2z+1/2}\big((m-1)!\big)^22^m(2z)(2z+1)\cdots(2z+2m-1)}{(2m)^{2z}(2m-1)!z(z+1)\cdots(z+m-1)(2z+1)(2z+3)\cdots(2z+2m-1)}= \\
    \lim_{m \longrightarrow +\infty} \frac{2^{2m-1}m^{1/2}\big((m-1)!\big)^2}{(2m-1)!}.
  \end{gather*}
  Questa quantità non dipende da $z$, perciò è costante. Per la proposizione \ref{gammaze1-z} si ha $\Gamma^2(1/2)=\frac{\pi}{\sin(\pi/2)}=\pi \implies \Gamma(1/2)=\sqrt{\pi}$; allora la costante è proprio $\sqrt{\pi}$, come voluto.
\end{proof}

\begin{oss}
  $\Gamma(1/2)\Gamma(1)=\sqrt{\pi}\Gamma(2) \implies \Gamma(1/2)=\sqrt{\pi}$.

  $\Gamma(3/2)=\frac{1}{2}\Gamma(1/2)=\sqrt{\pi}/2<1$.
\end{oss}

\begin{oss}
  \begin{gather*}
    \frac{\Gamma\left(\frac{1}{2}-\frac{z}{2}\right)}{\Gamma\left(\frac{z}{2}\right)}=\frac{\Gamma\left(\frac{1}{2}-\frac{z}{2}\right)\Gamma\left(1-\frac{z}{2}\right)}{\Gamma\left(\frac{z}{2}\right)\Gamma\left(1-\frac{z}{2}\right)}=\\
    =\frac{\sin(\pi z/2)}{\pi}\sqrt{\pi}2^{1-(1-z)}\Gamma(1-z)=\frac{\sin(\pi z/2)}{\sqrt{\pi}}2^z\Gamma(1-z).
  \end{gather*}
\end{oss}

\begin{prop} \label{Gammagamma}
  Fuori dai poli di $\Gamma$ si ha $\displaystyle \frac{\Gamma'}{\Gamma}(z)=-\gamma-\frac{1}{z}+\sum_{n=1}^{+\infty} \frac{z}{n(n+z)}$.
\end{prop}

\begin{proof}
  \begin{gather*}
    \Gamma(z)=\frac{1}{z}e^{-\gamma z}\prod_{n=1}^{+\infty}\left(1+\frac{z}{n}\right)^{-1}e^{z/n} \\
    \log\big(\Gamma(z)\big)=-\log{z}-\gamma z+\sum_{n=1}^{+\infty} \Bigg(\frac{z}{n}-\log\left(1+\frac{z}{n}\right)\Bigg) \\
    \frac{\Gamma'}{\Gamma}(z)=-\gamma-\frac{1}{z}+\sum_{n=1}^{+\infty} \left(\frac{1}{n}-\frac{1}{n+z}\right)=-\gamma-\frac{1}{z}+\sum_{n=1}^{+\infty} \frac{z}{n(n+z)}.
  \end{gather*}
\end{proof}

\begin{oss}
  Dalla proposizione \ref{Gammagamma} si ha
  $$\Gamma'(1)=-\gamma-1+\sum_{n=1}^{+\infty} \frac{1}{n(n+1)}=-\gamma.$$
\end{oss}

L'integrale $\displaystyle \int_0^{+\infty} e^{-x}x^{z-1}\diff x$ converge per $\mathfrak{Re}z>0$ e definisce una funzione olomorfa in $z$. Vale il seguente risultato.

\begin{thm}
  Per $\mathfrak{Re}z>0$ si ha
  \begin{equation}
    \Gamma(z)=\int_0^{+\infty} e^{-x}x^{z-1}\diff x.
  \end{equation}
\end{thm}

\begin{proof}
  Vogliamo dimostrare due cose:
  \begin{enumerate}
    \item $\displaystyle \lim_{n \longrightarrow +\infty} \int_0^n \left(1-\frac{x}{n}\right)^nx^{z-1}\diff x=\int_0^{+\infty} e^{-x}x^{z-1}\diff x$;
    \item $\displaystyle \int_0^n \left(1-\frac{x}{n}\right)^nx^{z-1}\diff x=\frac{n^zn!}{z(z+1)\cdots(z+n)}$.
  \end{enumerate}
  La tesi seguirà dalla formula di Gauss.
  \begin{enumerate}
    \item Il primo passo è mostrare la seguente catena di disuguaglianze:
    $$0 \le e^{-x}-\left(1-\frac{x}{n}\right)^n \le \frac{x^2}{n}e^{-x}. \qquad (\star)$$
    Notiamo che si ha
    \begin{gather*}
      1+\frac{x}{n} \le 1+\frac{x}{n}+\frac{x^2}{2!n^2}+\frac{x^3}{3!n^3}+\dots \le 1+\frac{x}{n}+\frac{x^2}{n^2}+\dots \iff \\
      \iff 1+\frac{x}{n} \le e^{x/n} \le \frac{1}{1-x/n} \implies \\
      \implies \left(1+\frac{x}{n}\right)^n \le e^x \text{ e } \left(1-\frac{x}{n}\right)^n \le e^{-x} \implies \\
      \implies e^{-x}-\left(1-\frac{x}{n}\right)^n=e^{-x}\Bigg(1-e^x\left(1-\frac{x}{n}\right)^n\Bigg) \le e^{-x}\Bigg(1-\left(1-\frac{x^2}{n^2}\right)^n\Bigg).
    \end{gather*}
    Ricordiamo la disuguaglianza di Bernoulli: per $\alpha \ge 0$ e $n \in \mathbb{N}\cup\{0\}$ abbiamo che $(1-\alpha)^n \ge 1-n\alpha$. Allora
    $$e^{-x}\Bigg(1-\left(1-\frac{x^2}{n^2}\right)^n\Bigg) \le e^{-x}\Bigg(1-\left(1-\frac{x^2}{n}\right)\Bigg)=\frac{x^2}{n}e^{-x}.$$
    Ne deduciamo che
    \begin{gather*}
      \int_0^n \left(1-\frac{x}{n}\right)^nx^{z-1}\diff x=\int_0^n e^{-x}x^{z-1}\diff x-\left(\int_0^n e^{-x}x^{z-1}\diff x-\int_0^n\left(1-\frac{x}{n}\right)^nx^{z-1}\diff x\right) \text{ e } \\
      \left|\int_0^n\Bigg(e^{-x}-\left(1-\frac{x}{n}\right)^n\Bigg)x^{z-1}\diff x\right| \le \int_0^n x^{\mathfrak{Re}z-1}\Bigg(e^{-x}-\left(1-\frac{x}{n}\right)^n\Bigg)\diff x \le \\
      \le \frac{1}{n} \int_0^n x^{\mathfrak{Re}z-1+2}e^{-x}\diff x \ll \frac{1}{n} \implies \\
      \implies \lim_{n \longrightarrow +\infty} \int_0^n \left(1-\frac{x}{n}\right)^nx^{z-1}\diff x=\int_0^{+\infty} e^{-x}x^{z-1}\diff x.
    \end{gather*}
    \item Effettuando il cambio di variabile $y=x/n$ troviamo
    $$\int_0^n \left(1-\frac{x}{n}\right)^nx^{z-1}\diff x=n^z\int_0^1(1-y)^ny^{z-1}\diff y;$$
    integrando per parti più volte si ottiene che
    \begin{gather*}
      \int_0^1 (1-y)^ny^{z-1}\diff y=\frac{n}{z}\int_0^1 (1-y)^{n-1}y^z\diff y=\dots= \\
      =\frac{n(n-1)\cdots 2\cdot 1}{z(z+1)\cdots(z+n-1)}\int_0^1 y^{z+n-1}\diff y=\frac{n!}{z(z+1)\cdots(z+n)}.
    \end{gather*}
  \end{enumerate}
\end{proof}

\begin{thm}
  (Stirling) Sia $\epsilon>0$. Si ha la seguente formula asintotica:
  \begin{equation}
    \log\big(\Gamma(z)\big)=\left(z-\frac{1}{2}\right)\log{z}-z+\log{\sqrt{2\pi}}+O_{\epsilon}\left(\frac{1}{|z|}\right)
  \end{equation}
  uniformemente per $|z| \ge \epsilon$ e $|\arg{z}| \le \pi-\epsilon$. In particolare,
  \begin{align*}
    \log{n!}&=(n+1/2)\Bigg(\log{n}+\frac{1}{n}+O\left(\frac{1}{n^2}\right)\Bigg)-n-1+\log{\sqrt{2\pi}}+O\left(\frac{1}{|n+1|}\right)= \\
    &=(n+1/2)\log{n}-n+\log{\sqrt{2\pi}}+O\left(\frac{1}{n}\right).
  \end{align*}
\end{thm}

\begin{proof}
  Abbiamo $\displaystyle \log\big(\Gamma(z)\big)=\lim_{n \longrightarrow +\infty} \log\left(\frac{n^zn!}{z(z+1)\cdots(z+n)}\right)$. Notiamo che
  $$\log\left(\frac{n^zn!}{z(z+1)\cdots(z+n)}\right)=z\log{n}-\log(z+n)-\sum_{k=1}^n \log\left(1+\frac{z-1}{k}\right).$$
  Applicando la formula di Eulero-Maclaurin alla funzione $f_z(x)=\log\left(1+\frac{z-1}{x}\right)$ troviamo
  \begin{gather*}
    \sum_{k=1}^n \log\left(1+\frac{z-1}{k}\right)=\int_1^n \log\left(1+\frac{z-1}{x}\right)\diff x+\\
    +\frac{1}{2}\big(\log{z}+\log(z+n-1)-\log{n}\big)+\int_1^n B_1(\{x\})\left(\frac{1}{z+x-1}-\frac{1}{x}\right)\diff x \implies \\
    \implies \log\left(\frac{n^zn!}{z(z+1)\cdots(z+n)}\right)=\dots=\left(z-\frac{1}{2}\right)\log{z}-\left(z+n+\frac{1}{2}\right)\log\left(1+\frac{z-1}{n}\right)+\\
    +\log\left(1-\frac{1}{z+n}\right)-\int_1^n B_1(\{x\})\left(\frac{1}{z+x-1}-\frac{1}{x}\right)\diff x;
  \end{gather*}
  poiché $\displaystyle \left(z+n+\frac{1}{2}\right)\log\left(1+\frac{z-1}{n}\right) \sim \frac{z-1}{n}\left(z+n+\frac{1}{2}\right) \overset{n \rightarrow +\infty}{\longrightarrow} z-1$ con un resto di $O_{\epsilon}(1/n)$, si ha
  \begin{gather*}
    \log\left(\frac{n^zn!}{z(z+1)\cdots(z+n)}\right)=\left(z-\frac{1}{2}\right)\log{z}-z+1+O_{\epsilon}\left(\frac{1}{n}\right)+\\
    -\int_1^{+\infty} B_1(\{x\})\left(\frac{1}{z+x-1}-\frac{1}{x}\right)\diff x+\int_n^{+\infty}B_1(\{x\})\left(\frac{1}{z+x-1}-\frac{1}{x}\right)\diff x=\\
    =\left(z-\frac{1}{2}\right)\log{z}-z+C+O_{\epsilon}\left(\frac{1}{n}\right)+R_z(n).
  \end{gather*}
  Ricordiamo che $DB_k(x)=kB_{k-1}(x)$ e che $\{x\}-1/2=x-\lfloor x\rfloor-1/2$, dunque una primitiva di $B_1(\{x\})$ è $\frac{(x-\lfloor x\rfloor)^2}{2}-\frac{x-\lfloor x\rfloor}{2}+\frac{1}{12}=\frac{1}{2}B_2(\{x\})$; allora integrando per parti si ha
  \begin{gather*}
    \int_1^{+\infty} \frac{B_1(\{x\})}{z+x-1}\diff x=\left[\frac{B_2(\{x\})}{2(z+x-1)}\right]_1^{+\infty}+\int_1^{+\infty} \frac{B_2(\{x\})}{2(z+x-1)^2}\diff x=\\
    =-\frac{1}{12z}+\int_1^{+\infty} \frac{B_2(\{x\})}{2(z+x-1)^2}\diff x.
  \end{gather*}
  Poniamo anche $\displaystyle C=1+\int_1^{+\infty} \frac{B_1(\{x\})}{x}\diff x$. Vogliamo vedere se
  $$-\frac{1}{2}\int_1^{+\infty}\frac{B_2(\{x\})}{(z+x-1)^2}\diff x=-\frac{1}{2}\int_0^{+\infty}\frac{B_2(\{x\})}{(z+x)^2}\diff x \overset{?}{\ll_{\epsilon}} \frac{1}{|z|}.$$
  Si ha
  $$\left|-\frac{1}{2}\int_0^{+\infty}\frac{B_2(\{x\})}{(z+x)^2}\diff x\right| \le \int_0^{+\infty} \frac{\diff x}{|z+x|^2}.$$
  Da $|\arg{z}| \le \pi-\epsilon$, abbiamo $|z+x|^2=|z|^2+x^2+2\langle z,x \rangle \ge |z|^2+x^2-2|z|x\cos{\epsilon}$; scrivendo $y=x-|z|\cos{\epsilon}$ e cambiando di variabile nei passaggi giusti troviamo
  \begin{gather*}
    \int_0^{+\infty} \frac{\diff x}{|z+x|^2} \le \int_{-|z|\cos{\epsilon}}^{+\infty} \frac{\diff y}{|z|^2\sin^2{\epsilon}+y^2} \le \int_{-\infty}^{+\infty} \frac{\diff y}{|z|^2\sin^2{\epsilon}+y^2}= \\
    =\frac{1}{|z|\sin{\epsilon}}\int_{-\infty}^{+\infty} \frac{\diff y}{y^2+1} \ll_{\epsilon} \frac{1}{|z|}.
  \end{gather*}
  Ricordiamo adesso che
  \begin{gather*}
    \Gamma(z)\Gamma(z+1/2)=\sqrt{\pi}2^{1-2z}\Gamma(2z) \implies \\
    \implies \log\big(\Gamma(z)\big)+\log\big(\Gamma(z+1/2)\big)=\frac{1}{2}\log{\pi}+(1-2z)\log{2}+\log\big(\Gamma(2z)\big);
  \end{gather*}
  si ha anche $\log\big(\Gamma(z)\big)=(z-1/2)\log{z}-z+C+O_{\epsilon}(1/|z|)$, da cui con un po' di conti si trova che $C-\frac{1}{2}\log(2\pi) \ll_{\epsilon} \frac{1}{|z|} \implies C=\frac{1}{2}\log(2\pi)$.
\end{proof}

\begin{cor} \label{gammaprimosu}
  Se $|z| \ge \epsilon$ e $|\arg{z}| \le \pi-\epsilon$ si ha $$\frac{\Gamma'}{\Gamma}(z)=\log{z}+O_{\epsilon}\left(\frac{1}{|z|}\right).$$
\end{cor}

\begin{proof}
  Sia $f(z)=\log\big(\Gamma(z)\big)-(z-1/2)\log{z}+z-\frac{1}{2}\log(2\pi)$. Preso $z_0$ che soddisfi le ipotesi e $z$ t.c. $|z-z_0|=\epsilon/2$, abbiamo $f(z) \ll_{\epsilon} \frac{1}{|z|} \ll \frac{1}{|z_0|}$. Per la formula integrale di Cauchy troviamo
  $$f'(z_0)=\frac{1}{2\pi i}\oint_{\gamma} \frac{f(z)}{(z-z_0)^2}\diff z \ll_{\epsilon} \frac{1}{|z_0|},$$
  dove $\gamma$ è il cerchio di centro $z_0$ e raggio $\epsilon/2$. La tesi segue notando che vale $f'(z)=\dfrac{\Gamma'}{\Gamma}(z)-\log{z}+\dfrac{1}{2z}$.
\end{proof}

\begin{cor} \label{gammaord}
  Per $|z| \ge \epsilon$ e $|\arg{z}| \le \pi-\epsilon$ si ha
  $$\Gamma(z)=\sqrt{\frac{2\pi}{z}}\left(\frac{z}{e}\right)^z\Bigg(1+O_{\epsilon}\left(\frac{1}{|z|}\right)\Bigg).$$
  In particolare $\displaystyle n!=\sqrt{2\pi n}\left(\frac{n}{e}\right)^n\Bigg(1+O\left(\frac{1}{n}\right)\Bigg)$.
\end{cor}

\begin{proof}
  Per esercizio.
\end{proof}

\begin{cor}
  Se $k \ge 1$, allora $(-1)^kB_{2k}=4\sqrt{\pi k}\left(\dfrac{k}{e\pi}\right)^{2k}\Bigg(1+O\left(\dfrac{1}{k}\right)\Bigg)$.
\end{cor}

\begin{proof}
  Si ha infatti $(-1)^kB_{2k}=\dfrac{2(2k)!}{(2\pi)^{2k}}\zeta(2k)$, ma
  $$\zeta(2k)=\sum_{n=1}^{+\infty} \frac{1}{n^{2k}} \le 1+\int_1^{+\infty} \frac{\diff x}{x^{2k}}=1+\frac{1}{2k-1}=1+O\left(\frac{1}{k}\right).$$
\end{proof}

\begin{oss}
  Si ritrova il raggio di convergenza di $\dfrac{z}{e^z-1}$.
\end{oss}

\begin{cor}
  Sia $z=x+iy$ con $x_1 \le x \le x_2$. Allora
  $$\left|\Gamma(x+iy)\right|=\sqrt{2\pi}|y|^{x-\frac{1}{2}}e^{-\frac{\pi}{2}|y|}\Bigg(1+O\left(\frac{1}{|y|}\right)\Bigg).$$
\end{cor}

\begin{proof}
  \begin{gather*}
    \log\big(\Gamma(x+iy)\big)=\left(x-\frac{1}{2}+iy\right)\log(x+iy)-x-iy+\frac{1}{2}\log(2\pi)+O\left(\frac{1}{|y|}\right) \\
    \mathfrak{Re}\Big(\log\big(\Gamma(x+iy)\big)\Big)=\mathfrak{Re}\Big((x-1/2+iy)\big(\log(iy)+\log(1-i\cdot x/y)\big)\Big)-x+\frac{1}{2}\log(2\pi)+O\left(\frac{1}{|y|}\right)= \\
    =(x-1/2)\log|y|+iy\left(\frac{\pi}{2}\cdot i\cdot\text{sgn}(y)\right)+iy\left(-\frac{ix}{y}\right)-x+\frac{1}{2}\log(2\pi)+O\left(\frac{1}{|y|}\right)=\\
    =(x-1/2)\log|y|-|y|\frac{\pi}{2}+\frac{1}{2}\log(2\pi)+O\left(\frac{1}{|y|}\right)
  \end{gather*}
\end{proof}
