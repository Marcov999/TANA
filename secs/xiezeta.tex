Dovevamo capire qual è l'ordine di $\xi$ come funzione intera. Avevamo già controllato per $\sigma>1$, quindi ci manca $\sigma \ge 1/2$. Ricordiamo che $\xi(s)=\dfrac{s(s-1)}{2}\pi^{-\frac{s}{2}}\Gamma\left(\dfrac{s}{2}\right)\zeta(s)$. Per il lemma \ref{lls} abbiamo già $\zeta(s) \ll |s|$ per $\sigma \ge 1/2$; inoltre, ricordando il corollario \ref{gammaord}, abbiamo
\begin{gather*}
  \Gamma(s) \ll_{\epsilon} e^{|s|^{1+\epsilon}} \text{ per ogni } \epsilon>0, \quad \pi^{-\frac{s}{2}} \ll 1, \quad \frac{s(s-1)}{2} \ll |s^2| \implies \\
  \implies \xi(s) \ll_{\epsilon} e^{|s|^{1+\epsilon}} \text{ per }\sigma \ge 1/2.
\end{gather*}
Da $\xi(s)=\xi(1-s)$, otteniamo che è vero per ogni $\sigma$. Scriviamo $\xi$ con il prodotto di Weierstrass:
$$\xi(s)=e^{a+As}\prod_{\rho}\left(1-\frac{s}{\rho}\right)e^{\frac{s}{\rho}};$$
dev'essere $\displaystyle a=\log\big(\xi(0)\big)=\log\left(\frac{1}{2}\right) \implies \xi(s)=\frac{1}{2}e^{As}\prod_{\rho}\left(1-\frac{s}{\rho}\right)e^{\frac{s}{\rho}}$.

Consideriamo invece $(s-1)\zeta(s)=\xi(s)\dfrac{\pi^{\frac{s}{2}}}{\frac{s}{2}\Gamma\left(\frac{s}{2}\right)}$. Dall'equazione appena scritta otteniamo che ha ordine $1$, dunque si ha (ricordando anche gli zeri banali)
$$(s-1)\zeta(s)=e^{b+Bs}\prod_{\rho}\left(1-\frac{s}{\rho}\right)e^{\frac{s}{\rho}}\prod_{n=1}^{+\infty}\left(1+\frac{s}{2n}\right)e^{-\frac{s}{2n}};$$
dev'essere $\displaystyle b=\Bigg(\log\big((s-1)\zeta(s)\big)\Bigg)_{s=0}=\log\left(\frac{1}{2}\right) \implies$
$$\implies (s-1)\zeta(s)=\frac{1}{2}e^{Bs}\prod_{\rho}\left(1-\frac{s}{\rho}\right)e^{\frac{s}{\rho}}\prod_{n=1}^{+\infty}\left(1+\frac{s}{2n}\right)e^{-\frac{s}{2n}}.$$

Confrontando i prodotti di Weierstrass per $\zeta$ e $\xi$ usando l'equazione che lega le due funzioni, ricordando anche la definizione di $\Gamma$, si ha $B=A+\frac{1}{2}\log{\pi}+\frac{\gamma}{2}$. Andiamo a calcolarci $A$ e $B$.

$$A=\frac{\xi'}{\xi}(0)=2\xi'(0), \quad B=\left(\frac{1}{s-1}+\frac{\zeta'}{\zeta}(0)\right)_{s=0}=-2\zeta'(0)-1.$$
Si ha anche
\begin{gather*}
  \frac{B}{2}=\frac{A}{2}+\frac{1}{4}\log{\pi}+\frac{\gamma}{4} \implies \\
  \implies \xi'(0)+\zeta'(0)=-\frac{1}{2}-\frac{1}{4}\log{\pi}-\frac{\gamma}{4}.
\end{gather*}
Vogliamo calcolare $\xi'(0)$ usando l'equazione funzionale per $\xi$. Dalla definizione abbiamo
\begin{gather*}
  \xi'(s)=\frac{(s-1)\zeta(s)}{2}\pi^{-\frac{s}{2}}\Gamma\left(\frac{s}{2}\right)-\frac{1}{4}\log{\pi}\cdot s(s-1)\zeta(s)\Gamma\left(\frac{s}{2}\right)\pi^{-\frac{s}{2}}+ \\
  +\frac{s}{4}\pi^{-\frac{s}{2}}\Gamma'\left(\frac{s}{2}\right)(s-1)\zeta(s)+\frac{s}{2}\pi^{-\frac{s}{2}}\Gamma\left(\frac{s}{2}\right)D\big(\zeta(s)(s-1)\big)
\end{gather*}

Nella dimostrazione del lemma \ref{lls} abbiamo visto che
$$\zeta(s)=\frac{1}{s-1}+1-s\int_1^{+\infty} \frac{\{u\}}{u^{s+1}}\diff u \implies \lim_{s \longrightarrow 1} \left(\zeta(s)-\frac{1}{s-1}\right)=\gamma,$$
dunque dev'essere $\zeta(s)=\dfrac{1}{s-1}+\gamma+(s-1)g(s)$, con $g$ una qualche funzione intera. Allora otteniamo
$$\xi'(1)=\frac{\Gamma\left(\frac{1}{2}\right)}{2\sqrt{\pi}}-\frac{1}{4}\log{\pi}\frac{\Gamma\left(\frac{1}{2}\right)}{\sqrt{\pi}}+\frac{\Gamma'\left(\frac{1}{2}\right)}{4\sqrt{\pi}}+\frac{\Gamma\left(\frac{1}{2}\right)}{2\sqrt{\pi}}\gamma;$$
dobbiamo calcolare $\Gamma'\left(\frac{1}{2}\right)$. Dalla proposizione \ref{Gammagamma} abbiamo
\begin{gather*}
  \frac{\Gamma'}{\Gamma}(z)=-\gamma-\frac{1}{z}+\sum_{n=1}^{+\infty} \frac{z}{n(n+z)} \implies \\
  \implies \frac{\Gamma'}{\Gamma}\left(\frac{1}{2}\right)=-\gamma-2+\sum_{n=1}^{+\infty} \frac{2}{2n(2n+1)}=-\gamma-2+2\sum_{n=1}^{+\infty} \left(\frac{1}{2n}-\frac{1}{2n+1}\right)= \\
  =-\gamma-2+2\left(\frac{1}{2}-\frac{1}{3}+\frac{1}{4}-\frac{1}{5}+\dots\right)=-\gamma-2+2-2\log{2}=-\gamma-2\log{2} \implies \\
  \implies \Gamma'\left(\frac{1}{2}\right)=-\sqrt{\pi}(\gamma+2\log{2}),
\end{gather*}
quindi
$$\xi'(1)=\frac{1}{2}-\frac{1}{4}\log{\pi}-\frac{\gamma}{4}-\frac{1}{2}\log{2}+\frac{\gamma}{2}=\frac{\gamma}{4}+\frac{1}{2}-\frac{1}{4}\log(4\pi).$$

Adesso usiamo l'equazione funzionale:
\begin{gather*}
  \xi(1-s)=\xi(s) \implies -\xi'(1-s)=\xi'(s) \implies \\
  \implies \xi'(0)=-\xi'(1)=\frac{1}{4}\log(4\pi)-\frac{\gamma}{4}-\frac{1}{2} \text{ e}\\
  \zeta'(0)=\frac{\gamma}{4}+\frac{1}{2}-\frac{1}{4}\log(4\pi)-\frac{1}{2}-\frac{\gamma}{4}-\frac{1}{4}\log{\pi}=\\
  =-\frac{1}{2}\log{\pi}-\frac{1}{2}\log{2}=-\frac{1}{2}\log(2\pi);
\end{gather*}
abbiamo quindi
$$B=-2\zeta'(0)-1=\log(2\pi)-1 \text{ e }A=2\xi'(0)=\frac{1}{2}\log(4\pi)-1-\frac{\gamma}{2}.$$

Ora, vogliamo dire che non ci sono zeri per $\sigma=1$ (e dalla funzionale, nemmeno per $\sigma=0$). Dal prodotto di Weierstrass troviamo
\begin{gather*}
  \frac{\xi'}{\xi}(s)=A+\sum_{\rho} \left(\frac{1}{s-\rho}+\frac{1}{\rho}\right)\text{ e} \\
  \frac{\zeta'}{\zeta}(s)=-\frac{1}{s-1}-\frac{1}{2}\cdot\frac{\Gamma'}{\Gamma}\left(\frac{s}{2}+1\right)+\frac{1}{2}\log{\pi}+\frac{\xi'}{\xi}(s)=\\
  =-\frac{1}{s-1}+A+\frac{1}{2}\log{\pi}+\sum_{\rho} \left(\frac{1}{s-\rho}+\frac{1}{\rho}\right)-\frac{1}{2}\cdot\frac{\Gamma'}{\Gamma}\left(\frac{s}{2}+1\right).
\end{gather*}
Supponiamo per assurdo che ci sia uno zero in $1+it$. Ci tornerà utile la seguente disuguaglianza, valida per ogni $\theta$ reale: $3+4\cos{\theta}+\cos(2\theta) \ge 0$. Infatti, la si riscrive come $2\cos^2{\theta}+4\cos{\theta}+2=2(\cos{\theta}+1)^2$. Sia ora $\sigma>1$, abbiamo
$$\log\big(\zeta(s)\big)=\log\Bigg(\prod_p\left(1-\frac{1}{p^s}\right)^{-1}\Bigg)=-\sum_p \log\left(1-\frac{1}{p^s}\right)=\sum_p\sum_{m=1}^{+\infty} \frac{1}{mp^{ms}},$$
da cui prendendo la parte reale
$$\log|\zeta(\sigma+it)|=\sum_p \sum_{m=1}^{+\infty} \frac{1}{mp^{\sigma m}}\cos(mt\log{p}).$$
Adesso, sfruttando la disuguaglianza vista sopra con $\theta=mt\log{p}$ al variare di $m$ e $p$, otteniamo
\begin{gather*}
  3\log|\zeta(\sigma)|+4\log|\zeta(\sigma+it)|+\log|\zeta(\sigma+i2t)|= \\
  =\sum_p \sum_{m=1}^{+\infty} \frac{1}{mp^{m\sigma}}\big(3+4\cos(mt\log{p})+\cos(2mt\log{p})\big) \ge 0 \implies \\
  \implies \zeta^3(\sigma)|\zeta(\sigma+it)|^4||\zeta(\sigma+i2t)| \ge 1.
\end{gather*}
Adesso, $\zeta$ ha un polo semplice in $1$, perciò $\zeta(\sigma)=\dfrac{1}{\sigma-1}+g(\sigma)$ con $g$ una qualche funzione intera; invece, se ci fosse uno zero in $1+it$, avremmo
$$\zeta(\sigma+it)=(\sigma+it-1-it)h(\sigma+it)=(\sigma-1)h(\sigma+it),$$
con $h$ una qualche funzione olomorfa in un intorno di $1+it$. Ma allora, poiché $4>3$, si ha
$$\lim_{\sigma \longrightarrow 1} \zeta^3(\sigma)\zeta(\sigma+it)^4\zeta(\sigma+i2t)=0,$$
in contraddizione con la disuguaglianza trovata.

Diamo un'altra dimostrazione di questo fatto, dovuta a Ingham (si veda \cite{T}, capitolo III paragrafo 3.4). Dati $a,b \in \mathbb{C}$, usiamo che
$$\frac{\zeta(s)\zeta(s-a)\zeta(s-b)\zeta(s-a-b)}{\zeta(2s-a-b)}=\sum_{n=1}^{+\infty} \frac{\sigma_a(n)\sigma_b(n)}{n^s},$$
dove $\displaystyle \sigma_{\alpha}(n)=\sum_{d \mid n} d^{\alpha}$. Prendendo $a=-b=i\gamma$ troviamo
$$\frac{\zeta^2(s)\zeta(s-i\gamma)\zeta(s+i\gamma)}{\zeta(2s)}=\sum_{n=1}^{+\infty} \frac{\sigma_{i\gamma}(n)\sigma_{-i\gamma}(n)}{n^s}=\sum_{n=1}^{+\infty} \frac{|\sigma_{i\gamma}(n)|^2}{n^s}.$$
Sia $\sigma_0$ l'ascissa di convergenza dell'ultima serie che abbiamo scritto. Poiché i coefficienti della stessa sono maggiori o uguali a zero, abbiamo che la funzione $\frac{\zeta^2(s)\zeta(s-i\gamma)\zeta(s+i\gamma)}{\zeta(2s)}$ ha un polo in $\sigma_0$. Supponiamo per assurdo che abbia uno zero in $1+i\gamma$, dunque anche in $1-i\gamma$; allora per $s=1$ c'è un doppio zero che annulla il polo doppio di $\zeta^2(s)$ al numeratore. Dunque il primo polo che si incontra dev'essere in uno zero del denominatore, cioè (ci stiamo restringendo alla retta reale perché i coefficienti della serie sono non negativi) in $s=-1$. Allora dev'essere $\sigma_0 \le -1$, ma prendendo $s=1/2$ troviamo facilmente un assurdo (la funzione si annulla grazie al polo al denominatore, la serie no).

Andiamo ora a cercare una regione in cui sono contenuti gli zeri non banali di $\zeta$. Abbiamo già visto che
$$\frac{\zeta'}{\zeta}(s)=-\frac{1}{s-1}+A+\frac{1}{2}\log{\pi}-\frac{1}{2}\cdot\frac{\Gamma'}{\Gamma}\left(\frac{s}{2}+1\right)+\sum_{\rho}\left(\frac{1}{s-\rho}+\frac{1}{\rho}\right).$$

\begin{prop}
  (de la Vallée-Poussin, 1899) Esiste una costante $C_0>0$ t.c., se $\rho=\beta+i\gamma$ con $\gamma>0$ è uno zero non banale, allora
  \begin{equation}
    \beta<1-\frac{C_0}{\log{\gamma}}.
  \end{equation}
\end{prop}

\begin{proof}
  Si ha
  $$-\mathfrak{Re}\left(\frac{\zeta'}{\zeta}(s)\right)=\frac{\sigma-1}{(\sigma-1)^2+t^2}-A_1+\frac{1}{2}\mathfrak{Re}\Bigg(\frac{\Gamma'}{\Gamma}\left(\frac{s}{2}+1\right)\Bigg)-\sum_{\rho}\mathfrak{Re}\left(\frac{1}{s-\rho}+\frac{1}{\rho}\right).$$
  Per $t \ge 2, 1<\sigma \le 2$, usando il corollario \ref{gammaprimosu} otteniamo
  \begin{gather*}
    -\mathfrak{Re}\left(\frac{\zeta'}{\zeta}(\sigma+it)\right)<C\log{t}-\sum_{\rho} \left(\frac{\sigma-\beta}{(\sigma-\beta)^2+(t-\gamma)^2}+\frac{\beta}{\beta^2+\gamma^2}\right)< \\
    <C\log{t}-\sum_{\rho} \frac{\sigma-\beta}{(\sigma-\beta)^2+(t-\gamma)^2}.
  \end{gather*}
  Adesso, tenendo eventualmente solo lo zero dell'ipotesi, abbiamo
  $$t=\begin{cases}
    \gamma & \implies -\mathfrak{Re}\left(\dfrac{\zeta'}{\zeta}(\sigma+i\gamma)\right)<C\log{\gamma}-\dfrac{1}{\sigma-\beta} \\
    2\gamma & \implies -\mathfrak{Re}\left(\dfrac{\zeta'}{\zeta}(\sigma+i2\gamma)\right)<C\log(2\gamma) \le C_1\log{\gamma} \\
    =0 & \implies -\mathfrak{Re}\left(\dfrac{\zeta'}{\zeta}(\sigma)\right)=\dfrac{1}{\sigma-1}+O(1).
\end{cases}$$
Poiché siamo nel giusto semipiano di convergenza, scriviamo
\begin{gather*}
  -\frac{\zeta'}{\zeta}(s)=\sum_{n=1}^{+\infty} \frac{\Lambda(n)}{n^s} \implies \\
  \implies -\mathfrak{Re}\left(\frac{\zeta'}{\zeta}(\sigma+it)\right)=\sum_{n=1}^{+\infty} \frac{\Lambda(n)}{n^{\sigma}}\cos(t\log{n}) \implies \\
  \implies -3\cdot\mathfrak{Re}\left(\frac{\zeta'}{\zeta}(\sigma)\right)-4\cdot\mathfrak{Re}\left(\frac{\zeta'}{\zeta}(\sigma+i\gamma)\right)-\mathfrak{Re}\left(\frac{\zeta'}{\zeta}(\sigma+i2\gamma)\right) \ge 0 \implies \\
  \implies 0 \le \frac{3}{\sigma-1}-\frac{4}{\sigma-\beta}+C_2\log{\gamma} \implies \frac{4}{\sigma-\beta} \le \frac{3}{\sigma-1}+C_2\log{\gamma}.
\end{gather*}
Prendendo $\sigma=1+\dfrac{\delta}{\log{\gamma}}$ con $\delta>0$, si ha la tesi.
\end{proof}

\begin{oss}
  Si può dire che il primo zero ha $\gamma \ge 6$. Prendendo anche $6 \le \gamma \le t$, dato che $1-\frac{C_0}{\log{\gamma}} \le 1-\frac{C_0}{\log{t}}$, quello che troviamo è che la regione $\sigma>1-\dfrac{C_0}{\log(|t|+2)}$ è libera da zeri di $\zeta$. Questo risultato è stato migliorato:
  $$\text{Littlewood, 1922: } \sigma>1-\frac{C_0\log\log(|t|+2)}{\log(|t|+2)};$$
  $$\text{Vinogradov, 1958: } \sigma>1-\frac{C_0(\epsilon)}{\big(\log(|t|+2)\big)^{2/3+\epsilon}} \text{ per ogni }\epsilon>0.$$
\end{oss}

Per $t \ge 2$, abbiamo visto che
$$-\mathfrak{Re}\left(\frac{\zeta'}{\zeta}(s)\right)<C\log{t}-\sum_{\rho} \mathfrak{Re}\left(\frac{1}{s-\rho}+\frac{1}{\rho}\right).$$
Prendendo $s=2+it$, la quantità $\frac{\zeta'}{\zeta}(s)$ è limitata, dunque a meno di cambiare costante troviamo
\begin{gather*}
  \mathfrak{Re}\sum_{\rho}\left(\frac{1}{(2-\beta)+i(t-\gamma)}+\frac{1}{\rho}\right)<C\log{t} \implies \\
  \implies \sum_{\rho} \frac{1}{4+(t-\gamma)^2} \le \sum_{\rho} \left(\frac{2-\beta}{(2-\beta)^2+(t-\gamma)^2}+\frac{\beta}{\beta^2+\gamma^2}\right)<C\log{t} \implies \\
  \implies \sum_{|\gamma-t| \ge 1} \frac{1}{4+(t-\gamma)^2} \le C\log{t}.
\end{gather*}
Otteniamo il seguente lemma.

\begin{lm} \label{zprimoz}
  Se $-1 \le \sigma \le 2$ si ha
  $$\frac{\zeta'}{\zeta}(s)=\sum_{|\gamma-t| \le 1} \frac{1}{s-\rho}+O\big(\log(|t|+2)\big).$$
\end{lm}

\begin{proof}
  Per $|t| \ge 2$ abbiamo che
  \begin{gather*}
    \frac{\zeta'}{\zeta}(s)=\sum_{\rho} \left(\frac{1}{s-\rho}+\frac{1}{\rho}\right)+O(\log{t}) \\
    \frac{\zeta'}{\zeta}(2+it)=\sum_{\rho}\left(\frac{1}{2+it-\rho}+\frac{1}{\rho}\right)+O(\log{t});
  \end{gather*}
  poiché $\frac{\zeta'}{\zeta}(2+it)$ è limitata, facendo la differenza otteniamo
  $$\frac{\zeta'}{\zeta}(s)=\sum_{\rho} \left(\frac{1}{s-\rho}-\frac{1}{2+it-\rho}\right)+O(\log{t})=\sum_{\rho} \frac{2-\sigma}{(s-\rho)(2+it-\rho)}+O(\log{t}).$$
  Ora osserviamo che
  \begin{gather*}
    \left| \sum_{|\gamma-t|>1} \frac{2-\sigma}{(s-\rho)(2+it-\rho)}\right| \le \sum_{|\gamma-t|>1} \frac{3}{(\gamma-t)^2} \ll \log{t} \implies \\
    \implies \frac{\zeta'}{\zeta}(s)=\sum_{|\gamma-t| \le 1} \left(\frac{1}{s-\rho}-\frac{1}{2+it-\rho}\right)+O(\log{t}) \text{ e} \\
    \sum_{|\gamma-t| \le 1} \frac{1}{|2+it-\rho|} \le \sum_{|\gamma-t| \le 1} \frac{1}{2-\beta} \le N(t+1)-N(t-1) \ll \log{t},
  \end{gather*}
  da cui la tesi.
\end{proof}

Torniamo su
\begin{gather*}
  S(T)=\frac{1}{\pi}\arg\Bigg(\zeta\left(\frac{1}{2}+iT\right)\Bigg)-\frac{1}{\pi}\arg\big(\zeta(2+iT)\big)= \\
  =\frac{1}{\pi}\mathfrak{Im}\left[\log\Bigg(\zeta\left(\frac{1}{2}+iT\right)\Bigg)-\log\big(\zeta(2+iT)\big)\right]= \\
  =-\frac{1}{\pi}\int_{1/2}^2 \mathfrak{Im}\left(\frac{\zeta'}{\zeta}(\sigma+iT)\right)\diff\sigma= \\
  =-\frac{1}{\pi}\int_{1/2}^2 \sum_{|\gamma-T| \le 1} \mathfrak{Im}\left(\frac{1}{\sigma+iT-\rho}\right)\diff\sigma+O(\log{T})= \\
  =-\frac{1}{\pi}\sum_{|\gamma-T| \le 1} \Delta_l\arg(s-\rho) +O(\log{T}) \ll \sum_{|\gamma-T| \le 1} 1 +O(\log{T}) \ll \log{T},
\end{gather*}
dove dalla terza alla quarta riga abbiamo usato il lemma \ref{zprimoz} ($l$ è il segmento da $\frac{1}{2}+iT$ a $2+iT$). Che valga $\Delta_l\arg(s-\rho) \ll 1$ per $|\gamma-T| \le 1$ segue da un po' di geometria.
