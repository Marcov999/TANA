Dovevamo capire qual è l'ordine di $\xi$ come funzione intera. Avevamo già controllato per $\sigma>1$, quindi ci manca $\sigma \ge 1/2$. Ricordiamo che $\xi(s)=\dfrac{s(s-1)}{2}\pi^{-\frac{s}{2}}\Gamma\left(\dfrac{s}{2}\right)\zeta(s)$. Per il lemma \ref{lls} abbiamo già $\zeta(s) \ll |s|$ per $\sigma \ge 1/2$; inoltre, ricordando il corollario \ref{gammaord}, abbiamo
\begin{gather*}
  \Gamma(s) \ll_{\epsilon} e^{|s|^{1+\epsilon}} \text{ per ogni } \epsilon>0, \quad \pi^{-\frac{s}{2}} \ll 1, \quad \frac{s(s-1)}{2} \ll |s^2| \implies \\
  \implies \xi(s) \ll_{\epsilon} e^{|s|^{1+\epsilon}} \text{ per }\sigma \ge 1/2.
\end{gather*}
Da $\xi(s)=\xi(1-s)$, otteniamo che è vero per ogni $\sigma$. Scriviamo $\xi$ con il prodotto di Weierstrass:
$$\xi(s)=e^{a+As}\prod_{\rho}\left(1-\frac{s}{\rho}\right)e^{\frac{s}{\rho}};$$
dev'essere $\displaystyle a=\log\big(\xi(0)\big)=\log\left(\frac{1}{2}\right) \implies \xi(s)=\frac{1}{2}e^{As}\prod_{\rho}\left(1-\frac{s}{\rho}\right)e^{\frac{s}{\rho}}$.

Consideriamo invece $(s-1)\zeta(s)=\xi(s)\dfrac{\pi^{\frac{s}{2}}}{\frac{s}{2}\Gamma\left(\frac{s}{2}\right)}$. Dall'equazione appena scritta otteniamo che ha ordine $1$, dunque si ha (ricordando anche gli zeri banali)
$$(s-1)\zeta(s)=e^{b+Bs}\prod_{\rho}\left(1-\frac{s}{\rho}\right)e^{\frac{s}{\rho}}\prod_{n=1}^{+\infty}\left(1+\frac{s}{2n}\right)e^{-\frac{s}{2n}};$$
dev'essere $\displaystyle b=\Bigg(\log\left(s-1)\zeta(s)\right)\Bigg)_{s=0}=\log\left(\frac{1}{2}\right) \implies$
$$\implies (s-1)\zeta(s)=\frac{1}{2}e^{Bs}\prod_{\rho}\left(1-\frac{s}{\rho}\right)e^{\frac{s}{\rho}}\prod_{n=1}^{+\infty}\left(1+\frac{s}{2n}\right)e^{-\frac{s}{2n}}.$$

Confrontando i prodotti di Weierstrass per $\zeta$ e $\xi$ usando l'equazione che lega le due funzioni, ricordando anche la definizione di $\Gamma$, si ha $B=A+\frac{1}{2}\log{\pi}+\frac{\gamma}{2}$. Andiamo a calcolarci $A$ e $B$.

$$A=\frac{\xi'}{\xi}(0)=2\xi'(0), \quad B=\left(\frac{1}{s-1}+\frac{\zeta'}{\zeta}(0)\right)_{s=0}=-2\zeta'(0)-1.$$
Si ha anche
\begin{gather*}
  \frac{B}{2}=\frac{A}{2}+\frac{1}{4}\log{\pi}+\frac{\gamma}{4} \implies \\
  \implies \xi'(0)+\zeta'(0)=-\frac{1}{2}-\frac{1}{4}\log{\pi}-\frac{\gamma}{4}.
\end{gather*}
Vogliamo calcolare $\xi'(0)$ usando l'equazione funzionale per $\xi$. Dalla definizione abbiamo
\begin{gather*}
  \xi'(s)=\frac{(s-1)\zeta(s)}{2}\pi^{-\frac{s}{2}}\Gamma\left(\frac{s}{2}\right)-\frac{1}{4}\log{\pi}\cdot s(s-1)\zeta(s)\Gamma\left(\frac{s}{2}\right)\pi^{-\frac{s}{2}}+ \\
  +\frac{s}{4}\pi^{-\frac{s}{2}}\Gamma'\left(\frac{s}{2}\right)(s-1)\zeta(s)+\frac{s}{2}\pi^{-\frac{s}{2}}\Gamma\left(\frac{s}{2}\right)D\big(\zeta(s)(s-1)\big)
\end{gather*}

Nella dimostrazione del lemma \ref{lls} abbiamo visto che
$$\zeta(s)=\frac{1}{s-1}+1-s\int_1^{+\infty} \frac{\{u\}}{u^{s+1}}\diff u \implies \lim_{s \longrightarrow 1} \left(\zeta(s)-\frac{1}{s-1}\right)=\gamma,$$
dunque dev'essere $\zeta(s)=\dfrac{1}{s-1}+\gamma+(s-1)g(s)$, con $g$ una qualche funzione intera. Allora otteniamo
$$\xi'(1)=\frac{\Gamma\left(\frac{1}{2}\right)}{2\sqrt{\pi}}-\frac{1}{4}\log{\pi}\frac{\Gamma\left(\frac{1}{2}\right)}{\sqrt{\pi}}+\frac{\Gamma'\left(\frac{1}{2}\right)}{4\sqrt{\pi}}+\frac{\Gamma\left(\frac{1}{2}\right)}{2\sqrt{\pi}}\gamma;$$
dobbiamo calcolare $\Gamma'\left(\frac{1}{2}\right)$. Dalla proposizione \ref{Gammagamma} abbiamo
\begin{gather*}
  \frac{\Gamma'}{\Gamma}(z)=-\gamma-\frac{1}{z}+\sum_{n=1}^{+\infty} \frac{z}{n(n+z)} \implies \\
  \implies \frac{\Gamma'}{\Gamma}\left(\frac{1}{2}\right)=-\gamma-2+\sum_{n=1}^{+\infty} \frac{2}{2n(2n+1)}=-\gamma-2+2\sum_{n=1}^{+\infty} \left(\frac{1}{2n}-\frac{1}{2n+1}\right)= \\
  =-\gamma-2+2\left(\frac{1}{2}-\frac{1}{3}+\frac{1}{4}-\frac{1}{5}+\dots\right)=-\gamma-2+2-2\log{2}=-\gamma-2\log{2} \implies \\
  \implies \Gamma'\left(\frac{1}{2}\right)=-\sqrt{\pi}(\gamma+2\log{2}),
\end{gather*}
quindi
$$\xi'(1)=\frac{1}{2}-\frac{1}{4}\log{\pi}-\frac{\gamma}{4}-\frac{1}{2}\log{2}+\frac{\gamma}{2}=\frac{\gamma}{4}+\frac{1}{2}-\frac{1}{4}\log(4\pi).$$

Adesso usiamo l'equazione funzionale:
\begin{gather*}
  \xi(1-s)=\xi(s) \implies -\xi'(1-s)=\xi'(s) \implies \\
  \implies \xi'(0)=-\xi'(1)=\frac{1}{4}\log(4\pi)-\frac{\gamma}{4}-\frac{1}{2} \text{ e}\\
  \zeta'(0)=\frac{\gamma}{4}+\frac{1}{2}-\frac{1}{4}\log(4\pi)-\frac{1}{2}-\frac{\gamma}{4}-\frac{1}{4}\log{\pi}=\\
  =-\frac{1}{2}\log{\pi}-\frac{1}{2}\log{2}=-\frac{1}{2}\log(2\pi);
\end{gather*}
abbiamo quindi
$$B=-2\zeta'(0)-1=\log(2\pi)-1 \text{ e }A=2\xi'(0)=\frac{1}{2}\log(4\pi)-1-\frac{\gamma}{2}.$$

Ora, vogliamo dire che non ci sono zeri per $\sigma=1$ (e dalla funzionale, nemmeno per $\sigma=0$). Dal prodotto di Weierstrass troviamo
\begin{gather*}
  \frac{\xi'}{\xi}(s)=A+\sum_{\rho} \left(\frac{1}{s-\rho}+\frac{1}{\rho}\right)\text{ e} \\
  \frac{\zeta'}{\zeta}(s)=-\frac{1}{s-1}-\frac{1}{2}\cdot\frac{\Gamma'}{\Gamma}\left(\frac{s}{2}+1\right)+\frac{1}{2}\log{\pi}+\frac{\xi'}{\xi}(s)=\\
  =-\frac{1}{s-1}+A+\frac{1}{2}\log{\pi}+\sum_{\rho} \left(\frac{1}{s-\rho}+\frac{1}{\rho}\right)-\frac{1}{2}\cdot\frac{\Gamma'}{\Gamma}\left(\frac{s}{2}+1\right).
\end{gather*}
Supponiamo per assurdo che ci sia uno zero in $1+it$. Ci tornerà utile la seguente disuguaglianza, valida per ogni $\theta$ reale: $3+4\cos{\theta}+\cos(2\theta) \ge 0$. Infatti, la si riscrive come $2\cos^2{\theta}+4\cos{\theta}+2=2(\cos{\theta}+1)^2$. Sia ora $\sigma>1$, abbiamo
$$\log\big(\zeta(s)\big)=\log\Bigg(\prod_p\left(1-\frac{1}{p^s}\right)^{-1}\Bigg)=-\sum_p \log\left(1-\frac{1}{p^s}\right)=\sum_p\sum_{m=1}^{+\infty} \frac{1}{mp^{ms}},$$
da cui prendendo la parte reale
$$\log|\zeta(\sigma+it)|=\sum_p \sum_{m=1}^{+\infty} \frac{1}{mp^{\sigma m}}\cos(mt\log{p}).$$
Adesso, sfruttando la disuguaglianza vista sopra con $\theta=mt\log{p}$ al variare di $m$ e $p$, otteniamo
\begin{gather*}
  3\log|\zeta(\sigma)|+4\log|\zeta(\sigma+it)|+\log|\zeta(\sigma+i2t)|= \\
  =\sum_p \sum_{m=1}^{+\infty} \frac{1}{mp^{m\sigma}}\big(3+4\cos(mt\log{p})+\cos(2mt\log{p})\big) \ge 0 \implies \\
  \implies \zeta^3(\sigma)|\zeta(\sigma+it)|^4||\zeta(\sigma+i2t)| \ge 1.
\end{gather*}
Adesso, $\zeta$ ha un polo semplice in $1$, perciò $\zeta(\sigma)=\dfrac{1}{\sigma-1}+g(\sigma)$ con $g$ una qualche funzione intera; invece, se ci fosse uno zero in $1+it$, avremmo
$$\zeta(\sigma+it)=(\sigma+it-1-it)h(\sigma+it)=(\sigma-1)h(\sigma+it),$$
con $h$ una qualche funzione olomorfa in un intorno di $1+it$. Ma allora, poiché $4>3$, si ha
$$\lim_{\sigma \longrightarrow 1} \zeta^3(\sigma)\zeta(\sigma+it)^4\zeta(\sigma+i2t)=0,$$
in contraddizione con la disuguaglianza trovata.
