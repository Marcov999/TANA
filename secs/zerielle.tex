Vediamo prima il caso $\chi$ complesso.

\begin{prop}
  Se $\chi$ modulo $q$ è complesso, allora esiste una costante $c_0>0$ t.c.
  $$\beta_\chi < 1-\frac{c_0}{\log\big(q(|\gamma_\chi|+2)\big)}=:1-\frac{c_0}{\mathcal{L}}.$$
\end{prop}

\begin{proof}
  Per $\sigma>1$ abbiamo
  \begin{gather*}
    -\sum_{n=1}^{+\infty} \frac{\Lambda(n)\chi(n)}{n^{\sigma+it}}=\frac{L'}{L}(\sigma+it,\chi), \\
    -\sum_{n=1}^{+\infty} \frac{\Lambda(n)\chi_0(n)}{n^{\sigma}}=\frac{L'}{L}(\sigma,\chi_0), \\
    -\sum_{n=1}^{+\infty} \frac{\Lambda(n)\chi^2(n)}{n^{\sigma+2it}}=\frac{L'}{L}(\sigma+2it,\chi^2).
  \end{gather*}
  Poiché $\chi$ è complesso, $\chi^2$ non è il carattere principale (che aggiungerebbe un polo), quindi possiamo fare come si è fatto per la $\zeta$ quando abbiamo usato la disuguaglianza $3+4\cos{\theta}+\cos(2\theta) \ge 0$. Troviamo così
  $$\frac{4}{\sigma-\beta_\chi}<\frac{3}{\sigma-1}+O\Big(\log\big(q(|t|+2)\big)\Big)$$
  e per concludere basta prendere $\sigma=1+\delta/\mathcal{L}$ con $\delta>0$ opportuno.
\end{proof}

Vediamo ora il caso $\chi$ reale, perciò $\frac{L'}{L}(s,\chi^2)=\frac{L'}{L}(s,\chi_0)$. Si ha
$$\left|\frac{L'}{L}(s,\chi_0)-\frac{\zeta'}{\zeta}(s)\right| \le \log{q},$$
quindi a meno di un termine trascurabile passiamo da $-\mathfrak{Re}\,\frac{L'}{L}(\sigma+2i\gamma,\chi_0)$ a $-\mathfrak{Re}\,\frac{\zeta'}{\zeta}(\sigma+2i\gamma) \le \mathfrak{Re}\,\frac{1}{\sigma-1+2i\gamma_\chi}+c\mathcal{L}$ (e questo perché abbiamo che vale la stima $-\mathfrak{Re}\,\frac{\zeta'}{\zeta}(s) \ll \mathfrak{Re}\,\frac{1}{s-1}+O\big(\log(|t|+2)\big)$). Si ha dunque
$$\frac{4}{\sigma-\beta_\chi}<\frac{3}{\sigma-1}+\mathfrak{Re}\left(\frac{1}{\sigma-1+2i\gamma_\chi}\right)+c\mathcal{L}.$$
Supponiamo che $|\gamma_\chi| \ge \frac{\delta}{\log{q}}$; allora prendendo $\sigma=1+\frac{\delta}{\mathcal{L}}$ otteniamo
\begin{gather*}
  \frac{4\mathcal{L}}{\delta+(1-\beta_\chi)\mathcal{L}}<\frac{3 \mathcal{L}}{\delta}+\frac{\delta\mathcal{L}}{5\delta^2}+c\mathcal{L} \implies \\
  \implies 1-\beta_\chi>\frac{4-5c\delta}{16+5c\delta}\cdot\frac{\delta}{\mathcal{L}}=\frac{c'}{\mathcal{L}},
\end{gather*}
dunque anche in questo caso $\beta_\chi<1-\frac{c_0}{\mathcal{L}}$. Ne segue la seguente proposizione.

\begin{prop}
  Se $\chi$ è reale modulo $q$ e $|\gamma_\chi| \ge \frac{1}{\log{q}}$, esiste $c_0>0$ t.c.
  $$\beta_\chi<1-\frac{c_0}{\mathcal{L}}.$$
\end{prop}

Mostriamo adesso la seguente.

\begin{prop}
  Se $\chi$ è reale modulo $q$ e $\rho_\chi=\beta_\chi+i\gamma_\chi$ con $\gamma_\chi \le \frac{1}{\log{q}}$ e $\beta_\chi>1-\frac{c}{\mathcal{L}}$, allora $\rho_\chi$ è reale e semplice. Inoltre, $\rho_\chi$ è unico.
\end{prop}

\begin{proof}
  $$-\frac{L'}{L}(\sigma,\chi)<c\log{q}-\sum_{\rho_\chi} \mathfrak{Re}\left(\frac{1}{\sigma-\rho_\chi}\right).$$
  Se fosse $L(\beta_0\pm i\gamma_0,\chi)=0$, avremmo
  $$-\frac{L'}{L}(\sigma,\chi)<c\log{q}-\frac{2(\sigma-\beta_0)}{(\sigma-\beta_0)^2+\gamma_0^2};$$
  si ha anche
  \begin{gather*}
    -\frac{L'}{L}(\sigma,\chi)=\sum_{n=1}^{+\infty} \frac{\chi(n)\Lambda(n)}{n^{\sigma}} \ge -\sum_{n=1}^{+\infty} \frac{\Lambda(n)}{n^{\sigma}}= \\
    =\frac{\zeta'}{\zeta}(\sigma)>-\frac{1}{\sigma-1}+c_1.
  \end{gather*}
  Concantenando le due disuguaglianze otteniamo
  $$-\frac{1}{\sigma-1}<c\log{q}-\frac{2(\sigma-\beta_0)}{(\sigma-\beta_0)^2+\gamma_0^2}.$$
  Prendendo $\sigma=1+\frac{2\delta}{\log{q}}$ otteniamo $|\gamma_0|<\frac{\delta}{\log{q}}=(\sigma-1)/2 \le (\sigma-\beta_0)/2$, da cui
  \begin{gather*}
    -\frac{1}{\sigma-1}<c\log{q}-\frac{8}{5(\sigma-\beta_0)} \\
    \frac{8}{5(\sigma-\beta_0)}<c\log{q}+\frac{1}{\sigma-1}=\left(c+\frac{1}{2\delta}\right)\log{q} \\
    \frac{8}{10\delta+5(1-\beta_0)\log{q}}<\frac{1+2\delta c}{2\delta} \\
    10\delta+5(1-\beta_0)\log{q}>\frac{16\delta}{1+2\delta c} \\
    1-\beta_0>\frac{6-20\delta c}{5+10\delta c}\cdot \frac{\delta}{\log{q}},
  \end{gather*}
  da cui con $\delta$ piccolo si ha l'assurdo.

  Il caso con $\gamma_0=0$ è pure più facile.
\end{proof}

Abbiamo così dimostrato la seguente proposizione.

\begin{prop}
  Esiste $c_0>0$ t.c. se $\chi$ è primitivo modulo $q$ e $\beta_\chi+i\gamma_\chi$ è uno zero di $L(s,\chi)$, allora
  $$\beta_\chi<\begin{cases}
    1-\dfrac{c_0}{\log(q|\gamma_\chi|)} &\mbox{se }|\gamma_\chi| \ge 1 \\
    1-\dfrac{c_0}{\log{q}} &\mbox{se } |\gamma_\chi| \le 1
\end{cases};$$
se inoltre $\chi$ è reale, esiste al più un solo zero $\beta_0$ reale semplice t.c. $L(\beta_0,\chi)=0$ e $\beta_0$ non soddisfa la disuguaglianza. Questi sono detti zeri di Siegel.
\end{prop}

Adesso un risultato che non dimostreremo.

\begin{prop} \label{landauq1q2}
  (Landau) Siano $\chi_1$ modulo $q_1$ e $\chi_2$ modulo $q_2$ reali primitivi con $\chi_1\not=\chi_2$, e siano $\beta_1,\beta_2$ t.c. $L(\beta_1,\chi_1)=0,L(\beta_2,\chi_2)=0$. Allora esiste $c_1>0$ t.c.
  $$\min\{\beta_1,\beta_2\}<1-\frac{c_1}{\log(q_1q_2)}.$$
\end{prop}

\begin{cor}
  Se $q_1=q_2=q$, allora $\min\{\beta_1,\beta_2\}<1-\dfrac{c_1}{2\log{q}}=1-\dfrac{c_2}{\log{q}}$.
\end{cor}

\begin{cor}
  Siano $q_1<q_2<\dots$ e siano $\chi_j$ modulo $q_j$ reali primitivi e $L(\beta_j,\chi_j)=0$ con $\beta_j>1-\dfrac{c_3}{\log{q_j}}$, dove $c_3=c_1/3$. Allora $q_{j+1}>q_j^2$.
\end{cor}

\begin{proof}
  Per assurdo $q_{j+1} \le q_j^2$. Allora
  $$\min\{\beta_j,\beta_{j+1}\}<1-\frac{c_1}{\log(q_jq_{j+1})} \le 1-\frac{c_1}{3\log{q_j}} < 1-\frac{c_3}{\log{q_{j+1}}}.$$
\end{proof}

\begin{cor}
  (lemma di Page) Esiste $c_4>0$ t.c. esiste un unico $q \le z$ ed un unico $\chi$ reale primitivo modulo $q$ con $L(\beta_\chi,\chi)=0$ e $\beta_\chi>1-\frac{c_4}{\log{z}}$.
\end{cor}

\begin{proof}
  Per assurdo siano $(\beta_1,\chi_1,q_1)$ e $(\beta_2,\chi_2,q_2)$ con $q_1,q_2 \le z$ t.c. $\beta_j>1-\frac{c_4}{\log{z}}$. Poiché $q_1q_2 \le z^2$, prendendo $c_4=c_1/2$ dalla proposizione \ref{landauq1q2} otteniamo l'assurdo $\min\{\beta_1,\beta_2\}<1-\frac{c_1}{2\log{z}}$.
\end{proof}

Un'altra proposizione che non dimostreremo.

\begin{prop}
  Sia $\chi$ reale primitivo modulo $q$. Vale la disuguaglianza
  $$L(1,\chi)>\frac{c_0}{\sqrt{q}}.$$
\end{prop}

\begin{oss}
  $L(\beta_0,\chi)=0 \implies L(1,\chi)=L(1,\chi)-L(\beta_0,\chi)=$\\
  $=L'(\sigma_0,\chi)(1-\beta_0)$ con $\beta_0 \le \sigma_0 \le 1$. Scegliendo un'opportuna costante $c$, se $\beta_0$ è uno zero di Siegel allora $\sigma_0 \ge \beta_0 \ge 1-c/\log{q}$, e in realtà possiamo prendere $c=1$ perché le costanti in gioco sono sempre più piccole di $1$.
  Allora per il lemma che vedremo tra poco $L'(\sigma_0,\chi)(1-\beta_0) \ll (1-\beta_0)\log^2{q}$ e mettendo tutto assieme troviamo $1-\beta_0 \ge \frac{c_1}{\sqrt{q}\log^2{q}}$.
\end{oss}

\begin{lm}
  Sia $1-(\log{q})^{-1} \le \sigma \le 1$. Si ha $L'(\sigma,\chi) \ll \log^2{q}$, dove $\chi$ modulo $q$ è un carattere qualunque.
\end{lm}

\begin{proof}
  Per $\sigma>1$ e sommando per parti,
  $$L'(\sigma,\chi)=-\sum_{n=1}^{+\infty} \frac{\chi(n)\log{n}}{n^\sigma}=\sigma\int_2^{+\infty} \left(\sum_{n \le u} \chi(u)\right) \frac{1+\sigma\log{u}}{u^{\sigma+1}}\diff u,$$
  che converge uniformemente per $\sigma \ge \epsilon$.
  In realtà, per stimare tronchiamo la somma a $q$ e integriamo per parti la coda. Il primo pezzo si maggiora così:
  $$\left| \sum_{n=1}^q \frac{\chi(n)\log{n}}{n^{\sigma}}\right| \le \frac{1}{e}\sum_{n=1}^q \frac{\log{n}}{n} \ll \log^2{q},$$
  dove la prima disuguaglianza segue da $n^\sigma \ge \exp\big((1-1/\log{q})\log{n}\big) \ge n/e$. Per l'altro pezzo usiamo che $(\log{n})/n^{\sigma}$ decresce per $n>q$, dunque la maggiorazione è
  $$\left|\sum_{n=1+1}^{+\infty} \chi(n)(\log{n})n^{-\sigma}\right| \le (\log{q})q^{-\sigma}\max_N \left|\sum_{n=q+1}^N \chi(n)\right| \le \log{q}\cdot eq^{-1}q.$$
\end{proof}
