In questa sezione sfrutteremo i risultati visti finora per calcolare i valori della $\zeta$ di Riemann sui pari.

\begin{defn}
  Data $F$ intera, il \textit{genere} è $g=\max\{p,q\}$.
\end{defn}

\begin{oss}
  $\alpha-1 \le g \le \alpha$. La seconda è ovvia. Se fosse $g<\alpha-1$, avremmo $p<\alpha-1 \implies \beta \le p+1<\alpha$, ma anche $q<\alpha-1$; dato che abbiamo $\alpha=\max\{\beta,q\}$, si ha un assurdo.
\end{oss}

\begin{ex}
  Sia $F(z)=\dfrac{\sin(\pi z)}{\pi z}$. Si ha
  $$\sin(\pi z)=\frac{e^{i\pi z}-e^{-i\pi z}}{2i} \ll_{\epsilon} e^{|z|^{1+\epsilon}} \text{ per ogni } \epsilon>0,$$
  dunque $F$ ha ordine $\alpha \le 1$. Gli zeri sono $z_n=\pm n$ per ogni $n \in \mathbb{N}$, quindi $\displaystyle \sum_{n=1}^{+\infty} \frac{1}{|z_n|^{1+\epsilon}}<+\infty$ per ogni $\epsilon>0$ ma chiaramente non per $\epsilon=0$, da cui $\beta=1$;
  $\alpha \ge \beta \implies \alpha=1$. Inoltre $p=1$. Studiamo questa funzione.
\end{ex}

I termini del prodotto di Weierstrass sono $E\left(\dfrac{z}{n},1\right)=\left(1-\dfrac{z}{n}\right)\exp\left(\dfrac{z}{n}\right)$. $E\left(\dfrac{z}{n},1\right) \cdot E\left(\dfrac{z}{-n},1\right)=\left(1-\dfrac{z^2}{n^2}\right)$.
Abbiamo quindi il prodotto $\displaystyle \prod_n \left(1-\frac{z^2}{n^2}\right)$. Per il teorema di Hadamard, il grado del polinomio $G$ è $q \le \alpha=1=p$. Per il corollario \ref{1.2.14}, $G(z)=\log{F(0)}+\dfrac{F'}{F}(0)z=\log{1}+0=0$.
Abbiamo allora $\displaystyle \sin(\pi z)=\pi z\prod_{n=1}^{+\infty} \left(1-\frac{z^2}{n^2}\right)$.

Consideriamo
$$\log\big(F(z)\big)=\log\left(\frac{\sin(\pi z)}{\pi z}\right)=\log\big(\sin(\pi z)\big)\log(\pi z);$$
derivando troviamo
$$\frac{F'}{F}(z)=\pi\frac{\cos(\pi z)}{\sin(\pi z)}-\frac{1}{z}=\pi\cot(\pi z)-\frac{1}{z}.$$
Per $k \ge 2$, dal corollario \ref{1.2.14} si ha $\cot$
$$\sum_n \frac{1}{n^k}+\sum_n \frac{1}{(-n)^k}=-\frac{D^{k-1}\big(\pi\cot(\pi z)-1/z\big)_{z=0}}{(k-1)!}.$$
Passando ai pari, per $k \ge 1$ si ha
\begin{gather*}
  \zeta(2k)=\sum_{n=1}^{+\infty} \frac{1}{n^{2k}}=-\frac{D^{2k-1}\big(\frac{\pi}{2}\cot(\pi z)-\frac{1}{2z}\big)_{z=0}}{(2k-1)!} \implies \\
  \implies \frac{1}{2z}-\frac{\pi}{2}\cot(\pi z)=\sum_{k=1}^{+\infty} \zeta(2k)z^{2k-1} \text{ per } |z|<1.
\end{gather*}
È un esercizio verificare che $\zeta(2k)=1+O(1/2^{2k}) \sim 1$ per $k \longrightarrow +\infty$. Segue che $\sqrt[k]{\zeta(2k)} \longrightarrow 1$ per $k \longrightarrow +\infty$, ma anche, in particolare, $\sqrt[2k-1]{\zeta(2k)} \longrightarrow 1$ per $k \longrightarrow +\infty$, dunque il raggio di convergenza della funzione $\dfrac{1}{2z}-\dfrac{\pi}{2}\cot(\pi z)$ è proprio $1$.

\begin{defn}
  Si dicono \textit{numeri di Bernoulli} quelli definiti nel modo seguente:
  $$B_n=D^n\left(\frac{z}{e^z-1}\right)_{z=0}.$$
\end{defn}
