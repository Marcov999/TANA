In questa sezione sfrutteremo i risultati visti finora per calcolare i valori della $\zeta$ di Riemann sui pari.

\begin{defn}
  Data $F$ intera, il \textit{genere} è $g=\max\{p,q\}$.
\end{defn}

\begin{oss}
  $\alpha-1 \le g \le \alpha$. La seconda è ovvia. Se fosse $g<\alpha-1$, avremmo $p<\alpha-1 \implies \beta \le p+1<\alpha$, ma anche $q<\alpha-1$; dato che abbiamo $\alpha=\max\{\beta,q\}$, si ha un assurdo.
\end{oss}

\begin{ex}
  Sia $F(z)=\dfrac{\sin(\pi z)}{\pi z}$. Si ha
  $$\sin(\pi z)=\frac{e^{i\pi z}-e^{-i\pi z}}{2i} \ll_{\epsilon} e^{|z|^{1+\epsilon}} \text{ per ogni } \epsilon>0,$$
  dunque $F$ ha ordine $\alpha \le 1$. Gli zeri sono $z_n=\pm n$ per ogni $n \in \mathbb{N}$, quindi $\displaystyle \sum_{n=1}^{+\infty} \frac{1}{|z_n|^{1+\epsilon}}<+\infty$ per ogni $\epsilon>0$ ma chiaramente non per $\epsilon=0$, da cui $\beta=1$;
  $\alpha \ge \beta \implies \alpha=1$. Inoltre $p=1$. Studiamo questa funzione.
\end{ex}

I termini del prodotto di Weierstrass sono $E\left(\dfrac{z}{n},1\right)=\left(1-\dfrac{z}{n}\right)\exp\left(\dfrac{z}{n}\right)$. $E\left(\dfrac{z}{n},1\right) \cdot E\left(\dfrac{z}{-n},1\right)=\left(1-\dfrac{z^2}{n^2}\right)$.
Abbiamo quindi il prodotto $\displaystyle \prod_n \left(1-\frac{z^2}{n^2}\right)$. Per il teorema di Hadamard, il grado del polinomio $G$ è $q \le \alpha=1=p$. Per il corollario \ref{1.2.14}, $G(z)=\log{F(0)}+\dfrac{F'}{F}(0)z=\log{1}+0=0$.
Abbiamo allora $\displaystyle \sin(\pi z)=\pi z\prod_{n=1}^{+\infty} \left(1-\frac{z^2}{n^2}\right)$.

Consideriamo
$$\log\big(F(z)\big)=\log\left(\frac{\sin(\pi z)}{\pi z}\right)=\log\big(\sin(\pi z)\big)\log(\pi z);$$
derivando troviamo
$$\frac{F'}{F}(z)=\pi\frac{\cos(\pi z)}{\sin(\pi z)}-\frac{1}{z}=\pi\cot(\pi z)-\frac{1}{z}.$$
Per $k \ge 2$, dal corollario \ref{1.2.14} si ha $\cot$
$$\sum_n \frac{1}{n^k}+\sum_n \frac{1}{(-n)^k}=-\frac{D^{k-1}\big(\pi\cot(\pi z)-1/z\big)_{z=0}}{(k-1)!}.$$
Passando ai pari, per $k \ge 1$ si ha
\begin{gather*}
  \zeta(2k)=\sum_{n=1}^{+\infty} \frac{1}{n^{2k}}=-\frac{D^{2k-1}\big(\frac{\pi}{2}\cot(\pi z)-\frac{1}{2z}\big)_{z=0}}{(2k-1)!} \implies \\
  \implies \frac{1}{2z}-\frac{\pi}{2}\cot(\pi z)=\sum_{k=1}^{+\infty} \zeta(2k)z^{2k-1} \text{ per } |z|<1.
\end{gather*}
È un esercizio verificare che $\zeta(2k)=1+O(1/2^{2k}) \sim 1$ per $k \longrightarrow +\infty$. Segue che $\sqrt[k]{\zeta(2k)} \longrightarrow 1$ per $k \longrightarrow +\infty$, ma anche, in particolare, $\sqrt[2k-1]{\zeta(2k)} \longrightarrow 1$ per $k \longrightarrow +\infty$, dunque il raggio di convergenza della funzione $\dfrac{1}{2z}-\dfrac{\pi}{2}\cot(\pi z)$ è proprio $1$.

\begin{defn}
  Si dicono \textit{numeri di Bernoulli} quelli definiti nel modo seguente:
  $$B_n=D^n\left(\frac{z}{e^z-1}\right)_{z=0}.$$
\end{defn}

\begin{oss}
  Poiché la funzione $\frac{z}{e^z-1}$ ha raggio di convergenza $2\pi$, dev'essere $\displaystyle \limsup \sqrt{\frac{B_n}{n!}}=\frac{1}{2\pi}$.
\end{oss}

\begin{oss}
  \begin{gather*}
    f(z)=\frac{z}{e^z-1}+\frac{z}{2}=\frac{z+ze^z}{2(e^z-1)}=\frac{z(1+e^z)}{2(e^z-1)} \\
    f(-z)=\frac{-ze^z}{1-e^z}-\frac{z}{2}=\frac{-ze^z-z}{2(1-e^z)}=\frac{z(1+e^z)}{2(e^z-1)}
  \end{gather*}
  Quindi $f$ è pari. Allora
  $$D^{2k-1}\left(\frac{z}{e^z-1}\right)_{z=0}=-D^{2k-1}\left(\frac{z}{2}\right)_{z=0}=\begin{cases}
    -1/2 & \mbox{se }k=1 \\ 0 & \mbox{se } k>1
\end{cases}$$,
dunque $B_1=-1/2$ e $B_{2n-1}=0$ per ogni $n>1$.
\end{oss}

\begin{oss}
  \begin{gather*}
    1=\left(\sum_{n=0}^{+\infty}\frac{B_n}{n!}z^n\right)\frac{e^z-1}{z}=\left(\sum_{n=0}^{+\infty}\frac{B_n}{n!}z^n\right)\left(\sum_{m=1}^{+\infty}\frac{z^{m-1}}{m!}\right)= \\
    =\sum_{n=1}^{+\infty}\left(\sum_{k=0}^{n-1}\frac{B_k}{k!}\frac{n!}{(n-k)!}\right)\frac{z^{n-1}}{n!}=\sum_{n=1}^{+\infty}\Bigg(\sum_{k=0}^{n-1}\binom{n}{k}B_k\Bigg)\frac{z^{n-1}}{n!}
  \end{gather*}
  Perciò $B_0=1$ e $\displaystyle \sum_{k=0}^{n-1}\binom{n}{k}B_k=0$ per ogni $n \ge 2$, che ci dà $\displaystyle \sum_{k=0}^n\binom{n}{k}B_k=B_n$ e
  $$B_{n-1}=-\frac{1+\binom{n}{1}B_1+\dots+\binom{n}{n-2}B_{n-2}}{\binom{n}{n-1}} \implies B_n \in \mathbb{Q}.$$
\end{oss}

\begin{oss}
  \begin{gather*}
    \cot(z)=i\frac{e^{iz}+e^{-iz}}{e^{iz}-e^{-iz}}=i\frac{e^{2iz}+1}{e^{2iz}-1}=i\left(1+\frac{2z}{z(e^{2iz}-1)}\right)= \\
    =i+\frac{1}{z}\cdot\frac{2iz}{e^{2iz}-1}=i+\frac{1}{z}\left(1-iz+\sum_{k=1}^{+\infty}\frac{(-1)^kB_{2k}(2z)^{2k}}{(2k)!}\right)= \\
    =\frac{1}{z}+\sum_{k=1}^{+\infty}\frac{2(-1)^kB_{2k}}{(2k)!}(2z)^{2k-1} \implies \\
    \implies \sum_{k=1}^{+\infty}\zeta(2k)z^{2k-1}=\frac{1}{2z}-\frac{\pi}{2}\cot(\pi z)=\frac{1}{2}\sum_{k=1}^{+\infty} \frac{(-1)^{k-1}(2\pi)^{2k}B_{2k}z^{2k-1}}{(2k)!} \implies \\
    \implies 1+O(1/4^k)=\zeta(2k)=\frac{(-1)^{k-1}(2\pi)^{2k}B_{2k}}{2(2k)!} \implies \\
    \implies (-1)^{k-1}B_{2k}=\frac{2(2k)!}{(2\pi)^{2k}}\zeta(2k).
  \end{gather*}
\end{oss}

\begin{oss}
  $B_{2k}(-1)^{k-1}>0$. Inoltre,
  $$\frac{2(2k)!}{(2\pi)^{2k}} \le |B_{2k}| \le \frac{2(2k)!\pi^2}{6(2\pi)^{2k}} \text{ per ogni } k \ge 1.$$
  Vogliamo vedere quando si ha
  \begin{gather*}
    |B_{2(k+1)}| \ge \frac{2\big(2(k+1)\big)!}{(2\pi)^{2(k+1)}} \stackrel{?}{\ge} \frac{2(2k)!\pi^2}{6(2\pi)^{2k}} \ge |B_{2k}| \\
    \frac{2(k+1)(2k+1)}{4\pi^2} \stackrel{?}{\ge} \frac{\pi^2}{6} \\
    \pi^4 \stackrel{?}{\le} 3(k+1)(2k+1),
  \end{gather*}
  che è vero per $k \ge 4$. Da lì in poi, i numeri di Bernoulli di indice pari hanno moduli crescenti.
\end{oss}

\begin{defn}
  Si dice \textit{polinomio di Bernoulli} $n$-esimo il seguente:
  $$B_n(x)=\sum_{k=0}^n \binom{n}{k}B_kx^{n-k}.$$
\end{defn}

\begin{ftt}
  \begin{itemize}
    \item $B_n(0)=B_n$
    \item $\displaystyle B_n(1)=\sum_{k=0}^n \binom{n}{k}B_k=(-1)^nB_n$
  \end{itemize}
\end{ftt}

\begin{oss}
  \begin{gather*}
    \frac{ze^{xz}}{e^z-1}=\left(\sum_{n=0}^{+\infty}\frac{B_n}{n!}z^n\right)\left(\sum_{m=0}^{+\infty}\frac{x^mz^m}{m!}\right)=\sum_{n=0}^{+\infty}\left(\sum_{k=0}^n\frac{B_k}{k!}\cdot \frac{x^{n-k}}{(n-k)!}\right)z^n= \\
    =\sum_{n=0}^{+\infty}\left(\sum_{k=0}^nB_k\binom{n}{k}x^{n-k}\right)\frac{z^n}{n!}=\sum_{n=0}^{+\infty}\frac{B_n(x)}{n!}z^n.
  \end{gather*}
\end{oss}

Adesso un po' di fatti che chi vuole può divertirsi a dimostrare per esercizio.

\begin{ftt}
  \begin{enumerate}
    \item $\displaystyle B_n(x+y)=\sum_{k=0}^n\binom{n}{k}B_k(x)y^{n-k}$. Per $y=1$ si ha $\displaystyle B_n(x+1)=\sum_{k=0}^n\binom{n}{k}B_k(x)$;
    \item $B_n(x+1)-B_n(x)=nx^{n-1}$;
    \item $\displaystyle \sum_{k=m}^{n-1} k^r=\frac{B_{r+1}(n)-B_{r+1}(m)}{r+1}$;
    \item $B'_n(x)=nB_{n-1}(x)$.
  \end{enumerate}
\end{ftt}

Adesso, un caso particolare di un teorema che non dimostreremo, che si utilizza per dimostrare una proposizione che dimostreremo in un altro modo.

\begin{thm}
  (formula di sommazione di Eulero) Sia $f:[a,b] \longrightarrow \mathbb{C}$ di classe $C^1$. Allora
  $$\sum_{a<k \le b}f(k)=\int_a^b f(x)\diff x-\bigg[B_1(\{x\})f(x)\bigg]_a^b+\int_a^b B_1(\{x\})f'(x)\diff x.$$
  Se $a=m, b=n$,
  $$\sum_{m<k \le n} f(k)=\int_m^nf(x)\diff x-B_1\big(f(n)-f(m)\big)+\int_m^n B_1(\{x\})f'(x)\diff x.$$
\end{thm}

\begin{prop}
  Esiste
  $$\lim_{n \longrightarrow +\infty} \sum_{k=1}^n \frac{1}{k}-\log{n}=\gamma,$$
  con $0<\gamma<1$.
\end{prop}
