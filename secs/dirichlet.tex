\begin{defn}
  Sia $q>1$ un intero. I \textit{caratteri} modulo $q$ sono i caratteri del gruppo abeliano $(\mathbb{Z}/q \mathbb{Z})^*$, definiti come stiamo per vedere.
\end{defn}

I caratteri modulo $q$ sono un gruppo isomorfo a $(\mathbb{Z}/q \mathbb{Z})^*$, quindi sono in numero $\phi(q)$, e sono funzioni $\chi:\mathbb{Z} \longrightarrow \mathbb{C}$ (se si restringe il dominio a $(\mathbb{Z}/q \mathbb{Z})^*$, si può osservare, in base alla definizione esplicita, che i caratteri sono omomorfismi a valori nelle radici $\phi(q)$-esime dell'unità). C'è un carattere principale $\chi_0$, che assume solo i valori $0$ e $1$ in base a una regola che andiamo a vedere. I caratteri devono soddisfare:
\begin{itemize}
  \item $(\chi_1\cdot\chi_2)(n)=\chi_1(n)\cdot\chi_2(n)$;
  \item $\chi^{-1}(n)=\bar{\chi}(n)$;
  \item $\chi(nm)=\chi(n)\chi(m)$ per ogni $n,m \in \mathbb{Z}$ (sono completamente moltiplicativi);
  \item $\chi$ ha periodo $q$;
  \item $|\chi(n)|=1$ se $(n,q)=1$, altrimenti è $0$.
\end{itemize}

Vediamo ora la definizione esplicita, procedendo per casi.

Caso $p^a$ con $p>2$ primo. $(\mathbb{Z}/p^a \mathbb{Z})^*$ è ciclico, sia $g$ un generatore. Prendiamo $\omega$ t.c. $\omega^{\phi(p^a)}=1$ (radice $\phi(p^a)$-esima dell'unità). Dato $n$ intero, sia $\nu(n)$ t.c. $g^{\nu(n)}=[n]_{\mathbb{Z}/p^a \mathbb{Z}}$.
Il carattere corrispondente a $\omega$ è $\chi_{p^a}(n)=\omega^{\nu(n)}=e^{\frac{2\pi im\nu(n)}{\phi(p^a)}}$.

Il caso $(\mathbb{Z}/2 \mathbb{Z})^*$ è banale.

Caso $2^2$: $\phi(4)=2$; il carattere principale è banale, per l'altro:
$$\chi_4(n)=\begin{cases}
  1 &\mbox{se } n \equiv 1 \pmod{4} \\
  -1 &\mbox{se } n \equiv 3 \pmod{4} \\
  0 &\mbox{se } n \equiv 0 \pmod{2}
\end{cases}$$

Caso $2^a$ con $a \ge 3$: $(\mathbb{Z}/2^a \mathbb{Z})^* \cong \mathbb{Z}/2 \mathbb{Z} \times \mathbb{Z}/2^{a-2} \mathbb{Z}$.
Sia $g$ un generatore della parte $\mathbb{Z}/2^{a-2} \mathbb{Z}$ e $\nu_0'$ t.c. $g^{\nu_0'} \equiv [n]$. Allora $n \equiv (-1)^{\nu_0}5^{\nu_0'} \pmod{2^a}$ e poniamo $\chi(n)=e^{\pi im_0\nu_0+\frac{2\pi im_0'\nu_0'}{2^{a-2}}}$.

Caso generale $q=2^a\cdot p_1^{a_1}\cdots p_r^{a_r}$:
$$\chi_q(n)=e^{\pi im_0\nu_0+\frac{2\pi im_0'\nu_0'}{2^{a-2}}+\frac{2\pi im_1\nu_1}{\phi(p_1^{a_1})}+\dots+\frac{2\pi im_r\nu_r}{\phi(p_r^{a_r})}}.$$
Alla scelta degli $m_j$ corrisponde il carattere, mentre la dipendenza da $n$ è data dai $\nu_j$.

Per esercizio, il lettore può dimostrare le seguenti formule di ortogonalità:
\begin{itemize}
  \item $\displaystyle \sum_{n=1}^q \chi(n)=\begin{cases}
    0 &\mbox{se }\chi\not=\chi_0 \\
    \phi(q) &\mbox{se }\chi=\chi_0
  \end{cases}$;
  \item $\displaystyle \sum_{\chi\text{ mod }q} \chi(n)=\begin{cases}
    0 &\mbox{se }n\not\equiv 1\pmod{q} \\
    \phi(q) &\mbox{se }n\equiv 1\pmod{q}
  \end{cases}$.
\end{itemize}
Osserviamo che $\bar{\chi}(a)=\chi^{-1}(a)=\chi(a^{-1})$. Dalla seguente formula di ortogonalità segue il seguente risultato: se $(a,q)=1$, allora
$$\frac{1}{\phi(q)}\sum_{\chi\text{ mod }q}\bar{\chi}(a)\chi(n)=\begin{cases}
  1 &\mbox{se }n\equiv a \pmod{q} \\
  0 &\mbox{altrimenti}
\end{cases};$$
questo permette in un qualche senso di isolare i numeri congrui ad $a$ modulo $q$, per poter così dimostrare il teorema di Dirichlet.

Andiamo adesso a definire una classe di funzioni molto importanti, non solo perché ci serviranno nella dimostrazione del suddetto teorema, ma anche perché sono una specie di generalizzazione della $\zeta$ di Riemann (se vogliamo, la $\zeta$ corrisponde alla funzione definita a partire dai ``caratteri modulo $1$'', anche se ovviamente quest'affermazione a poco o nessun senso).

\begin{defn}
  Sia $\chi$ un carattere non principale modulo $q$. Si dice funzione $L$ relativa al carattere $\chi$ la somma della seguente serie, definita per $\sigma>1$:
  $$L(s,\chi)=\sum_{n=1}^{+\infty} \frac{\chi(n)}{n^s}.$$
\end{defn}

\begin{prop}
  Se $\chi$ non è principale, la serie appena definita converge uniformemente per $\sigma \ge \epsilon>0$.
\end{prop}

\begin{proof}
  Usando la solita sommazione per parti,
  $$\sum_{n \le N} \frac{\chi(n)}{n^s}=\left(\sum_{n \le N} \chi(n)\right)N^{-s}+s\int_1^N\left(\sum_{n \le u} \chi(u)\right)\frac{\diff u}{u^{s+1}}.$$
  Dalla prima formula di ortogonalità segue che $\displaystyle \left|\sum_{n \le N} \chi(n)\right| \le q$, dunque per $\sigma>1$ si ha
  $$L(s,\chi)=s\int_1^{+\infty} \left(\sum_{n \le u}\chi(u)\right)\frac{\diff u}{u^{s+1}},$$
  e questo integrale converge uniformemente in $\sigma \ge \epsilon>0$.
\end{proof}

\begin{oss}
  Se prendiamo $\chi=\chi_0$ e consideriamo la serie associata, abbiamo $\displaystyle \sum_{n \le N} \chi_0(n) \le  \left(\left\lfloor \frac{N}{q}\right\rfloor+1\right)\phi(q)$, dunque non si ottiene convergenza uniforme per $\sigma \ge \epsilon>0$.
\end{oss}

\begin{oss}
  Per l'identità di Eulero si ha, per $\sigma>1$ e per ogni carattere modulo $q$:
  $$L(s,\chi)=\prod_p\left(1-\frac{\chi(p)}{p^s}\right)^{-1}.$$
  In particolare,
  \begin{gather*}
    L(s,\chi_0)=\prod_p\left(1-\frac{\chi_0(p)}{p^s}\right)^{-1}=\prod_{p \nmid q}\left(1-\frac{1}{p^s}\right)^{-1}= \\
    =\prod_p \left(1-\frac{1}{p^s}\right)^{-1}\prod_{p\mid q}\left(1-\frac{1}{p^s}\right)=\prod_{p\mid q}\left(1-\frac{1}{p^s}\right)\zeta(s),
  \end{gather*}
  da cui si ottiene che $L(s,\chi_0)$ è meromorfa con un polo semplice in $s=1$ con residuo $\displaystyle \prod_{p \mid q} \left(1-\frac{1}{p}\right)=\frac{\phi(q)}{q}$. Ha altri zeri oltre quelli di $\zeta$, che si ottengono quando $p^{-s}=1$, cioè $\sigma=0$ e $t=\frac{2k\pi}{\log{p}}$.
\end{oss}

\begin{oss} \label{logL}
  Per $\sigma>1$ abbiamo
  \begin{gather*}
    \log\big(L(s,\chi)\big)=-\sum_p \log\left(1-\frac{\chi(p)}{p^s}\right)=\sum_p\sum_{n=1}^{+\infty} \frac{\chi(p^n)}{np^{ns}} \implies \\
    \implies \frac{L'}{L}(s,\chi)=-\sum_{n=1}^{+\infty} \frac{\chi(n)\Lambda(n)}{n^s}.
  \end{gather*}
  In particolare,
  $$\frac{L'}{L}(s,\chi_0)=-\sum_{n=1}^{+\infty} \frac{\chi_0(n)\Lambda(n)}{n^s}=-\sum_{\substack{n=1, \\ (n,q)=1}}^{+\infty} \frac{\Lambda(n)}{n^s}=-\sum_{n=1}^{+\infty} \frac{\Lambda(n)}{n^s}+\sum_{\substack{p \mid q, \\a \ge 1}} \frac{\log{p}}{p^{as}}.$$
  Scriviamo adesso
  \begin{gather*}
    \log\big(L(s,\chi)\big)=\sum_p \frac{\chi(p)}{p^s}+O\left(\sum_p \sum_{n \ge 2} \frac{1}{p^{\sigma n}}\right)=\\
    =\sum_p \frac{\chi(p)}{p^s}+O\left(\sum_p \frac{1}{p^{\sigma}(p^{\sigma}-1)}\right)\overset{\sigma>1/2}{=}\sum_p \frac{\chi(p)}{p^s}+O(1) \implies \\
    \implies \sum_{\chi\text{ mod }q}\log\big(L(s,\chi)\big)=\sum_{\chi\text{ mod }q}\sum_p \frac{\chi(p)}{p^s}+O\big(\phi(q)\big).
  \end{gather*}
  Prendendo $s$ reale, e ricordando che i caratteri complessi sono accoppiati ciascuno con il proprio coniugato, si ha
  $$\log\left(\prod_{\chi\text{ mod }q}|L(s,\chi)|\right)=\sum_p\left(\frac{1}{p^s}+\sum_{\chi\not=\chi_0}\frac{\chi(p)}{p^s}\right)+O\big(\phi(q)\big).$$
\end{oss}

Ripetendo i passaggi appena visti, nell'ipotesi $(a,q)=1$ usando la formula di ortogonalità si dimostra anche che
$$\frac{1}{\phi(q)}\sum_{\chi\text{ mod }q} \bar{\chi}(a)\log\big(L(s,\chi)\big)\overset{(\star)}{=}\sum_{p \equiv a \pmod{q}} \frac{1}{p^s}+O(1).$$
Ricordiamo $\displaystyle L(s,\chi_0)=\zeta(s)\prod_{p \mid q} \left(1-\frac{1}{p^s}\right)$; perciò si ha $\log\big(L(s,\chi_0)\big) \overset{s \longrightarrow 1^+}{\longrightarrow} +\infty$ (prendendo $s \in \mathbb{R}$), e per la precisione c'è un polo semplice in $1$.
Se per $\chi\not=\chi_0$ si avesse $L(1,\chi)\not=0$, allora otterremmo $\log\big(L(s,\chi)\big)\overset{s \longrightarrow 1^+}{=}O(1)$; combinandolo con quello che sappiamo sul carattere principale, si avrebbe che il membro sinistro (e dunque anche il destro) di $(\star)$ tenderebbe a $+\infty$ per $s \longrightarrow 1^+$. Avremmo dunque il seguente risultato.

\begin{thm}
  (teorema di Dirichlet sui primi nelle progressioni aritmetiche) Siano $a,q$ interi con $(a,q)=1$. Esistono infiniti primi $p \equiv a \pmod{q}$.
\end{thm}

\begin{proof}
  Per quanto appena detto, dobbiamo verificare che le serie associate ai caratteri non principali non si annullano in $1$. Procediamo per assurdo.

  Caso $\chi$ complesso. Se $L(1,\chi)=0$, anche $L(1,\bar{\chi})=0$. Prendendo $a=1$ in $(\star)$, per $s \in \mathbb{R}, s>1$ e sfruttando l'espressione vista nella prima riga dell'osservazione \ref{logL}, avremmo
  $$\sum_{\chi\text{ mod }q} \log\big(L(s,\chi)\big)=\phi(q)\sum_{p^n \equiv 1 \pmod{q}} \frac{1}{np^{ns}} \in \mathbb{R}^+ \implies \prod_{\chi\text{ mod }q} |L(s,\chi)| \ge 1,$$
  assurdo perché in $1$ si avrebbero un polo semplice e due zeri.

  Se $\chi\not=\chi_0$ è reale e per assurdo $L(s,\chi)=0$, sia
  $$F(s)=\frac{L(s,\chi)L(s,\chi_0)}{L(2s,\chi_0)}.$$
  $F$ sarebbe regolare per $\sigma \ge 1/2$, in quanto il polo semplice di $L(s,\chi_0)$ in $1$ sarebbe compensato dall'eventuale $0$. Inoltre $\displaystyle \lim_{s \longrightarrow 1/2^+} F(s)=0$. Si ha
  $$F(s)=\prod_p \frac{\left(1-\frac{\chi(p)}{p^s}\right)^{-1}\left(1-\frac{\chi_0(p)}{p^s}\right)^{-1}}{\left(1-\frac{\chi_0(p)}{p^{2s}}\right)^{-1}}=\prod_{p\nmid q} \frac{\left(1-\frac{\chi(p)}{p^s}\right)^{-1}\left(1-\frac{1}{p^s}\right)^{-1}}{\left(1-\frac{1}{p^{2s}}\right)^{-1}}.$$
  Per i caratteri t.c. $\chi(p)=-1$ il fattore si semplifice e diventa $1$. Resta
  $$\prod_{\chi(p)=1} \frac{\left(1-\frac{1}{p^s}\right)^{-1}\left(1-\frac{1}{p^s}\right)^{-1}}{\left(1-\frac{1}{p^{2s}}\right)^{-1}}=\prod_{\chi(p)=1} \frac{p^s+1}{p^s-1}\overset{\sigma>1}{=} \sum_{n=1}^{+\infty} \frac{a_n}{n^s},$$
  dove $a_1=1$ e $a_n \ge 0$. La serie di Taylor di $F$ centrata in $s=2$ ha raggio almeno $3/2$. È $\displaystyle F(s)=\sum_{m=0}^{+\infty} \frac{b_m}{m!}(s-2)^m$ dove $\displaystyle b_m=(-1)^m\sum_{n=1}^{+\infty} \frac{a_n(\log{n})^m}{n^2}$, dunque
  $$F(s)=\sum_{m=0}^{+\infty} \frac{1}{m!} \sum_{n=1}^{+\infty} \frac{a_n(\log{n})^m}{n^2}(2-s)^m.$$
  Per $s$ reale minore di $2$ è tutto positivo, dunque $F(s) \ge a_1=1$ per $\sigma \ge 1/2$, assurdo.
\end{proof}

Sappiamo di più:
$$\sum_{\substack{p\equiv a\pmod{q}, \\ p \le x}} \frac{1}{p}=\frac{1}{\phi(q)}\log\log{x}+O_q(1).$$
Per mostrarlo, bisogna dire che $\displaystyle \sum_p \frac{\chi(p)}{p}$ converge per $\chi\not=\chi_0$. Mertens dimostrò questa cosa passando per la convergenza di $\displaystyle \sum_n \frac{\chi(n)\log{n}}{n}$ e usando che $\Lambda \star 1=L$, dove $L$ è la funzione aritmetica associata al logaritmo. Maggiori dettagli nel capitolo 7 di \cite{D}.
