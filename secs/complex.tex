\begin{lm} \label{mittagleffler}
  (Mittag-Leffler) Sia $(z_n)_{n \in \mathbb{N}}$ una successione di numeri complessi t.c. $|z_n| \longrightarrow \infty$ per $n \longrightarrow \infty$ e $0<|z_n| \le |z_{n+1}|$ per ogni $n$. Sia inoltre $(m_n)_{n \in \mathbb{N}}$ un'altra successione con $m_n \in \mathbb{C}^*$ per ogni $n$. Allora esistono $p_n \in \mathbb{N}\cup \{0\}$ t.c.
  $$f(z)=\sum_{n=1}^{+\infty} \left(\frac{z}{z_n}\right)^{p_n}\frac{m_n}{z-z_n}$$
  converge in $K \subset \mathbb{C}\setminus\{z_1, z_2, \dots\}$ compatto. Inoltre, se $|z|<|z_1|$, si ha
  $$f(z)=-\sum_{k=1}{+\infty} \left(\sum_{n:p_n<k}m_nz_n^{-k}\right)z^{k-1}.$$
\end{lm}

\begin{proof}
  Prendiamo $r_n$ reali positivi con $r_n \le r_{n+1}$ e $r_n \longrightarrow +\infty$ per $n \longrightarrow +\infty$, e t.c. $r_n<|z_n|$. Per $|z| \le r_n$ si ha
  $$\left|\frac{m_n}{z-z_n}\right| \le \frac{m_n}{|z_n|-r_n}, \quad \left|\frac{z}{z_n}\right|<\frac{|z|}{r_n} \le 1 \implies$$
  perciò si ha che esistono $p_n \in \mathbb{N}\cup\{0\}$ t.c. $\displaystyle \left|\frac{z}{z_n}\right|^{p_n} <\epsilon_n \frac{|z_n|-r_n}{|m_n|}$, con $\epsilon_n>0$ e $\displaystyle \sum_{n=1}^{+\infty} \epsilon_n<+\infty$.
  Abbiamo dunque $\displaystyle \left|\left(\frac{z}{z_n}\right)^{p_n}\frac{m_n}{z-z_n}\right| \le \left|\frac{z}{z_n}\right|^{p_n} \frac{|m_n|}{|z_n|-r_n}<\epsilon_n$. Fissiamo ora il compatto $K$ e consideriamo $N$ t.c. $|z| \le r_N$ per ogni $z \in K$.
  Poniamo $\displaystyle M_n=\max_{z \in K} \left|\left(\frac{z}{z_n}\right)^{p_n}\frac{m_n}{z-z_n}\right|$, per $n \le N-1$. Per ogni $z \in K$ si ha che
  $$\sum_{n=1}^{+\infty} \left|\left(\frac{z}{z_n}\right)^{p_n}\frac{m_n}{z-z_n}\right| \le \sum_{n=1}^{N-1} M_n+\sum_{n=N}^{+\infty} \epsilon_n<+\infty.$$

  Se $|z|<|z_1|$, possiamo scrivere
  \begin{align*}
    f(z) & =\sum_{n=1}^{+\infty} \left(\frac{z}{z_n}\right)^{p_n}\frac{m_n}{z-z_n} \\
    & =-\sum_{n=1}^{+\infty} \frac{m_nz^{p_n}}{z_n^{p_n+1}}\frac{1}{1-z/z_n}=-\sum_{n=1}^{+\infty}m_n\sum_{k=p_n+1}^{+\infty} \frac{z^{k-1}}{z_n^k}.
  \end{align*}
  Poiché nella prima parte della dimostrazione abbiamo visto che c'è convergenza totale, possiamo scambiare le due sommatorie ottenendo così la seconda parte della tesi.
\end{proof}

\begin{oss} \label{1.1.2}
  Per $|z| \le r_n$ si ha
  \begin{gather*}
    \left|\frac{z}{z_n}\right|=\frac{2|z|}{|z_n|+|z_n|}<\frac{2|z|}{|z_n|+r_n} \le \frac{2|z|}{|z-z_n|} \\
    |m_m| \left|\frac{z}{z_n}\right|^{p_n+1} \le \frac{2|z|}{|z-z_n|}\left|\frac{z}{z_n}\right|^{p_n}|m_n|=2|z|\left|\left(\frac{z}{z_n}\right)^{p_n}\frac{m_n}{z-z_n}\right| \\
    \sum_{n=1}^{+\infty} |m_n|\left|\frac{z}{z_n}\right|^{p_n+1} \le 2|z|\sum_{n=1}^{+\infty} \left|\left(\frac{z}{z_n}\right)^{p_n}\frac{m_n}{z-z_n}\right|<+\infty.
  \end{gather*}
\end{oss}

\begin{ex}
  Se $z_n=n$ e $m_n=1$ per ogni $n$, basta prendere $p_n=1$.

  Sia invece $\displaystyle |z_0|>\max_K |z|$ e consideriamo $|z_n|>|z_0|+1$. Si ha che
  \begin{align*}
    \left|\left(\frac{z}{z_n}\right)^{p_n}\frac{m_n}{z-z_n}\right| & \le \left|\frac{z_0}{z_n}\right|^{p_n+1}\frac{|m_n|}{|z_n|-|z_0|}\frac{|z_n|}{|z_0|} \\
    & \le \left|\frac{z_0}{z_n}\right|^{p_n+1}\frac{|z_0|+1}{|z_0|+1-|z_0|}\frac{|m_n|}{|z_0|}=|m_n|\left|\frac{z_0}{z_n}\right|^{p_n+1}\left(1+\frac{1}{|z_0|}\right),
  \end{align*}
  dove la seconda disugaglianza segue dal fatto che la funzione $\frac{t}{t-|z_0|}$ è decrescente. In questo caso, vale la maggiorazione opposta a quella dell'osservazione \ref{1.1.2}.
\end{ex}

\begin{lm} \label{1.1.4}
  Sia $f$ meromorfa con poli semplici nei punti $z_n\not=0$, con residui $m_n \in \mathbb{Z}$, e t.c. $|z_n| \longrightarrow +\infty$ per $n \longrightarrow +\infty$. Sia $\gamma(0,z)$ un cammino da $0$ a $z$ non passante per i punti $z_n$. Allora la funzione
  $$\varphi(z)=\exp\left(\int_{\gamma(0,z)} f(w)\diff w\right)$$
  è meromorfa, con zeri $z_n$ con molteplicità $m_n$ se $m_n>0$ e poli $z_n$ con molteplicità $-m_n$ se $m_n<0$.
\end{lm}

\begin{oss} \label{1.1.5}
  Sia $\gamma$ il cammino dell'integrale del lemma \ref{1.1.4} e $\gamma'$ un altro cammino, t.c. $\gamma' \cup -\gamma$ sia una curva di Jordan (piana, semplice, chiusa) contenuta in $\mathbb{C}\setminus\{z_1,z_2,\dots\}$. Per il teorema dei residui si ha
  $$\int_{\gamma}f(w)\diff w=\int_{\gamma'}f(w)\diff w+2\pi iR,$$
  con $R=\displaystyle\sum_{\substack{n:z_n \in A, \\ \partial A=\gamma'\cup-\gamma}} m_n$. Dunque $\displaystyle\varphi(z)=\exp\left(\int_{\gamma'}f(w)\diff w\right)e^{2\pi iR}$.
\end{oss}

\begin{proof}
  Poiché $m_n \in \mathbb{Z}$, per l'osservazione \ref{1.1.5} $\varphi$ non dipende dal cammino scelto. Consideriamo $f_1(z)=f(z)-\dfrac{m_1}{z-z_1}$. $f_1$ è olomorfa in $\mathbb{C}\setminus\{z_2,z_3,\dots\}$, quindi $\displaystyle \exp\left(\int_0^z f_1(w)\diff w\right)$ è olomorfa e mai nulla in $\mathbb{C}\setminus\{z_2,z_3,\dots\}$.
  \begin{gather*}
    \varphi(z)=\exp\left(\int_0^z f_1(w)\diff w+m_1\int_0^z\frac{\diff w}{w-z_1}\right), \int_0^z \frac{\diff w}{w-z_1}=\log\left(\frac{z-z_1}{-z_1}\right) \implies \\
    \implies \varphi(z)=\exp\left(\int_0^z f_1(w)\diff w\right)\cdot\exp\Bigg(m_1\log\left(\frac{z-z_1}{-z_1}\right)\Bigg)= \\
    =\exp\left(\int_0^z f_1(w)\diff w\right)\cdot(-z_1)^{-m_1}\cdot(z-z_1)^{m_1}=\varphi_1(z)(z-z_1)^{m_1},
  \end{gather*}
  dove $\varphi_1$ è una funzione olomorfa e mai nulla in $\mathbb{C}\setminus\{z_2,z_3,\dots\}$. Allora $\varphi$ ha uno zero o un polo dell'ordine voluto in $z_1$. Ripetendo per ogni $n$ si ha la tesi.
\end{proof}

Con i due lemmi appena mostrati si può costruire una funzione meromorfa su $\mathbb{C}$ con zeri e poli di molteplicità assegnata, assumendo che la successione degli stessi non abbia alcun limite finito.

\begin{thm} \label{wprod}
  (prodotto di Weierstrass) Sia $F$ meromorfa in $\mathbb{C}$ e siano $z_n \not=0$ gli zeri e i poli di molteplicità $|m_n|$. Esistono una successione $p_n \in \mathbb{N}\cup\{0\}$ e una funzione intera $G(z)$ t.c.
  \begin{equation} \label{wprodformula}
    F(z)=e^{G(z)}\prod_n\left(1-\frac{z}{z_n}\right)^{m_n}\exp\Bigg(m_n\sum_{k=1}^{p_n}\frac{1}{k}\left(\frac{z}{z_n}\right)^k\Bigg),
  \end{equation}
  dove il prodotto infinito converge uniformemente in ogni $K\subset \mathbb{C}\setminus\{z_1,z_2,\dots\}$ compatto. Inoltre $\displaystyle \sum_{n=1}^{+\infty} |m_n|\left|\frac{z}{z_n}\right|^{p_n+1}<+\infty$ per ogni $z \in K$.
\end{thm}

\begin{proof}
  Costruiamo $\displaystyle f(z)=\sum_{n=1}^{+\infty} \left(\frac{z}{z_n}\right)^{p_n}\frac{m_n}{z-z_n}$ come nel lemma \ref{mittagleffler} e $\displaystyle \varphi=\exp\left(\int_0^z f(w)\diff w\right)=\exp\Bigg(\int_0^z \sum_{n=1}^{+\infty}\left(\frac{w}{z_n}\right)^{p_n}\frac{m_n}{w-z_n}\diff w\Bigg)$ come nel lemma \ref{1.1.4}.
  Osserviamo che
  $$\left(\frac{w}{z_n}\right)^{p_n}\frac{1}{w-z_n}=\frac{1}{w-z_n}+\frac{1}{z_n}\sum_{k=0}^{p_n-1}\left(\frac{w}{z_n}\right)^n,$$
  quindi
  \begin{gather*}
    \varphi(z)=\prod_{n=1}^{+\infty} \exp\Bigg[\int_0^z\Bigg(\frac{m_n}{w-z_n}+\frac{m_n}{z_n}\sum_{k=0}^{p_n-1}\left(\frac{w}{z_n}\right)^k\Bigg)\diff w\Bigg]= \\
    =\prod_{n=1}^{+\infty}\exp\Bigg(m_n\log\left(\frac{z-z_n}{-z_n}\right)+m_n\sum_{k=1}^{p_n}\frac{1}{k}\left(\frac{z}{z_n}\right)^k\Bigg)= \\
    =\prod_{n=1}^{+\infty} \left(1-\frac{z}{z_n}\right)^{m_n}\exp\Bigg(m_n\sum_{k=1}^{p_n}\frac{1}{k}\left(\frac{z}{z_n}\right)^k\Bigg).
  \end{gather*}
  Osserviamo ora che $\frac{F(z)}{\varphi(z)}$ è intera e mai nulla in $\mathbb{C}$, dunque esiste $G(z)$ intera t.c. $\frac{F(z)}{\varphi(z)}=e^{G(z)} \implies F(z)=e^{G(z)}\varphi(z)$, da cui la tesi.
\end{proof}

\begin{oss}
  Se $F(z)$ ha uno zero o un polo di molteplicità $|m|$ in $0$, basta applicare il teorema \ref{wprod} alla funzione $\tilde{F}(z)=F(z)/z^m$.
\end{oss}
