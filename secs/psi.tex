Riemann congetturò che
$$\psi(x)=\sum_{n \le x} \Lambda(n)=x-\sum_{\rho} \frac{x^{\rho}}{\rho}-\frac{\zeta'}{\zeta}(0)-\frac{1}{2}\log\left(1-\frac{1}{x^2}\right).$$
Beh, non proprio: l'espressione a destra è continua, a differenza di $\psi$. Utilizzeremo $\psi_0(x)=\displaystyle \sum_{n<x} \Lambda(n)+\frac{1}{2}\Lambda(x)$, dove poniamo $\Lambda=0$ fuori dagli interi. L'idea di Riemann è di scrivere
$$\psi_0(x) \sim \frac{1}{2\pi i} \int_{c-i\infty}^{c+i\infty} -\frac{\zeta'}{\zeta}(s)\frac{x^s}{s}\diff s$$
con $c>1$ e applicare il teorema dei residui. Ma come ha fatto a derivare un'espressione tanto precisa? Ha guardato i poli dell'integranda.

\begin{lm}
  Dato $c>0$, sia $I(y,T)=\displaystyle \frac{1}{2\pi i}\int_{c-iT}^{c+iT} y^s\frac{\diff s}{s}$; allora
  $$|I(y,T)-\delta(y)| \le \begin{cases}
    y^c\min\left\{1,\frac{1}{T|\log{y}|}\right\} & \mbox{se } y\not=1 \\
    \frac{c}{T} & \mbox{se } y=1,
\end{cases}$$
dove $\delta(y)=\begin{cases}
  1 & \mbox{se } y>1 \\
  \frac{1}{2} & \mbox{se } y=1 \\
  0 & \mbox{se } 0<y<1
\end{cases}$. Di conseguenza, $\displaystyle \frac{1}{2\pi i}\int_{c-i\infty}^{c+i\infty} y^s\frac{\diff s}{s}=\delta(y)$.
\end{lm}

\begin{proof}
  Preso $0<y<1$, consideriamo l'integrale sul cammino in figura, percorso in senso orario.
  \begin{center}
    \begin{tikzpicture}[line cap=round,line join=round,>=triangle 45,x=1.0cm,y=1.0cm]
      \draw[->,color=black] (-2,0) -- (6.86,0);
      \foreach \x in {-1,1,2,3,4,5,6}
      \draw[shift={(\x,0)},color=black] (0pt,2pt) -- (0pt,-2pt);
      \draw[->,color=black] (0,-3.94) -- (0,3.98);
      \foreach \y in {-3,-2,-1,1,2,3}
      \draw[shift={(0,\y)},color=black] (2pt,0pt) -- (-2pt,0pt);
      \clip(-2,-3.94) rectangle (6.86,3.98);
      \draw (0.54,-2.47)-- (0.54,2.47);
      \draw [domain=0.54:6.859999999999997] plot(\x,{(--10.36-0*\x)/4.2});
      \draw [domain=0.54:6.859999999999997] plot(\x,{(-10.7-0*\x)/4.34});
      \begin{scriptsize}
        \fill [color=black] (0.54,0) circle (1.5pt);
        \draw[color=black] (0.8,0.28) node {$c$};
        \fill [color=black] (0.54,2.47) circle (1.5pt);
        \draw[color=black] (0.56,2.78) node {$c+iT$};
        \fill [color=black] (0.54,-2.47) circle (1.5pt);
        \draw[color=black] (0.56,-2.78) node {$c-iT$};
      \end{scriptsize}
    \end{tikzpicture}
  \end{center}
  In teoria dovremmo fare un integrale chiuso e poi mandare un lato a infinito, ma visto che non ci sono poli per $y^s/s$ nella regione considerata e sul lato che si manda a infinito la funzione va a $0$ uniformemente, saltiamo il passaggio. Per il teorema dei residui
  $$\frac{1}{2\pi i}\int_{c-iT}^{c+iT} \frac{y^s}{s}\diff s=\frac{1}{2\pi i}\left(\int_{c-iT}^{+\infty-iT}-\int_{c+iT}^{+\infty+iT}\right)y^s\frac{\diff s}{s}=\mathcal{I}_1-\mathcal{I}_2.$$
  Si ha $\displaystyle |\mathcal{I}_1| \le \frac{1}{T} \int_c^{+\infty} y^{\sigma}\diff\sigma=\frac{1}{T}\cdot\frac{y^c}{|\log{y}|}$ e $\mathcal{I}_2$ si stima allo stesso modo.

  Per l'altro argomento del minimo, consideriamo la circonferenza di centro l'origine e raggio $R=\sqrt{c^2+T^2}$ e prendiamo il cammino da $c-iT$ a $c+iT$ e ritorno che gira in senso orario (quindi il segmento passante per $c$ all'andata e l'arco destro al ritorno). Sia $\gamma$ l'arco percorso al ritorno.
  \begin{center}
\begin{tikzpicture}[line cap=round,line join=round,>=triangle 45,x=1.0cm,y=1.0cm]
\draw[->,color=black] (-3.14,0) -- (3.52,0);
\foreach \x in {-3,-2,-1,1,2,3}
\draw[shift={(\x,0)},color=black] (0pt,2pt) -- (0pt,-2pt);
\draw[->,color=black] (0,-3.24) -- (0,3.34);
\foreach \y in {-3,-2,-1,1,2,3}
\draw[shift={(0,\y)},color=black] (2pt,0pt) -- (-2pt,0pt);
\clip(-3.14,-3.24) rectangle (3.52,3.34);
\draw(0,0) circle (2.49cm);
\draw (0,0)-- (0.8,2.36);
\draw (0.8,-2.36)-- (0.8,2.36);
\begin{scriptsize}
\fill [color=black] (0.8,0) circle (1.5pt);
\draw[color=black] (1.02,0.2) node {$c$};
\fill [color=black] (0,2.36) circle (1.5pt);
\draw[color=black] (-0.22,2.3) node {$T$};
\fill [color=black] (0.8,2.36) circle (1.5pt);
\draw[color=black] (0.96,2.62) node {$c+iT$};
\fill [color=black] (0.8,-2.36) circle (1.5pt);
\draw[color=black] (0.96,-2.56) node {$c-iT$};
\end{scriptsize}
\end{tikzpicture}
  \end{center}
  Allora per il teorema dei residui
  \begin{gather*}
    \left|\frac{1}{2\pi i}\int_{c-iT}^{c+iT} \frac{y^s}{s}\diff s\right|=\left|\frac{1}{2\pi i}\int_\gamma \frac{y^s}{s}\diff s \right| \le \\
    \frac{y^c}{2\pi} \int_{\gamma} \frac{\diff|s|}{|s|} \le y^c\frac{\pi R}{2\pi R} \le \frac{y^c}{2}.
  \end{gather*}
  Per $y>1$, basta ripetere le stesse stime ma con i cammini ``di sinistra'' percorsi in senso antiorario, facendo attenzione al polo in $0$: $\displaystyle \underset{s=0}{\text{Res}}\,\frac{y^s}{s}=1$.

  Adesso facciamo $y=1$ (non saremo rigorosi, ma si può sistemare facilmente):
  \begin{gather*}
    \frac{1}{2\pi i}\int_{c-iT}^{c+iT}\frac{\diff s}{s}=\frac{1}{2\pi i}\int_{-T}^T \frac{i\diff t}{c+it}=\frac{1}{2\pi}\int_{-T}^T \frac{c-it}{c^2+t^2}\diff t=\\
    =\frac{1}{2\pi}\int_{-T}^T \frac{c}{c^2+t^2}\diff t=\frac{1}{\pi}\int_0^T \frac{c}{c^2+t^2}\diff t\overset{x=t/c}{=}\frac{1}{\pi}\int_0^{T/c} \frac{\diff x}{x^2+1}= \\
    =\frac{1}{\pi}\left(\int_0^{+\infty}\frac{\diff x}{x^2+1}-\int_{T/c}^{+\infty}\frac{\diff x}{x^2+1}\right) \le \frac{1}{2}+\frac{c}{T}.
  \end{gather*}
\end{proof}

Prendiamo ora $y=\frac{x}{n}$ con $n \in \mathbb{N}$. Si ha
$$\frac{1}{2\pi i}\int_{c-iT}^{c+iT} \frac{\Lambda(n)}{n^s}x^s\frac{\diff s}{s}=\Lambda(n)\cdot \begin{cases}
  1+O\Bigg(\left(\dfrac{x}{n}\right)^c\min\left\{1,\dfrac{1}{T\left|\log\left(\frac{x}{n}\right)\right|}\right\}\Bigg) & \mbox{se }n < x \\
  \dfrac{1}{2}+O\left(\dfrac{c}{T}\right) & \mbox{se }n=x \text{ (e }n=p^a\text{)} \\
  O\Bigg(\left(\dfrac{x}{n}\right)^c\min\left\{1,\dfrac{1}{T\left|\log\left(\frac{x}{n}\right)\right|}\right\}\Bigg) & \mbox{se }n>x.
\end{cases}$$
Di conseguenza, per $c>1$ abbiamo
\begin{gather*}
  \sum_{n \le x} \Lambda(n)+\frac{\Lambda(x)}{2}=\frac{1}{2\pi i}\int_{c-iT}^{c+iT} \left(\sum_{n=1}^{+\infty}\frac{\Lambda(n)}{n^s}\right)\frac{x^s}{s}\diff s+\\
  +O\left(\sum_{n=1,n\not=x}^{+\infty} \frac{\Lambda(n)}{n^c}x^c\min\left\{1,\frac{1}{T\left|\log\left(\frac{x}{n}\right)\right|}\right\}+\frac{c\Lambda(x)}{T}\right).
\end{gather*}
Vogliamo dare una stima del resto, facendo un po' di casi. Prendiamo anche $c=1+\dfrac{1}{\log{x}}$, per avere $x^c=ex \ll x$. Resterà così.
\begin{enumerate}
  \item Partiamo dal più semplice: $\dfrac{c\Lambda(x)}{T} \ll \dfrac{\log{x}}{T}$.
  \item Ora facciamo la somma per $n \le \frac{3}{4}x$ o $n \ge \frac{5}{4}x$, per cui $\left|\log\left(\frac{x}{n}\right)\right| \gg 1$. Dobbiamo dunque stimare
  $$\left(\sum_{n \le \frac{3}{4}x}+\sum_{n \ge \frac{5}{4}x}\right)\frac{\Lambda(n)}{n^c}\cdot\frac{x^c}{T} \ll \frac{x}{T}\sum_{n=1}^{+\infty} \frac{\Lambda(n)}{n^c} \ll \frac{x\log{x}}{T},$$
  dove l'ultimo passaggio segue scrivendo la somma come $-\frac{\zeta'}{\zeta}(c)$ e usando che $\frac{\zeta'}{\zeta}(\sigma) \ll \frac{1}{\sigma-1}$.
  \item Per $\frac{3}{4}x<n<x$, sia $\frac{3}{4}x<x_1<x$ la massima tra le potenze di un primo in quell'intervallo. Allora
  \begin{gather*}
    \log\left(\frac{x}{x_1}\right)=-\log\left(\frac{x_1}{x}\right)=-\log\left(1-\frac{x-x_1}{x}\right)=\\
    =\frac{x-x_1}{x}+\frac{1}{2}\left(\frac{x-x_1}{x}\right)^2+\dots>\frac{x-x_1}{x},
  \end{gather*}
  dunque la stima del termine $n=x_1$ diventa
  $$\implies \Lambda(x_1)\left(\frac{x}{x_1}\right)^c\frac{x}{T(x-x_1)} \ll \frac{x\log{x}}{T(x-x_1)}.$$
  E $\frac{3}{4}x<n<x_1$? Abbiamo
  \begin{gather*}
    \log\left(\frac{x}{n}\right) \ge \log\left(\frac{x_1}{n}\right)=-\log\left(\frac{n}{x_1}\right)=\\
    =-\log\left(1-\frac{x_1-n}{x_1}\right)
    \ge \frac{x_1-n}{x_1}=\frac{\nu}{x_1}
  \end{gather*}
  e troviamo
  $$\sum_{1 \le \nu <x_1} \frac{\Lambda(x_1-\nu)x^cx_1}{(x_1-\nu)^cT\nu} \ll \frac{x(\log{x})^2}{T}.$$
  Il caso $x<n<\frac{5}{4}x$ è analogo, prendendo $x_2$ la minima potenza di primo.
\end{enumerate}

Ora, definendo $\langle x\rangle$ come la distanza di $x$ dalla potenza di primo più vicina, mettendo assieme abbiamo trovato che
$$\psi_0(x)=\frac{1}{2\pi i}\int_{c-iT}^{c+iT} -\frac{\zeta'}{\zeta}(s)\frac{x^s}{s}\diff s+O\left(\frac{x\log^2{x}}{T}+\frac{x\log{x}}{T\langle x\rangle}\right);$$
dunque
$$\psi_0(x)=\frac{1}{2\pi i}\int_{c-iT}^{c+iT} -\frac{\zeta'}{\zeta}(s)\frac{x^s}{s}\diff s+R(x,T),$$
dove $R(x,T) \ll \dfrac{x\log^2{x}}{T}$ per $x \in \mathbb{N}$. Consideriamo ora il cammino in figura.
\begin{center}
\begin{tikzpicture}[line cap=round,line join=round,>=triangle 45,x=1.0cm,y=1.0cm]
\draw[->,color=black] (-3.36,0) -- (2.27,0);
\foreach \x in {-3,-2,-1,1,2}
\draw[shift={(\x,0)},color=black] (0pt,2pt) -- (0pt,-2pt);
\draw[->,color=black] (0,-2.01) -- (0,2.23);
\foreach \y in {-2,-1,1,2}
\draw[shift={(0,\y)},color=black] (2pt,0pt) -- (-2pt,0pt);
\clip(-3.36,-2.01) rectangle (2.27,2.23);
\draw (-2.13,-1.25)-- (1.2,-1.25);
\draw (1.2,-1.25)-- (1.2,1.25);
\draw (1.2,1.25)-- (-2.13,1.25);
\draw (-2.13,1.25)-- (-2.13,-1.25);
\draw [dash pattern=on 2pt off 2pt] (0.5,-1.25)-- (0.5,1.25);
\begin{scriptsize}
\fill [color=black] (1,0) circle (1.5pt);
\draw[color=black] (1,0.18) node {$1$};
\fill [color=black] (1.2,0) circle (1.5pt);
\draw[color=black] (1.3,0.12) node {$c$};
\fill [color=black] (0,1.25) circle (1.5pt);
\draw[color=black] (0.2,1.4) node {$T$};
\fill [color=black] (0,-1.25) circle (1.5pt);
\draw[color=black] (0.26,-1.4) node {$-T$};
\fill [color=black] (-2.13,0) circle (1.5pt);
\draw[color=black] (-2.4,0.12) node {$-U$};
\end{scriptsize}
\end{tikzpicture}
\end{center}
Prendiamo $T\not=\gamma$ per non passare dagli zeri non banali, allora usando ancora una volta il teorema dei residui si ha
\begin{gather*}
  \psi_0(x)=x-\sum_{|\gamma|<T} \frac{x^{\rho}}{\rho}-\frac{\zeta'}{\zeta}(0)-\sum_{1 \le n \le \frac{U}{2}} \frac{x^{-2n}}{-2n}+\\
  +\frac{1}{2\pi i}\left(\int_{-U+iT}^{c+iT}-\int_{-U-iT}^{c-iT}+\int_{-U-iT}^{-U+iT}\right)\left(-\frac{\zeta'}{\zeta}(s)\frac{x^s}{s}\diff s\right)+R(x,T).
\end{gather*}
Per il corollario della formula di Riemann-Von Mangoldt abbiamo
$$N(T+1)-N(T-1) \le C_0\log{T}.$$
Suddividiamo l'intervallo $[T-1,T+1]$ in $2C_0\log{T}$ intervalli, di modo che ce ne sia sempre uno senza zeri. Allora possiamo scegliere $T$, variandolo al più di una quantità minore o uguale a $1$, t.c. $|\gamma-T| \gg \dfrac{1}{\log{T}}$. Usando allora il lemma \ref{zprimoz} otteniamo
\begin{gather*}
  \frac{\zeta'}{\zeta}(\sigma+iT) \ll \sum_{|\gamma-T| \le 1} \frac{1}{|s-\rho|}+O(\log{T}) \ll \log{T}\sum_{|\gamma-T| \le 1} 1 \ll \log^2{T} \implies \\
  \implies \int_{-1 \pm +iT}^{c \pm iT} -\frac{\zeta'}{\zeta}(s)\frac{x^s}{s}\diff s \ll \frac{x^c\log^2{T}}{T} \ll \frac{x\log^2{T}}{T}.
\end{gather*}

\begin{oss} \label{zeeeta}
  Se $\sigma \le -1$ e $|s+2n| \ge 1/2$ per ogni $n \in \mathbb{N}$, come vedremo in seguito si ha
  \begin{gather*}
    \frac{\zeta'}{\zeta}(s) \ll \log(2|s|) \implies \\
    \implies \log(2|U\pm iT|)x^{-U}\int_0^T \frac{\diff t}{U+t} \ll \frac{\log(2|U\pm iT|)}{x^U}\log{T} \underset{U \longrightarrow +\infty}{\longrightarrow} 0;
  \end{gather*}
  questa era la parte ``verticale'' dell'integrale di $\psi_0$, con stime fatte alla buona. Mandando $U$ a infinito, rimangono da stimare i pezzi ``orizzontali'' da $-1$ a $-\infty$. Viene la seguente stima trascurabile:
  $$\frac{1}{T}\int_{-\infty}^{-1} \log(2|\sigma+iT|)x^{\sigma}\diff \sigma \ll \frac{x^{-1}\log{T}}{T\log{x}}.$$
  Anche questa stima è fatta alla buona. Si veda \cite{D} per una dimostrazione più precisa.
\end{oss}

Mettendo assieme quanto visto finora, otteniamo la seguente proposizione.

\begin{prop}
  \begin{equation*}
    \psi_0(x)=x-\sum_{|\gamma| < T} \frac{x^{\rho}}{\rho}-\frac{\zeta'}{\zeta}(0)-\frac{1}{2}\log\left(1-\frac{1}{x^2}\right)+R(x,T),
  \end{equation*}
  dove $R(x,T) \ll \dfrac{x\log^2(xT)}{T}$ per $x \in \mathbb{N}$ e $T \ge 2$.
\end{prop}

Bisogna stare attenti agli zeri mancanti (perché abbiamo cambiato un po' $T$, quindi dobbiamo sistemare), che però sono in un intervallo limitato, perciò sempre per il corollario di Riemann-Von Mangoldt un $O(\log{T})$. Allora il loro contributo è $O$-grande di:
$$\frac{x^{\rho}}{\rho}\log{T} \ll \frac{x^{\beta}\log{T}}{T} \ll \frac{x\log{T}}{T}.$$

Dimostriamo adesso l'asserzione usata all'inizio dell'osservazione \ref{zeeeta}, e per farlo passeremo dall'equazione funzionale per la $\zeta$. Si ha
\begin{gather*}
  \frac{s(s-1)}{2}\pi^{-\frac{s}{2}}\Gamma\left(\frac{s}{2}\right)\zeta(s)=\xi(s)= \\
  =\xi(1-s)=\frac{s(s-1)}{2}\pi^{\frac{s-1}{2}}\Gamma\left(\frac{1-s}{2}\right)\zeta(1-s) \implies \\
  \implies \zeta(1-s)=\pi^{\frac{1}{2}-s}\frac{\Gamma\left(\frac{s}{2}\right)}{\Gamma\left(\frac{1-s}{2}\right)}\zeta(s).
\end{gather*}
Usando la proposizione \ref{gammaze1-z} e la formula di duplicazione di Legendre, otteniamo
\begin{gather*}
  \frac{\Gamma\left(\frac{s}{2}\right)}{\Gamma\left(\frac{1-s}{2}\right)}=\frac{\Gamma\left(\frac{s}{2}\right)\Gamma\left(\frac{s}{2}+\frac{1}{2}\right)}{\Gamma\left(\frac{1-s}{2}\right)\Gamma\left(\frac{1+s}{2}\right)}= \\
  =\frac{\sqrt{\pi}2^{1-s}\Gamma(s)}{\pi/\sin\left(\frac{\pi}{2}-\frac{\pi s}{2}\right)}=2^{1-s}\pi^{-1/2}\cos\left(\frac{\pi s}{2}\right)\Gamma(s);
\end{gather*}
si ottiene dunque l'equazione funzionale per $\zeta$:
\begin{equation} \label{eqfunzeta}
  \zeta(1-s)=2^{1-s}\pi^{-s}\cos\left(\frac{\pi}{2}s\right)\Gamma(s)\zeta(s).
\end{equation}
Prendendo la derivata logaritmica si ha
$$\frac{\zeta'}{\zeta}(1-s)=\log{2}+\log{\pi}+\frac{\pi}{2}\tan\left(\frac{\pi}{2}s\right)-\frac{\Gamma'}{\Gamma}(s)-\frac{\zeta'}{\zeta}(s).$$
Per $\sigma \ge 2$, cioè $1-\sigma \le -1$, si ha
$$\frac{\zeta'}{\zeta}(s)=-\sum_{n=1}^{+\infty} \frac{\Lambda(n)}{n^s} \ll 1;$$
inoltre
\begin{gather*}
  \left|\tan\left(\frac{\pi}{2}s\right)\right|=\left|\frac{e^{i\frac{\pi}{2}s}-e^{-i\frac{\pi}{2}s}}{i(e^{i\frac{\pi}{2}s}+e^{-i\frac{\pi}{2}s})}\right|=\left|\frac{e^{\pi is}-1}{e^{\pi is}+1}\right| \le \\
  \le \frac{e^{\pi t}+1}{e^{\pi t}-1}=1+\frac{2}{e^{\pi t}-1} \le 3,
\end{gather*}
dove l'ultima disuguaglianza vale per $t \ge 1/2$, preso fuori da un intorno dei punti $-2n$. Infine, per il corollario \ref{gammaprimosu} abbiamo
$$\frac{\Gamma'}{\Gamma}(s) \ll \log{|s|}+O\left(\frac{1}{s}\right).$$
Allora, per $\sigma \le -1 \implies |s| \ge 1$, possiamo riarrangiare quanto trovato per ottenere
$$\frac{\zeta'}{\zeta}(s) \ll \log(2|1-s|) \le \log(2|s|),$$
in quanto $|1-s| \le 1+|s| \le 2|s|$.

Torniamo a $\psi_0$.
\begin{gather*}
  \psi_0(x)=\sum_{n<x} \Lambda(n)+\frac{1}{2}\Lambda(x) \implies \\
  \implies \psi_0(x)=x-\sum_{|\gamma|<T} \frac{x^{\rho}}{\rho}+O\left(\frac{x\log^2(xT)}{T}+\log{x}\cdot\min\left\{1,\frac{x}{T\langle x\rangle}\right\}\right).
\end{gather*}
Per de la Vallée-Poussin si ha che esiste $c_0>0$ t.c.
\begin{gather*}
  \beta<1-\frac{c_0}{\log{T}} \text{ per ogni } \rho=\beta+i\gamma \text{ con } |\gamma|<T \implies \\
  \implies \left|\sum_{|\gamma|<T} \frac{x^{\rho}}{\rho}\right| \le \max_{|\gamma|<T} x^{\beta} \sum_{|\gamma|<T} \frac{1}{|\rho|} \le x\exp\left(-\frac{c_0\log{x}}{\log{T}}\right)\sum_{|\gamma|<T} \frac{1}{|\rho|}.
\end{gather*}
Adesso, sommando per parti e usando Riemann-Von Mangoldt si ha
\begin{gather*}
  \sum_{|\gamma|<T} \frac{1}{|\rho|} \ll \sum_{2 \le \gamma \le T} \frac{1}{\gamma}=\left(\sum_{2 \le \gamma \le T}1\right)\frac{1}{T}-\int_2^T \left(\sum_{2 \le \gamma \le u}1\right)\frac{\diff u}{u^2}= \\
  =\frac{N(T)}{T}+\int_2^T \frac{N(u)}{u^2}\diff u \ll \frac{T\log{T}}{T}+\int_2^T \frac{\log{u}}{u}\diff u \ll \\
  \ll \log{T}+\log^2{T} \ll \log^2{T}.
\end{gather*}
Otteniamo dunque
\begin{gather*}
  \sum_{|\gamma|<T} \frac{x^{\rho}}{\rho} \ll x\log^2{T}\exp\left(-\frac{c_0\log{x}}{\log{T}}\right) \implies \\
  \implies \psi_0(x)-x \ll x\Bigg(\log^2{T}\exp\left(-\frac{c_0\log{x}}{\log{T}}\right)+\frac{\log^2(xT)}{T}\Bigg).
\end{gather*}
Prendendo $T=\exp(\sqrt{\log{x}}) \implies \log{x}=\log^2{T}$, troviamo che esiste $0<c_1<1$ t.c.
$$\psi_0(x)-x \ll x\exp(-c_1\sqrt{\log{x}}).$$
Questo è un risultato importante.

\begin{thm} \label{PNT}
  (Prime Number Theorem) esiste $0<c_1<1$ t.c.
  \begin{equation} \label{pnt}
    \psi(x)=x+O\big(x\exp(-c_1\sqrt{\log{x}})\big).
  \end{equation}
\end{thm}

\begin{oss}
  Littlewood, 1922: $\psi(x)=x+O\big(x\exp(-c_1\sqrt{\log{x}\log\log{x}})\big)$;
  Vinogradov-Korobov, 1958: $\psi(x)=x+O_{\epsilon}\Big(x\exp\big(-c_1(\log{x})^{3/5-\epsilon}\big)\Big)$. Questi seguono dai miglioramenti a de la Vallée-Poussin con dimostrazione analoga a quella del PNT.

  Allo stesso modo, l'Ipotesi di Riemann ci dà $\psi(x)=x+O(x^{1/2}\log^2{x})$. Infatti
  $$\left|\sum_{|\gamma|<T} \frac{x^{\rho}}{\rho}\right| \ll \sqrt{x}\sum_{|\gamma|<T} \frac{1}{|\gamma|} \ll \sqrt{x}\log^2{T},$$
  poi prendendo $T=\sqrt{x}$ si procede come nella dimostrazione del PNT.
\end{oss}

C'è anche la Quasi Ipotesi di Riemann (QRH): detto $\displaystyle \theta=\sup_{\rho} \beta$, allora si congettura che $1/2<\theta<1$.

\begin{exc}
  QRH $\implies \psi(x)=x+O(x^{\theta}\log^2{x})$.
\end{exc}

Congettura di Montgomery: il resto è $x^{1/2}(\log\log{x})^2$. Non possiamo portarlo a $\sqrt{x}$, o per meglio dire sarebbe utopistico: si sa che $\psi(x)-x=\Omega_{\pm}(\sqrt{x})$.

\begin{oss}
  Se $\psi(x)=x+O_{\epsilon}(x^{\theta+\epsilon})$ per ogni $\epsilon>0$, allora $\beta \le \theta$ per ogni $\rho$. Infatti, per $\sigma>1$ si ha per sommazione parziale
  \begin{gather*}
    \sum_{n \le N} \frac{\Lambda(n)}{n^s}=\frac{\psi(N)}{N^s}+s\int_2^N \frac{\psi(u)}{u^{s+1}}\diff u \implies \\
    \implies -\frac{\zeta'}{\zeta}(s)=s\int_2^{+\infty} \frac{\psi(u)}{u^{s+1}}\diff u=s\int_2^{+\infty} u^{-s}\diff u+s\int_2^{+\infty} \frac{R(u)}{u^{s+1}}\diff u= \\
    =\frac{s}{s-1}+s\int_2^{+\infty} \frac{R(u)}{u^{s+1}}\diff u.
  \end{gather*}
  Poiché $R(u)=O_{\epsilon}(u^{\theta+\epsilon})$, per $\sigma > \theta+2\epsilon$ l'integrale è finito e quindi troveremmo una funzione olomorfa con polo solo in $1$, cioè $\zeta$ non avrebbe zeri in quella regione $\implies \beta \le \theta$.
\end{oss}

Sia $\displaystyle \vartheta(x)=\sum_{p \le x} \log{p}$ (\textsc{non} è la funzione di Jacobi). Si ha
$$\psi(x)=\vartheta(x)+\vartheta(x^{1/2})+\dots+\vartheta(x^{1/N})\text{ con }x^{1/N} \ge 2 \implies N \ll \log{x}.$$
Poiché $\vartheta(y) \ll y$ per il teorema di Chebyshev, allora
$$\psi(x)=\vartheta(x)+O(\sqrt{x})+O(\sqrt[3]{x}\log{x})=\vartheta(x)+O(\sqrt{x}).$$
Abbiamo che RH $\implies \vartheta(x)=x+O(\sqrt{x}\log^2{x})$ e analogamente otteniamo che QRH $\implies \vartheta(x)=x+O(x^{\theta}\log^2{x})$. Dalla prima troviamo $\pi(x)=\li{x}+O(\sqrt{x}\log{x})$ e analogamente partendo da QRH.
Per farlo, si usa $\displaystyle \pi(x)=\frac{\vartheta(x)}{\log{x}}+\int_2^x\frac{\vartheta(u)}{u\log^2{u}}\diff u$, che è vera per sommazione parziale.
