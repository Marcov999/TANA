Riemann congetturò che
$$\psi(x)=\sum_{n \le x} \Lambda(n)=x-\sum_{\rho} \frac{x^{\rho}}{\rho}-\frac{\zeta'}{\zeta}(0)-\frac{1}{2}\log\left(1-\frac{1}{x^2}\right).$$
Beh, non proprio: l'espressione a destra è continua, a differenza di $\psi$. Utilizzeremo $\psi_0(x)=\displaystyle \sum_{n<x} \Lambda(n)+\frac{1}{2}\Lambda(x)$, dove poniamo $\Lambda=0$ fuori dagli interi. L'idea di Riemann è di scrivere
$$\psi_0(x) \sim \frac{1}{2\pi i} \int_{c-i\infty}^{c+i\infty} -\frac{\zeta'}{\zeta}(s)\frac{x^s}{s}\diff x$$
con $c>1$ e applicare il teorema dei residui. Ma come ha fatto a derivare un'espressione tanto precisa? Ha guardato i poli dell'integranda.

\begin{lm}
  Dato $c>0$, sia $I(y,T)=\displaystyle \frac{1}{2\pi i}\int_{c-iT}^{c+iT} y^s\frac{\diff s}{s}$; allora
  $$|I(y,T)-\delta(y)| \le \begin{cases}
    y^c\min\left\{1,\frac{1}{T|\log{y}|}\right\} & \mbox{se } y\not=1 \\
    \frac{c}{T} & \mbox{se } y=1,
\end{cases}$$
dove $\delta(y)=\begin{cases}
  1 & \mbox{se } y>1 \\
  \frac{1}{2} & \mbox{se } y=1 \\
  0 & \mbox{se } 0<y<1
\end{cases}$. Di conseguenza, $\displaystyle \frac{1}{2\pi i}\int_{c-i\infty}^{c+i\infty} y^s\frac{\diff s}{s}=\delta(y)$.
\end{lm}

\begin{proof}
  Preso $0<y<1$, consideriamo l'integrale sul cammino in figura, percorso in senso orario.
  \begin{center}
    \definecolor{uququq}{rgb}{0.25,0.25,0.25}
    \begin{tikzpicture}[line cap=round,line join=round,>=triangle 45,x=1.0cm,y=1.0cm]
      \draw[->,color=black] (-2,0) -- (6.86,0);
      \foreach \x in {-1,1,2,3,4,5,6}
      \draw[shift={(\x,0)},color=black] (0pt,2pt) -- (0pt,-2pt);
      \draw[->,color=black] (0,-3.94) -- (0,3.98);
      \foreach \y in {-3,-2,-1,1,2,3}
      \draw[shift={(0,\y)},color=black] (2pt,0pt) -- (-2pt,0pt);
      \clip(-2,-3.94) rectangle (6.86,3.98);
      \draw (0.54,-2.47)-- (0.54,2.47);
      \draw [domain=0.54:6.859999999999997] plot(\x,{(--10.36-0*\x)/4.2});
      \draw [domain=0.54:6.859999999999997] plot(\x,{(-10.7-0*\x)/4.34});
      \begin{scriptsize}
        \fill [color=black] (0.54,0) circle (1.5pt);
        \draw[color=black] (0.8,0.28) node {$c$};
        \fill [color=uququq] (0.54,2.47) circle (1.5pt);
        \draw[color=uququq] (0.56,2.78) node {$c+iT$};
        \fill [color=uququq] (0.54,-2.47) circle (1.5pt);
        \draw[color=uququq] (0.56,-2.78) node {$c-iT$};
      \end{scriptsize}
    \end{tikzpicture}
  \end{center}
  In teoria dovremmo fare un integrale chiuso e poi mandare un lato a infinito, ma visto che non ci sono poli per $y^s/s$ nella regione considerata e sul lato che si manda a infinito la funzione va a $0$ uniformemente, saltiamo il passaggio. Per il teorema dei residui
  $$\frac{1}{2\pi i}\int_{c-iT}^{c+iT} \frac{y^s}{s}\diff s=\frac{1}{2\pi i}\left(\int_{c-iT}^{+\infty-iT}-\int_{c+iT}^{+\infty+iT}\right)y^s\frac{\diff s}{s}=\mathcal{I}_1-\mathcal{I}_2.$$
  Si ha $\displaystyle |\mathcal{I}_1| \le \frac{1}{T} \int_c^{+\infty} y^{\sigma}\diff\sigma=\frac{1}{T}\cdot\frac{y^c}{|\log{y}|}$ e $\mathcal{I}_2$ si stima allo stesso modo.

  Per l'altro argomento del minimo, consideriamo la circonferenza di centro l'origine e raggio $R=\sqrt{c^2+T^2}$ e prendiamo il cammino da $c-iT$ a $c+iT$ e ritorno che gira in senso orario (quindi il segmento passante per $c$ all'andata e l'arco destro al ritorno). Sia $\gamma$ l'arco percorso al ritorno.
  \begin{center}
    \definecolor{uququq}{rgb}{0.25,0.25,0.25}
\begin{tikzpicture}[line cap=round,line join=round,>=triangle 45,x=1.0cm,y=1.0cm]
\draw[->,color=black] (-3.14,0) -- (3.52,0);
\foreach \x in {-3,-2,-1,1,2,3}
\draw[shift={(\x,0)},color=black] (0pt,2pt) -- (0pt,-2pt);
\draw[->,color=black] (0,-3.24) -- (0,3.34);
\foreach \y in {-3,-2,-1,1,2,3}
\draw[shift={(0,\y)},color=black] (2pt,0pt) -- (-2pt,0pt);
\clip(-3.14,-3.24) rectangle (3.52,3.34);
\draw(0,0) circle (2.49cm);
\draw (0,0)-- (0.8,2.36);
\draw (0.8,-2.36)-- (0.8,2.36);
\begin{scriptsize}
\fill [color=black] (0.8,0) circle (1.5pt);
\draw[color=black] (1.02,0.2) node {$c$};
\fill [color=black] (0,2.36) circle (1.5pt);
\draw[color=black] (-0.22,2.3) node {$T$};
\fill [color=uququq] (0.8,2.36) circle (1.5pt);
\draw[color=uququq] (0.96,2.62) node {$c+iT$};
\fill [color=uququq] (0.8,-2.36) circle (1.5pt);
\draw[color=uququq] (0.96,-2.56) node {$c-iT$};
\end{scriptsize}
\end{tikzpicture}
  \end{center}
  Allora per il teorema dei residui
  \begin{gather*}
    \left|\frac{1}{2\pi i}\int_{c-iT}^{c+iT} \frac{y^s}{s}\diff s\right|=\left|\frac{1}{2\pi i}\int_\gamma \frac{y^s}{s}\diff s \right| \le \\
    \frac{y^c}{2\pi} \int_{\gamma} \frac{\diff|s|}{|s|} \le y^c\frac{\pi R}{2\pi R} \le \frac{y^c}{2}.
  \end{gather*}
  Per $y>1$, basta ripetere le stesse stime ma con i cammini ``di sinistra'' percorsi in senso antiorario, facendo attenzione al polo in $0$: $\displaystyle \underset{s=0}{\text{Res}}\,\frac{y^s}{s}=1$.

  Adesso facciamo $y=1$ (non saremo rigorosi, ma si può sistemare facilmente):
  \begin{gather*}
    \frac{1}{2\pi i}\int_{c-iT}^{c+iT}\frac{\diff s}{s}=\frac{1}{2\pi i}\int_{-T}^T \frac{i\diff t}{c+it}=\frac{1}{2\pi}\int_{-T}^T \frac{c-it}{c^2+t^2}\diff t=\\
    =\frac{1}{2\pi}\int_{-T}^T \frac{c}{c^2+t^2}\diff t=\frac{1}{\pi}\int_0^T \frac{c}{c^2+t^2}\diff t\overset{x=t/c}{=}\frac{1}{\pi}\int_0^{T/c} \frac{\diff x}{x^2+1}= \\
    =\frac{1}{\pi}\left(\int_0^{+\infty}\frac{\diff x}{x^2+1}-\int_{T/c}^{+\infty}\frac{\diff x}{x^2+1}\right) \le \frac{1}{2}+\frac{c}{T}.
  \end{gather*}
\end{proof}

Prendiamo ora $y=\frac{x}{n}$ con $n \in \mathbb{N}$. Si ha
$$\frac{1}{2\pi i}\int_{c-iT}^{c+iT} \frac{\Lambda(n)}{n^s}x^s\frac{\diff s}{s}=\Lambda(n)\cdot \begin{cases}
  1+O\Bigg(\left(\dfrac{x}{n}\right)^c\min\left\{1,\dfrac{1}{T\left|\log\left(\frac{x}{n}\right)\right|}\right\}\Bigg) & \mbox{se }n < x \\
  \dfrac{1}{2}+O\left(\dfrac{c}{T}\right) & \mbox{se }n=x \text{ (e }n=p^a\text{)} \\
  O\Bigg(\left(\dfrac{x}{n}\right)^c\min\left\{1,\dfrac{1}{T\left|\log\left(\frac{x}{n}\right)\right|}\right\}\Bigg) & \mbox{se }n>x.
\end{cases}$$
Di conseguenza, per $c>1$ abbiamo
\begin{gather*}
  \sum_{n \le x} \Lambda(n)+\frac{\Lambda(x)}{2}=\frac{1}{2\pi i}\int_{c-iT}^{c+iT} \left(\sum_{n=1}^{+\infty}\frac{\Lambda(n)}{n^s}\right)\frac{x^s}{s}\diff s+\\
  +O\left(\sum_{n=1,n\not=x}^{+\infty} \frac{\Lambda(n)}{n^c}x^c\min\left\{1,\frac{1}{T\left|\log\left(\frac{x}{n}\right)\right|}\right\}+\frac{c\Lambda(x)}{T}\right).
\end{gather*}
Vogliamo dare una stima del resto, facendo un po' di casi. Prendiamo anche $c=1+\dfrac{1}{\log{x}}$, per avere $x^c=ex \ll x$. Resterà così.
\begin{enumerate}
  \item Partiamo dal più semplice: $\dfrac{c\Lambda(x)}{T} \ll \dfrac{\log{x}}{T}$.
  \item Ora facciamo la somma per $n \le \frac{3}{4}x$ o $n \ge \frac{5}{4}x$, per cui $\left|\log\left(\frac{x}{n}\right)\right| \gg 1$. Dobbiamo dunque stimare
  $$\left(\sum_{n \le \frac{3}{4}x}+\sum_{n \ge \frac{5}{4}x}\right)\frac{\Lambda(n)}{n^c}\cdot\frac{x^c}{T} \ll \frac{x}{T}\sum_{n=1}^{+\infty} \frac{\Lambda(n)}{n^c} \ll \frac{x\log{x}}{T},$$
  dove l'ultimo passaggio segue scrivendo la somma come $-\frac{\zeta'}{\zeta}(c)$ e usando che $\frac{\zeta'}{\zeta}(\sigma) \ll \frac{1}{\sigma-1}$.
  \item Per $\frac{3}{4}x<n<x$, sia $\frac{3}{4}x<x_1<x$ la massima tra le potenze di un primo in quell'intervallo. Allora
  \begin{gather*}
    \log\left(\frac{x}{x_1}\right)=-\log\left(\frac{x_1}{x}\right)=-\log\left(1-\frac{x-x_1}{x}\right)=\\
    =\frac{x-x_1}{x}+\frac{1}{2}\left(\frac{x-x_1}{x}\right)^2+\dots>\frac{x-x_1}{x},
  \end{gather*}
  dunque la stima del termine $n=x_1$ diventa
  $$\implies \Lambda(x_1)\left(\frac{x}{x_1}\right)^c\frac{x}{T(x-x_1)} \ll \frac{x\log{x}}{T(x-x_1)}.$$
  E $\frac{3}{4}x<n<x_1$? Abbiamo
  \begin{gather*}
    \log\left(\frac{x}{n}\right) \ge \log\left(\frac{x_1}{n}\right)=-\log\left(\frac{n}{x_1}\right)=\\
    =-\log\left(1-\frac{x_1-n}{x_1}\right)
    \ge \frac{x_1-n}{x_1}=\frac{\nu}{x_1}
  \end{gather*}
  e troviamo
  $$\sum_{1 \le \nu <x_1} \frac{\Lambda(x_1-\nu)x^cx_1}{(x_1-\nu)^cT\nu} \ll \frac{x(\log{x})^2}{T}.$$
  Il caso $x<n<\frac{5}{4}x$ è analogo, prendendo $x_2$ la minima potenza di primo.
\end{enumerate}

Ora, definendo $\langle x\rangle$ come la distanza di $x$ dalla potenza di primo più vicina, mettendo assieme abbiamo trovato che
$$\psi_0(x)=\frac{1}{2\pi i}\int_{c-iT}^{c+iT} -\frac{\zeta'}{\zeta}(s)\frac{x^s}{s}\diff s+O\left(\frac{x\log^2{x}}{T}+\frac{x\log{x}}{T\langle x\rangle}\right).$$
