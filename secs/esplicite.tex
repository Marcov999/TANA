\begin{defn}
  Dato $\chi$ un carattere modulo $q$, si pone $\displaystyle \psi(x,\chi)=\sum_{n \le x} \chi(n)\Lambda(n)$.
\end{defn}

\begin{oss}
  Se $\chi=\chi_0$, allora
  \begin{gather*}
    \psi(x,\chi_0)=\sum_{\substack{n \le x,\\ (n,q)=1}} \Lambda(n)=\psi(x)-\sum_{\substack{n \le x,\\ (n,q)>1}} \Lambda(n)= \\
    =\psi(x)+O\left(\sum_{p \mid q} \sum_{a \le \frac{\log{x}}{\log{p}}}\log{p}\right)=\psi(x)+O\left(\log{x}\sum_{p \mid q} 1\right)= \\
    =\psi(x)+O(\log{x}\log{q})=\psi(x)+O\big(\log^2(qx)\big).
  \end{gather*}
\end{oss}

Poniamo $\displaystyle \psi_0(x,\chi)=\sum_{n<x} \chi(n)\Lambda(n)+\frac{1}{2}\chi(x)\Lambda(x)$, dove al solito $\Lambda(x)=0$ se $x \not\in \mathbb{N}$. Se $\chi\not=\chi_0$, prendendo di nuovo $c=1+1/\log{x}$ e ripetendo quanto già fatto per $\psi_0(x)$ troviamo
$$\psi_0(x,\chi)=\frac{1}{2\pi i} \int_{c-iT}^{c+iT} -\frac{L'}{L}(s,\chi)\frac{x^s}{s}\diff s+O\left(\frac{x\log^2{x}}{T}+\log{x}\min\left\{1,\frac{x}{T\langle x\rangle}\right\}\right).$$

\begin{enumerate}
  \item Se $\chi(-1)=-1$ ($a=1$), i residui sono:
  $$-\frac{x^{\rho_\chi}}{\rho_\chi}, \,\,\, -\frac{L'}{L}(0,\chi), \,\,\, \frac{x^{-2m+1}}{2m-1}.$$
  \item Se $\chi(-1)=1$ ($a=0$), i residui sono:
  $$-\frac{x^{\rho_\chi}}{\rho_\chi}, \,\,\, -\log{x}-\lim_{s \longrightarrow 0} \left(\frac{L'}{L}(s,\chi)-\frac{1}{s}\right), \,\,\, \frac{x^{-2m}}{2m}.$$
\end{enumerate}

Consideriamo di nuovo il rettangolo con parte reale compresa tra $-U$ e $C$ e parte immaginaria tra $-T$ e $T$. Ripetendo i passaggi già visti, troviamo $\displaystyle \frac{L'}{L}(s,\chi)=\sum_{|\gamma_\chi| \le 1} \frac{1}{s-\rho_\chi}+O\big(\log(qT)\big)$. Imponendo $|\gamma_\chi\pm T| \gg 1/\log(qT)$ abbiamo $\frac{L'}{L}(s,\chi) \ll \log^2(qT)$.
Per gli altri pezzi, ripetendo le stime, si ha $\frac{L'}{L}(s,\chi) \ll \log(qT)$. Otteniamo dunque
$$\psi_0(x,\chi)=-\sum_{|\gamma_\chi|<T} \frac{x^{\rho_\chi}}{\rho_\chi}-b_\chi-(1-a)\log{x}-\sum_{m=1}^{+\infty} \frac{x^{-2m+a}}{a-2m}+O\left(\frac{x\log^2(qTx)}{T}+\log{x}\min\left\{1,\frac{x}{T\langle x\rangle}\right\}\right),$$
dove $b_\chi=\begin{cases}
  \dfrac{L'}{L}(0,\chi) \\ \displaystyle\lim_{s \longrightarrow 0} \left(\frac{L'}{L}(s,\chi)-\frac{1}{s}\right)
\end{cases}$.

Per calcolare il limite, ricordiamo che
$$\frac{L'}{L}(s,\chi)=-\frac{1}{2}\log\left(\frac{q}{\pi}\right)+A_\chi-\frac{1}{2}\cdot\frac{\Gamma'}{\Gamma}\left(\frac{s+a}{2}\right)+\sum_{\rho_\chi} \left(\frac{1}{s-\rho_\chi}+\frac{1}{\rho_\chi}\right);$$
portando il pezzo con $\Gamma$ a sinistra e sostituendo $s=0$, troviamo che vale
$b_\chi=-\frac{1}{2}\log\left(\frac{q}{\pi}\right)+A_\chi$. Scegliendo $s=2$, allora
\begin{gather*}
  \frac{L'}{L}(2,\chi)=-\frac{1}{2}\log\left(\frac{q}{\pi}\right)+O(1)+\sum_{\rho_\chi}\left(\frac{1}{2-\rho_\chi}+\frac{1}{\rho_\chi}\right)+A_\chi \implies \\
  \implies b_\chi=-\sum_{\rho_\chi}\left(\frac{1}{2-\rho_\chi}+\frac{1}{\rho_\chi}\right)+O(1).
\end{gather*}
Si ha
$$\left|\sum_{|\gamma_\chi| \ge 1} \left(\frac{1}{2-\rho_\chi}+\frac{1}{\rho_\chi}\right)\right| \le \sum_{|\gamma_\chi| \ge 1} \frac{2}{|\rho_\chi||2-\rho_\chi|} \ll \sum_{|\gamma_\chi| \ge 1} \frac{1}{1+\gamma_\chi^2} \ll \log{q}$$
e
$$\left|\sum_{|\gamma_\chi|<1} \frac{1}{2-\rho_\chi} \right| \ll \sum_{|\gamma_\chi|<1} 1 \ll \log{q},$$
da cui $\displaystyle b_\chi=-\sum_{|\gamma_\chi|<1} \frac{1}{\rho_\chi}+O(\log{q})$.
Indicando con $\beta_1$ l'eventuale zero di Siegel e scrivendo $\displaystyle \sum'$ per indicare che sommiamo su tutti gli altri zeri, per $x \in \mathbb{N}$ e $T \le x$ otteniamo
$$\psi(x,\chi)=-\sum_{|\gamma_\chi|<T}' \frac{x^{\rho_\chi}}{\rho_\chi}+\sum_{|\gamma_\chi|<1}' \frac{1}{\rho_\chi}-\left(\frac{x^{1-\beta_1}}{\beta_1}-\frac{1}{\beta_1}\right)-\left(\frac{x^{\beta_1}}{1-\beta_1}-\frac{1}{1-\beta_1}\right)+O\left(\frac{x\log^2(qx)}{T}\right).$$
Abbiamo $q \ge 3 \implies \log{q}>1$, inoltre esiste $0<c_0<1/4$ t.c. $\beta_1>1-c_0/\log{q}>3/4$, dunque $1/\beta_1=O(1)$. Inoltre esiste $0 \le \sigma_0 \le 1-\beta_1<1/4$ t.c.
$$\frac{x^{1-\beta_1}-1}{1-\beta_1}=x^{\sigma_0}\log{x} \ll x^{1/4}\log{x}.$$
Inoltre, usando che per gli zeri non di Siegel $\beta_\chi \ge c/\log{q}$ e che $\displaystyle \sum_{|\gamma_\chi|<1} 1 \ll \log{q}$, otteniamo
$$\sum_{|\gamma_\chi|<1}' \frac{1}{\rho_\chi} \ll \sum_{|\gamma_\chi|<1}' \frac{1}{\beta_\chi} \ll \log^2{q},$$
per cui dato $\chi$ un carattere primitivo modulo $q$ la formula esplicita per $\psi(x,\chi)$ è

\begin{equation} \label{expl}
  \psi(x,\chi)=-\sum_{|\gamma_\chi| \le T} \frac{x^{\rho_\chi}}{\rho_\chi}-\frac{x^{\beta_1}}{\beta_1}+O\left(\frac{x\log^2(qx)}{T}+x^{1/4}\log{x}\right).
\end{equation}

\begin{oss}
  Se $\chi$ è non primitivo e $\chi$ induce $\chi$, allora
  $$\left| \sum_{n \le x} \Lambda(n)\chi(n)-\sum_{n \le x}\Lambda(n)\chi_1(n)\right| \le \sum_{\substack{n \le x,\\ (n,q)>1}} \Lambda(n) \ll \log^2(qx).$$
  Allora l'equazione \eqref{expl} vale per ogni $\chi$ modulo $q$ non principale.
\end{oss}

\begin{cor}
  Se $(a,q)=1$, allora
  \begin{gather*}
    \psi(x;q,a):=\sum_{\substack{n \le x,\\ n \equiv a \pmod{q}}} \Lambda(n)=\frac{1}{\phi(q)} \sum_{\chi\textnormal{ modulo }q} \bar{\chi}(a)\psi(x,\chi)= \\
    =\frac{x}{\phi(q)}+\frac{1}{\phi(q)}\sum_{\chi\not=\chi_0} \bar{\chi}(a)\psi(x,\chi)+O\left(\frac{x\exp(-c\sqrt{\log x})}{\phi(q)}\right).
  \end{gather*}
  In realtà, non ho controllato in dettaglio ma credo che questa stima valga sotto le ipotesi aggiuntive su $q$ che introdurremo a breve.
\end{cor}

Vogliamo ora stimare la somma, che contiene $\psi(x,\chi)$, dunque la somma in \eqref{expl}. Abbiamo
$$\left|\sum_{|\gamma_\chi|<T}' \frac{x^{\rho_\chi}}{\rho_\chi}\right| \le \sum_{1 \le |\gamma_\chi|<T} \frac{x^{\beta_\chi}}{|\gamma_\chi|}+\sum_{|\gamma_\chi| < 1} \frac{x^{\beta_\chi}}{|\rho_\chi|}=\Sigma_1+\Sigma_2.$$
Per $\Sigma_1$, usiamo $x^{\beta_\chi} \le x^{1-\frac{c}{\log(qT)}}=x\exp\left(-\frac{c\log{x}}{\log(qT)}\right)$, mentre per la somma, sommando per parti e usando Riemann-Von Mangoldt, troviamo che è $\ll \log^2(qT) \ll \log^2(qx)$. Per $\Sigma_2$ abbiamo la stessa stima per $x^{\beta_\chi}$, mentre la somma avevamo già visto essere $\ll \log^2{q}$.
Prendendo allora $q \le \exp\big(C(\log{x})^{1/2}\big)=T$ con $C$ un'opportuna costante, stimiamo tutto con $x\exp(-c_1\sqrt{\log{x}})$. In conclusione si ha
$$\psi(x;q,a)=\frac{x}{\phi(q)}+O\big(x\exp(-c_1\sqrt{\log{x}})\big)-\frac{\chi(a)}{\phi(q)}\cdot\frac{x^{\beta_1}}{\beta_1},$$
dove abbiamo tolto il coniugio all'unico carattere per cui eventualmente compare $\beta_1$ perché sappiamo che deve essere reale. Ponendo $z=\exp\big(C(\log{x})^{1/2}\big)$, allora la stima è uniforme in $q$ per $q \le z$ e contiene, appunto, al più un unico termine con $\beta_1$ (e $\beta_1>1-c_0/\log{q}$).
Usando che $\beta_1 \le 1-\frac{c'}{\sqrt{q_1}\log^2{q_1}}$ dove $q_1$ è il modulo speciale per cui appare $\beta_1$, troviamo $x^{\beta_1} \le x\exp\left(-\frac{c'\log{x}}{\sqrt{q_1}\log^2{q_1}}\right)$.
Imponendo $q_1\log^4{q_1} \l \log{x}$ (con limiti molto costrittivi come ad esempio $q_1 \le (\log{x})^{1-\delta}$), otteniamo
$$\psi(x;q,a)=\frac{x}{\phi(q)}+O\big(x\exp(-c_1\sqrt{\log{x}})\big)$$
uniformemente per $q \le (\log{x})^{1-\delta}$. Questa è l'unica formula effettiva in tutte le variabili.

La GRH ($\beta_\chi=1/2$ per ogni $\rho_\chi$ e per ogni $\chi$ modulo $q$) ci darebbe $\psi(x,\chi)=O(x^{1/2}\log^2{x})$ prendendo $q \le \sqrt{x}=T$; allora avremmo $\psi(x;q,a)=\frac{x}{\phi(q)}+O(x^{1/2}\log^2{x})$.

\begin{prop}
  (Bombieri) Posto $E(x,q,a)=\psi(x;q,a)-\frac{x}{\phi(q)}$, si ha che per ogni $A>0$ esiste $B>0$ t.c.
  $$\sum_{q \le \frac{\sqrt{x}}{(\log{x})^B}} \max_{(a,q)=1}\max_{y \le x} |E(y,q,a)| \ll \frac{x}{(\log{x})^A}.$$
\end{prop}

\begin{oss}
  Usando GRH, avremmo
  $$\sum_{q \le \frac{\sqrt{x}}{(\log{x})^B}} \max_{(a,q)=1} |E(x,q,a)| \ll \sqrt{x}\log^2{x} \sum_{q \le \frac{\sqrt{x}}{(\log{x})^B}} 1 \ll \frac{x}{(\log{x})^{B-2}}.$$
  Allora basterebbe $B=A+2$; nel risultato dovuto a Bombieri, $B=A+5$.
\end{oss}

\begin{prop}
  Sia $\chi$ un carattere reale primitivo e $\epsilon>0$. Esiste $c(\epsilon)>0$ t.c. $L(1,\chi)>c(\epsilon)q^{-\epsilon}$.
\end{prop}

\begin{proof}
  Sia $F(s)=\zeta(s)L(s,\chi_1)L(s,\chi_2)L(s,\chi_1\chi_2)$ con $\chi_1$ e $\chi_2$ primitivi reali modulo $q_1$ e $q_2$. Poniamo $\displaystyle Res_{s=1}F(s)=\lambda=L(1,\chi_1)L(1,\chi_2)L(1,\chi_1\chi_2)$. Il grosso della dimostrazione sta nel seguente lemma.
  \begin{lm}
    $7/8<s<1 \implies F(s)>\frac{1}{2}-\frac{c\lambda}{1-s}(q_1q_2)^{8(1-s)}$.
  \end{lm}
  \begin{proof}
    $\displaystyle F(s)=\sum_{n=1}^{+\infty} \frac{a_n}{n^s}$ per $\sigma>1$; $a_1=1$, inoltre i fattori del prodotto di Eulero sono
    \begin{gather*}
      \left(1-\frac{1}{p^s}\right)^{-1}\left(1-\frac{\chi_1(p)}{p^s}\right)^{-1}\left(1-\frac{\chi_2(p)}{p^s}\right)^{-1}\left(1-\frac{\chi_1\cdot\chi_2(p)}{p^s}\right)^{-1} \implies \\
      \implies \log{F(s)}=\sum_p\sum_m \frac{1}{mp^{ms}} \big(1+\chi_1(p^m)\big)\big(1+\chi_2(p^m)\big).
    \end{gather*}
    Poiché $\big(1+\chi_1(p^m)\big)\big(1+\chi_2(p^m)\big) \ge 0$, abbiamo $a_n \ge 0$. Scriviamo $\displaystyle F(s)=\sum_{m=0}^{+\infty} b_m(2-s)^m$, dove
    $$b_m=\frac{(-1)^m}{m!}\cdot F^{(m)}(2)=\frac{1}{m!}\sum_{n=1}^{+\infty} \frac{a_n(\log{n})^m}{n^2} \ge 0$$
    e in più $b_1 \ge 1$. Espandendo $\frac{1}{s-1}=\frac{1}{1-(2-s)}$, per $|2-s| \le 3/2$ abbiamo
    $$F(s)-\frac{\lambda}{s-1}=\sum_{m=0}^{+\infty} (b_m-\lambda)(2-s)^m.$$
    Siccome $|L(s,\chi)| \ll q|s|$, tra $1/2$ e $1$ si ha $L(s,\chi_j) \ll q_j$ e $L(s,\chi_1\chi_2) \ll q_1q_2$. Restringendoci a $7/8 \le s \le 1$, per le disuguaglianze di Cauchy troviamo
    \begin{gather*}
      |b_m-\lambda|\cdot|2-s|^m \le cq_1^2q_2^2\cdot\left(\frac{2}{3}\right)^m\left(\frac{9}{8}\right)^m=c(q_1q_2)^2\left(\frac{3}{4}\right)^m \implies \\
      \implies \sum_{m=M}^{+\infty} |b_m-\lambda|\cdot|2-s|^m \le c\left(\frac{3}{4}\right)^M \implies \\
      \implies F(s)-\frac{\lambda}{s-1}=\sum_{m=0}^{M-1}+\sum_{m=M}^{+\infty} \ge 1-\lambda\sum_{m=0}^{M-1}(2-s)^m-c(q_1q_2)^2e^{-M/4}.
    \end{gather*}
    Scegliamo $M$ t.c. $\frac{1}{2}e^{-1/4}<c(q_1q_2)^2e^{-M/4}<1/2$, da cui anche $M<8\log(q_1q_2)+c$. Allora $F(s)-\frac{\lambda}{s-1}>\frac{1}{2}-\lambda\frac{(2-s)^M-1}{1-s}$ ma anche $(2-s)^M=e^{M\log(1+1-s)} \le e^{M(1-s)} \le c_0(q_1q_2)^{8(1-s)}$. Mettendo assieme si ha la tesi.
  \end{proof}
  Fissiamo adesso $\epsilon>0$. Primo caso: esiste $\chi_1$ modulo $q_1$ con $L(\beta_1,\chi_1)=0$ e $\beta_1>1-\epsilon/16$. Allora $F(\beta_1)=0$. Secondo caso: vicino a $1$ (a sinistra) abbiamo $L(\sigma,\chi)>0,\zeta(\sigma)<0$, dunque $F(\beta)<0$, dove possiamo scegliere $\beta$.
  In generale abbiamo che esiste $1-\epsilon/16<\beta<1$ t.c. $F(\beta) \le 0$, dunque per il lemma abbiamo $\frac{c_0\lambda}{1-s}(q_1q_2)^{8(1-s)}>1/2$. Prendendo $\chi_2$ modulo $q_2$ con $q_2>q_1$. Da vecchie stime possiamo anche dire $\lambda \le c_1\log{q_1}\log(q_1q_2)L(1,\chi_2)$. Allora
  $$L(1,\chi_2) \ge \frac{1-\beta_1}{2c}(q_1q_2)^{-8(1-\beta_1)}\big(\log{q_1}\log(q_1q_2)\big)^{-1} \ge C(\chi_1)q_2^{-\epsilon/3}.$$
\end{proof}

\begin{cor}
  A meno di cambiare costanti, si ha $\beta_1 \le 1-c(\epsilon)q^{-\epsilon}$.
\end{cor}

\begin{proof}
  Usare la stima vista per $L'(1,\chi)$.
\end{proof}

Conseguenza: $x^{\beta_1}<x\exp\left(\frac{-c(\epsilon)\log{x}}{q^{\epsilon}}\right)$. Scegliendo $\epsilon=\frac{1}{2N}$ e prendendo $q \le (\log{x})^N$, abbiamo $q^{\epsilon} \le (\log{x})^{1/2}$. Ne segue l'ultimo risultato del corso, ovvero il teorema dei numeri primi nelle progressioni aritmetiche nella versione di Siegel-Walfisz.

\begin{prop}
  (Siegel-Walfisz) Per ogni $N>0$ esiste $c(N)$ t.c.
  $$\psi(x;q,a)=\frac{x}{\phi(q)}+O\Big(x\exp\big(-c(N)(\log{x})^{1/2}\big)\Big)$$
  uniformemente per $q \le (\log{x})^N$ (ma $c(N)$ e la costante nell'$O$-grande non sono calcolabili).
\end{prop}
