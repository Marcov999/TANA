Questi appunti sono basati sul corso Teoria Analitica dei Numeri A tenuto dal professor Giuseppe Puglisi nel secondo semestre dell'anno accademico 2020/2021. Sono dati per buoni (si vedano i prerequisiti del corso) i corsi di Aritmetica, Analisi 1 e 2 e Teoria dei Numeri Elementare, più le base dell'analisi complessa in una variabile. Verranno omesse o soltanto hintate le dimostrazioni più semplici, ma si consiglia comunque di provare a svolgerle per conto proprio. Ogni tanto sarà commesso qualche abuso di notazione, facendo comunque in modo che il significato sia reso chiaro dal contesto. Inoltre, la notazione verrà alleggerita man mano, per evitare inutili ripetizioni e appesantimenti nella lettura. Si ricorda anche che questi appunti sono scritti non sempre subito dopo le lezioni, non sempre con appunti completi, ecc\dots. Spesso saranno rivisti, verranno aggiunte cose che mancavano perché c'era poco tempo (o voglia\dots), potrebbero mancare argomenti più o meno marginali\dots insomma, non è un libro di testo per il corso, ma vuole essere un valido supporto per aiutare gli studenti che seguono il corso. Spero di essere riuscito in questo intento.
