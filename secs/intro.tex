Questi appunti sono basati sul corso Teoria Analitica dei Numeri A tenuto dal professor Giuseppe Puglisi nel secondo semestre dell'anno accademico 2020/2021. Sono dati per buoni (si vedano i prerequisiti del corso) i corsi di Aritmetica, Analisi 1 e 2 e Teoria dei Numeri Elementare, più le base dell'analisi complessa in una variabile. Verranno omesse o soltanto hintate le dimostrazioni più semplici, ma si consiglia comunque di provare a svolgerle per conto proprio. Ogni tanto sarà commesso qualche abuso di notazione, facendo comunque in modo che il significato sia reso chiaro dal contesto. Inoltre, la notazione verrà alleggerita man mano, per evitare inutili ripetizioni e appesantimenti nella lettura. Si ricorda anche che questi appunti sono scritti non sempre subito dopo le lezioni, non sempre con appunti completi, ecc\dots. Spesso saranno rivisti, verranno aggiunte cose che mancavano perché c'era poco tempo (o voglia\dots), potrebbero mancare argomenti più o meno marginali\dots insomma, non è un libro di testo per il corso, ma vuole essere un valido supporto per aiutare gli studenti che seguono il corso. Spero di essere riuscito in questo intento.

\subsection{Introduzione al corso}

Definiamo alcune funzioni che ci interesserà studiare. Nel seguento, con $p$ indicheremo sempre un numero primo, salvo se diversamente specificato.

\begin{defn}
  $\displaystyle \pi(x):=\sum_{p \le x} 1$ è la \textit{prime counting function}, cioè la funzione che conta il numero di primi minori o uguali a $x$.
\end{defn}

Più avanti mostreremo il teorema dei numeri primi, una cui versione asserisce che $\pi(x) \sim \li(x)$. Con $\li(x)$ intendiamo la funzione logaritmo integrale, cioè $\displaystyle \li(x):=\int_2^x \frac{\diff t}{\log{t}} \sim \frac{x}{\log{x}}$. Una dimostrazione elementare di questo teorema è stata data da Selberg nel 1949.

\begin{defn}
  La \textit{funzione di Von Mongoldt} è data da $$\Lambda(n)=\begin{cases}
    \log{p} & \mbox{se }n=p^a, a \ge 1 \\ 0 & \mbox{altrimenti}
\end{cases}.$$
\end{defn}

\begin{defn}
  $\displaystyle\psi(x):=\sum_{n \le x} \Lambda(n)$ è la \textit{funzione di Chebyshev}.
\end{defn}

\begin{defn}
  La \textit{Zeta di Riemann} è la funzione definita, sui reali maggiori di $1$, come
  \begin{equation} \label{Zeta}
    \zeta(s)=\sum_{n=1}^{+\infty} \frac{1}{n^s}.
  \end{equation}
\end{defn}

La Zeta di Riemann è una delle funzioni di Dirichlet. Si può dimostrare che $\displaystyle \zeta(s)=\prod_p \left(1-\frac{1}{p^s}\right)^{-1}$.

Notazione: d'ora in avanti sottintenderemo $s=\sigma+it$.

Il capitolo 8 di \cite{D} tratta della Memoria di Riemann (1859), l'articolo che ha dato il via allo studio della funzione Zeta e di alcune sue proprietà. In esso, Riemann dimostra due risultati importanti (le dimostrazioni dei quali vedremo più avanti) e formula alcune congetture, tra cui anche la famosa Ipotesi.

Vediamo i due risultati mostrati da Riemann.
\begin{enumerate}
  \item $\zeta(s)$ è prolungabile analiticamente come funzione meromorfa a $\mathbb{C}$ con un unico polo semplice a $s=1$ di residuo uguale a $1$.
  \item Poniamo
  \begin{equation} \label{xi}
    \xi(s)=\pi^{-s/2}\frac{s(s-1)}{2}\Gamma(s/2)\zeta(s).
  \end{equation}
  Allora $\xi$ è una funzione intera e soddisfa $\xi(s)=\xi(1-s)$. Poiché $\Gamma(s)$ ha poli semplici in $-n, n \in \mathbb{N}$, si ha che $\Gamma(s/2)$ ce li ha in $-2n$, ed essendo $\xi$ intera ne deduciamo che $\zeta$ deve avere degli zeri, detti \textit{zeri banali}, per gli interi negativi pari.
\end{enumerate}

Vediamo invece adesso le varie congetture. Premettiamo una definizione.
\begin{defn}
  Si dice \textit{striscia critica} la regione $\{0 \le \sigma \le 1, t \in \mathbb{R}\}$.
\end{defn}
\begin{enumerate}
  \item $\zeta$ ha infiniti zeri nella striscia critica, distribuiti simmetricamente rispetto alla bisettrice $\sigma=1/2$ e all'asse reale $t=0$.
  \item Detto $N(T)=\{\rho=\beta+i\gamma \mid \zeta(\rho)=0, 0 \le \beta \le 1, 0<\gamma<T\}$, si ha che $N(T) \sim \frac{T}{2\pi}\log\left(\frac{T}{2\pi}\right)-\frac{T}{2\pi}$. Questo risultato è stato dimostrato da Von Mangoldt nel 1895, poi nel 1905 ha migliorato l'errore mostrando che $N(T)=\frac{T}{2\pi}\log\left(\frac{T}{2\pi}\right)-\frac{T}{2\pi}+O(\log{T})$.
  \item Vale la seguente formula esplicita (con $\rho$ indichiamo gli zeri della funzione Zeta nella striscia critica): $\displaystyle \psi(x)=x-\sum_{\rho} \frac{x^{\rho}}{\rho}-\frac{1}{2}\log(1-x^2)-\frac{\zeta'(0)}{\zeta(0)}$. Da essa si può dimostrare che $\psi(x) \sim x$, che è la versione del teorema dei numeri primi dimostrata indipendentemente da Hadamard e de la Valleé-Poussin nel 1896.

  Usando che $\left|\frac{x^{\rho}}{\rho}\right|=\frac{x^{\beta}}{|\rho|}$, se fosse $0<\beta \le \theta<1$ potremmo mostrare che $\displaystyle \left|\sum_{\gamma>0}\frac{x^{\rho}}{\rho}\right| \le x^{\theta}\cdot C$ dove $C$ è una costante.
  Hadamard e de la Valleé-Poussin hanno dimostrato che $\beta<1$, ma potrebbe essere che $\sup_{\gamma} \beta=1$.

  Nel 1899 hanno anche dimostrato che $\psi(x)=x+O\bigl(x\exp(-c\sqrt{\log{x}})\bigr)$, che è anche $x+o\bigl(x/(\log{x})^A\bigr)$ per ogni $A>0$.

  È stata affinata negli anni la stima della distanza minima (per gli zeri $\rho$) di $\beta$ da $1$ in funzione di $T$: $c/\log{T}$; $c\log\log{T}/\log{T}$ (Littlewood, 1922); $c/(\log{T})^{2/3+\epsilon}$ (Vinogradov, 1958).

  Ci sono anche le seguenti stime per $\psi(x)-x$: $O\bigl(x\exp(-c\sqrt{\log{x}\log\log{x}})\bigr)$ (Littlewood); $O\bigl(x\exp(-c(\log{x})^{3/5-\epsilon}\bigr)$ (Vinogradov).
  \item $\displaystyle \xi(s)=e^{A+Bs}\prod_{\rho}\left(1-\frac{s}{\rho}\right)e^{-s/\rho}$, il prodotto di Weierstrass.
  \item La famosa \textit{Ipotesi di Riemann} (RH): per ogni zero non banale, $\beta=1/2$.
\end{enumerate}
