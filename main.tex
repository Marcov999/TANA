\documentclass{article}
\usepackage{mstyle}
\usepackage{pgfplots}
\usetikzlibrary{intersections, pgfplots.fillbetween}

\title{Appunti di Teoria Analitica dei Numeri A}
\date{}
\author{Marco Vergamini}

\begin{document}
\maketitle
\newpage
\tableofcontents
\newpage


\section*{Introduzione}
\addcontentsline{toc}{section}{Introduzione}
Questi appunti sono basati sul corso Teoria Analitica dei Numeri A tenuto dal professor Giuseppe Puglisi nel secondo semestre dell'anno accademico 2020/2021. Sono dati per buoni (si vedano i prerequisiti del corso) i corsi di Aritmetica, Analisi 1 e 2 e Teoria dei Numeri Elementare, più le base dell'analisi complessa in una variabile. Verranno omesse o soltanto hintate le dimostrazioni più semplici, ma si consiglia comunque di provare a svolgerle per conto proprio. Ogni tanto sarà commesso qualche abuso di notazione, facendo comunque in modo che il significato sia reso chiaro dal contesto. Inoltre, la notazione verrà alleggerita man mano, per evitare inutili ripetizioni e appesantimenti nella lettura. Si ricorda anche che questi appunti sono scritti non sempre subito dopo le lezioni, non sempre con appunti completi, ecc\dots. Spesso saranno rivisti, verranno aggiunte cose che mancavano perché c'era poco tempo (o voglia\dots), potrebbero mancare argomenti più o meno marginali\dots insomma, non è un libro di testo per il corso, ma vuole essere un valido supporto per aiutare gli studenti che seguono il corso. Spero di essere riuscito in questo intento. \\

Ho inoltre deciso di omettere gli appunti delle prime due/tre ore di lezione, nelle quali sono stati esposti a grandi linee gli argomenti e i principali risultati trattati nel corso, poiché sono stati descritti in modo discorsivo e impreciso, e comunque verranno ovviamente trattati dettagliatamente nel seguito.


\newpage

\section{Funzioni intere e funzione $\Gamma$}

\subsection{Strumenti di analisi complessa}
\begin{lm}
  (Mittag-Leffler) Sia $(z_n)_{n \in \mathbb{N}}$ una successione di numeri complessi t.c. $|z_n| \longrightarrow \infty$ per $n \longrightarrow \infty$ e $0<|z_n| \le |z_{n+1}|$ per ogni $n$. Sia inoltre $(m_n)_{n \in \mathbb{N}}$ un'altra successione con $m_n \in \mathbb{C}^*$ per ogni $n$. Allora esistono $p_n \in \mathbb{N}\cup \{0\}$ t.c.
  $$f(z)=\sum_{n=1}^{+\infty} \left(\frac{z}{z_n}\right)^{p_n}\frac{m_n}{z-z_n}$$
  converge in $K \subset \mathbb{C}\setminus\{z_1, z_2, \dots\}$ compatto. Inoltre, se $|z|<|z_1|$, si ha
  $$f(z)=-\sum_{k=1}{+\infty} \left(\sum_{n:p_n<k}m_nz_n^{-k}\right)z^{k-1}.$$
\end{lm}

\begin{proof}
  Prendiamo $r_n$ reali positivi con $r_n \le r_{n+1}$ e $r_n \longrightarrow +\infty$ per $n \longrightarrow +\infty$, e t.c. $r_n<|z_n|$. Per $|z| \le r_n$ si ha
  $$\left|\frac{m_n}{z-z_n}\right| \le \frac{m_n}{|z_n|-r_n}, \quad \left|\frac{z}{z_n}\right|<\frac{|z|}{r_n} \le 1 \implies$$
  perciò si ha che esistono $p_n \in \mathbb{N}\cup\{0\}$ t.c. $\displaystyle \left|\frac{z}{z_n}\right|^{p_n} <\epsilon_n \frac{|z_n|-r_n}{|m_n|}$, con $\epsilon_n>0$ e $\displaystyle \sum_{n=1}^{+\infty} \epsilon_n<+\infty$.
  Abbiamo dunque $\displaystyle \left|\left(\frac{z}{z_n}\right)^{p_n}\frac{m_n}{z-z_n}\right| \le \left|\frac{z}{z_n}\right|^{p_n} \frac{|m_n|}{|z_n|-r_n}<\epsilon_n$. Fissiamo ora il compatto $K$ e consideriamo $N$ t.c. $|z| \le r_N$ per ogni $z \in K$.
  Poniamo $\displaystyle M_n=\max_{z \in K} \left|\left(\frac{z}{z_n}\right)^{p_n}\frac{m_n}{z-z_n}\right|$, per $n \le N-1$. Per ogni $z \in K$ si ha che
  $$\sum_{n=1}^{+\infty} \left|\left(\frac{z}{z_n}\right)^{p_n}\frac{m_n}{z-z_n}\right| \le \sum_{n=1}^{N-1} M_n+\sum_{n=N}^{+\infty} \epsilon_n<+\infty.$$

  Se $|z|<|z_1|$, possiamo scrivere
  \begin{align*}
    f(z) & =\sum_{n=1}^{+\infty} \left(\frac{z}{z_n}\right)^{p_n}\frac{m_n}{z-z_n} \\
    & =-\sum_{n=1}^{+\infty} \frac{m_nz^{p_n}}{z_n^{p_n+1}}\frac{1}{1-z/z_n}=-\sum_{n=1}^{+\infty}m_n\sum_{k=p_n+1}^{+\infty} \frac{z^{k-1}}{z_n^k}.
  \end{align*}
  Poiché nella prima parte della dimostrazione abbiamo visto che c'è convergenza totale, possiamo scambiare le due sommatorie ottenendo così la seconda parte della tesi.
\end{proof}

\begin{oss} \label{1.1.2}
  Per $|z| \le r_n$ si ha
  \begin{gather*}
    \left|\frac{z}{z_n}\right|=\frac{2|z|}{|z_n|+|z_n|}<\frac{2|z|}{|z_n|+r_n} \le \frac{2|z|}{|z-z_n|} \\
    |m_m| \left|\frac{z}{z_n}\right|^{p_n+1} \le \frac{2|z|}{|z-z_n|}\left|\frac{z}{z_n}\right|^{p_n}|m_n|=2|z|\left|\left(\frac{z}{z_n}\right)^{p_n}\frac{m_n}{z-z_n}\right| \\
    \sum_{n=1}^{+\infty} |m_n|\left|\frac{z}{z_n}\right|^{p_n+1} \le 2|z|\sum_{n=1}^{+\infty} \left|\left(\frac{z}{z_n}\right)^{p_n}\frac{m_n}{z-z_n}\right|<+\infty.
  \end{gather*}
\end{oss}

\begin{ex}
  Se $z_n=n$ e $m_n=1$ per ogni $n$, basta prendere $p_n=1$.

  Sia invece $\displaystyle |z_0|>\max_K |z|$ e consideriamo $|z_n|>|z_0|+1$. Si ha che
  \begin{align*}
    \left|\left(\frac{z}{z_n}\right)^{p_n}\frac{m_n}{z-z_n}\right| & \le \left|\frac{z_0}{z_n}\right|^{p_n+1}\frac{|m_n|}{|z_n|-|z_0|}\frac{|z_n|}{|z_0|} \\
    & \le \left|\frac{z_0}{z_n}\right|^{p_n+1}\frac{|z_0|+1}{|z_0|+1-|z_0|}\frac{|m_n|}{|z_0|}=|m_n|\left|\frac{z_0}{z_n}\right|^{p_n+1}\left(1+\frac{1}{|z_0|}\right),
  \end{align*}
  dove la seconda disugaglianza segue dal fatto che la funzione $\frac{t}{t-|z_0|}$ è decrescente. In questo caso, vale la maggiorazione opposta a quella dell'osservazione \ref{1.1.2}.
\end{ex}


\subsection{Funzioni intere di ordine finito}
Introduciamo ora un argomento importante: le funzioni intere di ordine finito.

Notazione: scriviamo che $f(z) \ll g(z)$ $(z \longrightarrow +\infty)$ $\iff$ $f(z)=O\big(g(z)\big)$. Inoltre, scriveremo $\ll_{\epsilon}$ per indicare che la costante dell'$O$-grande dipende da un parametro $\epsilon$.

\begin{defn}
  Data $F$ intera, si dice che ha \textit{ordine} $\alpha \ge 0$ se
  $$F(z) \ll_{\epsilon} e^{|z|^{\alpha+\epsilon}} \, (z \longrightarrow +\infty)$$
  per ogni $\epsilon>0$ e $\alpha$ è il minimo valore positivo per cui vale questa cosa. Equivalentemente, $\alpha=\inf\{A \ge 0 \mid F(z) \ll_A e^{|z|^A}\}$. Scriviamo $ord(F)=\alpha$.

  Se non vale $F(z) \ll_A e^{|z|^A}$ per nessun $A$ diciamo che $F$ ha ordine infinito.
\end{defn}

\begin{ex}
  \begin{enumerate}
    \item Sia $p$ un polinomio di grado $k$, allora $|p(z)| \le |a_k||z|^k\big(1+o(1)\big) \ll_{\epsilon} e^{|z|^{\epsilon}}$ per ogni $\epsilon>0$, dunque $p$ ha ordine $0$.
    \item $|e^{az+b}|=e^{\mathfrak{Re}(az+b)} \ll_{\epsilon} e^{|z|^{1+\epsilon}}$ per ogni $\epsilon>0$.
    \item Più in generale, dato $p_k$ un generico polinomio di grado $k$ abbiamo che $|e^{p_k(z)}|=e^{\mathfrak{Re}\big(p_k(z)\big)} \le e^{|p_k(z)|} \le e^{|a_k||z|^k\big(1+o(1)\big)} \ll_{\epsilon} e^{|z|^{k+\epsilon}}$ per ogni $\epsilon>0$.

    Inoltre, prendendo $z$ sulla retta $arg(z)=-arg(a_k)/k$, abbiamo che vale $|e^{p_k(z)}|=e^{\mathfrak{Re}\big(p_k(z)\big)}=e^{a_kz^k\big(1+o(1)\big)}=e^{|a_k||z|^k\big(1+o(1)\big)} \gg_{\epsilon} e^{|a_k||z|^k}$. Dunque l'ordine è $k$ e l'$\inf$ nella definizione non viene raggiunto.
  \end{enumerate}
\end{ex}

\begin{oss}
  Se $F_1$ e $F_2$ hanno ordine $\alpha_1$ e $\alpha_2$, allora $F_1+F_2$ e $F_1 \cdot F_2$ hanno ordine minore o uguale di $\max\{\alpha_1,\alpha_2\}$. La dimostrazione è lasciata come esercizio per il lettore.
\end{oss}

\begin{lm} \label{1.2.4}
  Sia $f$ olomorfa in $|z-z_0| \le R$ non costantemente nulla e sia $0<r<R$. Sia inoltre $N=\sharp\{z \in \mathbb{C} \mid |z-z_0| \le r, f(z)=0\}$ (ricordiamo che sono contati con molteplicità). Allora
  $$|f(z_0)| \le \left(\frac{r}{R}\right)^N\max_{|z-z_0|=R}|f(z)|.$$
\end{lm}

\begin{proof}
  Consideriamo senza perdita di generalità, a meno di una traslazione e di un'omotetia, $R=1,z_0=0$. Siano $z_n$ gli zeri di $f$ in $|z| \le r$ contati con molteplicità e sia $\displaystyle g(z)=f(z)\prod_{n=1}^N \frac{1-\bar{z}_nz}{z-z_n}$. Per $|z|=1$ scriviamo $z=e^{i\theta},\theta \in \mathbb{R}$.
  Allora $\displaystyle \left|\frac{1-\bar{z}_ne^{i\theta}}{e^{i\theta}-z_n}\right|=|e^{i\theta}|\left|\frac{\bar{z}_n-e^{-i\theta}}{z_n-e^{i\theta}}\right|=1,$ quindi, per il principio del massimo modulo per funzioni olomorfe (che d'ora in avanti useremo senza menzionarlo esplicitamente), $\displaystyle |g(z)| \le \max_{|z|=1} |f(z)|$ per $|z| \le 1$. Si ha dunque
  \begin{gather*}
    |f(w)|=|g(w)|\prod_{n=1}^N\left|\frac{w-z_n}{1-\bar{z}_nw}\right| \le \max_{|z|=1} |f(z)| \prod_{n=1}^N \left|\frac{w-z_n}{1-\bar{z}_nw}\right| \implies \\
    \implies |f(0)| \le \max_{|z|=1} |f(z)| \prod_{n=1}^N |z_n| \le r^N \max_{|z|=1} |f(z)|.
  \end{gather*}
\end{proof}

\begin{cor} \label{1.2.5}
  Siano $f, r, R, N$ come nel lemma \ref{1.2.4}. Se $f(z_0)\not=0$ allora
  $$N \le \frac{1}{\log(R/r)}\log\left(\frac{\max_{|z-z_0|=R}|f(z)|}{|f(z_0)|}\right).$$
\end{cor}

\begin{proof}
  Basta prendere la disugaglianza data dal lemma \ref{1.2.4}, portarla nella forma $\left(\frac{R}{r}\right)^N \le (\dots)$, prendere il logaritmo e dividere per $\log(R/r)$.
\end{proof}

\begin{thm} \label{1.2.6}
  Sia $F$ una funzione intera di ordine $\alpha<+\infty$ e consideriamo $N(r)=\sharp\{z \in \mathbb{C} \mid F(z)=0, |z| \le r\}$. Allora $N(r) \ll_{\epsilon} r^{\alpha+\epsilon}$ per ogni $\epsilon>0$.
\end{thm}

\begin{proof}
  Prendiamo $R=2r$, allora $\displaystyle \max_{|z|=R}|F(z)| \ll_{\epsilon} e^{(2r)^{\alpha+\epsilon}}$ per ogni $\epsilon>0$ $\implies$ $\displaystyle \log\left(\max_{|z|=R}|F(z)|\right) \ll_{\epsilon} r^{\alpha+\epsilon}$ per ogni $\epsilon>0$.
  Se $F(0)\not=0$, per il corollario \ref{1.2.5} abbiamo $N(r) \ll_{\epsilon} r^{\alpha+\epsilon}$ per ogni $\epsilon>0$. Se $F(0)=0$, consideriamo $\tilde{F}(z)=F(z)/z^m$ dove $m$ è la molteplicità di $0$ come zero. Per $|z| \le 1$, $\tilde{F} \ll 1$ per continuità.
  Per $|z|>1$, $\tilde{F}(z)=F(z)\frac{1}{z^n} \implies ord(\tilde{F}) \le \max\{\alpha,0\}=\alpha$. Allora si ripete la dimostrazione per $\tilde{F}$, poi si osserva che il numero di zeri di $F$ varia solo per la costante additiva $m$.
\end{proof}

\begin{defn}
  Sia $z_n\not=0$ una successione senza limiti finiti. Si dice \textit{esponente di convergenza} di $z_n$, se esiste, il numero $$\beta=\inf\{B>0 \mid \sum_{n=1}^{+\infty} \frac{1}{|z_n|^B}<+\infty\},$$
  ovvero $\displaystyle \sum_{n=1}^{+\infty} \frac{1}{|z_n|^{\beta+\epsilon}}<+\infty$ per ogni $\epsilon>0$ e $\beta$ è il minimo valore per cui è vero.
\end{defn}

\begin{ex}
  $z_n=\log{n}$ non ha esponente di convergenza finito.
\end{ex}

\begin{thm} \label{1.2.9}
  Sia $F$ una funzione intera di ordine $\alpha>0$, $F(0)\not=0$ t.c. la successione dei suoi zeri $z_n$ ha esponente di convergenza $\beta$. Allora $\beta \le \alpha$.
\end{thm}

\begin{proof}
  Se $ord(F)=\alpha<+\infty$, per il teorema \ref{1.2.6} si ha $N(r) \ll_{\epsilon} r^{\alpha+\epsilon}$ per ogni $\epsilon>0$.
  Prendendo $r_n=|z_n|$, ricordando che $z_n$ sono gli zeri di $F$ otteniamo $n \le N(r_n) \ll_{\epsilon} r_n^{\alpha+\epsilon}=|z_n|^{\alpha+\epsilon}=|z_n|^{\alpha+\epsilon}$ per ogni $\epsilon>0$ (non è $n=N(r_n)$ perché potrebbe esserci più di uno zero sul cerchio $|z|=r_n$). Allora $|z_n| \gg_{\epsilon} n^{\frac{1}{\alpha+\epsilon}}$ e dunque
  $$\sum_{n=1}^{+\infty} |z_n|^{-(\alpha+2\epsilon)} \ll_{\epsilon} \sum_{n=1}^{+\infty} n^{-\frac{\alpha+2\epsilon}{\alpha+\epsilon}}<+\infty.$$
  Perciò $\beta \le \inf\{\alpha+2\epsilon \mid \epsilon>0\}=\alpha$.
\end{proof}

\begin{thm} \label{wfatt}
  Sia $F$ intera di ordine finito, $F(0)\not=0$. Allora la fattorizzazione di Weierstrass si scrive come
  \begin{equation} \label{wfattoformula}
    F(z)=e^{G(z)} \prod_{n=1}^{+\infty} \left(1-\frac{z}{z_n}\right)\exp\Bigg(\sum_{k=1}^p\frac{1}{k}\left(\frac{z}{z_n}\right)^k\Bigg),
  \end{equation}
  dove $p \ge 0$ è indipendente da $n$ e t.c. $\displaystyle \sum_n |z_n|^{-(p+1)}<+\infty$. Talvolta si trova scritta con la notazione $\displaystyle E(z,p)=(1-z)\exp\left(\sum_{k=1}^p\frac{1}{k}z^k\right)$ o simili.
\end{thm}

\begin{proof}
  Diamo solamente una traccia. Osserviamo che
  $$\sum_{n=1}^{+\infty} |m_n|\left|\frac{z}{z_n}\right|^{p+1}<+\infty \iff \sum_{n=1}^{+\infty} |z_n|^{-(p+1)}<+\infty\text{ per ogni }z \in \mathbb{C}.$$
  Scegliendo $p+1=\alpha+\epsilon$, per il teorema \ref{1.2.9} si ha $\displaystyle \sum_{n=1}^{+\infty} |z_n|^{-(p+1)}<+\infty$.
\end{proof}

\begin{oss}
  Sia $\beta$ l'esponente di convergenza di $z_n$, zeri di una funzione $F$ di ordine $\alpha$ ($\beta \le \alpha$). Il $p$ migliore è:
  \begin{enumerate}
    \item se $\beta$ non è intero, $p=\lfloor\beta\rfloor$;
    \item se $\beta$ è intero e nella definizione con l'$\inf$ è in realtà un minimo, allora $p=\beta-1$;
    \item se $\beta$ è intero ma nella definizione con l'$\inf$ questo non viene raggiunto, cioè $\beta$ non è un minimo, allora $p=\beta$.
  \end{enumerate}
  Abbiamo dunque $\beta-1 \le p \le \beta \le \alpha$. Scrivendo la fattorizzazione di Weierstrass come in \eqref{wfattoformula} usando il miglior $p$ possibile si ha la forma canonica.
\end{oss}

\begin{thm}
  (Borel-Carathéodory) Siano $0<r<R$ e sia $f$ olomorfa in $|z-z_0| \le R$. Allora
  \begin{equation*}
    \max_{|z-z_0|=r}|f(z)| \le \frac{2r}{R-r}\max_{|z-z_0|=R}\mathfrak{Re}\big(f(z)\big)+\frac{R+r}{R-r}|f(z_0)|.
  \end{equation*}
  Se inoltre $\displaystyle \max_{|z-z_0|=R}\mathfrak{Re}\big(f(z)\big) \ge 0$, allora
  \begin{equation*}
    \max_{|z-z_0|=r}|f^{(n)}(z)| \le \frac{n!2^{n+2}R}{(R-r)^{n+1}}\Big(\max_{|z-z_0|=R}\mathfrak{Re}\big(f(z)\big)+|f(z_0)|\Big).
  \end{equation*}
\end{thm}

\begin{proof}
  Supponiamo senza perdita di generalità $z_0=0$. Se $f$ è costante la tesi è banale, consideriamo dunque $f$ non costante. Facciamo il caso $f(0)=0$. Sia $\displaystyle A=\max_{|z|=R}\mathfrak{Re}\big(f(z)\big)$. $A>0$, infatti basta osservare che $|e^{f(z)}|=e^{\mathfrak{Re}\big(f(z)\big)}$ e applicare il principio del massimo modulo a $e^{f(z)}$ (se il massimo non fosse maggiore di $1$, sarebbe proprio $1$ perché $f(0)=0$ e per lo stesso motivo sarebbe costante). Sia $\varphi(z)=\frac{f(z)}{2A-f(z)}$.
  Osserviamo che $\mathfrak{Re}\big(2A-f(z)\big)=2A-\mathfrak{Re}\big(f(z)\big) \ge A>0 \implies 2A-f(z)\not=0$, quindi $\varphi$ è olomorfa in $|z| \le R$.
  Posto $f(z)=u+iv$, allora $|\varphi(z)|^2=\frac{u^2+v^2}{(2A-u)^2+v^2} \le 1$. Infatti, $u \le A \implies u \le 2A-u$ e ovviamente $-2A+u \le u$, dunque $u^2 \le (2A-u)^2$.
  Poiché $\varphi(0)=0$, $\varphi(z)/z$ è olomorfa in $|z| \le R$ e
  \begin{gather*}
    \left|\frac{\varphi(z)}{z}\right| \le \max_{|z|=R} \left|\frac{\varphi(z)}{z}\right| \le R \implies
  \end{gather*}
\end{proof}


\subsection{Numeri e polinomi di Bernoulli e $\zeta(2k)$}
In questa sezione sfrutteremo i risultati visti finora per calcolare i valori della $\zeta$ di Riemann sui pari.

\begin{defn}
  Data $F$ intera, il \textit{genere} è $g=\max\{p,q\}$.
\end{defn}

\begin{oss}
  $\alpha-1 \le g \le \alpha$. La seconda è ovvia. Se fosse $g<\alpha-1$, avremmo $p<\alpha-1 \implies \beta \le p+1<\alpha$, ma anche $q<\alpha-1$; dato che abbiamo $\alpha=\max\{\beta,q\}$, si ha un assurdo.
\end{oss}

\begin{ex}
  Sia $F(z)=\dfrac{\sin(\pi z)}{\pi z}$. Si ha
  $$\sin(\pi z)=\frac{e^{i\pi z}-e^{-i\pi z}}{2i} \ll_{\epsilon} e^{|z|^{1+\epsilon}} \text{ per ogni } \epsilon>0,$$
  dunque $F$ ha ordine $\alpha \le 1$. Gli zeri sono $z_n=\pm n$ per ogni $n \in \mathbb{N}$, quindi $\displaystyle \sum_{n=1}^{+\infty} \frac{1}{|z_n|^{1+\epsilon}}<+\infty$ per ogni $\epsilon>0$ ma chiaramente non per $\epsilon=0$, da cui $\beta=1$;
  $\alpha \ge \beta \implies \alpha=1$. Inoltre $p=1$. Studiamo questa funzione.
\end{ex}

I termini del prodotto di Weierstrass sono $E\left(\dfrac{z}{n},1\right)=\left(1-\dfrac{z}{n}\right)\exp\left(\dfrac{z}{n}\right)$. $E\left(\dfrac{z}{n},1\right) \cdot E\left(\dfrac{z}{-n},1\right)=\left(1-\dfrac{z^2}{n^2}\right)$.
Abbiamo quindi il prodotto $\displaystyle \prod_n \left(1-\frac{z^2}{n^2}\right)$. Per il teorema di Hadamard, il grado del polinomio $G$ è $q \le \alpha=1=p$. Per il corollario \ref{1.2.14}, $G(z)=\log{F(0)}+\dfrac{F'}{F}(0)z=\log{1}+0=0$.
Abbiamo allora $\displaystyle \sin(\pi z)=\pi z\prod_{n=1}^{+\infty} \left(1-\frac{z^2}{n^2}\right)$.

Consideriamo
$$\log\big(F(z)\big)=\log\left(\frac{\sin(\pi z)}{\pi z}\right)=\log\big(\sin(\pi z)\big)\log(\pi z);$$
derivando troviamo
$$\frac{F'}{F}(z)=\pi\frac{\cos(\pi z)}{\sin(\pi z)}-\frac{1}{z}=\pi\cot(\pi z)-\frac{1}{z}.$$
Per $k \ge 2$, dal corollario \ref{1.2.14} si ha $\cot$
$$\sum_n \frac{1}{n^k}+\sum_n \frac{1}{(-n)^k}=-\frac{D^{k-1}\big(\pi\cot(\pi z)-1/z\big)_{z=0}}{(k-1)!}.$$
Passando ai pari, per $k \ge 1$ si ha
\begin{gather*}
  \zeta(2k)=\sum_{n=1}^{+\infty} \frac{1}{n^{2k}}=-\frac{D^{2k-1}\big(\frac{\pi}{2}\cot(\pi z)-\frac{1}{2z}\big)_{z=0}}{(2k-1)!} \implies \\
  \implies \frac{1}{2z}-\frac{\pi}{2}\cot(\pi z)=\sum_{k=1}^{+\infty} \zeta(2k)z^{2k-1} \text{ per } |z|<1.
\end{gather*}
È un esercizio verificare che $\zeta(2k)=1+O(1/2^{2k}) \sim 1$ per $k \longrightarrow +\infty$. Segue che $\sqrt[k]{\zeta(2k)} \longrightarrow 1$ per $k \longrightarrow +\infty$, ma anche, in particolare, $\sqrt[2k-1]{\zeta(2k)} \longrightarrow 1$ per $k \longrightarrow +\infty$, dunque il raggio di convergenza della funzione $\dfrac{1}{2z}-\dfrac{\pi}{2}\cot(\pi z)$ è proprio $1$.

\begin{defn}
  Si dicono \textit{numeri di Bernoulli} quelli definiti nel modo seguente:
  $$B_n=D^n\left(\frac{z}{e^z-1}\right)_{z=0}.$$
\end{defn}

\begin{oss}
  Poiché la funzione $\frac{z}{e^z-1}$ ha raggio di convergenza $2\pi$, dev'essere $\displaystyle \limsup \sqrt{\frac{B_n}{n!}}=\frac{1}{2\pi}$.
\end{oss}

\begin{oss}
  \begin{gather*}
    f(z)=\frac{z}{e^z-1}+\frac{z}{2}=\frac{z+ze^z}{2(e^z-1)}=\frac{z(1+e^z)}{2(e^z-1)} \\
    f(-z)=\frac{-ze^z}{1-e^z}-\frac{z}{2}=\frac{-ze^z-z}{2(1-e^z)}=\frac{z(1+e^z)}{2(e^z-1)}
  \end{gather*}
  Quindi $f$ è pari. Allora
  $$D^{2k-1}\left(\frac{z}{e^z-1}\right)_{z=0}=-D^{2k-1}\left(\frac{z}{2}\right)_{z=0}=\begin{cases}
    -1/2 & \mbox{se }k=1 \\ 0 & \mbox{se } k>1
\end{cases}$$,
dunque $B_1=-1/2$ e $B_{2n-1}=0$ per ogni $n>1$.
\end{oss}

\begin{oss}
  \begin{gather*}
    1=\left(\sum_{n=0}^{+\infty}\frac{B_n}{n!}z^n\right)\frac{e^z-1}{z}=\left(\sum_{n=0}^{+\infty}\frac{B_n}{n!}z^n\right)\left(\sum_{m=1}^{+\infty}\frac{z^{m-1}}{m!}\right)= \\
    =\sum_{n=1}^{+\infty}\left(\sum_{k=0}^{n-1}\frac{B_k}{k!}\frac{n!}{(n-k)!}\right)\frac{z^{n-1}}{n!}=\sum_{n=1}^{+\infty}\Bigg(\sum_{k=0}^{n-1}\binom{n}{k}B_k\Bigg)\frac{z^{n-1}}{n!}
  \end{gather*}
  Perciò $B_0=1$ e $\displaystyle \sum_{k=0}^{n-1}\binom{n}{k}B_k=0$ per ogni $n \ge 2$, che ci dà $\displaystyle \sum_{k=0}^n\binom{n}{k}B_k=B_n$ e
  $$B_{n-1}=-\frac{1+\binom{n}{1}B_1+\dots+\binom{n}{n-2}B_{n-2}}{\binom{n}{n-1}} \implies B_n \in \mathbb{Q}.$$
\end{oss}

\begin{oss}
  \begin{gather*}
    \cot(z)=i\frac{e^{iz}+e^{-iz}}{e^{iz}-e^{-iz}}=i\frac{e^{2iz}+1}{e^{2iz}-1}=i\left(1+\frac{2z}{z(e^{2iz}-1)}\right)= \\
    =i+\frac{1}{z}\cdot\frac{2iz}{e^{2iz}-1}=i+\frac{1}{z}\left(1-iz+\sum_{k=1}^{+\infty}\frac{(-1)^kB_{2k}(2z)^{2k}}{(2k)!}\right)= \\
    =\frac{1}{z}+\sum_{k=1}^{+\infty}\frac{2(-1)^kB_{2k}}{(2k)!}(2z)^{2k-1} \implies \\
    \implies \sum_{k=1}^{+\infty}\zeta(2k)z^{2k-1}=\frac{1}{2z}-\frac{\pi}{2}\cot(\pi z)=\frac{1}{2}\sum_{k=1}^{+\infty} \frac{(-1)^{k-1}(2\pi)^{2k}B_{2k}z^{2k-1}}{(2k)!} \implies \\
    \implies 1+O(1/4^k)=\zeta(2k)=\frac{(-1)^{k-1}(2\pi)^{2k}B_{2k}}{2(2k)!} \implies \\
    \implies (-1)^{k-1}B_{2k}=\frac{2(2k)!}{(2\pi)^{2k}}\zeta(2k).
  \end{gather*}
\end{oss}

\begin{oss}
  $B_{2k}(-1)^{k-1}>0$. Inoltre,
  $$\frac{2(2k)!}{(2\pi)^{2k}} \le |B_{2k}| \le \frac{2(2k)!\pi^2}{6(2\pi)^{2k}} \text{ per ogni } k \ge 1.$$
  Vogliamo vedere quando si ha
  \begin{gather*}
    |B_{2(k+1)}| \ge \frac{2\big(2(k+1)\big)!}{(2\pi)^{2(k+1)}} \stackrel{?}{\ge} \frac{2(2k)!\pi^2}{6(2\pi)^{2k}} \ge |B_{2k}| \\
    \frac{2(k+1)(2k+1)}{4\pi^2} \stackrel{?}{\ge} \frac{\pi^2}{6} \\
    \pi^4 \stackrel{?}{\le} 3(k+1)(2k+1),
  \end{gather*}
  che è vero per $k \ge 4$. Da lì in poi, i numeri di Bernoulli di indice pari hanno moduli crescenti.
\end{oss}

\begin{defn}
  Si dice \textit{polinomio di Bernoulli} $n$-esimo il seguente:
  $$B_n(x)=\sum_{k=0}^n \binom{n}{k}B_kx^{n-k}.$$
\end{defn}

\begin{ftt}
  \begin{itemize}
    \item $B_n(0)=B_n$
    \item $\displaystyle B_n(1)=\sum_{k=0}^n \binom{n}{k}B_k=(-1)^nB_n$
  \end{itemize}
\end{ftt}

\begin{oss}
  \begin{gather*}
    \frac{ze^{xz}}{e^z-1}=\left(\sum_{n=0}^{+\infty}\frac{B_n}{n!}z^n\right)\left(\sum_{m=0}^{+\infty}\frac{x^mz^m}{m!}\right)=\sum_{n=0}^{+\infty}\left(\sum_{k=0}^n\frac{B_k}{k!}\cdot \frac{x^{n-k}}{(n-k)!}\right)z^n= \\
    =\sum_{n=0}^{+\infty}\left(\sum_{k=0}^nB_k\binom{n}{k}x^{n-k}\right)\frac{z^n}{n!}=\sum_{n=0}^{+\infty}\frac{B_n(x)}{n!}z^n.
  \end{gather*}
\end{oss}

Adesso un po' di fatti che chi vuole può divertirsi a dimostrare per esercizio.

\begin{ftt}
  \begin{enumerate}
    \item $\displaystyle B_n(x+y)=\sum_{k=0}^n\binom{n}{k}B_k(x)y^{n-k}$. Per $y=1$ si ha $\displaystyle B_n(x+1)=\sum_{k=0}^n\binom{n}{k}B_k(x)$;
    \item $B_n(x+1)-B_n(x)=nx^{n-1}$;
    \item $\displaystyle \sum_{k=m}^{n-1} k^r=\frac{B_{r+1}(n)-B_{r+1}(m)}{r+1}$;
    \item $B'_n(x)=nB_{n-1}(x)$.
  \end{enumerate}
\end{ftt}

Adesso, un caso particolare di un teorema che non dimostreremo, che si utilizza per dimostrare una proposizione che dimostreremo in un altro modo.

\begin{thm}
  (formula di sommazione di Eulero) Sia $f:[a,b] \longrightarrow \mathbb{C}$ di classe $C^1$. Allora
  $$\sum_{a<k \le b}f(k)=\int_a^b f(x)\diff x-\bigg[B_1(\{x\})f(x)\bigg]_a^b+\int_a^b B_1(\{x\})f'(x)\diff x.$$
  Se $a=m, b=n$,
  $$\sum_{m<k \le n} f(k)=\int_m^nf(x)\diff x-B_1\big(f(n)-f(m)\big)+\int_m^n B_1(\{x\})f'(x)\diff x.$$
\end{thm}

\begin{prop}
  Esiste
  $$\lim_{n \longrightarrow +\infty} \sum_{k=1}^n \frac{1}{k}-\log{n}=\gamma,$$
  con $0<\gamma<1$.
\end{prop}


\subsection{La funzione $\Gamma$ di Eulero}
Adesso, un caso particolare di un teorema che non dimostreremo; lo utilizzeremo per dimostrare una proposizione che potrebbe essere dimostrata anche con il lemma di sommazione di Abel.

\begin{thm}
  (formula di sommazione di Eulero-Maclaurin) Consideriamo $f:[a,b] \longrightarrow \mathbb{C}$ di classe $C^1$. Allora
  $$\sum_{a<k \le b}f(k)=\int_a^b f(x)\diff x-\bigg[B_1(\{x\})f(x)\bigg]_a^b+\int_a^b B_1(\{x\})f'(x)\diff x.$$
  Se $a=m, b=n$,
  $$\sum_{m<k \le n} f(k)=\int_m^nf(x)\diff x-B_1\cdot\big(f(n)-f(m)\big)+\int_m^n B_1(\{x\})f'(x)\diff x.$$
\end{thm}

\begin{cor} \label{EuMac-cor}
  $$\sum_{k=m}^n f(k)=\int_m^n f(x)\diff x+\frac{f(m)+f(n)}{2}+\int_m^n B_1(\{x\})f'(x)\diff x,$$
  dove ricordiamo che $B_1(\{x\})=\{x\}-1/2$.
\end{cor}

Per completezza, riportiamo anche il lemma di Abel.

\begin{lm}
  (formula di sommazione di Abel)
  $$\sum_{k=m}^n a_kf(k)=\left(\sum_{k=m}^n a_k\right)f(n)-\int_m^n\left(\sum_{m \le k \le \lfloor x \rfloor}a_k\right)f'(x)\diff x.$$
\end{lm}

\begin{prop}
  Esiste
  $$\lim_{n \longrightarrow +\infty} \sum_{k=1}^n \frac{1}{k}-\log{n}=\gamma,$$
  con $0<\gamma<1$.
\end{prop}

\begin{proof}
  Applichiamo il corollario \ref{EuMac-cor} con $m=1$ e $f(x)=1/x$, otteniamo
  \begin{gather*}
    \sum_{k=1}^n \frac{1}{k}=\int_1^n \frac{\diff x}{x}+\frac{1}{2}+\frac{1}{2n}-\int_1^n \frac{\{x\}-1/2}{x^2}\diff x= \\
    =\log{n}+\frac{1}{2}+\frac{1}{2n}-\int_1^{+\infty} \frac{\{x\}-1/2}{x^2}\diff x+\int_n^{+\infty}\frac{\{x\}-1/2}{x^2}\diff x= \\
    =\log{n}+\frac{1}{2}+\frac{1}{2n}-\int_1^{+\infty} \frac{\{x\}}{x^2}\diff x+\frac{1}{2}+O\left(\int_n^{+\infty}\frac{\diff x}{x^2}\right)= \\
    =\log{n}+1-\int_1^{+\infty} \frac{\{x\}}{x^2}\diff x+O(1/n)=:\log{n}+\gamma+O(1/n).
  \end{gather*}
  Poiché $\displaystyle 0<\int_1^{+\infty} \frac{\{x\}}{x^2}\diff x<1$, si ha $0<\gamma<1$ con $\displaystyle \gamma=\lim_{n \longrightarrow +\infty} \sum_{k=1}^n \frac{1}{k}-\log{n}$.
\end{proof}

\begin{oss}
  Se $f:[1,+\infty) \longrightarrow \mathbb{R}$ è di classe $C^1$, infinitesima e non crescente, allora esiste $C>0$ t.c. $\displaystyle \sum_{1 \le k \le x} f(k)=\int_1^x f(y)\diff y+C+O\big(f(x)\big)$. La dimostrazione è lasciata per esercizio.
\end{oss}

\begin{oss} %controllare veridicità da appunti di Viola
  Si può dimostrare che
  $$\sum_{k=1}^n \frac{1}{k}=\log{n}+\gamma+\frac{1}{2n}-\sum_{r=1}^q \frac{B_{2r}}{2r}\cdot\frac{1}{n^{2r}}+O\left(\frac{1}{n^{2q+2}}\right).$$
\end{oss}

\begin{defn}
  Sia $\dfrac{1}{z\Gamma(z)}$ la funzione intera $F$ di ordine $1$ con zeri tutti semplici nei punti $-1,-2,-3,dots$ e t.c. $F(0)=1$ e $F'(0)=\gamma$.
\end{defn}

Deve essere $\deg{G} \le 1$, $\beta=1$ e $\alpha=1$. $G(z)=\log\big(F(0)\big)+\frac{F'}{F}(0)=\gamma z$, quindi $q=1$. Vale anche $p=1$ e $g=1$. Si ha dunque
$$\frac{1}{z\Gamma(z)}=e^{\gamma z}\prod_{n=1}^{+\infty}\left(1+\frac{z}{n}\right)e^{-z/n}.$$


\newpage

\section{A caccia di zeri}

\subsection{Lo spazio di Schwarz}
\begin{defn}
  Sia $f:\mathbb{R} \longrightarrow \mathbb{C}$; si dice che $f$ \textit{tende rapidamente} a $0$ per $|x| \longrightarrow +\infty$ se $\displaystyle \lim_{x \longrightarrow \pm \infty} |x|^nf(x)=0$ per ogni $n \in \mathbb{N}\cup\{0\}$.
\end{defn}

\begin{oss} \label{rapsse}
  $f$ tende rapidamente a $0$ se e solo se $f(x)|x|^n$ è limitata per ogni $n \in \mathbb{N}\cup\{0\}$.
\end{oss}

\begin{defn}
  Si dice \textit{spazio di Schwarz} $\mathcal{S}$ lo spazio su $\mathbb{C}$ delle funzioni $f \in C^{\infty}(\mathbb{R})$ (a valori complessi) tendenti rapidamente a $0$ insieme a tutte le loro derivate.
\end{defn}

\begin{oss}
  L'operatore $D^k$ manda $\mathcal{S}$ in sé per ogni $k \ge 0$.
\end{oss}

Notazione: indichiamo con $M^k$ l'operatore $(M^kf)(x)=x^kf(x)$; abbiamo che anche $M^k$ manda $\mathcal{S}$ in sé.

Consideriamo ora la trasformata di Fourier in $\mathcal{S}$ definita come
$$\hat{f}(\xi)=\int_{\mathbb{R}} f(x)e^{-2\pi\xi x}\diff x.$$
Si ha che è ben definita.

\begin{oss} \label{limFourier}
  $\displaystyle |\hat{f}(\xi)| \le \int_{\mathbb{R}} |f(x)|\diff x<+\infty$, quindi $\hat{f}$ è limitata se $f \in \mathcal{S}$.
\end{oss}

\begin{lm}
  L'operatore $\textasciicircum$ manda $\mathcal{S}$ in sé.
\end{lm}

\begin{proof}
  Derivando sotto il segno di integrale abbiamo
  \begin{gather*}
    \hat{f}'(\xi)=-2\pi i \int_{-\infty}^{+\infty} xf(x)e^{-2\pi i\xi x}\diff x=-2\pi i \widehat{Mf}(\xi) \implies \\
    \implies D\hat{f}=(-2\pi i)\widehat{Mf} \implies D^k\hat{f}=(-2\pi i)^k\widehat{M^kf} \implies \hat{f} \in C^{\infty}(\mathbb{R}).
  \end{gather*}
  Integrando per parti si ha
  \begin{gather*}
    \xi\hat{f}(\xi)=\int_{-\infty}^{+\infty} f(x)\xi e^{-2\pi i\xi x}\diff x= \\
    =\left[-\frac{1}{2\pi i}e^{-2\pi i\xi x}f(x)\right]_{-\infty}^{+\infty}+\frac{1}{2\pi i}\int_{-\infty}^{+\infty} f'(x)e^{-2\pi i\xi x}\diff x=\frac{1}{2\pi i}\widehat{Df}(\xi) \implies \\
    \implies M^k\hat{f}=\left(\frac{1}{2\pi i}\right)^k\widehat{D^kf}.
  \end{gather*}
  Vogliamo concludere applicando l'osservazione \ref{rapsse} alle funzioni $D^k\hat{f}$. Notiamo che
  $$M^hD^k\hat{f}=M^h(-2\pi i)^k\widehat{M^kf}=\left(\frac{1}{2\pi i}\right)^h(-2\pi i)^k\widehat{D^hM^kf};$$
  ci basta dunque mostrare che $\widehat{D^hM^kf}$ è limitata, ma questo segue dall'osservazione \ref{limFourier} e dal fatto che gli operatori $D$ e $M$ mandano $\mathcal{S}$ in sé.
\end{proof}

\begin{lm}
  (formula di Poisson) Se $f \in \mathcal{S}$ allora
  $$\sum_{n \in \mathbb{Z}} f(n)=\sum_{n \in \mathbb{Z}} \hat{f}(n).$$
\end{lm}

\begin{proof}
  Sia $\displaystyle g(x)=\sum_{n \in \mathbb{Z}} f(x+n)$. $g$ ha periodo $1$. Poiché $f \in \mathcal{S}$, abbiamo che $\displaystyle \sum_{n \in \mathbb{Z}} D^kf(x+n)$ converge uniformemente per ogni $k \ge 0$, dunque è uguale a $D^kg$, quindi $g \in C^{\infty}(\mathbb{R})$. Scriviamo $g$ in serie di Fourier: $\displaystyle g(x)=\sum_{m \in \mathbb{Z}} c_me^{2\pi imx}$. Si ha
  \begin{gather*}
    c_m=\int_0^1 g(x)e^{-2\pi imx}\diff x=\int_0^1 \sum_{n \in \mathbb{Z}} f(x+n)e^{-2\pi imx}\diff x=\\
    =\sum_{n \in \mathbb{Z}}\int_0^1 f(x+n)e^{-2\pi imx}\diff x \overset{y=x+n}{=} \sum_{n \in \mathbb{Z}} \int_n^{n+1} f(y)e^{-2\pi im(y-n)}\diff y=\\
    =\int_{\mathbb{R}} f(x)e^{-2\pi imx}\diff x=\hat{f}(m).
  \end{gather*}
  Basta allora guardare $g(0)$.
\end{proof}

\begin{lm}
  Sia $f(x)=e^{-\pi x^2}$ per $x \in \mathbb{R}$. Allora $f \in \mathcal{S}$ e inoltre $\hat{f}=f$.
\end{lm}

\begin{proof}
  Che $f \in \mathcal{S}$ è facile da dimostrare.
  \begin{gather*}
    \hat{f}(\xi)=\int_{\mathbb{R}} e^{-\pi x^2-2\pi i\xi x}\diff x \implies \\
    \implies D\hat{f}(\xi)=-2\pi i\int_{\mathbb{R}} xe^{-\pi x^2-2\pi i\xi x}\diff x \overset{\text{per parti}}{=} \\
    =\left[ie^{-\pi x^2-2\pi i\xi x}\right]_{-\infty}^{+\infty}+i(2\pi i\xi)\int_{\mathbb{R}} e^{-\pi x^2-2\pi i\xi x}\diff x=-2\pi \xi \hat{f}(\xi).
  \end{gather*}
  Abbiamo
  \begin{gather*}
    u'(\xi)=-2\pi\xi u(\xi) \implies \frac{u'}{u}(\xi)=-2\pi\xi \implies \\
    \implies \log\big(u(\xi)\big)=-\pi\xi^2+c \implies u(\xi)=Ce^{-\pi\xi^2} \implies \\
    \implies \hat{f}(\xi)=Ce^{-\pi\xi^2}.
  \end{gather*}
  Si ha anche $\displaystyle \hat{f}(0)=\int_{\mathbb{R}} e^{-\pi x^2}\diff x=1 \implies C=1$.
\end{proof}

\begin{oss}
  La serie $\displaystyle \sum_{n \in \mathbb{Z}} e^{-\pi n^2 z}$ converge totalmente per $\mathfrak{Re}\,z \ge \epsilon>0$.
\end{oss}


\subsection{La memoria di Riemann}
\begin{defn}
  Sia $z=x+iy$ con $x>0$. Si dice \textit{funzione $\vartheta$ di Jacobi} la seguente serie totalmente convergente:
  $$\vartheta(z)=\sum_{n \in \mathbb{Z}} e^{-\pi n^2z}.$$
\end{defn}

\begin{lm}
  Per $x=\mathfrak{Re}\,z>0$ si ha
  $$\vartheta(z)=\frac{1}{\sqrt{z}}\vartheta\left(\frac{1}{z}\right).$$
\end{lm}

\begin{proof}
  Possiamo dimostrare la formula per $z=x>0$, che valga in tutto il semipiano $\mathfrak{Re}\,z>0$ segue per prolungamento analitico. Sia $f(\xi)=e^{-\pi\xi^2}$ e sia $f_x(\xi)=f(\sqrt{x}\xi)=e^{-\pi x\xi^2}$. Allora
  \begin{gather*}
    \hat{f}_x(\xi)=\int_{\mathbb{R}} f(\sqrt{x}t)e^{-2\pi i\xi t}\diff t \overset{s=t\sqrt{x}}{=} \frac{1}{\sqrt{x}}\int_{\mathbb{R}} f(s)e^{-2\pi i\frac{\xi}{\sqrt{x}}s}\diff s= \\
    =\frac{1}{\sqrt{x}}\hat{f}\left(\frac{\xi}{\sqrt{x}}\right)=\frac{1}{\sqrt{x}}f\left(\frac{\xi}{\sqrt{x}}\right)=\frac{1}{\sqrt{x}}f_{\frac{1}{x}}(\xi).
  \end{gather*}
  Applicando la formula di Poisson abbiamo
  \begin{gather*}
    \sum_{n \in \mathbb{Z}} f_x(n)=\sum_{n \in \mathbb{Z}} \hat{f}_x(n)=\sum_{n \in \mathbb{Z}} \frac{1}{\sqrt{x}}f_{\frac{1}{x}}(n),
  \end{gather*}
  quindi
  $$\vartheta(x)=\sum_{n \in \mathbb{Z}} e^{-\pi n^2x}=\frac{1}{\sqrt{x}}\sum_{n \in \mathbb{Z}}e^{-\frac{\pi}{x}n^2}=\frac{1}{\sqrt{x}}\vartheta\left(\frac{1}{x}\right).$$
\end{proof}

Vogliamo usare quest'identità della funzione di Jacobi per dimostrare il teorema di Riemann sulla funzione $\zeta$. D'ora in avanti, per motivi storici la variabile sarà $s=\sigma+it$ con $\sigma, t \in \mathbb{R}$.

\begin{thm}
  (Riemann, 1860) La funzione $\zeta(s)$ è meromorfa in $\mathbb{C}$ con un polo semplice in $s=1$ con residuo $1$. Inoltre, posta
  $$\xi(s)=\frac{s(s-1)}{2}\pi^{-s/2}\Gamma\left(\frac{s}{2}\right)\zeta(s),$$
  allora $\xi$ è intera e fornisce il prolungamento analitico di $\zeta$. Si ha
  \begin{equation} \label{xifunctional}
    \xi(s)=\xi(1-s).
  \end{equation}
\end{thm}

\begin{proof}
  Se $\sigma>0$, allora
  \begin{gather*}
    \Gamma\left(\frac{s}{2}\right)=\int_0^{+\infty} e^{-x}x^{\frac{s}{2}}\frac{\diff x}{x} \implies \\
    \implies \frac{\pi^{-s/2}}{n^s}\Gamma\left(\frac{s}{2}\right)=\int_0^{+\infty} e^{-x}\left(\frac{x}{\pi n^2}\right)^{s/2}\frac{\diff x}{x}.
  \end{gather*}
  Se $\sigma>1$, allora
  \begin{gather*}
    \pi^{-s/2}\Gamma\left(\frac{s}{2}\right)\zeta(s)=\sum_{n=1}^{+\infty} \int_0^{+\infty} e^{-x}\left(\frac{x}{\pi n^2}\right)^{s/2}\frac{\diff x}{x} \overset{y=\frac{x}{\pi n^2}}{=} \\
    =\sum_{n=1}^{+\infty} \int_0^{+\infty} e^{-\pi n^2 y}y^{s/2}\frac{\diff y}{y}=\int_0^{+\infty} \sum_{n=1}^{+\infty} e^{-\pi n^2y}y^{s/2}\frac{\diff y}{y}= \\
    =\frac{1}{2}\int_0^{+\infty} \big(\vartheta(y)-1\big)y^{s/2}\frac{\diff y}{y}=\frac{1}{2}\left(\int_0^1+\int_1^{+\infty}\right)\big(\vartheta(y)-1\big)y^{s/2}\frac{\diff y}{y}.
  \end{gather*}
  Abbiamo che
  \begin{gather*}
    \frac{1}{2} \int_0^1 \big(\vartheta(y)-1\big)y^{s/2}\frac{\diff y}{y} \overset{x=\frac{1}{y}}{=} \frac{1}{2}\int_1^{+\infty} \Bigg(\vartheta\left(\frac{1}{x}\right)-1\Bigg)x^{-s/2}\frac{\diff x}{x}= \\
    =\frac{1}{2}\int_1^{+\infty} \big(\vartheta(x)-1\big)\sqrt{x}\cdot x^{-s/2}\frac{\diff x}{x}+\frac{1}{2}\int_1^{+\infty} x^{\frac{1-s}{2}}\frac{\diff x}{x}-\frac{1}{2}\int_1^{+\infty} x^{-\frac{s}{2}}\frac{\diff x}{x}= \\
    =\frac{1}{2}\int_1^{+\infty} \big(\vartheta(x)-1\big)\cdot x^{\frac{1-s}{2}}\frac{\diff x}{x}+\frac{1}{s-1}-\frac{1}{s}=\\
    =\frac{1}{2}\int_1^{+\infty} \big(\vartheta(x)-1\big)\cdot x^{\frac{1-s}{2}}\frac{\diff x}{x}+\frac{1}{s(s-1)}.
  \end{gather*}
  Si ha dunque
  \begin{gather*}
    \pi^{-s/2}\Gamma\left(\frac{s}{2}\right)\zeta(s)=\frac{1}{s(s-1)}+\frac{1}{2}\int_1^{+\infty} \big(\vartheta(x)-1\big)(x^{\frac{s}{2}}+x^{\frac{1-s}{2}})\frac{\diff x}{x} \implies \\
    \implies \xi(s)=\frac{s(s-1)}{2}\pi^{-s/2}\Gamma\left(\frac{s}{2}\right)\zeta(s)=\\
    =\frac{1}{2}+\frac{s(s-1)}{4}\int_1^{+\infty} \big(\vartheta(x)-1\big)(x^{\frac{s}{2}}+x^{\frac{1-s}{2}})\frac{\diff x}{x}.
  \end{gather*}
  Poiché
  $$\frac{1}{2}\big(\vartheta(x)-1\big)=\sum_{n=1}^{+\infty} e^{-\pi n^2x} \le \sum_{n=1}^{+\infty} e^{-\pi nx}=\frac{1}{e^{\pi x}-1} \ll e^{-\pi x},$$
  l'ultimo integrale nella formula per $\xi(s)$ converge uniformemente in ogni striscia $a \le \sigma \le b$; allora $\xi$ definita da quell'integrale è una funzione intera e la relazione con $\Gamma$ e $\zeta$ data nell'enunciato è valida per $\sigma>1$. Che $\xi(1-s)=\xi(s)$ è ovvio. Poiché $\Gamma$ non ha zeri e lo zero semplice di $s/2$ in $0$ è cancellato dal polo di $\Gamma$, questo definisce l'estensione analitica di $\zeta$ a $\mathbb{C}\setminus\{1\}$. Sappiamo già che sui reali $\displaystyle \lim_{s \longrightarrow +\infty} \zeta(s)=1$. Mostriamo che è un polo semplice di residuo $1$. Si ha
  \begin{gather*}
    \zeta(s)=\xi(s)\frac{\pi^{s/2}}{\frac{s}{2}\Gamma\left(\frac{s}{2}\right)}\frac{1}{s-1} \text{ e} \\
    \underset{s=1}{\text{Res}}\,\zeta=\lim_{s \longrightarrow 1} \big(\zeta(s)(s-1)\big)=\xi(1)\frac{\pi^{1/2}}{\frac{1}{2}\Gamma\left(\frac{1}{2}\right)}=\frac{1}{2}\cdot\frac{\sqrt{\pi}}{\frac{1}{2}\sqrt{\pi}}=1.
  \end{gather*}
\end{proof}

\begin{oss}
  $\xi$ non ha zeri fuori dalla striscia $0 \le \sigma \le 1$.
\end{oss}

\begin{cor}
  $\zeta(-2k)=0$ per ogni $k \ge 1$.
\end{cor}

\begin{proof}
  Guardare i poli di $\Gamma$.
\end{proof}

\begin{oss}
  All'interno della striscia  $0 \le \sigma \le 1$ abbiamo che $\zeta$ e $\xi$ si annullano insieme.
\end{oss}

\begin{oss}
  Se $\rho$ è uno zero, dalla \eqref{xifunctional} anche $1-\rho$ è uno zero. Poiché $\xi$ è reale sui reali, se $\rho$ è uno $0$ anche $\bar{\rho}$ è uno zero.
\end{oss}

\begin{oss}
  Definendo $\Xi(s)=\xi(is+1/2)$ si ha
  $$\Xi(-s)=\xi(1/2-is)=\xi(1/2+is)=\Xi(s).$$
  Se $s=x \in \mathbb{R}$, allora $\overline{\xi(1/2+ix)}=\xi(1/2-ix)=\xi(1/2+ix)$, quindi $\Xi(x)$ è reale.
\end{oss}

\begin{oss}
  $\zeta(s)\not=0$ per $\sigma>1$. Infatti
  \begin{gather*}
    \left|\zeta(s)\prod_{p \le N} \left(1-\frac{1}{p^s}\right)\right|=\left|\prod_{p>N} \left(1-\frac{1}{p^s}\right)^{-1}\right|= \\
    =\left|1+\sum_{p \mid n \implies p>N} \frac{1}{n^s}\right| \ge 1-\sum_{n>N} \frac{1}{n^{\sigma}}=1+O\left(\frac{1}{N^{\sigma-1}}\right).
  \end{gather*}
\end{oss}

\begin{oss}
  Prendiamo $0 \le \sigma \le 1$ e $t=0$. Si ha
  $$\xi(\sigma)=\frac{1}{2}+\frac{\sigma(\sigma-1)}{4}\int_1^{+\infty}\big(\vartheta(x)-1\big)(x^{\frac{\sigma}{2}}+x^{\frac{1-\sigma}{2}})\frac{\diff x}{x}.$$
  Ricordiamo che $\dfrac{\vartheta(x)-1}{2} \le \dfrac{1}{e^{\pi x}-1}$ e per $x \ge 1$ vale che $x \ge \sqrt{x}$, dunque $e^{\pi x}-1 \ge \pi x \ge 2\sqrt{x}$, perciò $\vartheta(x)-1 \le \dfrac{1}{\sqrt{x}}$. Inoltre
  $$\int_1^{+\infty} (x^{\frac{\sigma-1}{2}}+x^{\frac{-\sigma}{2}})\frac{\diff x}{x}=\frac{2}{\sigma(\sigma-1)}.$$
  Mettendo assieme, troviamo che
  $$\xi(\sigma) \ge \frac{1}{2}-\frac{\sigma(1-\sigma)}{4}\cdot\frac{2}{\sigma(1-\sigma)}=0.$$
\end{oss}

\begin{oss}
  $\xi(0)=1/2$ e $\displaystyle \lim_{s \longrightarrow 1} \frac{s}{2}\Gamma\left(\frac{s}{2}\right)=1$, dunque dalla formula che lega $\zeta$ e $\xi$ si ha $\zeta(0)=-1/2$.
\end{oss}

\begin{exc}
  Dimostrare che $\zeta(-1)=-1/12$.
\end{exc}


\subsection{Le funzioni intere $\xi(s)$ e $(s-1)\zeta(s)$}
Dovevamo capire qual è l'ordine di $\xi$ come funzione intera. Avevamo già controllato per $\sigma>1$, quindi ci manca $\sigma \ge 1/2$. Ricordiamo che $\xi(s)=\dfrac{s(s-1)}{2}\pi^{-\frac{s}{2}}\Gamma\left(\dfrac{s}{2}\right)\zeta(s)$. Per il lemma \ref{lls} abbiamo già $\zeta(s) \ll |s|$ per $\sigma \ge 1/2$; inoltre, ricordando il corollario \ref{gammaord}, abbiamo
\begin{gather*}
  \Gamma(s) \ll_{\epsilon} e^{|s|^{1+\epsilon}} \text{ per ogni } \epsilon>0, \quad \pi^{-\frac{s}{2}} \ll 1, \quad \frac{s(s-1)}{2} \ll |s^2| \implies \\
  \implies \xi(s) \ll_{\epsilon} e^{|s|^{1+\epsilon}} \text{ per }\sigma \ge 1/2.
\end{gather*}
Da $\xi(s)=\xi(1-s)$, otteniamo che è vero per ogni $\sigma$. Scriviamo $\xi$ con il prodotto di Weierstrass:
$$\xi(s)=e^{a+As}\prod_{\rho}\left(1-\frac{s}{\rho}\right)e^{\frac{s}{\rho}};$$
dev'essere $\displaystyle a=\log\big(\xi(0)\big)=\log\left(\frac{1}{2}\right) \implies \xi(s)=\frac{1}{2}e^{As}\prod_{\rho}\left(1-\frac{s}{\rho}\right)e^{\frac{s}{\rho}}$.

Consideriamo invece $(s-1)\zeta(s)=\xi(s)\dfrac{\pi^{\frac{s}{2}}}{\frac{s}{2}\Gamma\left(\frac{s}{2}\right)}$. Dall'equazione appena scritta otteniamo che ha ordine $1$, dunque si ha (ricordando anche gli zeri banali)
$$(s-1)\zeta(s)=e^{b+Bs}\prod_{\rho}\left(1-\frac{s}{\rho}\right)e^{\frac{s}{\rho}}\prod_{n=1}^{+\infty}\left(1+\frac{s}{2n}\right)e^{-\frac{s}{2n}};$$
dev'essere $\displaystyle b=\Bigg(\log\left(s-1)\zeta(s)\right)\Bigg)_{s=0}=\log\left(\frac{1}{2}\right) \implies$
$$\implies (s-1)\zeta(s)=\frac{1}{2}e^{Bs}\prod_{\rho}\left(1-\frac{s}{\rho}\right)e^{\frac{s}{\rho}}\prod_{n=1}^{+\infty}\left(1+\frac{s}{2n}\right)e^{-\frac{s}{2n}}.$$

Confrontando i prodotti di Weierstrass per $\zeta$ e $\xi$ usando l'equazione che lega le due funzioni, ricordando anche la definizione di $\Gamma$, si ha $B=A+\frac{1}{2}\log{\pi}+\frac{\gamma}{2}$. Andiamo a calcolarci $A$ e $B$.

$$A=\frac{\xi'}{\xi}(0)=2\xi'(0), \quad B=\left(\frac{1}{s-1}+\frac{\zeta'}{\zeta}(0)\right)_{s=0}=-2\zeta'(0)-1.$$
Si ha anche
\begin{gather*}
  \frac{B}{2}=\frac{A}{2}+\frac{1}{4}\log{\pi}+\frac{\gamma}{4} \implies \\
  \implies \xi'(0)+\zeta'(0)=-\frac{1}{2}-\frac{1}{4}\log{\pi}-\frac{\gamma}{4}.
\end{gather*}
Vogliamo calcolare $\xi'(0)$ usando l'equazione funzionale per $\xi$. Dalla definizione abbiamo
\begin{gather*}
  \xi'(s)=\frac{(s-1)\zeta(s)}{2}\pi^{-\frac{s}{2}}\Gamma\left(\frac{s}{2}\right)-\frac{1}{4}\log{\pi}\cdot s(s-1)\zeta(s)\Gamma\left(\frac{s}{2}\right)\pi^{-\frac{s}{2}}+ \\
  +\frac{s}{4}\pi^{-\frac{s}{2}}\Gamma'\left(\frac{s}{2}\right)(s-1)\zeta(s)+\frac{s}{2}\pi^{-\frac{s}{2}}\Gamma\left(\frac{s}{2}\right)D\big(\zeta(s)(s-1)\big)
\end{gather*}

Nella dimostrazione del lemma \ref{lls} abbiamo visto che
$$\zeta(s)=\frac{1}{s-1}+1-s\int_1^{+\infty} \frac{\{u\}}{u^{s+1}}\diff u \implies \lim_{s \longrightarrow 1} \left(\zeta(s)-\frac{1}{s-1}\right)=\gamma,$$
dunque dev'essere $\zeta(s)=\dfrac{1}{s-1}+\gamma+(s-1)g(s)$, con $g$ una qualche funzione intera. Allora otteniamo
$$\xi'(1)=\frac{\Gamma\left(\frac{1}{2}\right)}{2\sqrt{\pi}}-\frac{1}{4}\log{\pi}\frac{\Gamma\left(\frac{1}{2}\right)}{\sqrt{\pi}}+\frac{\Gamma'\left(\frac{1}{2}\right)}{4\sqrt{\pi}}+\frac{\Gamma\left(\frac{1}{2}\right)}{2\sqrt{\pi}}\gamma;$$
dobbiamo calcolare $\Gamma'\left(\frac{1}{2}\right)$. Dalla proposizione \ref{Gammagamma} abbiamo
\begin{gather*}
  \frac{\Gamma'}{\Gamma}(z)=-\gamma-\frac{1}{z}+\sum_{n=1}^{+\infty} \frac{z}{n(n+z)} \implies \\
  \implies \frac{\Gamma'}{\Gamma}\left(\frac{1}{2}\right)=-\gamma-2+\sum_{n=1}^{+\infty} \frac{2}{2n(2n+1)}=-\gamma-2+2\sum_{n=1}^{+\infty} \left(\frac{1}{2n}-\frac{1}{2n+1}\right)= \\
  =-\gamma-2+2\left(\frac{1}{2}-\frac{1}{3}+\frac{1}{4}-\frac{1}{5}+\dots\right)=-\gamma-2+2-2\log{2}=-\gamma-2\log{2} \implies \\
  \implies \Gamma'\left(\frac{1}{2}\right)=-\sqrt{\pi}(\gamma+2\log{2}),
\end{gather*}
quindi
$$\xi'(1)=\frac{1}{2}-\frac{1}{4}\log{\pi}-\frac{\gamma}{4}-\frac{1}{2}\log{2}+\frac{\gamma}{2}=\frac{\gamma}{4}+\frac{1}{2}-\frac{1}{4}\log(4\pi).$$

Adesso usiamo l'equazione funzionale:
\begin{gather*}
  \xi(1-s)=\xi(s) \implies -\xi'(1-s)=\xi'(s) \implies \\
  \implies \xi'(0)=-\xi'(1)=\frac{1}{4}\log(4\pi)-\frac{\gamma}{4}-\frac{1}{2} \text{ e}\\
  \zeta'(0)=\frac{\gamma}{4}+\frac{1}{2}-\frac{1}{4}\log(4\pi)-\frac{1}{2}-\frac{\gamma}{4}-\frac{1}{4}\log{\pi}=\\
  =-\frac{1}{2}\log{\pi}-\frac{1}{2}\log{2}=-\frac{1}{2}\log(2\pi);
\end{gather*}
abbiamo quindi
$$B=-2\zeta'(0)-1=\log(2\pi)-1 \text{ e }A=2\xi'(0)=\frac{1}{2}\log(4\pi)-1-\frac{\gamma}{2}.$$

Ora, vogliamo dire che non ci sono zeri per $\sigma=1$ (e dalla funzionale, nemmeno per $\sigma=0$). Dal prodotto di Weierstrass troviamo
\begin{gather*}
  \frac{\xi'}{\xi}(s)=A+\sum_{\rho} \left(\frac{1}{s-\rho}+\frac{1}{\rho}\right)\text{ e} \\
  \frac{\zeta'}{\zeta}(s)=-\frac{1}{s-1}-\frac{1}{2}\cdot\frac{\Gamma'}{\Gamma}\left(\frac{s}{2}+1\right)+\frac{1}{2}\log{\pi}+\frac{\xi'}{\xi}(s)=\\
  =-\frac{1}{s-1}+A+\frac{1}{2}\log{\pi}+\sum_{\rho} \left(\frac{1}{s-\rho}+\frac{1}{\rho}\right)-\frac{1}{2}\cdot\frac{\Gamma'}{\Gamma}\left(\frac{s}{2}+1\right).
\end{gather*}
Supponiamo per assurdo che ci sia uno zero in $1+it$. Ci tornerà utile la seguente disuguaglianza, valida per ogni $\theta$ reale: $3+4\cos{\theta}+\cos(2\theta) \ge 0$. Infatti, la si riscrive come $2\cos^2{\theta}+4\cos{\theta}+2=2(\cos{\theta}+1)^2$. Sia ora $\sigma>1$, abbiamo
$$\log\big(\zeta(s)\big)=\log\Bigg(\prod_p\left(1-\frac{1}{p^s}\right)^{-1}\Bigg)=-\sum_p \log\left(1-\frac{1}{p^s}\right)=\sum_p\sum_{m=1}^{+\infty} \frac{1}{mp^{ms}},$$
da cui prendendo la parte reale
$$\log|\zeta(\sigma+it)|=\sum_p \sum_{m=1}^{+\infty} \frac{1}{mp^{\sigma m}}\cos(mt\log{p}).$$
Adesso, sfruttando la disuguaglianza vista sopra con $\theta=mt\log{p}$ al variare di $m$ e $p$, otteniamo
\begin{gather*}
  3\log|\zeta(\sigma)|+4\log|\zeta(\sigma+it)|+\log|\zeta(\sigma+i2t)|= \\
  =\sum_p \sum_{m=1}^{+\infty} \frac{1}{mp^{m\sigma}}\big(3+4\cos(mt\log{p})+\cos(2mt\log{p})\big) \ge 0 \implies \\
  \implies \zeta^3(\sigma)|\zeta(\sigma+it)|^4||\zeta(\sigma+i2t)| \ge 1.
\end{gather*}
Adesso, $\zeta$ ha un polo semplice in $1$, perciò $\zeta(\sigma)=\dfrac{1}{\sigma-1}+g(\sigma)$ con $g$ una qualche funzione intera; invece, se ci fosse uno zero in $1+it$, avremmo
$$\zeta(\sigma+it)=(\sigma+it-1-it)h(\sigma+it)=(\sigma-1)h(\sigma+it),$$
con $h$ una qualche funzione olomorfa in un intorno di $1+it$. Ma allora, poiché $4>3$, si ha
$$\lim_{\sigma \longrightarrow 1} \zeta^3(\sigma)\zeta(\sigma+it)^4\zeta(\sigma+i2t)=0,$$
in contraddizione con la disuguaglianza trovata.


\subsection{La funzione $\psi$ e il teorema dei numeri primi}
Riemann congetturò che
$$\psi(x)=\sum_{n \le x} \Lambda(n)=x-\sum_{\rho} \frac{x^{\rho}}{\rho}-\frac{\zeta'}{\zeta}(0)-\frac{1}{2}\log\left(1-\frac{1}{x^2}\right).$$
Beh, non proprio: l'espressione a destra è continua, a differenza di $\psi$. Utilizzeremo $\psi_0(x)=\displaystyle \sum_{n<x} \Lambda(n)+\frac{1}{2}\Lambda(x)$, dove poniamo $\Lambda=0$ fuori dagli interi. L'idea di Riemann è di scrivere
$$\psi_0(x) \sim \frac{1}{2\pi i} \int_{c-i\infty}^{c+i\infty} -\frac{\zeta'}{\zeta}(s)\frac{x^s}{s}\diff x$$
con $c>1$ e applicare il teorema dei residui. Ma come ha fatto a derivare un'espressione tanto precisa? Ha guardato i poli dell'integranda.

\begin{lm}
  Dato $c>0$, sia $I(y,T)=\displaystyle \frac{1}{2\pi i}\int_{c-iT}^{c+iT} y^s\frac{\diff s}{s}$; allora
  $$|I(y,T)-\delta(y)| \le \begin{cases}
    y^c\min\left\{1,\frac{1}{T|\log{y}|}\right\} & \mbox{se } y\not=1 \\
    \frac{c}{T} & \mbox{se } y=1,
\end{cases}$$
dove $\delta(y)=\begin{cases}
  1 & \mbox{se } y>1 \\
  \frac{1}{2} & \mbox{se } y=1 \\
  0 & \mbox{se } 0<y<1
\end{cases}$. Di conseguenza, $\displaystyle \frac{1}{2\pi i}\int_{c-i\infty}^{c+i\infty} y^s\frac{\diff s}{s}=\delta(y)$.
\end{lm}

\begin{proof}
  Preso $0<y<1$, consideriamo l'integrale sul cammino in figura, percorso in senso orario.
  \begin{center}
    \begin{tikzpicture}[line cap=round,line join=round,>=triangle 45,x=1.0cm,y=1.0cm]
      \draw[->,color=black] (-2,0) -- (6.86,0);
      \foreach \x in {-1,1,2,3,4,5,6}
      \draw[shift={(\x,0)},color=black] (0pt,2pt) -- (0pt,-2pt);
      \draw[->,color=black] (0,-3.94) -- (0,3.98);
      \foreach \y in {-3,-2,-1,1,2,3}
      \draw[shift={(0,\y)},color=black] (2pt,0pt) -- (-2pt,0pt);
      \clip(-2,-3.94) rectangle (6.86,3.98);
      \draw (0.54,-2.47)-- (0.54,2.47);
      \draw [domain=0.54:6.859999999999997] plot(\x,{(--10.36-0*\x)/4.2});
      \draw [domain=0.54:6.859999999999997] plot(\x,{(-10.7-0*\x)/4.34});
      \begin{scriptsize}
        \fill [color=black] (0.54,0) circle (1.5pt);
        \draw[color=black] (0.8,0.28) node {$c$};
        \fill [color=black] (0.54,2.47) circle (1.5pt);
        \draw[color=black] (0.56,2.78) node {$c+iT$};
        \fill [color=black] (0.54,-2.47) circle (1.5pt);
        \draw[color=black] (0.56,-2.78) node {$c-iT$};
      \end{scriptsize}
    \end{tikzpicture}
  \end{center}
  In teoria dovremmo fare un integrale chiuso e poi mandare un lato a infinito, ma visto che non ci sono poli per $y^s/s$ nella regione considerata e sul lato che si manda a infinito la funzione va a $0$ uniformemente, saltiamo il passaggio. Per il teorema dei residui
  $$\frac{1}{2\pi i}\int_{c-iT}^{c+iT} \frac{y^s}{s}\diff s=\frac{1}{2\pi i}\left(\int_{c-iT}^{+\infty-iT}-\int_{c+iT}^{+\infty+iT}\right)y^s\frac{\diff s}{s}=\mathcal{I}_1-\mathcal{I}_2.$$
  Si ha $\displaystyle |\mathcal{I}_1| \le \frac{1}{T} \int_c^{+\infty} y^{\sigma}\diff\sigma=\frac{1}{T}\cdot\frac{y^c}{|\log{y}|}$ e $\mathcal{I}_2$ si stima allo stesso modo.

  Per l'altro argomento del minimo, consideriamo la circonferenza di centro l'origine e raggio $R=\sqrt{c^2+T^2}$ e prendiamo il cammino da $c-iT$ a $c+iT$ e ritorno che gira in senso orario (quindi il segmento passante per $c$ all'andata e l'arco destro al ritorno). Sia $\gamma$ l'arco percorso al ritorno.
  \begin{center}
\begin{tikzpicture}[line cap=round,line join=round,>=triangle 45,x=1.0cm,y=1.0cm]
\draw[->,color=black] (-3.14,0) -- (3.52,0);
\foreach \x in {-3,-2,-1,1,2,3}
\draw[shift={(\x,0)},color=black] (0pt,2pt) -- (0pt,-2pt);
\draw[->,color=black] (0,-3.24) -- (0,3.34);
\foreach \y in {-3,-2,-1,1,2,3}
\draw[shift={(0,\y)},color=black] (2pt,0pt) -- (-2pt,0pt);
\clip(-3.14,-3.24) rectangle (3.52,3.34);
\draw(0,0) circle (2.49cm);
\draw (0,0)-- (0.8,2.36);
\draw (0.8,-2.36)-- (0.8,2.36);
\begin{scriptsize}
\fill [color=black] (0.8,0) circle (1.5pt);
\draw[color=black] (1.02,0.2) node {$c$};
\fill [color=black] (0,2.36) circle (1.5pt);
\draw[color=black] (-0.22,2.3) node {$T$};
\fill [color=black] (0.8,2.36) circle (1.5pt);
\draw[color=black] (0.96,2.62) node {$c+iT$};
\fill [color=black] (0.8,-2.36) circle (1.5pt);
\draw[color=black] (0.96,-2.56) node {$c-iT$};
\end{scriptsize}
\end{tikzpicture}
  \end{center}
  Allora per il teorema dei residui
  \begin{gather*}
    \left|\frac{1}{2\pi i}\int_{c-iT}^{c+iT} \frac{y^s}{s}\diff s\right|=\left|\frac{1}{2\pi i}\int_\gamma \frac{y^s}{s}\diff s \right| \le \\
    \frac{y^c}{2\pi} \int_{\gamma} \frac{\diff|s|}{|s|} \le y^c\frac{\pi R}{2\pi R} \le \frac{y^c}{2}.
  \end{gather*}
  Per $y>1$, basta ripetere le stesse stime ma con i cammini ``di sinistra'' percorsi in senso antiorario, facendo attenzione al polo in $0$: $\displaystyle \underset{s=0}{\text{Res}}\,\frac{y^s}{s}=1$.

  Adesso facciamo $y=1$ (non saremo rigorosi, ma si può sistemare facilmente):
  \begin{gather*}
    \frac{1}{2\pi i}\int_{c-iT}^{c+iT}\frac{\diff s}{s}=\frac{1}{2\pi i}\int_{-T}^T \frac{i\diff t}{c+it}=\frac{1}{2\pi}\int_{-T}^T \frac{c-it}{c^2+t^2}\diff t=\\
    =\frac{1}{2\pi}\int_{-T}^T \frac{c}{c^2+t^2}\diff t=\frac{1}{\pi}\int_0^T \frac{c}{c^2+t^2}\diff t\overset{x=t/c}{=}\frac{1}{\pi}\int_0^{T/c} \frac{\diff x}{x^2+1}= \\
    =\frac{1}{\pi}\left(\int_0^{+\infty}\frac{\diff x}{x^2+1}-\int_{T/c}^{+\infty}\frac{\diff x}{x^2+1}\right) \le \frac{1}{2}+\frac{c}{T}.
  \end{gather*}
\end{proof}

Prendiamo ora $y=\frac{x}{n}$ con $n \in \mathbb{N}$. Si ha
$$\frac{1}{2\pi i}\int_{c-iT}^{c+iT} \frac{\Lambda(n)}{n^s}x^s\frac{\diff s}{s}=\Lambda(n)\cdot \begin{cases}
  1+O\Bigg(\left(\dfrac{x}{n}\right)^c\min\left\{1,\dfrac{1}{T\left|\log\left(\frac{x}{n}\right)\right|}\right\}\Bigg) & \mbox{se }n < x \\
  \dfrac{1}{2}+O\left(\dfrac{c}{T}\right) & \mbox{se }n=x \text{ (e }n=p^a\text{)} \\
  O\Bigg(\left(\dfrac{x}{n}\right)^c\min\left\{1,\dfrac{1}{T\left|\log\left(\frac{x}{n}\right)\right|}\right\}\Bigg) & \mbox{se }n>x.
\end{cases}$$
Di conseguenza, per $c>1$ abbiamo
\begin{gather*}
  \sum_{n \le x} \Lambda(n)+\frac{\Lambda(x)}{2}=\frac{1}{2\pi i}\int_{c-iT}^{c+iT} \left(\sum_{n=1}^{+\infty}\frac{\Lambda(n)}{n^s}\right)\frac{x^s}{s}\diff s+\\
  +O\left(\sum_{n=1,n\not=x}^{+\infty} \frac{\Lambda(n)}{n^c}x^c\min\left\{1,\frac{1}{T\left|\log\left(\frac{x}{n}\right)\right|}\right\}+\frac{c\Lambda(x)}{T}\right).
\end{gather*}
Vogliamo dare una stima del resto, facendo un po' di casi. Prendiamo anche $c=1+\dfrac{1}{\log{x}}$, per avere $x^c=ex \ll x$. Resterà così.
\begin{enumerate}
  \item Partiamo dal più semplice: $\dfrac{c\Lambda(x)}{T} \ll \dfrac{\log{x}}{T}$.
  \item Ora facciamo la somma per $n \le \frac{3}{4}x$ o $n \ge \frac{5}{4}x$, per cui $\left|\log\left(\frac{x}{n}\right)\right| \gg 1$. Dobbiamo dunque stimare
  $$\left(\sum_{n \le \frac{3}{4}x}+\sum_{n \ge \frac{5}{4}x}\right)\frac{\Lambda(n)}{n^c}\cdot\frac{x^c}{T} \ll \frac{x}{T}\sum_{n=1}^{+\infty} \frac{\Lambda(n)}{n^c} \ll \frac{x\log{x}}{T},$$
  dove l'ultimo passaggio segue scrivendo la somma come $-\frac{\zeta'}{\zeta}(c)$ e usando che $\frac{\zeta'}{\zeta}(\sigma) \ll \frac{1}{\sigma-1}$.
  \item Per $\frac{3}{4}x<n<x$, sia $\frac{3}{4}x<x_1<x$ la massima tra le potenze di un primo in quell'intervallo. Allora
  \begin{gather*}
    \log\left(\frac{x}{x_1}\right)=-\log\left(\frac{x_1}{x}\right)=-\log\left(1-\frac{x-x_1}{x}\right)=\\
    =\frac{x-x_1}{x}+\frac{1}{2}\left(\frac{x-x_1}{x}\right)^2+\dots>\frac{x-x_1}{x},
  \end{gather*}
  dunque la stima del termine $n=x_1$ diventa
  $$\implies \Lambda(x_1)\left(\frac{x}{x_1}\right)^c\frac{x}{T(x-x_1)} \ll \frac{x\log{x}}{T(x-x_1)}.$$
  E $\frac{3}{4}x<n<x_1$? Abbiamo
  \begin{gather*}
    \log\left(\frac{x}{n}\right) \ge \log\left(\frac{x_1}{n}\right)=-\log\left(\frac{n}{x_1}\right)=\\
    =-\log\left(1-\frac{x_1-n}{x_1}\right)
    \ge \frac{x_1-n}{x_1}=\frac{\nu}{x_1}
  \end{gather*}
  e troviamo
  $$\sum_{1 \le \nu <x_1} \frac{\Lambda(x_1-\nu)x^cx_1}{(x_1-\nu)^cT\nu} \ll \frac{x(\log{x})^2}{T}.$$
  Il caso $x<n<\frac{5}{4}x$ è analogo, prendendo $x_2$ la minima potenza di primo.
\end{enumerate}

Ora, definendo $\langle x\rangle$ come la distanza di $x$ dalla potenza di primo più vicina, mettendo assieme abbiamo trovato che
$$\psi_0(x)=\frac{1}{2\pi i}\int_{c-iT}^{c+iT} -\frac{\zeta'}{\zeta}(s)\frac{x^s}{s}\diff s+O\left(\frac{x\log^2{x}}{T}+\frac{x\log{x}}{T\langle x\rangle}\right);$$
dunque
$$\psi_0(x)=\frac{1}{2\pi i}\int_{c-iT}^{c+iT} -\frac{\zeta'}{\zeta}(s)\frac{x^s}{s}\diff s+R(x,T),$$
dove $R(x,T) \ll \dfrac{x\log^2{x}}{T}$ per $x \in \mathbb{N}$. Consideriamo ora il cammino in figura.
\begin{center}
\begin{tikzpicture}[line cap=round,line join=round,>=triangle 45,x=1.0cm,y=1.0cm]
\draw[->,color=black] (-3.36,0) -- (2.27,0);
\foreach \x in {-3,-2,-1,1,2}
\draw[shift={(\x,0)},color=black] (0pt,2pt) -- (0pt,-2pt);
\draw[->,color=black] (0,-2.01) -- (0,2.23);
\foreach \y in {-2,-1,1,2}
\draw[shift={(0,\y)},color=black] (2pt,0pt) -- (-2pt,0pt);
\clip(-3.36,-2.01) rectangle (2.27,2.23);
\draw (-2.13,-1.25)-- (1.2,-1.25);
\draw (1.2,-1.25)-- (1.2,1.25);
\draw (1.2,1.25)-- (-2.13,1.25);
\draw (-2.13,1.25)-- (-2.13,-1.25);
\draw [dash pattern=on 2pt off 2pt] (0.5,-1.25)-- (0.5,1.25);
\begin{scriptsize}
\fill [color=black] (1,0) circle (1.5pt);
\draw[color=black] (1,0.18) node {$1$};
\fill [color=black] (1.2,0) circle (1.5pt);
\draw[color=black] (1.3,0.12) node {$c$};
\fill [color=black] (0,1.25) circle (1.5pt);
\draw[color=black] (0.2,1.4) node {$T$};
\fill [color=black] (0,-1.25) circle (1.5pt);
\draw[color=black] (0.26,-1.4) node {$-T$};
\fill [color=black] (-2.13,0) circle (1.5pt);
\draw[color=black] (-2.4,0.12) node {$-U$};
\end{scriptsize}
\end{tikzpicture}
\end{center}
Prendiamo $T\not=\gamma$ per non passare dagli zeri non banali, allora usando ancora una volta il teorema dei residui si ha
\begin{gather*}
  \psi_0(x)=x-\sum_{|\gamma|<T} \frac{x^{\rho}}{\rho}-\frac{\zeta'}{\zeta}(0)-\sum_{1 \le n \le \frac{U}{2}} \frac{x^{-2n}}{-2n}+\\
  +\frac{1}{2\pi i}\left(\int_{-U+iT}^{c+iT}-\int_{-U-iT}^{c-iT}+\int_{-U-iT}^{-U+iT}\right)\left(-\frac{\zeta'}{\zeta}(s)\frac{x^s}{s}\diff s\right)+R(x,T).
\end{gather*}
Per il corollario della formula di Riemann-Von Mangoldt abbiamo
$$N(T+1)-N(T-1) \le C_0\log{T}.$$
Suddividiamo l'intervallo $[T-1,T+1]$ in $2C_0\log{T}$ intervalli, di modo che ce ne sia sempre uno senza zeri. Allora possiamo scegliere $T$, variandolo al più di una quantità minore o uguale a $1$, t.c. $|\gamma-T| \gg \dfrac{1}{\log{T}}$. Usando allora il lemma \ref{zprimoz} otteniamo
\begin{gather*}
  \frac{\zeta'}{\zeta}(\sigma+iT) \ll \sum_{|\gamma-T| \le 1} \frac{1}{|s-\rho|}+O(\log{T}) \ll \log{T}\sum_{|\gamma-T| \le 1} 1 \ll \log^2{T} \implies \\
  \implies \int_{-1 \pm +iT}^{c \pm iT} -\frac{\zeta'}{\zeta}(s)\frac{x^s}{s}\diff s \ll \frac{x^c\log^2{T}}{T} \ll \frac{x\log^2{T}}{T}.
\end{gather*}

\begin{oss} \label{zeeeta}
  Se $\sigma \le -1$ e $|s+2n| \ge 1/2$ per ogni $n \in \mathbb{N}$, come vedremo in seguito si ha
  \begin{gather*}
    \frac{\zeta'}{\zeta}(s) \ll \log(2|s|) \implies \\
    \implies \log(2|U\pm iT|)x^{-U}\int_0^T \frac{\diff t}{U+t} \ll \frac{\log(2|U\pm iT|)}{x^U}\log{T} \underset{U \longrightarrow +\infty}{\longrightarrow} 0;
  \end{gather*}
  questa era la parte ``verticale'' dell'integrale di $\psi_0$, con stime fatte alla buona. Mandando $U$ a infinito, rimangono da stimare i pezzi ``orizzontali'' da $-1$ a $-\infty$. Viene la seguente stima trascurabile:
  $$\frac{1}{T}\int_{-\infty}^{-1} \log(2|\sigma+iT|)x^{\sigma}\diff \sigma \ll \frac{x^{-1}\log{T}}{T\log{x}}.$$
  Anche questa stima è fatta alla buona. Si veda \cite{D} per una dimostrazione più precisa.
\end{oss}

Mettendo assieme quanto visto finora, otteniamo la seguente proposizione.

\begin{prop}
  \begin{equation*}
    \psi_0(x)=x-\sum_{|\gamma| < T} \frac{x^{\rho}}{\rho}-\frac{\zeta'}{\zeta}(0)-\frac{1}{2}\log\left(1-\frac{1}{x^2}\right)+R(x,T),
  \end{equation*}
  dove $R(x,T) \ll \dfrac{x\log^2(xT)}{T}$ per $x \in \mathbb{N}$ e $T \ge 2$.
\end{prop}

Bisogna stare attenti agli zeri mancanti (perché abbiamo cambiato un po' $T$, quindi dobbiamo sistemare), che però sono in un intervallo limitato, perciò sempre per il corollario di Riemann-Von Mangoldt un $O(\log{T})$. Allora il loro contributo è $O$-grande di:
$$\frac{x^{\rho}}{\rho}\log{T} \ll \frac{x^{\beta}\log{T}}{T} \ll \frac{x\log{T}}{T}.$$

Dimostriamo adesso l'asserzione usata all'inizio dell'osservazione \ref{zeeeta}, e per farlo passeremo dall'equazione funzionale per la $\zeta$. Si ha
\begin{gather*}
  \frac{s(s-1)}{2}\pi^{-\frac{s}{2}}\Gamma\left(\frac{s}{2}\right)\zeta(s)=\xi(s)= \\
  =\xi(1-s)=\frac{s(s-1)}{2}\pi^{\frac{s-1}{2}}\Gamma\left(\frac{1-s}{2}\right)\zeta(1-s) \implies \\
  \implies \zeta(1-s)=\pi^{\frac{1}{2}-s}\frac{\Gamma\left(\frac{s}{2}\right)}{\Gamma\left(\frac{1-s}{2}\right)}\zeta(s).
\end{gather*}
Usando la proposizione \ref{gammaze1-z} e la formula di duplicazione di Legendre, otteniamo
\begin{gather*}
  \frac{\Gamma\left(\frac{s}{2}\right)}{\Gamma\left(\frac{1-s}{2}\right)}=\frac{\Gamma\left(\frac{s}{2}\right)\Gamma\left(\frac{s}{2}+\frac{1}{2}\right)}{\Gamma\left(\frac{1-s}{2}\right)\Gamma\left(\frac{1+s}{2}\right)}= \\
  =\frac{\sqrt{\pi}2^{1-s}\Gamma(s)}{\pi/\sin\left(\frac{\pi}{2}-\frac{\pi s}{2}\right)}=2^{1-s}\pi^{-1/2}\cos\left(\frac{\pi s}{2}\right)\Gamma(s);
\end{gather*}
si ottiene dunque l'equazione funzionale per $\zeta$:
\begin{equation} \label{eqfunzeta}
  \zeta(1-s)=2^{1-s}\pi^{-s}\cos\left(\frac{\pi}{2}s\right)\Gamma(s)\zeta(s).
\end{equation}
Prendendo la derivata logaritmica si ha
$$\frac{\zeta'}{\zeta}(1-s)=\log{2}+\log{\pi}+\frac{\pi}{2}\tan\left(\frac{\pi}{2}s\right)-\frac{\Gamma'}{\Gamma}(s)-\frac{\zeta'}{\zeta}(s).$$
Per $\sigma \ge 2$, cioè $1-\sigma \le -1$, si ha
$$\frac{\zeta'}{\zeta}(s)=-\sum_{n=1}^{+\infty} \frac{\Lambda(n)}{n^s} \ll 1;$$
inoltre
\begin{gather*}
  \left|\tan\left(\frac{\pi}{2}s\right)\right|=\left|\frac{e^{i\frac{\pi}{2}s}-e^{-i\frac{\pi}{2}s}}{i(e^{i\frac{\pi}{2}s}+e^{-i\frac{\pi}{2}s})}\right|=\left|\frac{e^{\pi is}-1}{e^{\pi is}+1}\right| \le \\
  \le \frac{e^{\pi t}+1}{e^{\pi t}-1}=1+\frac{2}{e^{\pi t}-1} \le 3,
\end{gather*}
dove l'ultima disuguaglianza vale per $t \ge 1/2$, preso fuori da un intorno dei punti $-2n$. Infine, per il corollario \ref{gammaprimosu} abbiamo
$$\frac{\Gamma'}{\Gamma}(s) \ll \log{|s|}+O\left(\frac{1}{s}\right).$$
Allora, per $\sigma \le -1 \implies |s| \ge 1$, possiamo riarrangiare quanto trovato per ottenere
$$\frac{\zeta'}{\zeta}(s) \ll \log(2|1-s|) \le \log(2|s|),$$
in quanto $|1-s| \le 1+|s| \le 2|s|$.

Torniamo a $\psi_0$.
\begin{gather*}
  \psi_0(x)=\sum_{n<x} \Lambda(n)+\frac{1}{2}\Lambda(x) \implies \\
  \implies \psi_0(x)=x-\sum_{|\gamma|<T} \frac{x^{\rho}}{\rho}+O\left(\frac{x\log^2(xT)}{T}+\log{x}\cdot\min\left\{1,\frac{x}{T\langle x\rangle}\right\}\right).
\end{gather*}
Per de la Vallée-Poussin si ha che esiste $c_0>0$ t.c.
\begin{gather*}
  \beta<1-\frac{c_0}{\log{T}} \text{ per ogni } \rho=\beta+i\gamma \text{ con } |\gamma|<T \implies \\
  \implies \left|\sum_{|\gamma|<T} \frac{x^{\rho}}{\rho}\right| \le \max_{|\gamma|<T} x^{\beta} \sum_{|\gamma|<T} \frac{1}{|\rho|} \le x\exp\left(-\frac{c_0\log{x}}{\log{T}}\right)\sum_{|\gamma|<T} \frac{1}{|\rho|}.
\end{gather*}
Adesso, sommando per parti e usando Riemann-Von Mangoldt si ha
\begin{gather*}
  \sum_{|\gamma|<T} \frac{1}{|\rho|} \ll \sum_{2 \le \gamma \le T} \frac{1}{\gamma}=\left(\sum_{2 \le \gamma \le T}1\right)\frac{1}{T}-\int_2^T \left(\sum_{2 \le \gamma \le u}1\right)\frac{\diff u}{u^2}= \\
  =\frac{N(T)}{T}+\int_2^T \frac{N(u)}{u^2}\diff u \ll \frac{T\log{T}}{T}+\int_2^T \frac{\log{u}}{u}\diff u \ll \\
  \ll \log{T}+\log^2{T} \ll \log^2{T}.
\end{gather*}
Otteniamo dunque
\begin{gather*}
  \sum_{|\gamma|<T} \frac{x^{\rho}}{\rho} \ll x\log^2{T}\exp\left(-\frac{c_0\log{x}}{\log{T}}\right) \implies \\
  \implies \psi_0(x)-x \ll x\Bigg(\log^2{T}\exp\left(-\frac{c_0\log{x}}{\log{T}}\right)+\frac{\log^2(xT)}{T}\Bigg).
\end{gather*}
Prendendo $T=\exp(\sqrt{\log{x}}) \implies \log{x}=\log^2{T}$, troviamo che esiste $0<c_1<1$ t.c.
$$\psi_0(x)-x \ll x\exp(-c_1\sqrt{\log{x}}).$$
Questo è un risultato importante.

\begin{thm} \label{PNT}
  (Prime Number Theorem) esiste $0<c_1<1$ t.c.
  \begin{equation} \label{pnt}
    \psi(x)=x+O\big(x\exp(-c_1\sqrt{\log{x}})\big).
  \end{equation}
\end{thm}

\begin{oss}
  Littlewood, 1922: $\psi(x)=x+O\big(x\exp(-c_1\sqrt{\log{x}\log\log{x}})\big)$;
  Vinogradov-Korobov, 1958: $\psi(x)=x+O_{\epsilon}\Big(x\exp\big(-c_1(\log{x})^{3/5-\epsilon}\big)\Big)$. Questi seguono dai miglioramenti a de la Vallée-Poussin con dimostrazione analoga a quella del PNT.

  Allo stesso modo, l'Ipotesi di Riemann ci dà $\psi(x)=x+O(x^{1/2}\log^2{x})$. Infatti
  $$\left|\sum_{|\gamma|<T} \frac{x^{\rho}}{\rho}\right| \ll \sqrt{x}\sum_{|\gamma|<T} \frac{1}{|\gamma|} \ll \sqrt{x}\log^2{T},$$
  poi prendendo $T=\sqrt{x}$ si procede come nella dimostrazione del PNT.
\end{oss}

C'è anche la Quasi Ipotesi di Riemann (QRH): detto $\displaystyle \theta=\sup_{\rho} \beta$, allora si congettura che $1/2<\theta<1$.

\begin{exc}
  QRH $\implies \psi(x)=x+O(x^{\theta}\log^2{x})$.
\end{exc}

Congettura di Montgomery: il resto è $x^{1/2}(\log\log{x})^2$. Non possiamo portarlo a $\sqrt{x}$, o per meglio dire sarebbe utopistico: si sa che $\psi(x)-x=\Omega_{\pm}(\sqrt{x})$.

\begin{oss}
  Se $\psi(x)=x+O_{\epsilon}(x^{\theta+\epsilon})$ per ogni $\epsilon>0$, allora $\beta \le \theta$ per ogni $\rho$. Infatti, per $\sigma>1$ si ha per sommazione parziale
  \begin{gather*}
    \sum_{n \le N} \frac{\Lambda(n)}{n^s}=\frac{\psi(N)}{N^s}+s\int_2^N \frac{\psi(u)}{u^{s+1}}\diff u \implies \text{[}R(u)=O_{\epsilon}(u^{\theta+\epsilon})\text{]}\\
    \implies -\frac{\zeta'}{\zeta}(s)=s\int_2^{+\infty} \frac{\psi(u)}{u^{s+1}}\diff u=s\int_2^{+\infty} u^{-s}\diff u+s\int_2^{+\infty} \frac{R(u)}{u^{s+1}}\diff u= \\
    =
  \end{gather*}
\end{oss}


\subsection{La trasformata di Mellin e alcune conseguenze}
\begin{defn}
  Sia $f:\mathbb{R}^+ \longrightarrow \mathbb{R}$ t.c. $\displaystyle \int_0^{+\infty} |f(x)|x^{\sigma}\frac{\diff x}{x}<+\infty$ per $\sigma \in \mathbb{R}$ fissato. Si dice \textit{trasformata di Mellin} di $f$ la seguente:
  $$\wideparen{f}(s)=\int_0^{+\infty} f(x)x^s\frac{\diff x}{x}\text{ con }s=\sigma+it.$$
\end{defn}

\begin{ex}
  Se $f(x)=e^{-x}$ e $\sigma>0$, allora $\wideparen{f}(s)=\Gamma(s)$.
\end{ex}

\begin{oss}
  Sia $x=e^{-2\pi u} \implies \frac{\diff x}{x}=-2\pi\diff u$. Allora
  \begin{gather*}
    \wideparen{f}(\sigma+it)=2\pi\int_{-\infty}^{+\infty} f(e^{-2\pi u})e^{-2\pi u\sigma-2\pi itu}\diff u= \\
    =2\pi \int_{-\infty}^{+\infty} \phi_{\sigma}(u)e^{-2\pi itu}\diff u=2\pi\hat{\phi}_{\sigma}(t).
  \end{gather*}
\end{oss}

\begin{prop}
  Se $f$ è di classe $C^1$ e $\displaystyle \int_0^{+\infty} |f(x)|x^{\sigma}\frac{\diff x}{x}<+\infty$, si ha
  $$f(x)=\frac{1}{2\pi i} \int_{\sigma-i\infty}^{\sigma+i\infty} \wideparen{f}(s)x^{-s}\diff s.$$
\end{prop}

\begin{proof}
  Da scrivere.
\end{proof}


\newpage

\section{Titolo da decidere}

\subsection{I caratteri e il teorema di Dirichlet}
\begin{defn}
  Sia $q>1$ un intero. I \textit{caratteri} modulo $q$ sono i caratteri del gruppo abeliano $(\mathbb{Z}/q \mathbb{Z})^*$, definiti come stiamo per vedere.
\end{defn}

I caratteri modulo $q$ sono un gruppo isomorfo a $(\mathbb{Z}/q \mathbb{Z})^*$, quindi sono in numero $\phi(q)$, e sono funzioni $\chi:\mathbb{Z} \longrightarrow \mathbb{C}$ (se si restringe il dominio a $(\mathbb{Z}/q \mathbb{Z})^*$, si può osservare, in base alla definizione esplicita, che i caratteri sono omomorfismi a valori nelle radici $\phi(q)$-esime dell'unità). C'è un carattere principale $\chi_0$, che assume solo i valori $0$ e $1$ in base a una regola che andiamo a vedere. I caratteri devono soddisfare:
\begin{itemize}
  \item $(\chi_1\cdot\chi_2)(n)=\chi_1(n)\cdot\chi_2(n)$;
  \item $\chi^{-1}(n)=\bar{\chi}(n)$;
  \item $\chi(nm)=\chi(n)\chi(m)$ per ogni $n,m \in \mathbb{Z}$ (sono completamente moltiplicativi);
  \item $\chi$ ha periodo $q$;
  \item $|\chi(n)|=1$ se $(n,q)=1$, altrimenti è $0$.
\end{itemize}

Vediamo ora la definizione esplicita, procedendo per casi.

Caso $p^a$ con $p>2$ primo. $(\mathbb{Z}/p^a \mathbb{Z})^*$ è ciclico, sia $g$ un generatore. Prendiamo $\omega$ t.c. $\omega^{\phi(p^a)}=1$ (radice $\phi(p^a)$-esima dell'unità). Dato $n$ intero, sia $\nu(n)$ t.c. $g^{\nu(n)}=[n]_{\mathbb{Z}/p^a \mathbb{Z}}$.
Il carattere corrispondente a $\omega$ è $\chi_{p^a}(n)=\omega^{\nu(n)}=e^{\frac{2\pi im\nu(n)}{\phi(p^a)}}$.

Il caso $(\mathbb{Z}/2 \mathbb{Z})^*$ è banale.

Caso $2^2$: $\phi(4)=2$; il carattere principale è banale, per l'altro:
$$\chi_4(n)=\begin{cases}
  1 &\mbox{se } n \equiv 1 \pmod{4} \\
  -1 &\mbox{se } n \equiv 3 \pmod{4} \\
  0 &\mbox{se } n \equiv 0 \pmod{2}
\end{cases}$$

Caso $2^a$ con $a \ge 3$: $(\mathbb{Z}/2^a \mathbb{Z})^* \cong \mathbb{Z}/2 \mathbb{Z} \times \mathbb{Z}/2^{a-2} \mathbb{Z}$.
Sia $g$ un generatore della parte $\mathbb{Z}/2^{a-2} \mathbb{Z}$ e $\nu_0'$ t.c. $g^{\nu_0'} \equiv [n]$. Allora $n \equiv (-1)^{\nu_0}5^{\nu_0'} \pmod{2^a}$ e poniamo $\chi(n)=e^{\pi im_0\nu_0+\frac{2\pi im_0'\nu_0'}{2^{a-2}}}$.

Caso generale $q=2^a\cdot p_1^{a_1}\cdots p_r^{a_r}$:
$$\chi_q(n)=e^{\pi im_0\nu_0+\frac{2\pi im_0'\nu_0'}{2^{a-2}}+\frac{2\pi im_1\nu_1}{\phi(p_1^{a_1})}+\dots+\frac{2\pi im_r\nu_r}{\phi(p_r^{a_r})}}.$$
Alla scelta degli $m_j$ corrisponde il carattere, mentre la dipendenza da $n$ è data dai $\nu_j$.

Per esercizio, il lettore può dimostrare le seguenti formule di ortogonalità:
\begin{itemize}
  \item $\displaystyle \sum_{n=1}^q \chi(n)=\begin{cases}
    0 &\mbox{se }\chi\not=\chi_0 \\
    \phi(q) &\mbox{se }\chi=\chi_0
  \end{cases}$;
  \item $\displaystyle \sum_{\chi\text{ mod }q} \chi(n)=\begin{cases}
    0 &\mbox{se }n\not\equiv 1\pmod{q} \\
    \phi(q) &\mbox{se }n\equiv 1\pmod{q}
  \end{cases}$.
\end{itemize}
Osserviamo che $\bar{\chi}(a)=\chi^{-1}(a)=\chi(a^{-1})$. Dalla seguente formula di ortogonalità segue il seguente risultato: se $(a,q)=1$, allora
$$\frac{1}{\phi(q)}\sum_{\chi\text{ mod }q}\bar{\chi}(a)\chi(n)=\begin{cases}
  1 &\mbox{se }n\equiv a \pmod{q} \\
  0 &\mbox{altrimenti}
\end{cases};$$
questo permette in un qualche senso di isolare i numeri congrui ad $a$ modulo $q$, per poter così dimostrare il teorema di Dirichlet.

Andiamo adesso a definire una classe di funzioni molto importanti, non solo perché ci serviranno nella dimostrazione del suddetto teorema, ma anche perché sono una specie di generalizzazione della $\zeta$ di Riemann (se vogliamo, la $\zeta$ corrisponde alla funzione definita a partire dai ``caratteri modulo $1$'', anche se ovviamente quest'affermazione a poco o nessun senso).

\begin{defn}
  Sia $\chi$ un carattere non principale modulo $q$. Si dice funzione $L$ relativa al carattere $\chi$ la somma della seguente serie, definita per $\sigma>1$:
  $$L(s,\chi)=\sum_{n=1}^{+\infty} \frac{\chi(n)}{n^s}.$$
\end{defn}

\begin{prop}
  Se $\chi$ non è principale, la serie appena definita converge uniformemente per $\sigma \ge \epsilon>0$.
\end{prop}

\begin{proof}
  Usando la solita sommazione per parti,
  $$\sum_{n \le N} \frac{\chi(n)}{n^s}=\left(\sum_{n \le N} \chi(n)\right)N^{-s}+s\int_1^N\left(\sum_{n \le u} \chi(u)\right)\frac{\diff u}{u^{s+1}}.$$
  Dalla prima formula di ortogonalità segue che $\displaystyle \left|\sum_{n \le N} \chi(n)\right| \le q$, dunque per $\sigma>1$ si ha
  $$L(s,\chi)=s\int_1^{+\infty} \left(\sum_{n \le u}\chi(u)\right)\frac{\diff u}{u^{s+1}},$$
  e questo integrale converge uniformemente in $\sigma \ge \epsilon>0$.
\end{proof}

\begin{oss}
  Se prendiamo $\chi=\chi_0$ e consideriamo la serie associata, abbiamo $\displaystyle \sum_{n \le N} \chi_0(n) \le  \left(\left\lfloor \frac{N}{q}\right\rfloor+1\right)\phi(q)$, dunque non si ottiene convergenza uniforme per $\sigma \ge \epsilon>0$.
\end{oss}

\begin{oss}
  Per l'identità di Eulero si ha, per $\sigma>1$ e per ogni carattere modulo $q$:
  $$L(s,\chi)=\prod_p\left(1-\frac{\chi(p)}{p^s}\right)^{-1}.$$
  In particolare,
  \begin{gather*}
    L(s,\chi_0)=\prod_p\left(1-\frac{\chi_0(p)}{p^s}\right)^{-1}=\prod_{p \nmid q}\left(1-\frac{1}{p^s}\right)^{-1}= \\
    =\prod_p \left(1-\frac{1}{p^s}\right)^{-1}\prod_{p\mid q}\left(1-\frac{1}{p^s}\right)=\prod_{p\mid q}\left(1-\frac{1}{p^s}\right)\zeta(s),
  \end{gather*}
  da cui si ottiene che $L(s,\chi_0)$ è meromorfa con un polo semplice in $s=1$ con residuo $\displaystyle \prod_{p \mid q} \left(1-\frac{1}{p}\right)=\frac{\phi(q)}{q}$. Ha altri zeri oltre quelli di $\zeta$, che si ottengono quando $p^{-s}=1$, cioè $\sigma=0$ e $t=\frac{2k\pi}{\log{p}}$.
\end{oss}

\begin{oss} \label{logL}
  Per $\sigma>1$ abbiamo
  \begin{gather*}
    \log\big(L(s,\chi)\big)=-\sum_p \log\left(1-\frac{\chi(p)}{p^s}\right)=\sum_p\sum_{n=1}^{+\infty} \frac{\chi(p^n)}{np^{ns}} \implies \\
    \implies \frac{L'}{L}(s,\chi)=-\sum_{n=1}^{+\infty} \frac{\chi(n)\Lambda(n)}{n^s}.
  \end{gather*}
  In particolare,
  $$\frac{L'}{L}(s,\chi_0)=-\sum_{n=1}^{+\infty} \frac{\chi_0(n)\Lambda(n)}{n^s}=-\sum_{\substack{n=1, \\ (n,q)=1}}^{+\infty} \frac{\Lambda(n)}{n^s}=-\sum_{n=1}^{+\infty} \frac{\Lambda(n)}{n^s}+\sum_{\substack{p \mid q, \\a \ge 1}} \frac{\log{p}}{p^{as}}.$$
  Scriviamo adesso
  \begin{gather*}
    \log\big(L(s,\chi)\big)=\sum_p \frac{\chi(p)}{p^s}+O\left(\sum_p \sum_{n \ge 2} \frac{1}{p^{\sigma n}}\right)=\\
    =\sum_p \frac{\chi(p)}{p^s}+O\left(\sum_p \frac{1}{p^{\sigma}(p^{\sigma}-1)}\right)\overset{\sigma>1/2}{=}\sum_p \frac{\chi(p)}{p^s}+O(1) \implies \\
    \implies \sum_{\chi\text{ mod }q}\log\big(L(s,\chi)\big)=\sum_{\chi\text{ mod }q}\sum_p \frac{\chi(p)}{p^s}+O\big(\phi(q)\big).
  \end{gather*}
  Prendendo $s$ reale, e ricordando che i caratteri complessi sono accoppiati ciascuno con il proprio coniugato, si ha
  $$\log\left(\prod_{\chi\text{ mod }q}|L(s,\chi)|\right)=\sum_p\left(\frac{1}{p^s}+\sum_{\chi\not=\chi_0}\frac{\chi(p)}{p^s}\right)+O\big(\phi(q)\big).$$
\end{oss}

Ripetendo i passaggi appena visti, nell'ipotesi $(a,q)=1$ usando la formula di ortogonalità si dimostra anche che
$$\frac{1}{\phi(q)}\sum_{\chi\text{ mod }q} \bar{\chi}(a)\log\big(L(s,\chi)\big)\overset{(\star)}{=}\sum_{p \equiv a \pmod{q}} \frac{1}{p^s}+O(1).$$
Ricordiamo $\displaystyle L(s,\chi_0)=\zeta(s)\prod_{p \mid q} \left(1-\frac{1}{p^s}\right)$; perciò si ha $\log\big(L(s,\chi_0)\big) \overset{s \longrightarrow 1^+}{\longrightarrow} +\infty$ (prendendo $s \in \mathbb{R}$), e per la precisione c'è un polo semplice in $1$.
Se per $\chi\not=\chi_0$ si avesse $L(1,\chi)\not=0$, allora otterremmo $\log\big(L(s,\chi)\big)\overset{s \longrightarrow 1^+}{=}O(1)$; combinandolo con quello che sappiamo sul carattere principale, si avrebbe che il membro sinistro (e dunque anche il destro) di $(\star)$ tenderebbe a $+\infty$ per $s \longrightarrow 1^+$. Avremmo dunque il seguente risultato.

\begin{thm}
  (teorema di Dirichlet sui primi nelle progressioni aritmetiche) Siano $a,q$ interi con $(a,q)=1$. Esistono infiniti primi $p \equiv a \pmod{q}$.
\end{thm}

\begin{proof}
  Per quanto appena detto, dobbiamo verificare che le serie associate ai caratteri non principali non si annullano in $1$. Procediamo per assurdo.

  Caso $\chi$ complesso. Se $L(1,\chi)=0$, anche $L(1,\bar{\chi})=0$. Prendendo $a=1$ in $(\star)$, per $s \in \mathbb{R}, s>1$ e sfruttando l'espressione vista nella prima riga dell'osservazione \ref{logL}, avremmo
  $$\sum_{\chi\text{ mod }q} \log\big(L(s,\chi)\big)=\phi(q)\sum_{p^n \equiv 1 \pmod{q}} \frac{1}{np^{ns}} \in \mathbb{R}^+ \implies \prod_{\chi\text{ mod }q} |L(s,\chi)| \ge 1,$$
  assurdo perché in $1$ si avrebbero un polo semplice e due zeri.

  Se $\chi\not=\chi_0$ è reale e per assurdo $L(s,\chi)=0$, sia
  $$F(s)=\frac{L(s,\chi)L(s,\chi_0)}{L(2s,\chi_0)}.$$
  $F$ sarebbe regolare per $\sigma \ge 1/2$, in quanto il polo semplice di $L(s,\chi_0)$ in $1$ sarebbe compensato dall'eventuale $0$. Inoltre $\displaystyle \lim_{s \longrightarrow 1/2^+} F(s)=0$. Si ha
  $$F(s)=\prod_p \frac{\left(1-\frac{\chi(p)}{p^s}\right)^{-1}\left(1-\frac{\chi_0(p)}{p^s}\right)^{-1}}{\left(1-\frac{\chi_0(p)}{p^{2s}}\right)^{-1}}=\prod_{p\nmid q} \frac{\left(1-\frac{\chi(p)}{p^s}\right)^{-1}\left(1-\frac{1}{p^s}\right)^{-1}}{\left(1-\frac{1}{p^{2s}}\right)^{-1}}.$$
  Per i caratteri t.c. $\chi(p)=-1$ il fattore si semplifice e diventa $1$. Resta
  $$\prod_{\chi(p)=1} \frac{\left(1-\frac{1}{p^s}\right)^{-1}\left(1-\frac{1}{p^s}\right)^{-1}}{\left(1-\frac{1}{p^{2s}}\right)^{-1}}=\prod_{\chi(p)=1} \frac{p^s+1}{p^s-1}\overset{\sigma>1}{=} \sum_{n=1}^{+\infty} \frac{a_n}{n^s},$$
  dove $a_1=1$ e $a_n \ge 0$. La serie di Taylor di $F$ centrata in $s=2$ ha raggio almeno $3/2$. È $\displaystyle F(s)=\sum_{m=0}^{+\infty} \frac{b_m}{m!}(s-2)^m$ dove $\displaystyle b_m=(-1)^m\sum_{n=1}^{+\infty} \frac{a_n(\log{n})^m}{n^2}$, dunque
  $$F(s)=\sum_{m=0}^{+\infty} \frac{1}{m!} \sum_{n=1}^{+\infty} \frac{a_n(\log{n})^m}{n^2}(2-s)^m.$$
  Per $s$ reale minore di $2$ è tutto positivo, dunque $F(s) \ge a_1=1$ per $\sigma \ge 1/2$, assurdo.
\end{proof}

Sappiamo di più:
$$\sum_{\substack{p\equiv a\pmod{q}, \\ p \le x}} \frac{1}{p}=\frac{1}{\phi(q)}\log\log{x}+O_q(1).$$
Per mostrarlo, bisogna dire che $\displaystyle \sum_p \frac{\chi(p)}{p}$ converge per $\chi\not=\chi_0$. Mertens dimostrò questa cosa passando per la convergenza di $\displaystyle \sum_n \frac{\chi(n)\log{n}}{n}$ e usando che $\Lambda \star 1=L$, dove $L$ è la funzione aritmetica associata al logaritmo. Maggiori dettagli nel capitolo 7 di \cite{D}.


\subsection{Caratteri primitivi e somme di Gauss}
\begin{ex}
  Prendiamo $q=6$, per cui $(\mathbb{Z}/6 \mathbb{Z})^*=\{1,5\}$. C'è un solo carattere non principale, e sugli interi $1,2,3,4,5,6$ assume i valori $1,0,0,0,-1,0$. Consideriamo adesso $q_1=3$. $(\mathbb{Z}/3 \mathbb{Z})^*=\{1,2\}$.
  Valutandolo sugli stessi interi, assume i valori $1,-1,0,1-1,0$. In qualche senso, il carattere modulo $6$ si può ottenere annullando alcuni dei valori del carattere modulo $3$.
\end{ex}

\begin{defn}
  $\chi$ modulo $q$ si dice \textit{non primitivo} se esiste $q_1 \mid q$ e $\chi_1$ modulo $q_1$ t.c.
  $$\chi(n)=\begin{cases}
    \chi_1(n) &\mbox{se }(n,q)=1 \\
    0 &\mbox{se }(n,q)>1.
\end{cases}$$
Si dice \textit{primitivo} altrimenti.
\end{defn}

Nel caso di un carattere non primitivo, se $q_1$ è il minimo divisore di $q$ t.c. esiste un carattere $\chi_1$ modulo $q$ che soddisfa la condizione della definizione, $q_1$ si dice conduttore di $\chi$ modulo $q$ e diciamo che $\chi_1$ induce $\chi$. Per trovare $\chi_1$ partendo da $\chi$, se $(n,q)>1$ ma $(n,q_1)=1$, prendiamo $r$ t.c. $(n+rq_1,q)=1$ e poniamo $\chi_1(n)=\chi(n+rq_1)$. Si veda anche l'esempio. Maggiori dettagli aritmetici sul perché funziona nel capitolo 5 di \cite{D}.

\begin{oss}
  Se $\chi$ modulo $q$ è indotto da $\chi_1$ modulo $q_1$ con $q_1 \mid q$, per $\sigma>1$ si ha
  \begin{gather*}
    L(s,\chi)=\prod_{p\nmid q} \left(1-\frac{\chi(p)}{p^s}\right)^{-1}=\prod_{p\nmid q}\left(1-\frac{\chi_1(p)}{p^s}\right)^{-1}=\\
    =\prod_p \left(1-\frac{\chi_1(p)}{p^s}\right)^{-1}\prod_{p\mid q} \left(1-\frac{\chi_1(p)}{p^s}\right)=L(s,\chi_1)\prod_{p\mid q} \left(1-\frac{\chi_1(p)}{p^s}\right).
  \end{gather*}
  L'ultimo prodotto ci dà, per ogni primo $p \mid q$, infiniti zeri disposti periodicamente lungo $\sigma=0$, che non danno problemi. Possiamo notare che il legame tra un carattere non primitivo e il carattere primitivo che lo induce è analogo al legame tra la funzione $L$ associata a un carattere principale e la $\zeta$.
\end{oss}

\begin{defn}
  Sia $\chi$ modulo $q$ un carattere. Si definisce \textit{somma di Gauss} relativa a $\chi$ la quantità
  $$\tau(\chi)=\sum_{n=1}^q\chi(n)e^{\frac{2\pi in}{q}}.$$
\end{defn}

\begin{oss}
  Scriviamo $\displaystyle \chi(n)\tau(\bar{\chi})=\sum_{m=1}^q \chi(n)\bar{\chi}(m)e^{\frac{2\pi im}{q}}$.
  Supponendo $(n,q)=1$, sia $mn^{-1}=h$ in $\mathbb{Z}/q \mathbb{Z}$, ovvero $m \equiv nh \pmod{q}$ da cui abbiamo $\chi(n)\bar{\chi}(m)=\chi(n)\bar{\chi}(nh)=\chi(n)\bar{\chi}(n)\bar{\chi}(h)=\bar{\chi}(h)$. Allora troviamo
  $$\chi(n)\tau(\bar{\chi})\overset{(\star\star)}{=}\sum_{h=1}^q \bar{\chi}(h)e^{\frac{2\pi inh}{q}}.$$
\end{oss}

\begin{prop}
  Se $\chi$ modulo $q$ è primitivo, si ha che vale $(\star\star)$ anche se $(n,q)>1$; inoltre, $\tau(\bar{\chi})\not=0$.
\end{prop}

\begin{proof}
  Dimostriamo solo la seconda asserzione, dando per buona la prima.

  Prendiamo il coniugio in $(\star\star)$ e facciamo il prodotto con $(\star\star)$ stessa per ottenere
  $$|\chi(n)|^2|\tau(\bar{\chi})|^2=\sum_{h_1,h_2=1}^q \bar{\chi}(h_1)\chi(h_2)e^{\frac{2\pi in(h_1-h_2)}{q}}.$$
  Sommando su $n$ troviamo
  \begin{gather*}
    |\tau(\bar{\chi})|^2\phi(q)=\sum_{n=1}^q |\chi(n)|^2|\tau(\bar{\chi})|^2=\sum_{n=1}^q \sum_{h_1,h_2=1}^q \bar{\chi}(h_1)\chi(h_2)e^{\frac{2\pi in(h_1-h_2)}{q}}= \\
    \sum_{h_1,h_2=1}^q \sum_{n=1}^q \bar{\chi}(h_1)\chi(h_2)e^{\frac{2\pi in(h_1-h_2)}{q}}=\sum_{h=1}^q \sum_{n=1}^q |\chi(h)|^2=q\phi(q),
  \end{gather*}
  perciò $|\tau(\bar{\chi})|=\sqrt{q}$.
\end{proof}

Si ottiene dunque $\displaystyle \chi(n)=\frac{1}{\tau(\bar{\chi})}\sum_{m=1}^q \bar{\chi}(m)e^{\frac{2\pi imn}{q}}$.

\begin{oss}
  Si può dimostrare che se $\chi_1$ modulo $q_1$ induce $\chi$ modulo $q$, allora $\tau(\chi)=\tau(\chi_1)\mu\left(\frac{q}{q_1}\right)\chi_1\left(\frac{q}{q_1}\right)$.
\end{oss}

Andiamo adesso a dimostrare una generalizzazione dell'equazione funzionale per la $\vartheta$ di Jacobi.

\begin{prop}
  Sia $\alpha \in \mathbb{R}$ fissato, allora per $\mathfrak{Re}\,z>0$ si ha
  \begin{equation} \label{poiprimo}
    \sum_{n \in \mathbb{Z}} e^{-\pi n^2z-2\pi i\alpha n}=\frac{1}{\sqrt{z}}\sum_{n \in \mathbb{Z}} e^{-\frac{\pi(n+\alpha)^2}{z}}.
  \end{equation}
\end{prop}

\begin{proof}
  Come per la funzione di Jacobi, basta mostrarlo nel caso reale. Sia $f(\xi)=e^{-\pi\xi^2-2\pi i\alpha\xi/\sqrt{x}}$. Allora
  \begin{gather*}
    \hat{f}(\xi)=\int_{\mathbb{R}} e^{-\pi t^2-2\pi i\alpha t/\sqrt{x}-2\pi i\xi t}\diff t \implies \\
    \implies \hat{f}'(\xi)=-2\pi i\int_{\mathbb{R}} te^{-\pi t^2-2\pi i\alpha t/\sqrt{x}-2\pi i\xi t}\diff t=\text{(per parti)}\\
    =-2\pi\left(\xi+\frac{\alpha}{\sqrt{x}}\right)\int_{\mathbb{R}} e^{-\pi t^2-2\pi i\alpha t/\sqrt{x}-2\pi i\xi t}\diff t=-2\pi\left(\xi+\frac{\alpha}{\sqrt{x}}\right)\hat{f}(\xi).
  \end{gather*}
  Risolvendo l'equazione differenziale, dev'essere $\hat{f}(\xi)=Ce^{-\pi\left(\xi+\frac{\alpha}{\sqrt{x}}\right)^2}$. Prendendo $\xi=-\frac{\alpha}{\sqrt{x}}$ nel calcolo di $\hat{f}$ come trasformata, otteniamo $C=1$. Poniamo $f_x(\xi)=f(\sqrt{x}\xi)=e^{-\pi\xi^2x-2\pi i\alpha\xi}$. Cambiando di variabile $y=\sqrt{x}t$, si ha
  \begin{gather*}
    \hat{f}_x(\xi)=\int_{\mathbb{R}} f(\sqrt{x}t)e^{-2\pi it\xi}\diff t=\frac{1}{\sqrt{x}} \int_{\mathbb{R}} f(y) e^{-2\pi i\frac{y}{\sqrt{x}}\xi}\diff y=\\
    =\frac{1}{\sqrt{x}}\hat{f}\left(\frac{\xi}{\sqrt{x}}\right) \implies \hat{f}_x(\xi)=\frac{1}{\sqrt{x}}e^{-\pi(\xi+\alpha)^2/x}.
  \end{gather*}
  Per la formula di Poisson si ha la tesi.
\end{proof}

Derivando la formula \eqref{poiprimo} in $\alpha$ (cambiando i segni convenientemente a sinistra, che tanto si può fare), otteniamo
$$2\pi i \sum_{n \in \mathbb{Z}} ne^{-\pi n^2z+2\pi i\alpha n}=-\frac{2\pi}{z\sqrt{z}}\sum_{n \in \mathbb{Z}} (n+\alpha)e^{-\pi(n+\alpha)^2/z}, \text{ da cui}$$
\begin{equation} \label{poisecondo}
  \sum_{n \in \mathbb{Z}} ne^{-\pi n^2z+2\pi i\alpha n}=\frac{i}{z\sqrt{z}}\sum_{n \in \mathbb{Z}} (n+\alpha)e^{-\pi(n+\alpha)^2/z}.
\end{equation}


\subsection{Equazione funzionale e regioni libere da zeri per le funzioni $L$}
\begin{prop}
  Sia $\chi$ primitivo modulo $q$ e sia
  $$\xi(s,\chi)=\left(\frac{\pi}{q}\right)^{-\frac{s+a}{2}}\Gamma\left(\frac{s+a}{2}\right)L(s,\chi),$$
  con $a=\begin{cases}
    0 &\mbox{se }\chi(-1)=1 \\
    1 &\mbox{se }\chi(-1)=-1
\end{cases}.$ Allora
\begin{equation} \label{funelle}
  \xi(1-s,\bar{\chi})=\frac{i^a\sqrt{q}}{\tau(\chi)}\xi(s,\chi).
\end{equation}
In particolare, $\xi(s,\chi)$ è intera (di ordine $1$).
\end{prop}

\begin{proof}
  Facciamo il caso $\chi(-1)=1$ ($\implies \chi(-1)=\chi(1), \chi(-n)=\chi(n)$). Con un cambio di variabile $u=\pi n^2x/q$, per $\sigma>1$ si ha
  \begin{gather*}
    \Gamma(s/2)=\int_0^{+\infty} e^{-u}u^{s/2}\frac{\diff u}{u}=\int_0^{+\infty} e^{-\pi n^2x/q}x^{s/2}\left(\frac{q}{\pi}\right)^{-s/2}n^s\frac{\diff x}{x} \implies \\
    \implies \left(\frac{\pi}{q}\right)^{-s/2}\Gamma\left(\frac{s}{2}\right)\sum_{n=1}^{+\infty}\frac{\chi(n)}{n^s}=\int_0^{+\infty} \sum_{n=1}^{+\infty} \chi(n)e^{-\pi n^2x/q}x^{s/2}\frac{\diff x}{x}= \\
    =\frac{1}{2}\int_0^{+\infty} \vartheta(x,\chi)x^{s/2}\frac{\diff x}{x},
  \end{gather*}
  dove $\displaystyle \vartheta(x,\chi)=\sum_{n \in \mathbb{Z}} \chi(n)e^{-\pi n^2x/q}$. Abbiamo anche
  \begin{gather*}
    \chi(n)=\frac{1}{\tau(\bar{\chi})}\sum_{m=1}^q \bar{\chi}(m)e^{2\pi imn/q} \implies \\
    \implies \sum_{n \in \mathbb{Z}} \chi(n)e^{-\pi n^2 x/q}=\frac{1}{\tau(\bar{\chi})} \sum_{m=1}^q \bar{\chi}(m) \sum_{n \in \mathbb{Z}} e^{-\pi n^2x/q+2\pi imn/q} \implies \\
    \implies \vartheta(x,\chi)\tau(\bar{\chi})=\sum_{m=1}^q \bar{\chi}(m) \sum_{n \in \mathbb{Z}} e^{-\pi n^2x/q+2\pi imn/q}.
  \end{gather*}
  Usando la \eqref{poiprimo} con $z=x/q$ e $\alpha=m/q$ troviamo
  \begin{gather*}
    \vartheta(x,\chi)\tau(\bar{\chi})=\sqrt{\frac{q}{x}}\sum_{m=1}^q \bar{\chi}(m) \sum_{n \in \mathbb{Z}} e^{-\pi\left(n+\frac{m}{q}\right)^2q/x}= \\
    =\sqrt{\frac{q}{x}}\sum_{m=1}^q \bar{\chi}(m) \sum_{n \in \mathbb{Z}} e^{-\frac{\pi(nq+m)^2}{qx}}=\sqrt{\frac{q}{x}} \sum_{l \in \mathbb{Z}} \bar{\chi}(l)e^{-\frac{\pi l^2}{xq}}=\sqrt{\frac{q}{x}}\vartheta\left(\frac{1}{x},\bar{\chi}\right) \implies \\
    \implies \xi(s,\chi)=\frac{1}{2}\int_1^{+\infty} \vartheta(x,\chi)x^{s/2}\frac{\diff x}{x}+\frac{1}{2}\int_1^{+\infty} \vartheta\left(\frac{1}{x},\chi\right)x^{-s/2}\frac{\diff x}{x}= \\
    =\frac{1}{2}\int_1^{+\infty} \vartheta(x,\chi)x^{s/2}\frac{\diff x}{x}+\frac{1}{2}\cdot\frac{\sqrt{q}}{\tau(\bar{\chi})}\int_1^{+\infty} \vartheta(x,\bar{\chi})x^{\frac{1-s}{2}}\frac{\diff x}{x} \implies \\
    \implies \xi(1-s,\bar{\chi})=\frac{1}{2}\int_1^{+\infty} \vartheta(x,\bar{\chi})x^{\frac{1-s}{2}}\frac{\diff x}{x}+\frac{1}{2}\cdot\frac{\sqrt{q}}{\tau(\chi)}\int_1^{+\infty} \vartheta(x,\chi)x^{\frac{s}{2}}\frac{\diff x}{x}= \\
    =\frac{\sqrt{q}}{\tau(\chi)}\xi(s,\chi),
  \end{gather*}
  dove nell'ultimo passaggio abbiamo usato che, in questo caso, si può mostrare che $\tau(\bar{\chi})=\overline{\tau(\chi)}$ (e ricordiamo $|\tau(\chi)|^2=q$). Si ottiene anche che $L(s,\chi)$ ha zeri banali nei pari negativi e in $0$ (nell'altro caso sono i dispari negativi).

  Caso $\chi(-1)=1$: diamo una breve traccia, i passaggi sono analoghi al caso precedente.
  \begin{gather*}
    \left(\frac{\pi}{q}\right)^{-\frac{s+1}{2}}\Gamma\left(\frac{s+1}{2}\right)L(s,\chi)=\int_0^{+\infty}\sum_{n=0}^{+\infty} n\chi(n)e^{-\pi n^2x}x^{\frac{s+1}{2}}\frac{\diff x}{x}= \\
    =\frac{1}{2}\int_1^{+\infty}\vartheta_1(x,\chi)x^{\frac{s+1}{2}}\frac{\diff x}{x}+\frac{1}{2}\int_1^{+\infty} \vartheta_1\left(\frac{1}{x},\chi\right)x^{-\frac{s+1}{2}}\frac{\diff x}{x},
  \end{gather*}
  dove $\displaystyle \vartheta_1(x,\chi)=\sum_{n \in \mathbb{Z}} n\chi(n)e^{-\pi n^2x}$. Usando \eqref{poisecondo} si trova $\vartheta_1(x,\chi)\tau(\bar{\chi})=\dfrac{i\sqrt{q}}{x^{3/2}}\vartheta_1\left(\dfrac{1}{x},\bar{\chi}\right)$,
  da cui si ottiene $\xi(1-s,\bar{\chi})=\frac{i\sqrt{q}}{\tau(\chi)}\xi(s,\chi)$, usando che in questo caso vale $\chi(-n)=-\chi(n)$ e $\tau(\bar{\chi})=-\overline{\tau(\chi)}$.
\end{proof}

Vogliamo studiare l'ordine di $\xi(s,\chi)$.

Per $\sigma \ge 1/2$, $\Gamma(s+1/2) \ll e^{c_0|s|\log{|s|}} \ll_{\epsilon} e^{|s|^{1+\epsilon}}$ per ogni $\epsilon>0$. Per $\sigma>1$, $|L(s,\chi)| \ll \zeta(\sigma)$. Possiamo dire di più: per $\sigma \ge \epsilon$, per sommazione parziale
$$\sum_{n \le x} \frac{\chi(n)}{n^s}=\left(\sum_{n \le x}\chi(n)\right)x^{-s}s\int_1^x\left(\sum_{n \le u}\chi(n)\right)u^{-(s+1)}\diff u;$$
poiché le sommatorie sono $O(q)$, troviamo
$$L(s,\chi)=s\int_1^{+\infty} \left(\sum_{n \le u}\chi(n)\right)u^{-(s+1)}\diff u.$$
Per $\sigma \ge 1/2$ questo ci dà $L(s,\chi) \ll 2|s|q$. Per le stime nell'altro semipiano basta, al solito, sfruttare l'equazione funzionale.

Esistono infiniti zeri $\rho_\chi=\beta_\chi+i\gamma_\chi$ della funzione $L(s,\chi)$ con $0 \le \beta_\chi \le 1$. Si ha $\displaystyle \sum_{\rho_\chi} \frac{1}{|\rho_\chi|^{1+\epsilon}}<+\infty$ per ogni $\epsilon>0$ e $\displaystyle \sum_{\rho_\chi} \frac{1}{|\rho_\chi|}=+\infty$.
Non ci sono le stesse simmetrie della $\zeta$: $L(\bar{s},\chi)=\overline{L(s,\bar{\chi})}\not=\overline{L(s,\chi)}$. Da questo otteniamo che, se $\rho_\chi$ è uno zero di $L(s,\chi)$, allora $\bar{\rho}_\chi=\rho'_{\bar{\chi}}$ è uno zero di $L(s,\bar{\chi})$; dall'equazione funzionale troviamo anche che $1-\rho'_{\bar{\chi}}=\rho''_\chi$ è un altro zero di $L(s,\chi)$.
Un altro problema, che vedremo in dettaglio tra poco, è che $\gamma_\chi$ può essere molto piccolo: questo renderà le stime che andremo a fare, analoghe a quelle per la $\zeta$, dipendenti anche da $q$.

Non mostriamo invece la seguente proposizione, che dovrebbe ormai essere immediata.

\begin{prop}
  Si ha il seguente prodotto di Weierstrass:
  $$\xi(s,\chi)=e^{a+As}\prod_{\rho_\chi}\left(1-\frac{s}{\rho_\chi}\right)e^{\frac{s}{\rho_\chi}}.$$
\end{prop}

Si ha $e^a=\xi(0,\chi)$, che è collegato a $L(0,\chi)$, che è collegato a $L(1,\chi) \ll \log{q}$. Vedremo che $\displaystyle \mathfrak{Re}\,A=-\frac{1}{2}\sum_{\rho_\chi}\left(\frac{1}{\rho_\chi}+\frac{1}{\bar{\rho}_\chi}\right)=-\sum_{\rho_\chi} \mathfrak{Re}\left(\frac{1}{\rho_\chi}\right)$, che converge. Le stime per $A$ non sono buonissime ($\ll \sqrt{q}$).

\begin{cor} \label{derlogxielle}
  Sia $K$ un compatto contenuto in $\mathbb{C} \setminus \displaystyle \bigcup_{\rho_\chi} \{\rho_\chi\}$. Per $s \in K$ si ha
  $$\frac{\xi'}{\xi}(s,\chi)=A+\sum_{\rho_\chi} \left(\frac{1}{s-\rho_\chi}+\frac{1}{\rho_\chi}\right), \text{ da cui}$$
  $$\frac{L'}{L}(s,\chi)=\frac{1}{2}\log\left(\frac{\pi}{q}\right)-\frac{1}{2}\cdot\frac{\Gamma'}{\Gamma}\left(\frac{s+a}{2}\right)+A+\sum_{\rho_\chi}\left(\frac{1}{s-\rho_\chi}+\frac{1}{\rho_\chi}\right).$$
\end{cor}

\begin{prop}
  (formula di Riemann-Von Mangoldt) Sia $\chi$ primitivo modulo $q$ e $T \ge 2$. Posto $N(T,\chi)=\frac{1}{2}\sharp\{\rho_\chi=\beta_\chi+i\gamma_\chi \mid 0 \le \beta_\chi \le 1, |\gamma_\chi|<T\}$, si ha
  $$N(T,\chi)=\frac{T}{2\pi}\log\left(\frac{qT}{2\pi}\right)-\frac{T}{2\pi}+O\big(\log(qT)\big).$$
\end{prop}

\begin{proof}
  La strategia generale della dimostrazione è la stessa che per la $\zeta$. Dobbiamo fare alcuni accorgimenti: percorreremo in senso antiorario il rettangolo $R$ di vertici $\{5/2-iT,5/2+iT, -3/2+iT, -3/2-iT\}$; questo perché non c'è simmetria tra sotto e sopra (da cui il $\pm iT$), inoltre si becca uno zero banale ($0$ o $1$), che comunque si tratta di un $\pm 1$ nella formula e quindi possiamo ignorarlo. Dall'equazione funzionale,
  $$\arg\xi(\sigma+it,\chi)=c+\arg\xi(1-\sigma-it,\bar{\chi})=\arg\overline{\xi(1-\sigma+it,\chi)}+c.$$
  Allora la metà sinistra e quella destra di $R$ danno contributi che variano solo per una costante, perciò ci basterò stimare $2\cdot\frac{1}{2\pi}\Delta_L$ dove $L$ è la metà destra di $R$. Riprendendo la definizione di $\xi$, ripetendo le stime viste per la $\zeta$ ma facendo attenzione che questa volta il percorso è doppio (sia sopra che sotto), troviamo
  $$\Delta_L \arg\left(\frac{\pi}{q}\right)^{-\frac{s+a}{2}}=T\log(q/\pi), \quad \Delta_L\arg\Gamma\left(\frac{s+a}{2}\right)=T\log\frac{T}{2}-T+O\left(\frac{1}{T}\right).$$
  Rimane da stimare il termine dovuto a $L(s,\chi)$ e vedere che ci dà il resto voluto. Poiché è reale sui reali, dobbiamo stimare $\frac{1}{2\pi}\arg L(1/2\pm iT,\chi)$. Dal corollario \ref{derlogxielle} si ha
  $$\frac{L'}{L}(s,\chi)=-\frac{1}{2}\log\left(\frac{q}{\pi}\right)+A_\chi-\frac{1}{2}\cdot\frac{\Gamma'}{\Gamma}\left(\frac{s+a}{2}\right)+\sum_{\rho_\chi} \left(\frac{1}{s-\rho_\chi}+\frac{1}{\rho_\chi}\right)$$
  e $\frac{\xi'}{\xi}(0,\chi)=A_\chi$. Dall'espressione con $L'/L$ valutata in $s=0$ e coniugando opportunamente si trova $A_{\bar{\chi}}=\bar{A}_\chi$. Allora dall'equazione funzionale si ha
  \begin{gather*}
    A_\chi=\frac{\xi'}{\xi}(0,\chi)=-\frac{\xi'}{\xi}(1,\bar{\chi})=-A_{\bar{\chi}}-\sum_{\rho_{\bar{\chi}}} \left(\frac{1}{1-\rho_{\bar{\chi}}}+\frac{1}{\rho_{\bar{\chi}}}\right)= \\
    =-A_{\bar{\chi}}-\sum_{\rho_\chi}\left(\frac{1}{1-\bar{\rho}_\chi}+\frac{1}{\bar{\rho}_\chi}\right)=-A_{\bar{\chi}}-\sum_{\rho_\chi}\left(\frac{1}{\rho_\chi}+\frac{1}{\bar{\rho}_\chi}\right) \implies \\
    \implies \mathfrak{Re}\,A_\chi=-\sum_{\rho_\chi} \mathfrak{Re}\left(\frac{1}{\rho_\chi}\right).
  \end{gather*}
  Si ha dunque
  \begin{gather*}
    -\mathfrak{Re}\,\frac{L'}{L}(s,\chi)=\frac{1}{2}\log\left(\frac{q}{\pi}\right)-\mathfrak{Re}\,A_\chi+\frac{1}{2}\mathfrak{Re}\,\frac{\Gamma'}{\Gamma}\left(\frac{s+a}{2}\right)-\sum_{\rho_\chi}\Bigg(\mathfrak{Re}\left(\frac{1}{s-\rho_\chi}\right)+\mathfrak{Re}\left(\frac{1}{\rho_\chi}\right)\Bigg)=\\
    =\frac{1}{2}\log\left(\frac{q}{\pi}\right)+\frac{1}{2}\mathfrak{Re}\,\frac{\Gamma'}{\Gamma}\left(\frac{s+a}{2}\right)-\sum_{\rho_\chi}\mathfrak{Re}\left(\frac{1}{s-\rho_\chi}\right).
  \end{gather*}
  Prendendo $s=\sigma+it$ e stimando $\Gamma'/\Gamma$, abbiamo
  $$-\mathfrak{Re}\,\frac{L'}{L}(\sigma+it,\chi) \le C\cdot\log\big(q(|t|+2)\big)-\sum_{\rho_\chi} \frac{\sigma-\beta_\chi}{(\sigma-\beta_\chi)^2+(t-\gamma_\chi)^2};$$
  ragionando come abbiamo fatto a suo tempo per la $\zeta$ troviamo
  $$\sum_{|\gamma_\chi-t|<1} 1, \,\, \sum_{\gamma_\chi}\frac{1}{1+(t-\gamma_\chi)^2} \ll \log\big(q(|t|+2)\big), \text{ da cui}$$
  $$\frac{L'}{L}(s,\chi)=\sum_{|\gamma_\chi-t|<1} \frac{1}{s-\rho_\chi}+O\Big(\log\big(q(|t|+2)\big)\Big).$$
  Di nuovo con ragionamenti analoghi a quelli visti per la $\zeta$, osserviamo che ci interessano solo i tratti da $1/2$ a $1$, ma per quanto ottenuto finora si ha
  $$O(1)+\arg\frac{L'}{L}(1/2 \pm iT,\chi)=\int_{1/2}^1 \mathfrak{Im}\,\frac{L'}{L}(\sigma \pm iT,\chi)\diff\sigma \ll \log\big(q(|T|+2)\big),$$
  da cui la tesi.
\end{proof}

\begin{oss}
  Se $\chi$ non è primitivo modulo $q$, sia $\chi_1$ modulo $q_1$ che induce $\chi$, allora $\displaystyle L(s,\chi)=L(s,\chi_1)\prod_{p \mid q}\left(1-\frac{\chi_1(p)}{p^s}\right)$ in $\mathbb{C}$, da cui per $\sigma \ge 1/2$
  \begin{gather*}
    \frac{L'}{L}(s,\chi)=\frac{L'}{L}(s,\chi_1)+\sum_{p \mid q} \frac{p^{-s}\log{p}\cdot\chi_1(p)}{1-\frac{\chi_1(p)}{p^s}} \implies \\
    \implies \left|\frac{L'}{L}(s,\chi)-\frac{L'}{L}(s,\chi_1)\right| \le \sum_{p \mid q} \frac{\log{p}}{p^{\sigma}-1} \ll \sum_{p \mid q} \log{p} \le \log{q}.
  \end{gather*}
  Ci sono anche gli zeri $0+it$ t.c. $p^{-it}=\pm 1$, cioè $t=\frac{\pi(2k+1)}{\log{p}}$; quelli fino a $T$ si stimano con $T\log{p}$ e quindi abbiamo $\displaystyle \sum_{p \mid q} T\log{p} \le T\log{q}$. Dunque nel caso $\chi$ non primitivo si ha $N(T,\chi)=\frac{T}{2\pi}\log{T}+O(T\log{q})$.
\end{oss}


\newpage

\begin{thebibliography}{widest entry}
  \bibitem[JBG]{JBG} J. B. Garnett: \textbf{Bounded Analytic Functions (Revised First Edition)}. Springer, New York, 2007
  \bibitem[NN]{NN} R. Narasimhan, Y. Nievergelt: \textbf{Complex analysis in one variable (2nd edition)}. Springer, New York, 2001
\end{thebibliography}


\section*{Ringraziamenti}
\addcontentsline{toc}{section}{Ringraziamenti}
Da scrivere alla fine del corso. \\
JBG e NN sopra nella bibliografia sono degli esempi lasciati per ricordarsi qual è il modo giusto di scriverli, da togliere dopo aver inserito la bibliografia giusta per queste dispense.


\end{document}
