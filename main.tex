\documentclass{article}
\usepackage{mstyle}
\usepackage{pgfplots}
\usetikzlibrary{intersections, pgfplots.fillbetween}

\title{Appunti di Teoria Analitica dei Numeri A}
\date{}
\author{Marco Vergamini}

\begin{document}
\maketitle
\newpage
\tableofcontents
\newpage


\section*{Introduzione}
\addcontentsline{toc}{section}{Introduzione}
Questi appunti sono basati sul corso Teoria Analitica dei Numeri A tenuto dal professor Giuseppe Puglisi nel secondo semestre dell'anno accademico 2020/2021. Sono dati per buoni (si vedano i prerequisiti del corso) i corsi di Aritmetica, Analisi 1 e 2 e Teoria dei Numeri Elementare, più le base dell'analisi complessa in una variabile. Verranno omesse o soltanto hintate le dimostrazioni più semplici, ma si consiglia comunque di provare a svolgerle per conto proprio. Ogni tanto sarà commesso qualche abuso di notazione, facendo comunque in modo che il significato sia reso chiaro dal contesto. Inoltre, la notazione verrà alleggerita man mano, per evitare inutili ripetizioni e appesantimenti nella lettura. Si ricorda anche che questi appunti sono scritti non sempre subito dopo le lezioni, non sempre con appunti completi, ecc\dots. Spesso saranno rivisti, verranno aggiunte cose che mancavano perché c'era poco tempo (o voglia\dots), potrebbero mancare argomenti più o meno marginali\dots insomma, non è un libro di testo per il corso, ma vuole essere un valido supporto per aiutare gli studenti che seguono il corso. Spero di essere riuscito in questo intento. \\

Ho inoltre deciso di omettere gli appunti delle prime due/tre ore di lezione, nelle quali sono stati esposti a grandi linee gli argomenti e i principali risultati trattati nel corso, poiché sono stati descritti in modo discorsivo e impreciso, e comunque verranno ovviamente trattati dettagliatamente nel seguito.


\newpage

\section{Funzioni intere e funzione $\Gamma$}

\subsection{Strumenti di analisi complessa}
\begin{lm} \label{mittagleffler}
  (Mittag-Leffler) Sia $(z_n)_{n \in \mathbb{N}}$ una successione di numeri complessi t.c. $|z_n| \longrightarrow \infty$ per $n \longrightarrow \infty$ e $0<|z_n| \le |z_{n+1}|$ per ogni $n$. Sia inoltre $(m_n)_{n \in \mathbb{N}}$ un'altra successione con $m_n \in \mathbb{C}^*$ per ogni $n$. Allora esistono $p_n \in \mathbb{N}\cup \{0\}$ t.c.
  $$f(z)=\sum_{n=1}^{+\infty} \left(\frac{z}{z_n}\right)^{p_n}\frac{m_n}{z-z_n}$$
  converge in $K \subset \mathbb{C}\setminus\{z_1, z_2, \dots\}$ compatto. Inoltre, se $|z|<|z_1|$, si ha
  $$f(z)=-\sum_{k=1}{+\infty} \left(\sum_{n:p_n<k}m_nz_n^{-k}\right)z^{k-1}.$$
\end{lm}

\begin{proof}
  Prendiamo $r_n$ reali positivi con $r_n \le r_{n+1}$ e $r_n \longrightarrow +\infty$ per $n \longrightarrow +\infty$, e t.c. $r_n<|z_n|$. Per $|z| \le r_n$ si ha
  $$\left|\frac{m_n}{z-z_n}\right| \le \frac{m_n}{|z_n|-r_n}, \quad \left|\frac{z}{z_n}\right|<\frac{|z|}{r_n} \le 1 \implies$$
  perciò si ha che esistono $p_n \in \mathbb{N}\cup\{0\}$ t.c. $\displaystyle \left|\frac{z}{z_n}\right|^{p_n} <\epsilon_n \frac{|z_n|-r_n}{|m_n|}$, con $\epsilon_n>0$ e $\displaystyle \sum_{n=1}^{+\infty} \epsilon_n<+\infty$.
  Abbiamo dunque $\displaystyle \left|\left(\frac{z}{z_n}\right)^{p_n}\frac{m_n}{z-z_n}\right| \le \left|\frac{z}{z_n}\right|^{p_n} \frac{|m_n|}{|z_n|-r_n}<\epsilon_n$. Fissiamo ora il compatto $K$ e consideriamo $N$ t.c. $|z| \le r_N$ per ogni $z \in K$.
  Poniamo $\displaystyle M_n=\max_{z \in K} \left|\left(\frac{z}{z_n}\right)^{p_n}\frac{m_n}{z-z_n}\right|$, per $n \le N-1$. Per ogni $z \in K$ si ha che
  $$\sum_{n=1}^{+\infty} \left|\left(\frac{z}{z_n}\right)^{p_n}\frac{m_n}{z-z_n}\right| \le \sum_{n=1}^{N-1} M_n+\sum_{n=N}^{+\infty} \epsilon_n<+\infty.$$

  Se $|z|<|z_1|$, possiamo scrivere
  \begin{align*}
    f(z) & =\sum_{n=1}^{+\infty} \left(\frac{z}{z_n}\right)^{p_n}\frac{m_n}{z-z_n} \\
    & =-\sum_{n=1}^{+\infty} \frac{m_nz^{p_n}}{z_n^{p_n+1}}\frac{1}{1-z/z_n}=-\sum_{n=1}^{+\infty}m_n\sum_{k=p_n+1}^{+\infty} \frac{z^{k-1}}{z_n^k}.
  \end{align*}
  Poiché nella prima parte della dimostrazione abbiamo visto che c'è convergenza totale, possiamo scambiare le due sommatorie ottenendo così la seconda parte della tesi.
\end{proof}

\begin{oss} \label{1.1.2}
  Per $|z| \le r_n$ si ha
  \begin{gather*}
    \left|\frac{z}{z_n}\right|=\frac{2|z|}{|z_n|+|z_n|}<\frac{2|z|}{|z_n|+r_n} \le \frac{2|z|}{|z-z_n|} \\
    |m_m| \left|\frac{z}{z_n}\right|^{p_n+1} \le \frac{2|z|}{|z-z_n|}\left|\frac{z}{z_n}\right|^{p_n}|m_n|=2|z|\left|\left(\frac{z}{z_n}\right)^{p_n}\frac{m_n}{z-z_n}\right| \\
    \sum_{n=1}^{+\infty} |m_n|\left|\frac{z}{z_n}\right|^{p_n+1} \le 2|z|\sum_{n=1}^{+\infty} \left|\left(\frac{z}{z_n}\right)^{p_n}\frac{m_n}{z-z_n}\right|<+\infty.
  \end{gather*}
\end{oss}

\begin{ex}
  Se $z_n=n$ e $m_n=1$ per ogni $n$, basta prendere $p_n=1$.

  Sia invece $\displaystyle |z_0|>\max_K |z|$ e consideriamo $|z_n|>|z_0|+1$. Si ha che
  \begin{align*}
    \left|\left(\frac{z}{z_n}\right)^{p_n}\frac{m_n}{z-z_n}\right| & \le \left|\frac{z_0}{z_n}\right|^{p_n+1}\frac{|m_n|}{|z_n|-|z_0|}\frac{|z_n|}{|z_0|} \\
    & \le \left|\frac{z_0}{z_n}\right|^{p_n+1}\frac{|z_0|+1}{|z_0|+1-|z_0|}\frac{|m_n|}{|z_0|}=|m_n|\left|\frac{z_0}{z_n}\right|^{p_n+1}\left(1+\frac{1}{|z_0|}\right),
  \end{align*}
  dove la seconda disugaglianza segue dal fatto che la funzione $\frac{t}{t-|z_0|}$ è decrescente. In questo caso, vale la maggiorazione opposta a quella dell'osservazione \ref{1.1.2}.
\end{ex}

\begin{lm} \label{1.1.4}
  Sia $f$ meromorfa con poli semplici nei punti $z_n\not=0$, con residui $m_n \in \mathbb{Z}$, e t.c. $|z_n| \longrightarrow +\infty$ per $n \longrightarrow +\infty$. Sia $\gamma(0,z)$ un cammino da $0$ a $z$ non passante per i punti $z_n$. Allora la funzione
  $$\varphi(z)=\exp\left(\int_{\gamma(0,z)} f(w)\diff w\right)$$
  è meromorfa, con zeri $z_n$ con molteplicità $m_n$ se $m_n>0$ e poli $z_n$ con molteplicità $-m_n$ se $m_n<0$.
\end{lm}

\begin{oss} \label{1.1.5}
  Sia $\gamma$ il cammino dell'integrale del lemma \ref{1.1.4} e $\gamma'$ un altro cammino, t.c. $\gamma' \cup -\gamma$ sia una curva di Jordan (piana, semplice, chiusa) contenuta in $\mathbb{C}\setminus\{z_1,z_2,\dots\}$. Per il teorema dei residui si ha
  $$\int_{\gamma}f(w)\diff w=\int_{\gamma'}f(w)\diff w+2\pi iR,$$
  con $R=\displaystyle\sum_{\substack{n:z_n \in A, \\ \partial A=\gamma'\cup-\gamma}} m_n$. Dunque $\displaystyle\varphi(z)=\exp\left(\int_{\gamma'}f(w)\diff w\right)e^{2\pi iR}$.
\end{oss}

\begin{proof}
  Poiché $m_n \in \mathbb{Z}$, per l'osservazione \ref{1.1.5} $\varphi$ non dipende dal cammino scelto. Consideriamo $f_1(z)=f(z)-\dfrac{m_1}{z-z_1}$. $f_1$ è olomorfa in $\mathbb{C}\setminus\{z_2,z_3,\dots\}$, quindi $\displaystyle \exp\left(\int_0^z f_1(w)\diff w\right)$ è olomorfa e mai nulla in $\mathbb{C}\setminus\{z_2,z_3,\dots\}$.
  \begin{gather*}
    \varphi(z)=\exp\left(\int_0^z f_1(w)\diff w+m_1\int_0^z\frac{\diff w}{w-z_1}\right), \int_0^z \frac{\diff w}{w-z_1}=\log\left(\frac{z-z_1}{-z_1}\right) \implies \\
    \implies \varphi(z)=\exp\left(\int_0^z f_1(w)\diff w\right)\cdot\exp\Bigg(m_1\log\left(\frac{z-z_1}{-z_1}\right)\Bigg)= \\
    =\exp\left(\int_0^z f_1(w)\diff w\right)\cdot(-z_1)^{-m_1}\cdot(z-z_1)^{m_1}=\varphi_1(z)(z-z_1)^{m_1},
  \end{gather*}
  dove $\varphi_1$ è una funzione olomorfa e mai nulla in $\mathbb{C}\setminus\{z_2,z_3,\dots\}$. Allora $\varphi$ ha uno zero o un polo dell'ordine voluto in $z_1$. Ripetendo per ogni $n$ si ha la tesi.
\end{proof}

Con i due lemmi appena mostrati si può costruire una funzione meromorfa su $\mathbb{C}$ con zeri e poli di molteplicità assegnata, assumendo che la successione degli stessi non abbia alcun limite finito.

\begin{thm} \label{wprod}
  (prodotto di Weierstrass) Sia $F$ meromorfa in $\mathbb{C}$ e siano $z_n \not=0$ gli zeri e i poli di molteplicità $|m_n|$. Esistono una successione $p_n \in \mathbb{N}\cup\{0\}$ e una funzione intera $G(z)$ t.c.
  \begin{equation} \label{wprodformula}
    F(z)=e^{G(z)}\prod_n\left(1-\frac{z}{z_n}\right)^{m_n}\exp\Bigg(m_n\sum_{k=1}^{p_n}\frac{1}{k}\left(\frac{z}{z_n}\right)^k\Bigg),
  \end{equation}
  dove il prodotto infinito converge uniformemente in ogni $K\subset \mathbb{C}\setminus\{z_1,z_2,\dots\}$ compatto. Inoltre $\displaystyle \sum_{n=1}^{+\infty} |m_n|\left|\frac{z}{z_n}\right|^{p_n+1}<+\infty$ per ogni $z \in K$.
\end{thm}

\begin{proof}
  Costruiamo $\displaystyle f(z)=\sum_{n=1}^{+\infty} \left(\frac{z}{z_n}\right)^{p_n}\frac{m_n}{z-z_n}$ come nel lemma \ref{mittagleffler} e $\displaystyle \varphi=\exp\left(\int_0^z f(w)\diff w\right)=\exp\Bigg(\int_0^z \sum_{n=1}^{+\infty}\left(\frac{w}{z_n}\right)^{p_n}\frac{m_n}{w-z_n}\diff w\Bigg)$ come nel lemma \ref{1.1.4}.
  Osserviamo che
  $$\left(\frac{w}{z_n}\right)^{p_n}\frac{1}{w-z_n}=\frac{1}{w-z_n}+\frac{1}{z_n}\sum_{k=0}^{p_n-1}\left(\frac{w}{z_n}\right)^n,$$
  quindi
  \begin{gather*}
    \varphi(z)=\prod_{n=1}^{+\infty} \exp\Bigg[\int_0^z\Bigg(\frac{m_n}{w-z_n}+\frac{m_n}{z_n}\sum_{k=0}^{p_n-1}\left(\frac{w}{z_n}\right)^k\Bigg)\diff w\Bigg]= \\
    =\prod_{n=1}^{+\infty}\exp\Bigg(m_n\log\left(\frac{z-z_n}{-z_n}\right)+m_n\sum_{k=1}^{p_n}\frac{1}{k}\left(\frac{z}{z_n}\right)^k\Bigg)= \\
    =\prod_{n=1}^{+\infty} \left(1-\frac{z}{z_n}\right)^{m_n}\exp\Bigg(m_n\sum_{k=1}^{p_n}\frac{1}{k}\left(\frac{z}{z_n}\right)^k\Bigg).
  \end{gather*}
  Osserviamo ora che $\frac{F(z)}{\varphi(z)}$ è intera e mai nulla in $\mathbb{C}$, dunque esiste $G(z)$ intera t.c. $\frac{F(z)}{\varphi(z)}=e^{G(z)} \implies F(z)=e^{G(z)}\varphi(z)$, da cui la tesi.
\end{proof}

\begin{oss}
  Se $F(z)$ ha uno zero o un polo di molteplicità $|m|$ in $0$, basta applicare il teorema \ref{wprod} alla funzione $\tilde{F}(z)=F(z)/z^m$.
\end{oss}

\begin{cor}
  Sia $F$ come nel teorema \ref{wprod}, indichiamo con $D^k$ la derivata $k$-esima. Si ha
  $$G(z)=\log{F(0)}+\sum_{k=1}^{+\infty}\Bigg(\frac{1}{k!}D^{k-1}\left(\frac{F'}{F}(z)\right)_{z=0}+\frac{1}{k}\sum_{n:p_n<k}m_nz_n^{-k}\Bigg)z^k.$$
\end{cor}

\begin{proof}
  Dalla dimostrazione del teorema \ref{wprod} e prendendo il logaritmo otteniamo
  $$G(z)+\int_0^z\sum_{n=1}^{+\infty}\left(\frac{w}{z_n}\right)^{p_n}\frac{m_n}{w-z_n}\diff w=\log{F(z)}.$$
  Ovviamente $\displaystyle \log{F(z)}=\log{F(0)}+\sum_{k=1}^{+\infty} \frac{D^{k-1}\left(\frac{F'}{F}(z)\right)_{z=0}}{k!}z^k$. Inoltre
  \begin{gather*}
    -\int_0^z \sum_{n=1}^{+\infty} \left(\frac{w}{z_n}\right)^{p_n}\frac{m_n}{w-z_n}\diff w=\int_0^z \sum_{k=1}^{+\infty} \sum_{n: p_n<k} (m_nz_n^{-k})w^{k-1}\diff w= \\
    =\sum_{k=1}^{+\infty} \frac{1}{k}\left(\sum_{n: p_n<k}m_mz_n^{-k}\right)z^k.
  \end{gather*}
\end{proof}

Avviciniamoci alla notazione canonica del prodotto di Weierstrass per una funzione intera: $\displaystyle F(z)=e^{G(z)}\prod_{n=1}^{+\infty}\left(1-\frac{z}{z_n}\right)\exp\Bigg(\sum_{k=1}^{p_n}\left(\frac{z}{z_n}\right)^k\Bigg)$, dove gli zeri vengono contati con molteplicità (e d'ora in avanti si sottindenderà sempre così).


\subsection{Funzioni intere di ordine finito}
Introduciamo ora un argomento importante: le funzioni intere di ordine finito.

Notazione: scriviamo che $f(z) \ll g(z)$ $(z \longrightarrow +\infty)$ $\iff$ $f(z)=O\big(g(z)\big)$. Inoltre, scriveremo $\ll_{\epsilon}$ per indicare che la costante dell'$O$-grande dipende da un parametro $\epsilon$.

\begin{defn}
  Data $F$ intera, si dice che ha \textit{ordine} $\alpha \ge 0$ se
  $$F(z) \ll_{\epsilon} e^{|z|^{\alpha+\epsilon}} \, (z \longrightarrow +\infty)$$
  per ogni $\epsilon>0$ e $\alpha$ è il minimo valore positivo per cui vale questa cosa. Equivalentemente, $\alpha=\inf\{A \ge 0 \mid F(z) \ll_A e^{|z|^A}\}$. Scriviamo $ord(F)=\alpha$.

  Se non vale $F(z) \ll_A e^{|z|^A}$ per nessun $A$ diciamo che $F$ ha ordine infinito.
\end{defn}

\begin{ex}
  \begin{enumerate}
    \item Sia $p$ un polinomio di grado $k$, allora $|p(z)| \le |a_k||z|^k\big(1+o(1)\big) \ll_{\epsilon} e^{|z|^{\epsilon}}$ per ogni $\epsilon>0$, dunque $p$ ha ordine $0$.
    \item $|e^{az+b}|=e^{\mathfrak{Re}(az+b)} \ll_{\epsilon} e^{|z|^{1+\epsilon}}$ per ogni $\epsilon>0$.
    \item Più in generale, dato $p_k$ un generico polinomio di grado $k$ abbiamo che $|e^{p_k(z)}|=e^{\mathfrak{Re}\big(p_k(z)\big)} \le e^{|p_k(z)|} \le e^{|a_k||z|^k\big(1+o(1)\big)} \ll_{\epsilon} e^{|z|^{k+\epsilon}}$ per ogni $\epsilon>0$.

    Inoltre, prendendo $z$ sulla retta $arg(z)=-arg(a_k)/k$, abbiamo che vale $|e^{p_k(z)}|=e^{\mathfrak{Re}\big(p_k(z)\big)}=e^{a_kz^k\big(1+o(1)\big)}=e^{|a_k||z|^k\big(1+o(1)\big)} \gg_{\epsilon} e^{|a_k||z|^k}$. Dunque l'ordine è $k$ e l'$\inf$ nella definizione non viene raggiunto.
  \end{enumerate}
\end{ex}

\begin{oss}
  Se $F_1$ e $F_2$ hanno ordine $\alpha_1$ e $\alpha_2$, allora $F_1+F_2$ e $F_1 \cdot F_2$ hanno ordine minore o uguale di $\max\{\alpha_1,\alpha_2\}$. La dimostrazione è lasciata come esercizio per il lettore.
\end{oss}

\begin{lm} \label{1.2.4}
  Sia $f$ olomorfa in $|z-z_0| \le R$ non costantemente nulla e sia $0<r<R$. Sia inoltre $N=\sharp\{z \in \mathbb{C} \mid |z-z_0| \le r, f(z)=0\}$ (ricordiamo che sono contati con molteplicità). Allora
  $$|f(z_0)| \le \left(\frac{r}{R}\right)^N\max_{|z-z_0|=R}|f(z)|.$$
\end{lm}

\begin{proof}
  Consideriamo senza perdita di generalità, a meno di una traslazione e di un'omotetia, $R=1,z_0=0$. Siano $z_n$ gli zeri di $f$ in $|z| \le r$ contati con molteplicità e sia $\displaystyle g(z)=f(z)\prod_{n=1}^N \frac{1-\bar{z}_nz}{z-z_n}$. Per $|z|=1$ scriviamo $z=e^{i\theta},\theta \in \mathbb{R}$.
  Allora $\displaystyle \left|\frac{1-\bar{z}_ne^{i\theta}}{e^{i\theta}-z_n}\right|=|e^{i\theta}|\left|\frac{\bar{z}_n-e^{-i\theta}}{z_n-e^{i\theta}}\right|=1,$ quindi, per il principio del massimo modulo per funzioni olomorfe (che d'ora in avanti useremo senza menzionarlo esplicitamente), $\displaystyle |g(z)| \le \max_{|z|=1} |f(z)|$ per $|z| \le 1$. Si ha dunque
  \begin{gather*}
    |f(w)|=|g(w)|\prod_{n=1}^N\left|\frac{w-z_n}{1-\bar{z}_nw}\right| \le \max_{|z|=1} |f(z)| \prod_{n=1}^N \left|\frac{w-z_n}{1-\bar{z}_nw}\right| \implies \\
    \implies |f(0)| \le \max_{|z|=1} |f(z)| \prod_{n=1}^N |z_n| \le r^N \max_{|z|=1} |f(z)|.
  \end{gather*}
\end{proof}

\begin{cor} \label{1.2.5}
  Siano $f, r, R, N$ come nel lemma \ref{1.2.4}. Se $f(z_0)\not=0$ allora
  $$N \le \frac{1}{\log(R/r)}\log\left(\frac{\max_{|z-z_0|=R}|f(z)|}{|f(z_0)|}\right).$$
\end{cor}

\begin{proof}
  Basta prendere la disugaglianza data dal lemma \ref{1.2.4}, portarla nella forma $\left(\frac{R}{r}\right)^N \le (\dots)$, prendere il logaritmo e dividere per $\log(R/r)$.
\end{proof}

\begin{thm} \label{1.2.6}
  Sia $F$ una funzione intera di ordine $\alpha<+\infty$ e consideriamo $N(r)=\sharp\{z \in \mathbb{C} \mid F(z)=0, |z| \le r\}$. Allora $N(r) \ll_{\epsilon} r^{\alpha+\epsilon}$ per ogni $\epsilon>0$.
\end{thm}

\begin{proof}
  Prendiamo $R=2r$, allora $\displaystyle \max_{|z|=R}|F(z)| \ll_{\epsilon} e^{(2r)^{\alpha+\epsilon}}$ per ogni $\epsilon>0$ $\implies$ $\displaystyle \log\left(\max_{|z|=R}|F(z)|\right) \ll_{\epsilon} r^{\alpha+\epsilon}$ per ogni $\epsilon>0$.
  Se $F(0)\not=0$, per il corollario \ref{1.2.5} abbiamo $N(r) \ll_{\epsilon} r^{\alpha+\epsilon}$ per ogni $\epsilon>0$. Se $F(0)=0$, consideriamo $\tilde{F}(z)=F(z)/z^m$ dove $m$ è la molteplicità di $0$ come zero. Per $|z| \le 1$, $\tilde{F} \ll 1$ per continuità.
  Per $|z|>1$, $\tilde{F}(z)=F(z)\frac{1}{z^n} \implies ord(\tilde{F}) \le \max\{\alpha,0\}=\alpha$. Allora si ripete la dimostrazione per $\tilde{F}$, poi si osserva che il numero di zeri di $F$ varia solo per la costante additiva $m$.
\end{proof}

\begin{defn}
  Sia $z_n\not=0$ una successione senza limiti finiti. Si dice \textit{esponente di convergenza} di $z_n$, se esiste, il numero $$\beta=\inf\{B>0 \mid \sum_{n=1}^{+\infty} \frac{1}{|z_n|^B}<+\infty\},$$
  ovvero $\displaystyle \sum_{n=1}^{+\infty} \frac{1}{|z_n|^{\beta+\epsilon}}<+\infty$ per ogni $\epsilon>0$ e $\beta$ è il minimo valore per cui è vero.
\end{defn}

\begin{ex}
  $z_n=\log{n}$ non ha esponente di convergenza finito.
\end{ex}

\begin{thm} \label{1.2.9}
  Sia $F$ una funzione intera di ordine $\alpha>0$, $F(0)\not=0$ t.c. la successione dei suoi zeri $z_n$ ha esponente di convergenza $\beta$. Allora $\beta \le \alpha$.
\end{thm}

\begin{proof}
  Se $ord(F)=\alpha<+\infty$, per il teorema \ref{1.2.6} si ha $N(r) \ll_{\epsilon} r^{\alpha+\epsilon}$ per ogni $\epsilon>0$.
  Prendendo $r_n=|z_n|$, ricordando che $z_n$ sono gli zeri di $F$ otteniamo $n \le N(r_n) \ll_{\epsilon} r_n^{\alpha+\epsilon}=|z_n|^{\alpha+\epsilon}=|z_n|^{\alpha+\epsilon}$ per ogni $\epsilon>0$ (non è $n=N(r_n)$ perché potrebbe esserci più di uno zero sul cerchio $|z|=r_n$). Allora $|z_n| \gg_{\epsilon} n^{\frac{1}{\alpha+\epsilon}}$ e dunque
  $$\sum_{n=1}^{+\infty} |z_n|^{-(\alpha+2\epsilon)} \ll_{\epsilon} \sum_{n=1}^{+\infty} n^{-\frac{\alpha+2\epsilon}{\alpha+\epsilon}}<+\infty.$$
  Perciò $\beta \le \inf\{\alpha+2\epsilon \mid \epsilon>0\}=\alpha$.
\end{proof}

\begin{thm} \label{wfatt}
  Sia $F$ intera di ordine finito, $F(0)\not=0$. Allora la fattorizzazione di Weierstrass si scrive come
  \begin{equation} \label{wfattoformula}
    F(z)=e^{G(z)} \prod_{n=1}^{+\infty} \left(1-\frac{z}{z_n}\right)\exp\Bigg(\sum_{k=1}^p\frac{1}{k}\left(\frac{z}{z_n}\right)^k\Bigg),
  \end{equation}
  dove $p \ge 0$ è indipendente da $n$ e t.c. $\displaystyle \sum_n |z_n|^{-(p+1)}<+\infty$. Talvolta si trova scritta con la notazione $\displaystyle E(z,p)=(1-z)\exp\left(\sum_{k=1}^p\frac{1}{k}z^k\right)$ o simili.
\end{thm}

\begin{proof}
  Diamo solamente una traccia. Osserviamo che
  $$\sum_{n=1}^{+\infty} |m_n|\left|\frac{z}{z_n}\right|^{p+1}<+\infty \iff \sum_{n=1}^{+\infty} |z_n|^{-(p+1)}<+\infty\text{ per ogni }z \in \mathbb{C}.$$
  Scegliendo $p+1=\alpha+\epsilon$, per il teorema \ref{1.2.9} si ha $\displaystyle \sum_{n=1}^{+\infty} |z_n|^{-(p+1)}<+\infty$.
\end{proof}

\begin{oss}
  Sia $\beta$ l'esponente di convergenza di $z_n$, zeri di una funzione $F$ di ordine $\alpha$ ($\beta \le \alpha$). Il $p$ migliore è:
  \begin{enumerate}
    \item se $\beta$ non è intero, $p=\lfloor\beta\rfloor$;
    \item se $\beta$ è intero e nella definizione con l'$\inf$ è in realtà un minimo, allora $p=\beta-1$;
    \item se $\beta$ è intero ma nella definizione con l'$\inf$ questo non viene raggiunto, cioè $\beta$ non è un minimo, allora $p=\beta$.
  \end{enumerate}
  Abbiamo dunque $\beta-1 \le p \le \beta \le \alpha$. Scrivendo la fattorizzazione di Weierstrass come in \eqref{wfattoformula} usando il miglior $p$ possibile si ha la forma canonica.
\end{oss}


\subsection{Numeri e polinomi di Bernoulli e $\zeta(2k)$}
In questa sezione sfrutteremo i risultati visti finora per calcolare i valori della $\zeta$ di Riemann sui pari.

\begin{defn}
  Data $F$ intera, il \textit{genere} è $g=\max\{p,q\}$.
\end{defn}

\begin{oss}
  $\alpha-1 \le g \le \alpha$. La seconda è ovvia. Se fosse $g<\alpha-1$, avremmo $p<\alpha-1 \implies \beta \le p+1<\alpha$, ma anche $q<\alpha-1$; dato che abbiamo $\alpha=\max\{\beta,q\}$, si ha un assurdo.
\end{oss}

\begin{ex}
  Sia $F(z)=\dfrac{\sin(\pi z)}{\pi z}$. Si ha
  $$\sin(\pi z)=\frac{e^{i\pi z}-e^{-i\pi z}}{2i} \ll_{\epsilon} e^{|z|^{1+\epsilon}} \text{ per ogni } \epsilon>0,$$
  dunque $F$ ha ordine $\alpha \le 1$. Gli zeri sono $z_n=\pm n$ per ogni $n \in \mathbb{N}$, quindi $\displaystyle \sum_{n=1}^{+\infty} \frac{1}{|z_n|^{1+\epsilon}}<+\infty$ per ogni $\epsilon>0$ ma chiaramente non per $\epsilon=0$, da cui $\beta=1$;
  $\alpha \ge \beta \implies \alpha=1$. Inoltre $p=1$. Studiamo questa funzione.
\end{ex}

I termini del prodotto di Weierstrass sono $E\left(\dfrac{z}{n},1\right)=\left(1-\dfrac{z}{n}\right)\exp\left(\dfrac{z}{n}\right)$. $E\left(\dfrac{z}{n},1\right) \cdot E\left(\dfrac{z}{-n},1\right)=\left(1-\dfrac{z^2}{n^2}\right)$.
Abbiamo quindi il prodotto $\displaystyle \prod_n \left(1-\frac{z^2}{n^2}\right)$. Per il teorema di Hadamard, il grado del polinomio $G$ è $q \le \alpha=1=p$. Per il corollario \ref{1.2.14}, $G(z)=\log{F(0)}+\dfrac{F'}{F}(0)z=\log{1}+0=0$.
Abbiamo allora $\displaystyle \sin(\pi z)=\pi z\prod_{n=1}^{+\infty} \left(1-\frac{z^2}{n^2}\right)$.

Consideriamo
$$\log\big(F(z)\big)=\log\left(\frac{\sin(\pi z)}{\pi z}\right)=\log\big(\sin(\pi z)\big)\log(\pi z);$$
derivando troviamo
$$\frac{F'}{F}(z)=\pi\frac{\cos(\pi z)}{\sin(\pi z)}-\frac{1}{z}=\pi\cot(\pi z)-\frac{1}{z}.$$
Per $k \ge 2$, dal corollario \ref{1.2.14} si ha $\cot$
$$\sum_n \frac{1}{n^k}+\sum_n \frac{1}{(-n)^k}=-\frac{D^{k-1}\big(\pi\cot(\pi z)-1/z\big)_{z=0}}{(k-1)!}.$$
Passando ai pari, per $k \ge 1$ si ha
\begin{gather*}
  \zeta(2k)=\sum_{n=1}^{+\infty} \frac{1}{n^{2k}}=-\frac{D^{2k-1}\big(\frac{\pi}{2}\cot(\pi z)-\frac{1}{2z}\big)_{z=0}}{(2k-1)!} \implies \\
  \implies \frac{1}{2z}-\frac{\pi}{2}\cot(\pi z)=\sum_{k=1}^{+\infty} \zeta(2k)z^{2k-1} \text{ per } |z|<1.
\end{gather*}
È un esercizio verificare che $\zeta(2k)=1+O(1/2^{2k}) \sim 1$ per $k \longrightarrow +\infty$. Segue che $\sqrt[k]{\zeta(2k)} \longrightarrow 1$ per $k \longrightarrow +\infty$, ma anche, in particolare, $\sqrt[2k-1]{\zeta(2k)} \longrightarrow 1$ per $k \longrightarrow +\infty$, dunque il raggio di convergenza della funzione $\dfrac{1}{2z}-\dfrac{\pi}{2}\cot(\pi z)$ è proprio $1$.

\begin{defn}
  Si dicono \textit{numeri di Bernoulli} quelli definiti nel modo seguente:
  $$B_n=D^n\left(\frac{z}{e^z-1}\right)_{z=0}.$$
\end{defn}

\begin{oss}
  Poiché la funzione $\frac{z}{e^z-1}$ ha raggio di convergenza $2\pi$, dev'essere $\displaystyle \limsup \sqrt{\frac{B_n}{n!}}=\frac{1}{2\pi}$.
\end{oss}

\begin{oss}
  \begin{gather*}
    f(z)=\frac{z}{e^z-1}+\frac{z}{2}=\frac{z+ze^z}{2(e^z-1)}=\frac{z(1+e^z)}{2(e^z-1)} \\
    f(-z)=\frac{-ze^z}{1-e^z}-\frac{z}{2}=\frac{-ze^z-z}{2(1-e^z)}=\frac{z(1+e^z)}{2(e^z-1)}
  \end{gather*}
  Quindi $f$ è pari. Allora
  $$D^{2k-1}\left(\frac{z}{e^z-1}\right)_{z=0}=-D^{2k-1}\left(\frac{z}{2}\right)_{z=0}=\begin{cases}
    -1/2 & \mbox{se }k=1 \\ 0 & \mbox{se } k>1
\end{cases},$$
dunque $B_1=-1/2$ e $B_{2n-1}=0$ per ogni $n>1$.
\end{oss}

\begin{oss}
  \begin{gather*}
    1=\left(\sum_{n=0}^{+\infty}\frac{B_n}{n!}z^n\right)\frac{e^z-1}{z}=\left(\sum_{n=0}^{+\infty}\frac{B_n}{n!}z^n\right)\left(\sum_{m=1}^{+\infty}\frac{z^{m-1}}{m!}\right)= \\
    =\sum_{n=1}^{+\infty}\left(\sum_{k=0}^{n-1}\frac{B_k}{k!}\frac{n!}{(n-k)!}\right)\frac{z^{n-1}}{n!}=\sum_{n=1}^{+\infty}\Bigg(\sum_{k=0}^{n-1}\binom{n}{k}B_k\Bigg)\frac{z^{n-1}}{n!}
  \end{gather*}
  Perciò $B_0=1$ e $\displaystyle \sum_{k=0}^{n-1}\binom{n}{k}B_k=0$ per ogni $n \ge 2$, che ci dà $\displaystyle \sum_{k=0}^n\binom{n}{k}B_k=B_n$ e
  $$B_{n-1}=-\frac{1+\binom{n}{1}B_1+\dots+\binom{n}{n-2}B_{n-2}}{\binom{n}{n-1}} \implies B_n \in \mathbb{Q}.$$
\end{oss}

\begin{oss}
  \begin{gather*}
    \cot(z)=i\frac{e^{iz}+e^{-iz}}{e^{iz}-e^{-iz}}=i\frac{e^{2iz}+1}{e^{2iz}-1}=i\left(1+\frac{2z}{z(e^{2iz}-1)}\right)= \\
    =i+\frac{1}{z}\cdot\frac{2iz}{e^{2iz}-1}=i+\frac{1}{z}\left(1-iz+\sum_{k=1}^{+\infty}\frac{(-1)^kB_{2k}(2z)^{2k}}{(2k)!}\right)= \\
    =\frac{1}{z}+\sum_{k=1}^{+\infty}\frac{2(-1)^kB_{2k}}{(2k)!}(2z)^{2k-1} \implies \\
    \implies \sum_{k=1}^{+\infty}\zeta(2k)z^{2k-1}=\frac{1}{2z}-\frac{\pi}{2}\cot(\pi z)=\frac{1}{2}\sum_{k=1}^{+\infty} \frac{(-1)^{k-1}(2\pi)^{2k}B_{2k}z^{2k-1}}{(2k)!} \implies \\
    \implies 1+O(1/4^k)=\zeta(2k)=\frac{(-1)^{k-1}(2\pi)^{2k}B_{2k}}{2(2k)!} \implies \\
    \implies (-1)^{k-1}B_{2k}=\frac{2(2k)!}{(2\pi)^{2k}}\zeta(2k).
  \end{gather*}
\end{oss}

\begin{oss}
  $B_{2k}(-1)^{k-1}>0$. Inoltre,
  $$\frac{2(2k)!}{(2\pi)^{2k}} \le |B_{2k}| \le \frac{2(2k)!\pi^2}{6(2\pi)^{2k}} \text{ per ogni } k \ge 1.$$
  Vogliamo vedere quando si ha
  \begin{gather*}
    |B_{2(k+1)}| \ge \frac{2\big(2(k+1)\big)!}{(2\pi)^{2(k+1)}} \stackrel{?}{\ge} \frac{2(2k)!\pi^2}{6(2\pi)^{2k}} \ge |B_{2k}| \\
    \frac{2(k+1)(2k+1)}{4\pi^2} \stackrel{?}{\ge} \frac{\pi^2}{6} \\
    \pi^4 \stackrel{?}{\le} 3(k+1)(2k+1),
  \end{gather*}
  che è vero per $k \ge 4$. Da lì in poi, i numeri di Bernoulli di indice pari hanno moduli crescenti.
\end{oss}

\begin{defn}
  Si dice \textit{polinomio di Bernoulli} $n$-esimo il seguente:
  $$B_n(x)=\sum_{k=0}^n \binom{n}{k}B_kx^{n-k}.$$
\end{defn}

\begin{ftt}
  \begin{itemize}
    \item $B_n(0)=B_n$
    \item $\displaystyle B_n(1)=\sum_{k=0}^n \binom{n}{k}B_k=(-1)^nB_n$
  \end{itemize}
\end{ftt}

\begin{oss}
  \begin{gather*}
    \frac{ze^{xz}}{e^z-1}=\left(\sum_{n=0}^{+\infty}\frac{B_n}{n!}z^n\right)\left(\sum_{m=0}^{+\infty}\frac{x^mz^m}{m!}\right)=\sum_{n=0}^{+\infty}\left(\sum_{k=0}^n\frac{B_k}{k!}\cdot \frac{x^{n-k}}{(n-k)!}\right)z^n= \\
    =\sum_{n=0}^{+\infty}\left(\sum_{k=0}^nB_k\binom{n}{k}x^{n-k}\right)\frac{z^n}{n!}=\sum_{n=0}^{+\infty}\frac{B_n(x)}{n!}z^n.
  \end{gather*}
\end{oss}

Adesso un po' di fatti che chi vuole può divertirsi a dimostrare per esercizio.

\begin{ftt}
  \begin{enumerate}
    \item $\displaystyle B_n(x+y)=\sum_{k=0}^n\binom{n}{k}B_k(x)y^{n-k}$. Per $y=1$ si ha $\displaystyle B_n(x+1)=\sum_{k=0}^n\binom{n}{k}B_k(x)$;
    \item $B_n(x+1)-B_n(x)=nx^{n-1}$;
    \item $\displaystyle \sum_{k=m}^{n-1} k^r=\frac{B_{r+1}(n)-B_{r+1}(m)}{r+1}$;
    \item $B'_n(x)=nB_{n-1}(x)$.
  \end{enumerate}
\end{ftt}


\subsection{La funzione $\Gamma$ di Eulero}
Adesso, un caso particolare di un teorema che non dimostreremo; lo utilizzeremo per dimostrare una proposizione che potrebbe essere dimostrata anche con il lemma di sommazione di Abel.

\begin{thm}
  (formula di sommazione di Eulero-Maclaurin) Consideriamo $f:[a,b] \longrightarrow \mathbb{C}$ di classe $C^1$. Allora
  $$\sum_{a<k \le b}f(k)=\int_a^b f(x)\diff x-\bigg[B_1(\{x\})f(x)\bigg]_a^b+\int_a^b B_1(\{x\})f'(x)\diff x.$$
  Se $a=m, b=n$,
  $$\sum_{m<k \le n} f(k)=\int_m^nf(x)\diff x-B_1\cdot\big(f(n)-f(m)\big)+\int_m^n B_1(\{x\})f'(x)\diff x.$$
\end{thm}

\begin{cor} \label{EuMac-cor}
  $$\sum_{k=m}^n f(k)=\int_m^n f(x)\diff x+\frac{f(m)+f(n)}{2}+\int_m^n B_1(\{x\})f'(x)\diff x,$$
  dove ricordiamo che $B_1(\{x\})=\{x\}-1/2$.
\end{cor}

Riportiamo anche il lemma di Abel.

\begin{lm}
  (formula di sommazione di Abel)
  $$\sum_{k=m}^n a_kf(k)=\left(\sum_{k=m}^n a_k\right)f(n)-\int_m^n\left(\sum_{m \le k \le \lfloor x \rfloor}a_k\right)f'(x)\diff x.$$
\end{lm}

\begin{prop}
  Esiste
  $$\lim_{n \longrightarrow +\infty} \sum_{k=1}^n \frac{1}{k}-\log{n}=\gamma,$$
  con $0<\gamma<1$.
\end{prop}

\begin{proof}
  Applichiamo il corollario \ref{EuMac-cor} con $m=1$ e $f(x)=1/x$, otteniamo
  \begin{gather*}
    \sum_{k=1}^n \frac{1}{k}=\int_1^n \frac{\diff x}{x}+\frac{1}{2}+\frac{1}{2n}-\int_1^n \frac{\{x\}-1/2}{x^2}\diff x= \\
    =\log{n}+\frac{1}{2}+\frac{1}{2n}-\int_1^{+\infty} \frac{\{x\}-1/2}{x^2}\diff x+\int_n^{+\infty}\frac{\{x\}-1/2}{x^2}\diff x= \\
    =\log{n}+\frac{1}{2}+\frac{1}{2n}-\int_1^{+\infty} \frac{\{x\}}{x^2}\diff x+\frac{1}{2}+O\left(\int_n^{+\infty}\frac{\diff x}{x^2}\right)= \\
    =\log{n}+1-\int_1^{+\infty} \frac{\{x\}}{x^2}\diff x+O(1/n)=:\log{n}+\gamma+O(1/n).
  \end{gather*}
  Poiché $\displaystyle 0<\int_1^{+\infty} \frac{\{x\}}{x^2}\diff x<1$, si ha $0<\gamma<1$ con $\displaystyle \gamma=\lim_{n \longrightarrow +\infty} \sum_{k=1}^n \frac{1}{k}-\log{n}$.
\end{proof}

\begin{oss}
  Se $f:[1,+\infty) \longrightarrow \mathbb{R}$ è di classe $C^1$, infinitesima e non crescente, allora esiste $C>0$ t.c. $\displaystyle \sum_{1 \le k \le x} f(k)=\int_1^x f(y)\diff y+C+O\big(f(x)\big)$. La dimostrazione è lasciata per esercizio.
\end{oss}

\begin{oss}
  Si può dimostrare che
  $$\sum_{k=1}^n \frac{1}{k}=\log{n}+\gamma+\frac{1}{2n}-\sum_{r=1}^q \frac{B_{2r}}{2r}\cdot\frac{1}{n^{2r}}+O\left(\frac{1}{n^{2q+2}}\right).$$
\end{oss}

\begin{defn}
  Sia $\dfrac{1}{z\Gamma(z)}$ la funzione intera $F$ di ordine $1$ con zeri tutti semplici nei punti $-1,-2,-3,\dots$ e t.c. $F(0)=1$ e $F'(0)=\gamma$.
\end{defn}

Deve essere $q=\deg{G} \le 1$, $\beta=1$ e $\alpha=1$. $G(z)=\log\big(F(0)\big)+\frac{F'}{F}(0)=\gamma z$, quindi $q=1$. Vale anche $p=1$ e $g=1$. Si ha dunque
$$\frac{1}{z\Gamma(z)}=e^{\gamma z}\prod_{n=1}^{+\infty}\left(1+\frac{z}{n}\right)e^{-z/n}.$$

\begin{prop}
  (formula di Gauss) Per $z\not=-m$ con $m \in \mathbb{N}\cup\{0\}$ si ha
  \begin{equation}
    \Gamma(z)=\lim_{n \longrightarrow +\infty} \frac{n^zn!}{z(z+1)\cdots(z+n)}.
  \end{equation}
\end{prop}

\begin{proof}
  \begin{gather*}
    \frac{1}{\Gamma(z)}=ze^{\gamma z}\prod_{n=1}^{+\infty}\left(1+\frac{z}{n}\right)e^{-z/n}=\\
    =z\lim_{n \longrightarrow +\infty} e^{\left(\sum_{k=1}^n\frac{1}{k}-\log{n}\right)z}\prod_{k=1}^n\left(1+\frac{z}{n}\right)e^{-z/k}=z\lim_{n \longrightarrow +\infty} \prod_{k=1}^n\left(1+\frac{z}{k}\right)n^{-z} \implies \\
    \implies \Gamma(z)=\lim_{n \longrightarrow +\infty} \frac{n^z\prod_{k=1}^n k}{z\prod_{k=1}^n (k+z)}=\lim_{n \longrightarrow +\infty} \frac{n^zn!}{z(z+1)\cdots(z+n)}.
  \end{gather*}
\end{proof}

\begin{prop}
  Per $k \in \mathbb{N}\cup\{0\}$ si ha $\underset{z=-k}{\normalfont{\text{Res}}}\Gamma(z)=\dfrac{(-1)^k}{k!}$.
\end{prop}

\begin{proof}
  Fissato $k$, consideriamo il limite a partire da $n \ge k$. Vogliamo calcolare $\big(\Gamma(z)(z+k)\big)_{z=-k}$. Si ha
  \begin{gather*}
    \left(\frac{n^zn!(z+k)}{z(z+1)\cdots(z+n)}\right)_{z=-k}=\frac{n^{-k}n!}{-k(-k+1)\cdots(-2)\cdot(-1)\cdot1\cdot2\cdots(n-k)}= \\
    =\frac{(-1)^k}{k!}\left(\frac{n!}{(n-k)!n^k}\right) \longrightarrow \frac{(-1)^k}{k!} \text{ per } n \longrightarrow +\infty.
  \end{gather*}
\end{proof}

\begin{prop} \label{eulgamma}
  (Eulero) Per $z \not=-m$ con $m \in \mathbb{N}\cup\{0\}$ si ha la seguente formula:
  \begin{equation}
    \Gamma(z)=\frac{1}{z}\prod_{n=1}^{+\infty}\left(1+\frac{1}{n}\right)^z\left(1+\frac{z}{n}\right)^{-1}.
  \end{equation}
\end{prop}

\begin{proof}
  Dalla dimostrazione della formula di Gauss abbiamo trovato $\displaystyle \frac{1}{\Gamma(z)}=z\lim_{n \longrightarrow +\infty}\prod_{k=1}^n\left(1+\frac{z}{k}\right)n^{-z}$. Da $\displaystyle n=\prod_{k=1}^{n-1} \frac{k+1}{k}$ otteniamo $\displaystyle n^{-z}=\prod_{k=1}^{n-1}\left(1+\frac{1}{k}\right)^{-z}$. Si ha dunque
  \begin{gather*}
    z\lim_{n \longrightarrow +\infty}\prod_{k=1}^n\left(1+\frac{z}{k}\right)n^{-z}=z\lim_{n \longrightarrow +\infty} \prod_{k=1}^{n-1}\left(1+\frac{1}{k}\right)^{-z}\prod_{k=1}^n\left(1+\frac{z}{k}\right)=\\
    =z\lim_{n \longrightarrow +\infty} \prod_{k=1}^n \left(1+\frac{1}{k}\right)^{-z}\left(1+\frac{z}{k}\right)\left(1+\frac{1}{n}\right)^z=z\prod_{n=1}^{+\infty} \left(1+\frac{1}{n}\right)^{-z}\left(1+\frac{z}{n}\right).
  \end{gather*}
  Perciò
  $$\Gamma(z)=\frac{1}{z}\prod_{n=1}^{+\infty} \left(1+\frac{1}{n}\right)^z\left(1+\frac{z}{n}\right)^{-1}.$$
\end{proof}

\begin{prop} \label{fatt}
  Si ha $\Gamma(z+1)=z\Gamma(z)$ per $z \in \mathbb{C}$.
\end{prop}

\begin{proof}
  Dalla proposizione \ref{eulgamma} abbiamo
  \begin{gather*}
    \Gamma(z+1)=\frac{1}{z+1}\prod_{n=1}^{+\infty} \left(1+\frac{1}{n}\right)^z\left(1+\frac{1}{n}\right)\left(1+\frac{z+1}{n}\right)^{-1}= \\
    =\frac{1}{z+1}\prod_{n=1}^{+\infty}\left(1+\frac{1}{n}\right)^z\left(\frac{n+1+z}{n}\cdot\frac{n}{n+1}\right)^{-1}=\\
    =\frac{1}{z+1}\prod_{n=1}^{+\infty}\left(1+\frac{1}{n}\right)^z\left(1+\frac{z}{n+1}\right)^{-1}=\\
    =\lim_{n \longrightarrow +\infty} \prod_{k=1}^n \left(1+\frac{1}{k}\right)^z\frac{1}{z+1}\prod_{k=1}^n\left(1+\frac{z}{k+1}\right)^{-1}=\\
    =\lim_{n \longrightarrow +\infty}\prod_{k=1}^n\left(1+\frac{1}{k}\right)^z\frac{1}{z+1}\prod_{k=2}^{n+1}\left(1+\frac{z}{k}\right)^{-1}=\\
    =\lim_{n \longrightarrow +\infty}\prod_{k=1}^n\left(1+\frac{1}{k}\right)^z\prod_{k=1}^{n+1}\left(1+\frac{z}{k}\right)^{-1}= \\
    =\lim_{n \longrightarrow +\infty}\prod_{k=1}^n\left(1+\frac{1}{k}\right)^z\prod_{k=1}^n\left(1+\frac{z}{k}\right)^{-1}\frac{n+1}{n+1+z}=\\
    =\lim_{n \longrightarrow +\infty}\prod_{k=1}^n\left(1+\frac{1}{k}\right)^z\prod_{k=1}^n\left(1+\frac{z}{k}\right)^{-1}=\prod_{n=1}^{+\infty}\left(1+\frac{1}{n}\right)^z\left(1+\frac{z}{n}\right)^{-1}=z\Gamma(z).
  \end{gather*}
\end{proof}

\begin{oss}
  Abbiamo
  $$\Gamma(n+1)=n\Gamma(n)=n(n-1)\Gamma(n-1)=\dots=n(n-1)\cdots2\cdot\Gamma(1)=n!.$$
\end{oss}

\begin{prop} \label{gammaze1-z}
  Per $z \in \mathbb{C}$ si ha la relazione
  \begin{equation}
    \Gamma(z)\Gamma(1-z)=\frac{\pi}{\sin(\pi z)}.
  \end{equation}
\end{prop}

\begin{proof}
  Per le proposizioni \ref{fatt} e \ref{eulgamma} si ha
  \begin{gather*}
    z\Gamma(z)\Gamma(1-z)=\Gamma(1+z)\Gamma(1-z)=\\
    =\frac{1}{1-z^2}\prod_{n=1}^{+\infty} \left(1+\frac{1}{n}\right)^{z+1}\left(1+\frac{1}{n}\right)^{1-z}\left(1+\frac{z+1}{n}\right)^{-1}\left(1+\frac{1-z}{n}\right)^{-1}=\\
    =\frac{1}{1-z^2}\prod_{n=1}^{+\infty}\left(1+\frac{1}{n}\right)^2\Bigg(\left(1+\frac{1}{n}\right)^2-\frac{z^2}{n^2}\Bigg)^{-1}=\frac{1}{1-z^2}\prod_{n=1}^{+\infty} \left(\frac{\left(1+\frac{1}{n}\right)^2-\frac{z^2}{n^2}}{\left(1+\frac{1}{n}\right)^2}\right)^{-1}=\\
    =\frac{1}{1-z^2}\prod_{n=1}^{+\infty} \left(1-\frac{z^2}{(n+1)^2}\right)^{-1}=\frac{1}{1-z^2}\prod_{n=2}^{+\infty}\left(1-\frac{z^2}{n^2}\right)^{-1}=\\
    =\prod_{n=1}^{+\infty}\left(1-\frac{z^2}{n^2}\right)^{-1}=\frac{\pi z}{\sin(\pi z)}.
  \end{gather*}
\end{proof}

Per $n \ge 1$, vale la seguente formula di moltiplicazione dovuta a Gauss:
$$\prod_{k=0}^{n-1}\Gamma\left(z+\frac{k}{n}\right)=(2\pi)^{\frac{n-1}{2}}n^{\frac{1}{2}-nz}\Gamma(nz).$$
Per i nostri scopi ci servirà solo un caso particolare, che è quello che dimostreremo.

\begin{prop}
  (formula di duplicazione di Legendre)
  \begin{equation}
    \Gamma(z)\Gamma(z+1/2)=\sqrt{\pi}2^{1-2z}\Gamma(2z).
  \end{equation}
\end{prop}

\begin{proof}
  Dalla formula di Gauss abbiamo
  \begin{gather*}
    \Gamma(z)=\lim_{m \longrightarrow +\infty} \frac{m^zm!}{z(z+1)\cdots(z+m)}= \\
    =\lim_{m \longrightarrow+\infty}\frac{m^z(m-1)!}{z(z+1)\dots(z+m-1)}\cdot\frac{m}{z+m}=\lim_{m \longrightarrow+\infty}\frac{m^z(m-1)!}{z(z+1)\dots(z+m-1)}.
  \end{gather*}
  Dunque
  \begin{gather*}
    \frac{2^{2z-1}\Gamma(z)\Gamma\left(z+\frac{1}{2}\right)}{\Gamma(2z)}=\\
    =\lim_{m \longrightarrow +\infty} \frac{2^{2z-1}}{\frac{(2m)^{2z}(2m-1)!}{2z(2z+1)\cdots(2z+2m-1)}}\cdot\frac{m^z(m-1)!}{z(z+1)\cdots(z+m-1)}\cdot\frac{m^{z+1/2}(m-1)!}{(z+1/2)(z+3/2)\cdots(z+m-1/2)}=\\
    =\lim_{m \longrightarrow +\infty} \frac{2^{2z-1}m^{2z+1/2}\big((m-1)!\big)^22^m(2z)(2z+1)\cdots(2z+2m-1)}{(2m)^{2z}(2m-1)!z(z+1)\cdots(z+m-1)(2z+1)(2z+3)\cdots(2z+2m-1)}= \\
    \lim_{m \longrightarrow +\infty} \frac{2^{2m-1}m^{1/2}\big((m-1)!\big)^2}{(2m-1)!}.
  \end{gather*}
  Questa quantità non dipende da $z$, perciò è costante. Per la proposizione \ref{gammaze1-z} si ha $\Gamma^2(1/2)=\frac{\pi}{\sin(\pi/2)}=\pi \implies \Gamma(1/2)=\sqrt{\pi}$; allora la costante è proprio $\sqrt{\pi}$, come voluto.
\end{proof}

\begin{oss}
  $\Gamma(1/2)\Gamma(1)=\sqrt{\pi}\Gamma(2) \implies \Gamma(1/2)=\sqrt{\pi}$.

  $\Gamma(3/2)=\frac{1}{2}\Gamma(1/2)=\sqrt{\pi}/2<1$.
\end{oss}

\begin{oss}
  \begin{gather*}
    \frac{\Gamma\left(\frac{1}{2}-\frac{z}{2}\right)}{\Gamma\left(\frac{z}{2}\right)}=\frac{\Gamma\left(\frac{1}{2}-\frac{z}{2}\right)\Gamma\left(1-\frac{z}{2}\right)}{\Gamma\left(\frac{z}{2}\right)\Gamma\left(1-\frac{z}{2}\right)}=\\
    =\frac{\sin(\pi z/2)}{\pi}\sqrt{\pi}2^{1-(1-z)}\Gamma(1-z)=\frac{\sin(\pi z/2)}{\sqrt{\pi}}2^z\Gamma(1-z).
  \end{gather*}
\end{oss}

\begin{prop} \label{Gammagamma}
  Fuori dai poli di $\Gamma$ si ha $\displaystyle \frac{\Gamma'}{\Gamma}(z)=-\gamma-\frac{1}{z}+\sum_{n=1}^{+\infty} \frac{z}{n(n+z)}$.
\end{prop}

\begin{proof}
  \begin{gather*}
    \Gamma(z)=\frac{1}{z}e^{-\gamma z}\prod_{n=1}^{+\infty}\left(1+\frac{z}{n}\right)^{-1}e^{z/n} \\
    \log\big(\Gamma(z)\big)=-\log{z}-\gamma z+\sum_{n=1}^{+\infty} \Bigg(\frac{z}{n}-\log\left(1+\frac{z}{n}\right)\Bigg) \\
    \frac{\Gamma'}{\Gamma}(z)=-\gamma-\frac{1}{z}+\sum_{n=1}^{+\infty} \left(\frac{1}{n}-\frac{1}{n+z}\right)=-\gamma-\frac{1}{z}+\sum_{n=1}^{+\infty} \frac{z}{n(n+z)}.
  \end{gather*}
\end{proof}

\begin{oss}
  Dalla proposizione \ref{Gammagamma} si ha
  $$\Gamma'(1)=-\gamma-1+\sum_{n=1}^{+\infty} \frac{1}{n(n+1)}=-\gamma.$$
\end{oss}

L'integrale $\displaystyle \int_0^{+\infty} e^{-x}x^{z-1}\diff x$ converge per $\mathfrak{Re}z>0$ e definisce una funzione olomorfa in $z$. Vale il seguente risultato.

\begin{thm}
  Per $\mathfrak{Re}z>0$ si ha
  \begin{equation}
    \Gamma(z)=\int_0^{+\infty} e^{-x}x^{z-1}\diff x.
  \end{equation}
\end{thm}

\begin{proof}
  Vogliamo dimostrare due cose:
  \begin{enumerate}
    \item $\displaystyle \lim_{n \longrightarrow +\infty} \int_0^n \left(1-\frac{x}{n}\right)^nx^{z-1}\diff x=\int_0^{+\infty} e^{-x}x^{z-1}\diff x$;
    \item $\displaystyle \int_0^n \left(1-\frac{x}{n}\right)^nx^{z-1}\diff x=\frac{n^zn!}{z(z+1)\cdots(z+n)}$.
  \end{enumerate}
  La tesi seguirà dalla formula di Gauss.
  \begin{enumerate}
    \item Il primo passo è mostrare la seguente catena di disuguaglianze:
    $$0 \le e^{-x}-\left(1-\frac{x}{n}\right)^n \le \frac{x^2}{n}e^{-x}. \qquad (\star)$$
    Notiamo che si ha
    \begin{gather*}
      1+\frac{x}{n} \le 1+\frac{x}{n}+\frac{x^2}{2!n^2}+\frac{x^3}{3!n^3}+\dots \le 1+\frac{x}{n}+\frac{x^2}{n^2}+\dots \iff \\
      \iff 1+\frac{x}{n} \le e^{x/n} \le \frac{1}{1-x/n} \implies \\
      \implies \left(1+\frac{x}{n}\right)^n \le e^x \text{ e } \left(1-\frac{x}{n}\right)^n \le e^{-x} \implies \\
      \implies e^{-x}-\left(1-\frac{x}{n}\right)^n=e^{-x}\Bigg(1-e^x\left(1-\frac{x}{n}\right)^n\Bigg) \le e^{-x}\Bigg(1-\left(1-\frac{x^2}{n^2}\right)^n\Bigg).
    \end{gather*}
    Ricordiamo la disuguaglianza di Bernoulli: per $\alpha \ge 0$ e $n \in \mathbb{N}\cup\{0\}$ abbiamo che $(1-\alpha)^n \ge 1-n\alpha$. Allora
    $$e^{-x}\Bigg(1-\left(1-\frac{x^2}{n^2}\right)^n\Bigg) \le e^{-x}\Bigg(1-\left(1-\frac{x^2}{n}\right)\Bigg)=\frac{x^2}{n}e^{-x}.$$
    Ne deduciamo che
    \begin{gather*}
      \int_0^n \left(1-\frac{x}{n}\right)^nx^{z-1}\diff x=\int_0^n e^{-x}x^{z-1}\diff x-\left(\int_0^n e^{-x}x^{z-1}\diff x-\int_0^n\left(1-\frac{x}{n}\right)^nx^{z-1}\diff x\right) \text{ e } \\
      \left|\int_0^n\Bigg(e^{-x}-\left(1-\frac{x}{n}\right)^n\Bigg)x^{z-1}\diff x\right| \le \int_0^n x^{\mathfrak{Re}z-1}\Bigg(e^{-x}-\left(1-\frac{x}{n}\right)^n\Bigg)\diff x \le \\
      \le \frac{1}{n} \int_0^n x^{\mathfrak{Re}z-1+2}e^{-x}\diff x \ll \frac{1}{n} \implies \\
      \implies \lim_{n \longrightarrow +\infty} \int_0^n \left(1-\frac{x}{n}\right)^nx^{z-1}\diff x=\int_0^{+\infty} e^{-x}x^{z-1}\diff x.
    \end{gather*}
    \item Effettuando il cambio di variabile $y=x/n$ troviamo
    $$\int_0^n \left(1-\frac{x}{n}\right)^nx^{z-1}\diff x=n^z\int_0^1(1-y)^ny^{z-1}\diff y;$$
    integrando per parti più volte si ottiene che
    \begin{gather*}
      \int_0^1 (1-y)^ny^{z-1}\diff y=\frac{n}{z}\int_0^1 (1-y)^{n-1}y^z\diff y=\dots= \\
      =\frac{n(n-1)\cdots 2\cdot 1}{z(z+1)\cdots(z+n-1)}\int_0^1 y^{z+n-1}\diff y=\frac{n!}{z(z+1)\cdots(z+n)}.
    \end{gather*}
  \end{enumerate}
\end{proof}

\begin{thm}
  (Stirling) Sia $\epsilon>0$. Si ha la seguente formula asintotica:
  \begin{equation}
    \log\big(\Gamma(z)\big)=\left(z-\frac{1}{2}\right)\log{z}-z+\log{\sqrt{2\pi}}+O_{\epsilon}\left(\frac{1}{|z|}\right)
  \end{equation}
  uniformemente per $|z| \ge \epsilon$ e $|\arg{z}| \le \pi-\epsilon$. In particolare,
  \begin{align*}
    \log{n!}&=(n+1/2)\Bigg(\log{n}+\frac{1}{n}+O\left(\frac{1}{n^2}\right)\Bigg)-n-1+\log{\sqrt{2\pi}}+O\left(\frac{1}{|n+1|}\right)= \\
    &=(n+1/2)\log{n}-n+\log{\sqrt{2\pi}}+O\left(\frac{1}{n}\right).
  \end{align*}
\end{thm}

\begin{proof}
  Abbiamo $\displaystyle \log\big(\Gamma(z)\big)=\lim_{n \longrightarrow +\infty} \log\left(\frac{n^zn!}{z(z+1)\cdots(z+n)}\right)$. Notiamo che
  $$\log\left(\frac{n^zn!}{z(z+1)\cdots(z+n)}\right)=z\log{n}-\log(z+n)-\sum_{k=1}^n \log\left(1+\frac{z-1}{k}\right).$$
  Applicando la formula di Eulero-Maclaurin alla funzione $f_z(x)=\log\left(1+\frac{z-1}{x}\right)$ troviamo
  \begin{gather*}
    \sum_{k=1}^n \log\left(1+\frac{z-1}{k}\right)=\int_1^n \log\left(1+\frac{z-1}{x}\right)\diff x+\\
    +\frac{1}{2}\big(\log{z}+\log(z+n-1)-\log{n}\big)+\int_1^n B_1(\{x\})\left(\frac{1}{z+x-1}-\frac{1}{x}\right)\diff x \implies \\
    \implies \log\left(\frac{n^zn!}{z(z+1)\cdots(z+n)}\right)=\dots=\left(z-\frac{1}{2}\right)\log{z}-\left(z+n+\frac{1}{2}\right)\log\left(1+\frac{z-1}{n}\right)+\\
    +\log\left(1-\frac{1}{z+n}\right)-\int_1^n B_1(\{x\})\left(\frac{1}{z+x-1}-\frac{1}{x}\right)\diff x;
  \end{gather*}
  poiché $\displaystyle \left(z+n+\frac{1}{2}\right)\log\left(1+\frac{z-1}{n}\right) \sim \frac{z-1}{n}\left(z+n+\frac{1}{2}\right) \overset{n \rightarrow +\infty}{\longrightarrow} z-1$ con un resto di $O_{\epsilon}(1/n)$, si ha
  \begin{gather*}
    \log\left(\frac{n^zn!}{z(z+1)\cdots(z+n)}\right)=\left(z-\frac{1}{2}\right)\log{z}-z+1+O_{\epsilon}\left(\frac{1}{n}\right)+\\
    -\int_1^{+\infty} B_1(\{x\})\left(\frac{1}{z+x-1}-\frac{1}{x}\right)\diff x+\int_n^{+\infty}B_1(\{x\})\left(\frac{1}{z+x-1}-\frac{1}{x}\right)\diff x=\\
    =\left(z-\frac{1}{2}\right)\log{z}-z+C+O_{\epsilon}\left(\frac{1}{n}\right)+R_z(n).
  \end{gather*}
  Ricordiamo che $DB_k(x)=kB_{k-1}(x)$ e che $\{x\}-1/2=x-\lfloor x\rfloor-1/2$, dunque una primitiva di $B_1(\{x\})$ è $\frac{(x-\lfloor x\rfloor)^2}{2}-\frac{x-\lfloor x\rfloor}{2}+\frac{1}{12}=\frac{1}{2}B_2(\{x\})$; allora integrando per parti si ha
  \begin{gather*}
    \int_1^{+\infty} \frac{B_1(\{x\})}{z+x-1}\diff x=\left[\frac{B_2(\{x\})}{2(z+x-1)}\right]_1^{+\infty}+\int_1^{+\infty} \frac{B_2(\{x\})}{2(z+x-1)^2}\diff x=\\
    =-\frac{1}{12z}+\int_1^{+\infty} \frac{B_2(\{x\})}{2(z+x-1)^2}\diff x.
  \end{gather*}
  Poniamo anche $\displaystyle C=1+\int_1^{+\infty} \frac{B_1(\{x\})}{x}\diff x$. Vogliamo vedere se
  $$-\frac{1}{2}\int_1^{+\infty}\frac{B_2(\{x\})}{(z+x-1)^2}\diff x=-\frac{1}{2}\int_0^{+\infty}\frac{B_2(\{x\})}{(z+x)^2}\diff x \overset{?}{\ll_{\epsilon}} \frac{1}{|z|}.$$
  Si ha
  $$\left|-\frac{1}{2}\int_0^{+\infty}\frac{B_2(\{x\})}{(z+x)^2}\diff x\right| \le \int_0^{+\infty} \frac{\diff x}{|z+x|^2}.$$
  Da $|\arg{z}| \le \pi-\epsilon$, abbiamo $|z+x|^2=|z|^2+x^2+2\langle z,x \rangle \ge |z|^2+x^2-2|z|x\cos{\epsilon}$; scrivendo $y=x-|z|\cos{\epsilon}$ e cambiando di variabile nei passaggi giusti troviamo
  \begin{gather*}
    \int_0^{+\infty} \frac{\diff x}{|z+x|^2} \le \int_{-|z|\cos{\epsilon}}^{+\infty} \frac{\diff y}{|z|^2\sin^2{\epsilon}+y^2} \le \int_{-\infty}^{+\infty} \frac{\diff y}{|z|^2\sin^2{\epsilon}+y^2}= \\
    =\frac{1}{|z|\sin{\epsilon}}\int_{-\infty}^{+\infty} \frac{\diff y}{y^2+1} \ll_{\epsilon} \frac{1}{|z|}.
  \end{gather*}
  Ricordiamo adesso che
  \begin{gather*}
    \Gamma(z)\Gamma(z+1/2)=\sqrt{\pi}2^{1-2z}\Gamma(2z) \implies \\
    \implies \log\big(\Gamma(z)\big)+\log\big(\Gamma(z+1/2)\big)=\frac{1}{2}\log{\pi}+(1-2z)\log{2}+\log\big(\Gamma(2z)\big);
  \end{gather*}
  si ha anche $\log\big(\Gamma(z)\big)=(z-1/2)\log{z}-z+C+O_{\epsilon}(1/|z|)$, da cui con un po' di conti si trova che $C-\frac{1}{2}\log(2\pi) \ll_{\epsilon} \frac{1}{|z|} \implies C=\frac{1}{2}\log(2\pi)$.
\end{proof}

\begin{cor} \label{gammaprimosu}
  Se $|z| \ge \epsilon$ e $|\arg{z}| \le \pi-\epsilon$ si ha $$\frac{\Gamma'}{\Gamma}(z)=\log{z}+O_{\epsilon}\left(\frac{1}{|z|}\right).$$
\end{cor}

\begin{proof}
  Sia $f(z)=\log\big(\Gamma(z)\big)-(z-1/2)\log{z}+z-\frac{1}{2}\log(2\pi)$. Preso $z_0$ che soddisfi le ipotesi e $z$ t.c. $|z-z_0|=\epsilon/2$, abbiamo $f(z) \ll_{\epsilon} \frac{1}{|z|} \ll \frac{1}{|z_0|}$. Per la formula integrale di Cauchy troviamo
  $$f'(z_0)=\frac{1}{2\pi i}\oint_{\gamma} \frac{f(z)}{(z-z_0)^2}\diff z \ll_{\epsilon} \frac{1}{|z_0|},$$
  dove $\gamma$ è il cerchio di centro $z_0$ e raggio $\epsilon/2$. La tesi segue notando che vale $f'(z)=\dfrac{\Gamma'}{\Gamma}(z)-\log{z}+\dfrac{1}{2z}$.
\end{proof}

\begin{cor} \label{gammaord}
  Per $|z| \ge \epsilon$ e $|\arg{z}| \le \pi-\epsilon$ si ha
  $$\Gamma(z)=\sqrt{\frac{2\pi}{z}}\left(\frac{z}{e}\right)^z\Bigg(1+O_{\epsilon}\left(\frac{1}{|z|}\right)\Bigg).$$
  In particolare $\displaystyle n!=\sqrt{2\pi n}\left(\frac{n}{e}\right)^n\Bigg(1+O\left(\frac{1}{n}\right)\Bigg)$.
\end{cor}

\begin{proof}
  Per esercizio.
\end{proof}

\begin{cor}
  Se $k \ge 1$, allora $(-1)^kB_{2k}=4\sqrt{\pi k}\left(\dfrac{k}{e\pi}\right)^{2k}\Bigg(1+O\left(\dfrac{1}{k}\right)\Bigg)$.
\end{cor}

\begin{proof}
  Si ha infatti $(-1)^kB_{2k}=\dfrac{2(2k)!}{(2\pi)^{2k}}\zeta(2k)$, ma
  $$\zeta(2k)=\sum_{n=1}^{+\infty} \frac{1}{n^{2k}} \le 1+\int_1^{+\infty} \frac{\diff x}{x^{2k}}=1+\frac{1}{2k-1}=1+O\left(\frac{1}{k}\right).$$
\end{proof}

\begin{oss}
  Si ritrova il raggio di convergenza di $\dfrac{z}{e^z-1}$.
\end{oss}

\begin{cor}
  Sia $z=x+iy$ con $x_1 \le x \le x_2$. Allora
  $$\left|\Gamma(x+iy)\right|=\sqrt{2\pi}|y|^{x-\frac{1}{2}}e^{-\frac{\pi}{2}|y|}\Bigg(1+O\left(\frac{1}{|y|}\right)\Bigg).$$
\end{cor}

\begin{proof}
  \begin{gather*}
    \log\big(\Gamma(x+iy)\big)=\left(x-\frac{1}{2}+iy\right)\log(x+iy)-x-iy+\frac{1}{2}\log(2\pi)+O\left(\frac{1}{|y|}\right) \\
    \mathfrak{Re}\Big(\log\big(\Gamma(x+iy)\big)\Big)=\mathfrak{Re}\Big((x-1/2+iy)\big(\log(iy)+\log(1-i\cdot x/y)\big)\Big)-x+\frac{1}{2}\log(2\pi)+O\left(\frac{1}{|y|}\right)= \\
    =(x-1/2)\log|y|+iy\left(\frac{\pi}{2}\cdot i\cdot\text{sgn}(y)\right)+iy\left(-\frac{ix}{y}\right)-x+\frac{1}{2}\log(2\pi)+O\left(\frac{1}{|y|}\right)=\\
    =(x-1/2)\log|y|-|y|\frac{\pi}{2}+\frac{1}{2}\log(2\pi)+O\left(\frac{1}{|y|}\right)
  \end{gather*}
\end{proof}


\newpage

\section{A caccia di zeri}

\subsection{Lo spazio di Schwarz}
\begin{defn}
  Sia $f:\mathbb{R} \longrightarrow \mathbb{C}$; si dice che $f$ \textit{tende rapidamente} a $0$ per $|x| \longrightarrow +\infty$ se $\displaystyle \lim_{x \longrightarrow \pm \infty} |x|^nf(x)=0$ per ogni $n \in \mathbb{N}\cup\{0\}$.
\end{defn}

\begin{oss} \label{rapsse}
  $f$ tende rapidamente a $0$ se e solo se $f(x)|x|^n$ è limitata per ogni $n \in \mathbb{N}\cup\{0\}$.
\end{oss}

\begin{defn}
  Si dice \textit{spazio di Schwarz} $\mathcal{S}$ lo spazio su $\mathbb{C}$ delle funzioni $f \in C^{\infty}(\mathbb{R})$ (a valori complessi) tendenti rapidamente a $0$ insieme a tutte le loro derivate.
\end{defn}

\begin{oss}
  L'operatore $D^k$ manda $\mathcal{S}$ in sé per ogni $k \ge 0$.
\end{oss}

Notazione: indichiamo con $M^k$ l'operatore $(M^kf)(x)=x^kf(x)$; abbiamo che anche $M^k$ manda $\mathcal{S}$ in sé.

Consideriamo ora la trasformata di Fourier in $\mathcal{S}$ definita come
$$\hat{f}(\xi)=\int_{\mathbb{R}} f(x)e^{-2\pi\xi x}\diff x.$$
Si ha che è ben definita.

\begin{oss} \label{limFourier}
  $\displaystyle |\hat{f}(\xi)| \le \int_{\mathbb{R}} |f(x)|\diff x<+\infty$, quindi $\hat{f}$ è limitata se $f \in \mathcal{S}$.
\end{oss}

\begin{lm}
  L'operatore $\textasciicircum$ manda $\mathcal{S}$ in sé.
\end{lm}

\begin{proof}
  Derivando sotto il segno di integrale abbiamo
  \begin{gather*}
    \hat{f}'(\xi)=-2\pi i \int_{-\infty}^{+\infty} xf(x)e^{-2\pi i\xi x}\diff x=-2\pi i \widehat{Mf}(\xi) \implies \\
    \implies D\hat{f}=(-2\pi i)\widehat{Mf} \implies D^k\hat{f}=(-2\pi i)^k\widehat{M^kf} \implies \hat{f} \in C^{\infty}(\mathbb{R}).
  \end{gather*}
  Integrando per parti si ha
  \begin{gather*}
    \xi\hat{f}(\xi)=\int_{-\infty}^{+\infty} f(x)\xi e^{-2\pi i\xi x}\diff x= \\
    =\left[-\frac{1}{2\pi i}e^{-2\pi i\xi x}f(x)\right]_{-\infty}^{+\infty}+\frac{1}{2\pi i}\int_{-\infty}^{+\infty} f'(x)e^{-2\pi i\xi x}\diff x=\frac{1}{2\pi i}\widehat{Df}(\xi) \implies \\
    \implies M^k\hat{f}=\left(\frac{1}{2\pi i}\right)^k\widehat{D^kf}.
  \end{gather*}
  Vogliamo concludere applicando l'osservazione \ref{rapsse} alle funzioni $D^k\hat{f}$. Notiamo che
  $$M^hD^k\hat{f}=M^h(-2\pi i)^k\widehat{M^kf}=\left(\frac{1}{2\pi i}\right)^h(-2\pi i)^k\widehat{D^hM^kf};$$
  ci basta dunque mostrare che $\widehat{D^hM^kf}$ è limitata, ma questo segue dall'osservazione \ref{limFourier} e dal fatto che gli operatori $D$ e $M$ mandano $\mathcal{S}$ in sé.
\end{proof}

\begin{lm}
  (formula di Poisson) Se $f \in \mathcal{S}$ allora
  $$\sum_{n \in \mathbb{Z}} f(n)=\sum_{n \in \mathbb{Z}} \hat{f}(n).$$
\end{lm}

\begin{proof}
  Sia $\displaystyle g(x)=\sum_{n \in \mathbb{Z}} f(x+n)$. $g$ ha periodo $1$. Poiché $f \in \mathcal{S}$, abbiamo che $\displaystyle \sum_{n \in \mathbb{Z}} D^kf(x+n)$ converge uniformemente per ogni $k \ge 0$, dunque è uguale a $D^kg$, quindi $g \in C^{\infty}(\mathbb{R})$. Scriviamo $g$ in serie di Fourier: $\displaystyle g(x)=\sum_{m \in \mathbb{Z}} c_me^{2\pi imx}$. Si ha
  \begin{gather*}
    c_m=\int_0^1 g(x)e^{-2\pi imx}\diff x=\int_0^1 \sum_{n \in \mathbb{Z}} f(x+n)e^{-2\pi imx}\diff x=\\
    =\sum_{n \in \mathbb{Z}}\int_0^1 f(x+n)e^{-2\pi imx}\diff x \overset{y=x+n}{=} \sum_{n \in \mathbb{Z}} \int_n^{n+1} f(y)e^{-2\pi im(y-n)}\diff y=\\
    =\int_{\mathbb{R}} f(x)e^{-2\pi imx}\diff x=\hat{f}(m).
  \end{gather*}
  Basta allora guardare $g(0)$.
\end{proof}

\begin{lm}
  Sia $f(x)=e^{-\pi x^2}$ per $x \in \mathbb{R}$. Allora $f \in \mathcal{S}$ e inoltre $\hat{f}=f$.
\end{lm}

\begin{proof}
  Che $f \in \mathcal{S}$ è facile da dimostrare.
  \begin{gather*}
    \hat{f}(\xi)=\int_{\mathbb{R}} e^{-\pi x^2-2\pi i\xi x}\diff x \implies \\
    \implies D\hat{f}(\xi)=-2\pi i\int_{\mathbb{R}} xe^{-\pi x^2-2\pi i\xi x}\diff x \overset{\text{per parti}}{=} \\
    =\left[ie^{-\pi x^2-2\pi i\xi x}\right]_{-\infty}^{+\infty}+i(2\pi i\xi)\int_{\mathbb{R}} e^{-\pi x^2-2\pi i\xi x}\diff x=-2\pi \xi \hat{f}(\xi).
  \end{gather*}
  Abbiamo
  \begin{gather*}
    u'(\xi)=-2\pi\xi u(\xi) \implies \frac{u'}{u}(\xi)=-2\pi\xi \implies \\
    \implies \log\big(u(\xi)\big)=-\pi\xi^2+c \implies u(\xi)=Ce^{-\pi\xi^2} \implies \\
    \implies \hat{f}(\xi)=Ce^{-\pi\xi^2}.
  \end{gather*}
  Si ha anche $\displaystyle \hat{f}(0)=\int_{\mathbb{R}} e^{-\pi x^2}\diff x=1 \implies C=1$.
\end{proof}

\begin{oss}
  La serie $\displaystyle \sum_{n \in \mathbb{Z}} e^{-\pi n^2 z}$ converge totalmente per $\mathfrak{Re}\,z \ge \epsilon>0$.
\end{oss}

\begin{defn}
  Sia $z=x+iy$ con $x>0$. Si dice \textit{funzione $\vartheta$ di Jacobi} la seguente serie totalmente convergente:
  $$\vartheta(z)=\sum_{n \in \mathbb{Z}} e^{-\pi n^2z}.$$
\end{defn}

\begin{lm}
  Per $x=\mathfrak{Re}\,z>0$ si ha
  $$\vartheta(z)=\frac{1}{\sqrt{z}}\vartheta\left(\frac{1}{z}\right).$$
\end{lm}

\begin{proof}
  Possiamo dimostrare la formula per $z=x>0$, che valga in tutto il semipiano $\mathfrak{Re}\,z>0$ segue per prolungamento analitico. Sia $f(\xi)=e^{-\pi\xi^2}$ e sia $f_x(\xi)=f(\sqrt{x}\xi)=e^{-\pi x\xi^2}$. Allora
  \begin{gather*}
    \hat{f}_x(\xi)=\int_{\mathbb{R}} f(\sqrt{x}t)e^{-2\pi i\xi t}\diff t \overset{s=t\sqrt{x}}{=} \frac{1}{\sqrt{x}}\int_{\mathbb{R}} f(s)e^{-2\pi i\frac{\xi}{\sqrt{x}}s}\diff s= \\
    =\frac{1}{\sqrt{x}}\hat{f}\left(\frac{\xi}{\sqrt{x}}\right)=\frac{1}{\sqrt{x}}f\left(\frac{\xi}{\sqrt{x}}\right)=\frac{1}{\sqrt{x}}f_{\frac{1}{x}}(\xi).
  \end{gather*}
  Applicando la formula di Poisson abbiamo
  \begin{gather*}
    \sum_{n \in \mathbb{Z}} f_x(n)=\sum_{n \in \mathbb{Z}} \hat{f}_x(n)=\sum_{n \in \mathbb{Z}} \frac{1}{\sqrt{x}}f_{\frac{1}{x}}(n),
  \end{gather*}
  quindi
  $$\vartheta(x)=\sum_{n \in \mathbb{Z}} e^{-\pi n^2x}=\frac{1}{\sqrt{x}}\sum_{n \in \mathbb{Z}}e^{-\frac{\pi}{x}n^2}=\frac{1}{\sqrt{x}}\vartheta\left(\frac{1}{x}\right).$$
\end{proof}


\subsection{La memoria di Riemann}
\begin{defn}
  Sia $z=x+iy$ con $x>0$. Si dice \textit{funzione $\vartheta$ di Jacobi} la seguente serie totalmente convergente:
  $$\vartheta(z)=\sum_{n \in \mathbb{Z}} e^{-\pi n^2z}.$$
\end{defn}

\begin{lm}
  Per $x=\mathfrak{Re}\,z>0$ si ha
  $$\vartheta(z)=\frac{1}{\sqrt{z}}\vartheta\left(\frac{1}{z}\right).$$
\end{lm}

\begin{proof}
  Possiamo dimostrare la formula per $z=x>0$, che valga in tutto il semipiano $\mathfrak{Re}\,z>0$ segue per prolungamento analitico. Sia $f(\xi)=e^{-\pi\xi^2}$ e sia $f_x(\xi)=f(\sqrt{x}\xi)=e^{-\pi x\xi^2}$. Allora
  \begin{gather*}
    \hat{f}_x(\xi)=\int_{\mathbb{R}} f(\sqrt{x}t)e^{-2\pi i\xi t}\diff t \overset{s=t\sqrt{x}}{=} \frac{1}{\sqrt{x}}\int_{\mathbb{R}} f(s)e^{-2\pi i\frac{\xi}{\sqrt{x}}s}\diff s= \\
    =\frac{1}{\sqrt{x}}\hat{f}\left(\frac{\xi}{\sqrt{x}}\right)=\frac{1}{\sqrt{x}}f\left(\frac{\xi}{\sqrt{x}}\right)=\frac{1}{\sqrt{x}}f_{\frac{1}{x}}(\xi).
  \end{gather*}
  Applicando la formula di Poisson abbiamo
  \begin{gather*}
    \sum_{n \in \mathbb{Z}} f_x(n)=\sum_{n \in \mathbb{Z}} \hat{f}_x(n)=\sum_{n \in \mathbb{Z}} \frac{1}{\sqrt{x}}f_{\frac{1}{x}}(n),
  \end{gather*}
  quindi
  $$\vartheta(x)=\sum_{n \in \mathbb{Z}} e^{-\pi n^2x}=\frac{1}{\sqrt{x}}\sum_{n \in \mathbb{Z}}e^{-\frac{\pi}{x}n^2}=\frac{1}{\sqrt{x}}\vartheta\left(\frac{1}{x}\right).$$
\end{proof}

Vogliamo usare quest'identità della funzione di Jacobi per dimostrare il teorema di Riemann sulla funzione $\zeta$. D'ora in avanti, per motivi storici la variabile sarà $s=\sigma+it$ con $\sigma, t \in \mathbb{R}$.

\begin{thm}
  (Riemann, 1860) La funzione $\zeta(s)$ è meromorfa in $\mathbb{C}$ con un polo semplice in $s=1$ con residuo $1$. Inoltre, posta
  $$\xi(s)=\frac{s(s-1)}{2}\pi^{-s/2}\Gamma\left(\frac{s}{2}\right)\zeta(s),$$
  allora $\xi$ è intera e fornisce il prolungamento analitico di $\zeta$. Si ha
  \begin{equation} \label{xifunctional}
    \xi(s)=\xi(1-s).
  \end{equation}
\end{thm}

\begin{proof}
  Se $\sigma>0$, allora
  \begin{gather*}
    \Gamma\left(\frac{s}{2}\right)=\int_0^{+\infty} e^{-x}x^{\frac{s}{2}}\frac{\diff x}{x} \implies \\
    \implies \frac{\pi^{-s/2}}{n^s}\Gamma\left(\frac{s}{2}\right)=\int_0^{+\infty} e^{-x}\left(\frac{x}{\pi n^2}\right)^{s/2}\frac{\diff x}{x}.
  \end{gather*}
  Se $\sigma>1$, allora
  \begin{gather*}
    \pi^{-s/2}\Gamma\left(\frac{s}{2}\right)\zeta(s)=\sum_{n=1}^{+\infty} \int_0^{+\infty} e^{-x}\left(\frac{x}{\pi n^2}\right)^{s/2}\frac{\diff x}{x} \overset{y=\frac{x}{\pi n^2}}{=} \\
    =\sum_{n=1}^{+\infty} \int_0^{+\infty} e^{-\pi n^2 y}y^{s/2}\frac{\diff y}{y}=\int_0^{+\infty} \sum_{n=1}^{+\infty} e^{-\pi n^2y}y^{s/2}\frac{\diff y}{y}= \\
    =\frac{1}{2}\int_0^{+\infty} \big(\vartheta(y)-1\big)y^{s/2}\frac{\diff y}{y}=\frac{1}{2}\left(\int_0^1+\int_1^{+\infty}\right)\big(\vartheta(y)-1\big)y^{s/2}\frac{\diff y}{y}.
  \end{gather*}
  Abbiamo che
  \begin{gather*}
    \frac{1}{2} \int_0^1 \big(\vartheta(y)-1\big)y^{s/2}\frac{\diff y}{y} \overset{x=\frac{1}{y}}{=} \frac{1}{2}\int_1^{+\infty} \Bigg(\vartheta\left(\frac{1}{x}\right)-1\Bigg)x^{-s/2}\frac{\diff x}{x}= \\
    =\frac{1}{2}\int_1^{+\infty} \big(\vartheta(x)-1\big)\sqrt{x}\cdot x^{-s/2}\frac{\diff x}{x}+\frac{1}{2}\int_1^{+\infty} x^{\frac{1-s}{2}}\frac{\diff x}{x}-\frac{1}{2}\int_1^{+\infty} x^{-\frac{s}{2}}\frac{\diff x}{x}= \\
    =\frac{1}{2}\int_1^{+\infty} \big(\vartheta(x)-1\big)\cdot x^{\frac{1-s}{2}}\frac{\diff x}{x}+\frac{1}{s-1}-\frac{1}{s}=\\
    =\frac{1}{2}\int_1^{+\infty} \big(\vartheta(x)-1\big)\cdot x^{\frac{1-s}{2}}\frac{\diff x}{x}+\frac{1}{s(s-1)}.
  \end{gather*}
  Si ha dunque
  \begin{gather*}
    \pi^{-s/2}\Gamma\left(\frac{s}{2}\right)\zeta(s)=\frac{1}{s(s-1)}+\frac{1}{2}\int_1^{+\infty} \big(\vartheta(x)-1\big)(x^{\frac{s}{2}}+x^{\frac{1-s}{2}})\frac{\diff x}{x} \implies \\
    \implies \xi(s)=\frac{s(s-1)}{2}\pi^{-s/2}\Gamma\left(\frac{s}{2}\right)\zeta(s)=\\
    =\frac{1}{2}+\frac{s(s-1)}{4}\int_1^{+\infty} \big(\vartheta(x)-1\big)(x^{\frac{s}{2}}+x^{\frac{1-s}{2}})\frac{\diff x}{x}.
  \end{gather*}
  Poiché
  $$\frac{1}{2}\big(\vartheta(x)-1\big)=\sum_{n=1}^{+\infty} e^{-\pi n^2x} \le \sum_{n=1}^{+\infty} e^{-\pi nx}=\frac{1}{e^{\pi x}-1} \ll e^{-\pi x},$$
  l'ultimo integrale nella formula per $\xi(s)$ converge uniformemente in ogni striscia $a \le \sigma \le b$; allora $\xi$ definita da quell'integrale è una funzione intera e la relazione con $\Gamma$ e $\zeta$ data nell'enunciato è valida per $\sigma>1$. Che $\xi(1-s)=\xi(s)$ è ovvio. Poiché $\Gamma$ non ha zeri e lo zero semplice di $s/2$ in $0$ è cancellato dal polo di $\Gamma$, questo definisce l'estensione analitica di $\zeta$ a $\mathbb{C}\setminus\{1\}$. Sappiamo già che sui reali $\displaystyle \lim_{s \longrightarrow 1} \zeta(s)=+\infty$. Mostriamo che è un polo semplice di residuo $1$. Si ha
  \begin{gather*}
    \zeta(s)=\xi(s)\frac{\pi^{s/2}}{\frac{s}{2}\Gamma\left(\frac{s}{2}\right)}\frac{1}{s-1} \text{ e} \\
    \underset{s=1}{\text{Res}}\,\zeta=\lim_{s \longrightarrow 1} \big(\zeta(s)(s-1)\big)=\xi(1)\frac{\pi^{1/2}}{\frac{1}{2}\Gamma\left(\frac{1}{2}\right)}=\frac{1}{2}\cdot\frac{\sqrt{\pi}}{\frac{1}{2}\sqrt{\pi}}=1.
  \end{gather*}
\end{proof}

\begin{oss}
  $\xi$ non ha zeri fuori dalla striscia $0 \le \sigma \le 1$.
\end{oss}

\begin{cor}
  $\zeta(-2k)=0$ per ogni $k \ge 1$.
\end{cor}

\begin{proof}
  Guardare i poli di $\Gamma$.
\end{proof}

\begin{oss}
  All'interno della striscia  $0 \le \sigma \le 1$ abbiamo che $\zeta$ e $\xi$ si annullano insieme.
\end{oss}

\begin{oss}
  Se $\rho$ è uno zero, dalla \eqref{xifunctional} anche $1-\rho$ è uno zero. Poiché $\xi$ è reale sui reali, se $\rho$ è uno $0$ anche $\bar{\rho}$ è uno zero.
\end{oss}

\begin{oss}
  Definendo $\Xi(s)=\xi(is+1/2)$ si ha
  $$\Xi(-s)=\xi(1/2-is)=\xi(1/2+is)=\Xi(s).$$
  Se $s=x \in \mathbb{R}$, allora $\overline{\xi(1/2+ix)}=\xi(1/2-ix)=\xi(1/2+ix)$, quindi $\Xi(x)$ è reale.
\end{oss}

\begin{oss}
  $\zeta(s)\not=0$ per $\sigma>1$. Infatti
  \begin{gather*}
    \left|\zeta(s)\prod_{p \le N} \left(1-\frac{1}{p^s}\right)\right|=\left|\prod_{p>N} \left(1-\frac{1}{p^s}\right)^{-1}\right|= \\
    =\left|1+\sum_{p \mid n \implies p>N} \frac{1}{n^s}\right| \ge 1-\sum_{n>N} \frac{1}{n^{\sigma}}=1+O\left(\frac{1}{N^{\sigma-1}}\right).
  \end{gather*}
\end{oss}

\begin{oss}
  Prendiamo $0 \le \sigma \le 1$ e $t=0$. Si ha
  $$\xi(\sigma)=\frac{1}{2}+\frac{\sigma(\sigma-1)}{4}\int_1^{+\infty}\big(\vartheta(x)-1\big)(x^{\frac{\sigma}{2}}+x^{\frac{1-\sigma}{2}})\frac{\diff x}{x}.$$
  Ricordiamo che $\dfrac{\vartheta(x)-1}{2} \le \dfrac{1}{e^{\pi x}-1}$ e per $x \ge 1$ vale che $x \ge \sqrt{x}$, dunque $e^{\pi x}-1 \ge \pi x \ge 2\sqrt{x}$, perciò $\vartheta(x)-1 \le \dfrac{1}{\sqrt{x}}$. Inoltre
  $$\int_1^{+\infty} (x^{\frac{\sigma-1}{2}}+x^{\frac{-\sigma}{2}})\frac{\diff x}{x}=\frac{2}{\sigma(\sigma-1)}.$$
  Mettendo assieme, troviamo che
  $$\xi(\sigma) \ge \frac{1}{2}-\frac{\sigma(1-\sigma)}{4}\cdot\frac{2}{\sigma(1-\sigma)}=0.$$
\end{oss}

\begin{oss}
  $\xi(0)=1/2$ e $\displaystyle \lim_{s \longrightarrow 1} \frac{s}{2}\Gamma\left(\frac{s}{2}\right)=1$, dunque dalla formula che lega $\zeta$ e $\xi$ si ha $\zeta(0)=-1/2$.
\end{oss}

\begin{exc}
  Dimostrare che $\zeta(-1)=-1/12$.
\end{exc}

Vogliamo ora capire qual è l'ordine di $\xi$. Abbiamo $\xi(s)=(s-1)\Gamma\left(\frac{s}{2}+1\right)\pi^{-\frac{s}{2}}\zeta(s)$. Analizziamo i vari fattori.

Con $s=2n$ si ha $\Gamma\left(\dfrac{s}{2}+1\right)=\Gamma(n+1)=n!\sim e^{\left(n+\frac{1}{2}\right)\log{n}-n+(\dots)} \ll_{\epsilon} e^{n^{1+\epsilon}}$. Più in particolare, per il corollario \ref{gammaord} abbiamo che questo vale in generale per $\sigma>1$.
Nella stessa regione, $\zeta(s)$ e $\pi^{-\frac{s}{2}}$ sono limitate a infinito e $s-1$ è ``piccola''. Questo ci dice che dev'essere $ord(\xi) \ge 1$ e, per quanto visto su $\Gamma$, se valesse l'uguale l'ordine sarebbe un $\inf$ e non un $\min$.

Per simmetria, ci resta da verificare solo la regione $1/2 \le \sigma \le 1$. Vediamo che, se l'ordine fosse proprio $1$, per il teorema di Hadamard avremmo che $\xi$ ha un prodotto di Weierstrass della forma $\displaystyle \xi(s)=e^{as+b}\prod_p (\dots)$. Se non ci fossero zeri, il prodotto vuoto sarebbe $1$ e $\xi(s) \ll e^{a|s|}$, assurdo per via del contributo $e^{s\log{s}}$ dovuto a $\Gamma$. Per lo stesso motivo, gli zeri devono essere infiniti.

\begin{lm} \label{lls}
  Per $\sigma \ge \epsilon$ si ha $\zeta(s) \ll_{\epsilon} |s|$ (o $|t|$) uniformemente.
\end{lm}

\begin{proof}
  Sia $\sigma>1$. Per sommazione parziale,
  \begin{gather*}
    \sum_{n \le x} \frac{1}{n^s}=\frac{\lfloor x\rfloor}{x^s}+s\int_1^x \frac{\lfloor u\rfloor}{u^{s+1}}\diff u=\frac{\lfloor x\rfloor}{x^s}+s\int_1^x \frac{\diff u}{u^s}-s\int_1^x \frac{\{u\}}{u^{s+1}}\diff u= \\
    =\frac{\lfloor x\rfloor}{x^s}+\frac{s}{s-1}-s\int_1^x \frac{\{u\}}{u^{s+1}}\diff u \implies \zeta(s)=\frac{1}{s-1}+1-s\int_1^{+\infty} \frac{\{u\}}{u^{s+1}}\diff u;
  \end{gather*}
  questa è l'estensione di $\zeta$ a $\sigma>0$ (l'integrale converge uniformemente per $\sigma \ge \epsilon$).
  \begin{oss}
    Prendendo $s=1$, otteniamo
    \begin{gather*}
      \sum_{n \le x} \frac{1}{n}=1-\frac{\{x\}}{x}+\log{x}-\int_1^x \frac{\{u\}}{u^2}\diff u \implies \\
      \lim_{x \longrightarrow +\infty} \left(\sum_{n \le x} \frac{1}{n}-\log{x}\right)=1-\int_1^{+\infty} \frac{\{u\}}{u^2}\diff u=\gamma.
    \end{gather*}
    Ma passiamo oltre.
  \end{oss}

  Utilizzando i polinomio di Bernoulli, abbiamo
  \begin{gather*}
    \zeta(s)=\frac{1}{s-1}+1-\frac{s}{2}\int_1^{+\infty} \frac{\diff u}{u^{s+1}}-s\int_1^{+\infty} \frac{B_1(\{u\})}{u^{s+1}}\diff u= \\
    =\frac{1}{s-1}+1-\frac{1}{2}-s\int_1^{+\infty} \frac{B_1(\{u\})}{u^{s+1}}\diff u.
  \end{gather*}

  \begin{oss}
    Ricordando la relazione tra polinomi di Bernoulli successivi e le loro derivate, si può integrare per parti fino a ottenere il prolungamento di $\zeta$ in un semipiano che inizia arbitrariamente a sinistra. Ma, senza l'equazione funzionale, questa costruzione è abbastanza inutile.
  \end{oss}

  Tornando a noi, per $\sigma \ge \epsilon$ e usando che $B_1(\{u\})$ è limitato, abbiamo
  $$|\zeta(s)| \ll 1+|s|\int_1^{+\infty} \frac{\diff u}{u^{1+\epsilon}} \ll \frac{|s|}{\epsilon} \ll_{\epsilon} |s|.$$
\end{proof}

Curiosità: la congettura di Lindelöf ipotizza che $\zeta\left(\dfrac{1}{2}+it\right) \ll_{\epsilon} t^{\epsilon}$ per ogni $\epsilon>0$.

\begin{prop}
  (formula di Riemann-Von Mangoldt) Contando con moltepicità, sia
  $$N(T)=\sharp\{\rho=\beta+i\gamma \mid \zeta(\rho)=0, 0 \le \beta \le 1, 0<\gamma<T\},$$
  allora per $T \longrightarrow +\infty$ si ha
  \begin{equation} \label{rievonmformula}
    N(T)=\frac{T}{2\pi}\log\left(\frac{T}{2\pi}\right)-\frac{T}{2\pi}+O(\log{T}).
  \end{equation}
\end{prop}

\begin{proof}
  Consideriamo il bordo rettangolare $R$, in verde nella figura.
  \begin{center}
    \definecolor{ttccqq}{rgb}{0.2,0.8,0}
    \definecolor{uququq}{rgb}{0.25,0.25,0.25}
    \begin{tikzpicture}[line cap=round,line join=round,>=triangle 45,x=1.0cm,y=1.0cm]
        \draw[->,color=black] (-2.09,0) -- (3.18,0);
        \foreach \x in {-2,-1,1,2,3}
        \draw[shift={(\x,0)},color=black] (0pt,2pt) -- (0pt,-2pt);
        \draw[->,color=black] (0,-0.31) -- (0,6.22);
        \foreach \y in {,1,2,3,4,5,6}
        \draw[shift={(0,\y)},color=black] (2pt,0pt) -- (-2pt,0pt);
        \clip(-2.09,-0.31) rectangle (3.18,6.22);
        \draw [line width=1pt,color=ttccqq] (-1,0)-- (2,0);
        \draw [line width=1pt,color=ttccqq] (2,0)-- (2,5.58);
        \draw [line width=1pt,color=ttccqq] (-1,5.58)-- (2,5.58);
        \draw [line width=1pt,color=ttccqq] (-1,5.58)-- (-1,0);
        \draw (1,5.58)-- (1,0);
        \draw [dash pattern=on 3pt off 3pt] (0.5,5.58)-- (0.5,0);
        \begin{scriptsize}
        \fill [color=black] (2,0) circle (1.5pt);
        \draw[color=black] (2,-0.2) node {$2$};
        \fill [color=black] (1,0) circle (1.5pt);
        \draw[color=black] (1,-0.2) node {$1$};
        \fill [color=black] (0.5,0) circle (1.5pt);
        \draw[color=black] (0.5,-0.2) node {$1/2$};
        \fill [color=black] (-1,0) circle (1.5pt);
        \draw[color=black] (-1,-0.2) node {$-1$};
        \fill [color=uququq] (0.5,5.58) circle (1.5pt);
        \draw[color=uququq] (0.5,5.8) node {$\frac{1}{2}+iT$};
        \fill [color=uququq] (2,5.58) circle (1.5pt);
        \draw[color=uququq] (2.1,5.8) node {$2+iT$};
        \fill [color=black] (0,0) circle (1.5pt);
        \draw[color=black] (-0.15,-0.2) node {$0$};
        \draw[color=black] (3, -0.2) node {$\sigma$};
        \draw[color=black] (-0.2, 6) node {$t$};
        \draw[color=ttccqq] (2.5, 2.9) node {$R$};
      \end{scriptsize}
    \end{tikzpicture}
  \end{center}

  Poiché $\xi$ è una funzione intera senza zeri fuori dalla striscia critica, all'interno della quale ha invece gli stessi zeri di $\zeta$ con la stessa moltepicità, abbiamo che
  $$N(T)=\frac{1}{2\pi i} \oint_R \frac{\xi'(s)}{\xi(s)}\diff s=\frac{1}{2\pi}\Delta_R \arg\big(\xi(s)\big).$$
  Adesso notiamo che, per quello che sappiamo di $\xi$ tra $0$ e $1$ e fuori dalla striscia critica e usando l'equazione funzionale, essa è sempre reale e mai nulla tra $-1$ e $2$, dunque l'argomento non cambia e possiamo trascurare quel segmento. Dato che $\xi$ è reale sui reali, si ha anche $\xi(s)=\xi(1-s)=\overline{\xi(1-\bar{s})}$; perciò la variazione da $\frac{1}{2}+iT$ a $-1$ è la stessa che tra $2$ e $\frac{1}{2}+iT$.
  Detto allora $L$ il sottotratto di $R$ formato dai due segmenti da $2$ a $2+iT$ e da $2+iT$ e $\frac{1}{2}+iT$, si ha $N(T)=\dfrac{1}{\pi}\Delta_L \arg\big(\xi(s)\big)$. Ricordiamo ora la definizione di $\xi$:
  $$\xi(s)=\frac{s(s-1)}{2}\pi^{-\frac{s}{2}}\Gamma\left(\frac{s}{2}\right)\zeta(s)=(s-1)\pi^{-\frac{s}{2}}\Gamma\left(\frac{s}{2}+1\right)\zeta(s);$$
  si ha dunque
  $$\Delta_L\arg\big(\xi(s)\big)=\Delta_L\arg(s-1)+\Delta_L\arg\left(\pi^{-\frac{s}{2}}\right)+\Delta_L\arg\Bigg(\Gamma\left(\frac{s}{2}+1\right)\Bigg)+\Delta_L\arg\big(\zeta(s)\big).$$
  Studiamo i singoli pezzi.
  \begin{gather*}
    \Delta_L\arg(s-1)=\arg\left(-\frac{1}{2}+iT\right)=\frac{\pi}{2}+O\left(\frac{1}{T}\right) \text{ e}\\
    \Delta_L\arg\left(\pi^{-\frac{s}{2}}\right)=\Delta_L\arg\left(e^{-\frac{s}{2}\log{\pi}}\right)=-\frac{T}{2}\log{\pi}.
  \end{gather*}
  Ricordiamo la formula di Stirling:
  $$\log\big(\Gamma(z)\big)=\left(z-\frac{1}{2}\right)\log{z}-z+\log{\sqrt{2\pi}}+O\left(\frac{1}{|z|}\right);$$
  abbiamo quindi che
  \begin{gather*}
    \Delta_L\arg\Bigg(\Gamma\left(\frac{s}{2}+1\right)\Bigg)=\mathfrak{Im}\,\log\Bigg(\Gamma\left(\frac{5}{4}+i\frac{T}{2}\right)\Bigg)= \\
    =\mathfrak{Im}\Bigg(\left(\frac{3}{4}+i\frac{T}{2}\right)\log\left(\frac{5}{4}+i\frac{T}{2}\right)-\frac{5}{4}-i\frac{T}{2}+O\left(\frac{1}{T}\right)\Bigg)= \\
    =\frac{3}{4}\Bigg(\frac{\pi}{2}+O\left(\frac{1}{T}\right)\Bigg)+\frac{T}{2}\log\sqrt{\frac{T^2}{4}+\frac{25}{16}}-\frac{T}{2}+O\left(\frac{1}{T}\right)=\\
    =\frac{3}{8}\pi+\frac{T}{2}\log\left(\frac{T}{2}\right)+\frac{T}{2}\log\left(1+\frac{25}{4T^2}\right)-\frac{T}{2}+O\left(\frac{1}{T}\right)=\\
    =\frac{3}{8}\pi+\frac{T}{2}\log\left(\frac{T}{2}\right)-\frac{T}{2}+O\left(\frac{1}{T}\right).
  \end{gather*}
  Mettendo assieme si ha $N(T)=\dfrac{T}{2\pi}\log\left(\dfrac{T}{2\pi}\right)+\dfrac{7}{8}-\dfrac{T}{2\pi}+S(T)+O\left(\dfrac{1}{T}\right)$, dove poniamo $S(T)=\dfrac{1}{\pi}\arg\Bigg(\zeta\left(\dfrac{1}{2}+iT\right)\Bigg)$. Si conclude con il lemma seguente.
\end{proof}

\begin{lm} \label{Slllog}
  Sia $S(T)=\dfrac{1}{\pi}\arg\Bigg(\zeta\left(\dfrac{1}{2}+iT\right)\Bigg)$; allora
  $$S(T) \ll \log{T}.$$
\end{lm}

\begin{proof}
  Scriviamo $$\arg\Bigg(\zeta\left(\frac{1}{2}+iT\right)\Bigg)=\arg\big(\zeta(2+iT)\big)+\Bigg[\arg\Bigg(\zeta\left(\frac{1}{2}+iT\right)\Bigg)-\arg\big(\zeta(2+iT)\big)\Bigg]$$
  e stimiamo i due addendi. Si ha
  \begin{gather*}
    \mathfrak{Re}\,\zeta(2+iT)=1+\sum_{n=2}^{+\infty} \mathfrak{Re}\frac{1}{n^{2+iT}} \ge 1-\sum_{n=2}^{+\infty} \frac{1}{n^2}=1-\left(\frac{\pi^2}{6}-1\right)>\frac{1}{3} \implies \\
    \implies |\arg\big(\zeta(2+iT)\big)| \le \pi/2.
  \end{gather*}
  Sia $m=\sharp\{\sigma_j \in [1/2,2] \mid \mathfrak{Re}\,\zeta(\sigma_j+iT)=0\}$.
  Per come sono definiti, tra un $\sigma_j$ e il successivo l'argomento di $\zeta$ cambia al più di $\pi$. Allora
  $$\left|\arg\Bigg(\zeta\left(\frac{1}{2}+iT\right)\Bigg)-\arg\big(\zeta(2+iT)\big)\right| \le (m+1)\pi.$$
  Dobbiamo stimare $m$. Definiamo $f(s)=\zeta(s+iT)+\zeta(s-iT)$, che è una funzione olomorfa per $s+iT\not=1$, quindi lo è in particolare per $T$ sufficientemente grande in modulo ($\not=0$), che è quello che ci interessa. Per motivi di coniugio ($\zeta$ è reale sui reali), abbiamo che $f(\sigma)=2\mathfrak{Re}\,\zeta(\sigma+iT)$. Perciò otteniamo che $m=\sharp\{\sigma_j \in [1/2,2] \mid f(\sigma_j)=0\}$; vale dunque che
  $$m \le M=\sharp\{s \in \mathbb{C} \mid |s-2| \le 3/2, f(s)=0\}.$$
  Per il corollario \ref{1.2.5} con $r=3/2$ e $R=7/4$, troviamo
  \begin{gather*}
    M \le \frac{1}{\log(R/r)}\log\left(\frac{\max_{|s-2| \le 7/4}|f(s)|}{f(2)}\right) \le \\
    \le \frac{1}{\log(7/6)}\log\left(\frac{\max_{|s-2| \le 7/4} 2|\zeta(s+iT)|}{2/3}\right) \le C_1\log{T}+C_2 \ll \log{T},
  \end{gather*}
  dove abbiamo stimato $f(2)$ usando quanto trovato per $\mathfrak{Re}\,\zeta(2+iT)$, mentre il massimo dentro al logaritmo è stimato usando che $|\zeta(\sigma+iT)| \ll T$ per il lemma \ref{lls} con $\sigma \ge 1/4$.
\end{proof}

\begin{ftt}
  Il termine che è $O(\log{T})$ in \eqref{rievonmformula} è comunemente chiamato $\mathcal{R}$. L'ipotesi di Lindelöf implica $\mathcal{R}=o(\log{T})$, mentre l'ipotesi di Riemann implica $\mathcal{R}=O\left(\dfrac{\log{T}}{\log{\log{T}}}\right)$.
\end{ftt}

Dalla dimostrazione della formula di Riemann-Von Mangoldt abbiamo che $N(T)-\dfrac{T}{2\pi}\log\left(\dfrac{T}{2\pi}\right)+\dfrac{T}{2\pi}=\dfrac{7}{8}+S(T)+O\left(\dfrac{1}{T}\right)$;
Littlewood ha dimostrato che $\displaystyle S_1(T)=\int_0^T S(t)\diff t \ll \log{T}$ (se il risultato del lemma \ref{Slllog} fosse ottimale, questo significherebbe che $S$ cambia spesso di segno, quindi ha molti zeri). Si ha dunque
$$\lim_{T \longrightarrow +\infty} \frac{1}{T} \int_0^T \left[N(t)-\frac{t}{2\pi}\left(\frac{t}{2\pi}\right)-\frac{t}{2\pi}\right]\diff t=\frac{7}{8},$$
che ci dice che la quantità $\frac{7}{8}$ nella formula è relativamente importante, quindi non possiamo trascurarla con troppa leggerezza.

\begin{cor}
  uniformemente in $T$ e $H$ si ha che
  $$N(T+H)-N(T) \ll (H+1)\log(T+H).$$
  Inoltre, esiste $H_0$ t.c. per $H \ge H_0$ si ha
  $$N(T+H)-N(T) \gg H\log{T}.$$
\end{cor}

\begin{proof}
  \begin{gather*}
    N(T+H)-N(T)= \\
    =\frac{T+H}{2\pi}\log\left(\frac{T+H}{2\pi}\right)-\frac{T+H}{2\pi}-\Bigg(\frac{T}{2\pi}\log\left(\frac{T}{2\pi}\right)-\frac{T}{2\pi}\Bigg)+O\big(\log(T+H)\big)= \\
    =\int_{\frac{T}{2\pi}}^{\frac{T+H}{2\pi}} \log{t}\diff t+O\big(\log(T+H)\big).
  \end{gather*}
  Per il teorema del valor medio, esiste $0 \le \delta \le 1$ t.c.
  $$N(T+H)-N(T)=\frac{H}{2\pi}\log\left(\frac{T+\delta H}{2\pi}\right)+O\big(\log(T+H)\big);$$
  la prima parte della tesi segue immediatamente. Per la seconda, notiamo che
  $$\frac{H}{2\pi}\log\left(\frac{T+\delta H}{2\pi}\right)+O\big(\log(T+H)\big)>\frac{H}{2\pi}\log\left(\frac{T}{2\pi}\right)+O\big(\log(T+H)\big) \gg H\log{T},$$
  dove serve $H \ge H_0$ per evitare che domini il termine $O$-grande, che non potremmo stimare dal basso.
\end{proof}

\begin{cor}
  Siano $\rho_n=\beta_n+i\gamma_n$ gli zeri non banali di $\zeta$, ordinati per parte immaginaria crescente (consideriamo solo $\gamma_n>0$) e contati con moltepicità; allora si ha $\gamma_n \sim \dfrac{2\pi n}{\log n}$.

  (Littlewood: $\gamma_{n+1}-\gamma_n \ll \dfrac{1}{\log{\log{\log{\gamma_n}}}}$)
\end{cor}

\begin{proof}
  Si ha
  \begin{gather*}
    \frac{\gamma_n}{2\pi}\log{\gamma_n} \sim \frac{\gamma_n+1}{2\pi}\log\left(\frac{\gamma_n+1}{2\pi}\right) \sim N(\gamma_n+1) \ge n \ge \\
    \ge N(\gamma_n-1) \sim \frac{\gamma_n-1}{2\pi}\log\left(\frac{\gamma_n-1}{2\pi}\right) \sim \frac{\gamma_n}{2\pi}\log{\gamma_n} \implies \\
    \implies n \sim \frac{\gamma_n}{2\pi}\log{\gamma_n} \implies \log{n} \sim \log{\gamma_n}+\log\log{\gamma_n}-\log{2\pi} \sim \log{\gamma_n} \implies \\
    \implies \gamma_n \sim \frac{2\pi n}{\log{\gamma_n}} \sim \frac{2\pi n}{\log{n}}.
  \end{gather*}
\end{proof}

\begin{cor}
  La successione $\rho_n=\beta_n+i\gamma_n$ ha esponente di convergenza $1$.
\end{cor}

\begin{proof}
  Si ha
  \begin{gather*}
    \sum_{n=1}^{+\infty} \frac{1}{|\rho_n|^{1+\epsilon}} \le \sum_{n=1}^{+\infty} \frac{1}{|\gamma_n|^{1+\epsilon}} \le c_1 \sum_{n=1}^{+\infty} \frac{(\log{n})^{1+\epsilon}}{n^{1+\epsilon}} \le c_2 \sum_{n=1}^{+\infty} \frac{1}{n^{1+\epsilon/2}}<+\infty,
  \end{gather*}
  ma, poiché $|\beta_n| \le 1$, abbiamo che
  \begin{gather*}
    \sum_{n=1}^{+\infty} \frac{1}{|\rho_n|}>\sum_{n=1}^{+\infty} \frac{1}{1+|\gamma_n|} \ge \sum_{n=1}^{+\infty} \frac{1}{1+\frac{c_3n}{\log{n}}}=\sum_{n=1}^{+\infty} \frac{\log{n}}{\log{n}+c_3n}=+\infty.
  \end{gather*}
\end{proof}


\subsection{Le funzioni intere $\xi(s)$ e $(s-1)\zeta(s)$}
Dovevamo capire qual è l'ordine di $\xi$ come funzione intera. Avevamo già controllato per $\sigma>1$, quindi ci manca $\sigma \ge 1/2$. Ricordiamo che $\xi(s)=\dfrac{s(s-1)}{2}\pi^{-\frac{s}{2}}\Gamma\left(\dfrac{s}{2}\right)\zeta(s)$. Per il lemma \ref{lls} abbiamo già $\zeta(s) \ll |s|$ per $\sigma \ge 1/2$; inoltre, ricordando il corollario \ref{gammaord}, abbiamo
\begin{gather*}
  \Gamma(s) \ll_{\epsilon} e^{|s|^{1+\epsilon}} \text{ per ogni } \epsilon>0, \quad \pi^{-\frac{s}{2}} \ll 1, \quad \frac{s(s-1)}{2} \ll |s^2| \implies \\
  \implies \xi(s) \ll_{\epsilon} e^{|s|^{1+\epsilon}} \text{ per }\sigma \ge 1/2.
\end{gather*}
Da $\xi(s)=\xi(1-s)$, otteniamo che è vero per ogni $\sigma$. Scriviamo $\xi$ con il prodotto di Weierstrass:
$$\xi(s)=e^{a+As}\prod_{\rho}\left(1-\frac{s}{\rho}\right)e^{\frac{s}{\rho}};$$
dev'essere $\displaystyle a=\log\big(\xi(0)\big)=\log\left(\frac{1}{2}\right) \implies \xi(s)=\frac{1}{2}e^{As}\prod_{\rho}\left(1-\frac{s}{\rho}\right)e^{\frac{s}{\rho}}$.

Consideriamo invece $(s-1)\zeta(s)=\xi(s)\dfrac{\pi^{\frac{s}{2}}}{\frac{s}{2}\Gamma\left(\frac{s}{2}\right)}$. Dall'equazione appena scritta otteniamo che ha ordine $1$, dunque si ha (ricordando anche gli zeri banali)
$$(s-1)\zeta(s)=e^{b+Bs}\prod_{\rho}\left(1-\frac{s}{\rho}\right)e^{\frac{s}{\rho}}\prod_{n=1}^{+\infty}\left(1+\frac{s}{2n}\right)e^{-\frac{s}{2n}};$$
dev'essere $\displaystyle b=\Bigg(\log\left(s-1)\zeta(s)\right)\Bigg)_{s=0}=\log\left(\frac{1}{2}\right) \implies$
$$\implies (s-1)\zeta(s)=\frac{1}{2}e^{Bs}\prod_{\rho}\left(1-\frac{s}{\rho}\right)e^{\frac{s}{\rho}}\prod_{n=1}^{+\infty}\left(1+\frac{s}{2n}\right)e^{-\frac{s}{2n}}.$$

Confrontando i prodotti di Weierstrass per $\zeta$ e $\xi$ usando l'equazione che lega le due funzioni, ricordando anche la definizione di $\Gamma$, si ha $B=A+\frac{1}{2}\log{\pi}+\frac{\gamma}{2}$. Andiamo a calcolarci $A$ e $B$.

$$A=\frac{\xi'}{\xi}(0)=2\xi'(0), \quad B=\left(\frac{1}{s-1}+\frac{\zeta'}{\zeta}(0)\right)_{s=0}=-2\zeta'(0)-1.$$
Si ha anche
\begin{gather*}
  \frac{B}{2}=\frac{A}{2}+\frac{1}{4}\log{\pi}+\frac{\gamma}{4} \implies \\
  \implies \xi'(0)+\zeta'(0)=-\frac{1}{2}-\frac{1}{4}\log{\pi}-\frac{\gamma}{4}.
\end{gather*}
Vogliamo calcolare $\xi'(0)$ usando l'equazione funzionale per $\xi$. Dalla definizione abbiamo
\begin{gather*}
  \xi'(s)=\frac{(s-1)\zeta(s)}{2}\pi^{-\frac{s}{2}}\Gamma\left(\frac{s}{2}\right)-\frac{1}{4}\log{\pi}\cdot s(s-1)\zeta(s)\Gamma\left(\frac{s}{2}\right)\pi^{-\frac{s}{2}}+ \\
  +\frac{s}{4}\pi^{-\frac{s}{2}}\Gamma'\left(\frac{s}{2}\right)(s-1)\zeta(s)+\frac{s}{2}\pi^{-\frac{s}{2}}\Gamma\left(\frac{s}{2}\right)D\big(\zeta(s)(s-1)\big)
\end{gather*}

Nella dimostrazione del lemma \ref{lls} abbiamo visto che
$$\zeta(s)=\frac{1}{s-1}+1-s\int_1^{+\infty} \frac{\{u\}}{u^{s+1}}\diff u \implies \lim_{s \longrightarrow 1} \left(\zeta(s)-\frac{1}{s-1}\right)=\gamma,$$
dunque dev'essere $\zeta(s)=\dfrac{1}{s-1}+\gamma+(s-1)g(s)$, con $g$ una qualche funzione intera. Allora otteniamo
$$\xi'(1)=\frac{\Gamma\left(\frac{1}{2}\right)}{2\sqrt{\pi}}-\frac{1}{4}\log{\pi}\frac{\Gamma\left(\frac{1}{2}\right)}{\sqrt{\pi}}+\frac{\Gamma'\left(\frac{1}{2}\right)}{4\sqrt{\pi}}+\frac{\Gamma\left(\frac{1}{2}\right)}{2\sqrt{\pi}}\gamma;$$
dobbiamo calcolare $\Gamma'\left(\frac{1}{2}\right)$. Dalla proposizione \ref{Gammagamma} abbiamo
\begin{gather*}
  \frac{\Gamma'}{\Gamma}(z)=-\gamma-\frac{1}{z}+\sum_{n=1}^{+\infty} \frac{z}{n(n+z)} \implies \\
  \implies \frac{\Gamma'}{\Gamma}\left(\frac{1}{2}\right)=-\gamma-2+\sum_{n=1}^{+\infty} \frac{2}{2n(2n+1)}=-\gamma-2+2\sum_{n=1}^{+\infty} \left(\frac{1}{2n}-\frac{1}{2n+1}\right)= \\
  =-\gamma-2+2\left(\frac{1}{2}-\frac{1}{3}+\frac{1}{4}-\frac{1}{5}+\dots\right)=-\gamma-2+2-2\log{2}=-\gamma-2\log{2} \implies \\
  \implies \Gamma'\left(\frac{1}{2}\right)=-\sqrt{\pi}(\gamma+2\log{2}).
\end{gather*}


\subsection{La funzione $\psi$ e il teorema dei numeri primi}
Riemann congetturò che
$$\psi(x)=\sum_{n \le x} \Lambda(n)=x-\sum_{\rho} \frac{x^{\rho}}{\rho}-\frac{\zeta'}{\zeta}(0)-\frac{1}{2}\log\left(1-\frac{1}{x^2}\right).$$
Beh, non proprio: l'espressione a destra è continua, a differenza di $\psi$. Utilizzeremo $\psi_0(x)=\displaystyle \sum_{n<x} \Lambda(n)+\frac{1}{2}\Lambda(x)$, dove poniamo $\Lambda=0$ fuori dagli interi. L'idea di Riemann è di scrivere
$$\psi_0(x) \sim \frac{1}{2\pi i} \int_{c-i\infty}^{c+i\infty} -\frac{\zeta'}{\zeta}(s)\frac{x^s}{s}\diff x$$
con $c>1$ e applicare il teorema dei residui. Ma come ha fatto a derivare un'espressione tanto precisa? Ha guardato i poli dell'integranda.

\begin{lm}
  Dato $c>0$, sia $I(y,T)=\displaystyle \frac{1}{2\pi i}\int_{c-iT}^{c+iT} y^s\frac{\diff s}{s}$; allora
  $$|I(y,T)-\delta(y)| \le \begin{cases}
    y^c\min\left\{1,\frac{1}{T|\log{y}|}\right\} & \mbox{se } y\not=1 \\
    \frac{c}{T} & \mbox{se } y=1,
\end{cases}$$
dove $\delta(y)=\begin{cases}
  1 & \mbox{se } y>1 \\
  \frac{1}{2} & \mbox{se } y=1 \\
  0 & \mbox{se } 0<y<1
\end{cases}$. Di conseguenza, $\displaystyle \frac{1}{2\pi i}\int_{c-i\infty}^{c+i\infty} y^s\frac{\diff s}{s}=\delta(y)$.
\end{lm}

\begin{proof}
  Preso $0<y<1$, consideriamo l'integrale sul cammino in figura, percorso in senso orario.
  \begin{center}
    \definecolor{uququq}{rgb}{0.25,0.25,0.25}
    \begin{tikzpicture}[line cap=round,line join=round,>=triangle 45,x=1.0cm,y=1.0cm]
      \draw[->,color=black] (-2,0) -- (6.86,0);
      \foreach \x in {-1,1,2,3,4,5,6}
      \draw[shift={(\x,0)},color=black] (0pt,2pt) -- (0pt,-2pt);
      \draw[->,color=black] (0,-3.94) -- (0,3.98);
      \foreach \y in {-3,-2,-1,1,2,3}
      \draw[shift={(0,\y)},color=black] (2pt,0pt) -- (-2pt,0pt);
      \clip(-2,-3.94) rectangle (6.86,3.98);
      \draw (0.54,-2.47)-- (0.54,2.47);
      \draw [domain=0.54:6.859999999999997] plot(\x,{(--10.36-0*\x)/4.2});
      \draw [domain=0.54:6.859999999999997] plot(\x,{(-10.7-0*\x)/4.34});
      \begin{scriptsize}
        \fill [color=black] (0.54,0) circle (1.5pt);
        \draw[color=black] (0.8,0.28) node {$c$};
        \fill [color=uququq] (0.54,2.47) circle (1.5pt);
        \draw[color=uququq] (0.56,2.78) node {$c+iT$};
        \fill [color=uququq] (0.54,-2.47) circle (1.5pt);
        \draw[color=uququq] (0.56,-2.78) node {$c-iT$};
      \end{scriptsize}
    \end{tikzpicture}
  \end{center}
  In teoria dovremmo fare un integrale chiuso e poi mandare un lato a infinito, ma visto che non ci sono poli per $y^s/s$ nella regione considerata e sul lato che si manda a infinito la funzione va a $0$ uniformemente, saltiamo il passaggio. Per il teorema dei residui
  $$\frac{1}{2\pi i}\int_{c-iT}^{c+iT} \frac{y^s}{s}\diff s=\frac{1}{2\pi i}\left(\int_{c-iT}^{+\infty-iT}-\int_{c+iT}^{+\infty+iT}\right)y^s\frac{\diff s}{s}=\mathcal{I}_1-\mathcal{I}_2.$$
  Si ha $\displaystyle |\mathcal{I}_1| \le \frac{1}{T} \int_c^{+\infty} y^{\sigma}\diff\sigma=\frac{1}{T}\cdot\frac{y^c}{|\log{y}|}$ e $\mathcal{I}_2$ si stima allo stesso modo.

  Per l'altro argomento del minimo, consideriamo la circonferenza di centro l'origine e raggio $R=\sqrt{c^2+T^2}$ e prendiamo il cammino da $c-iT$ a $c+iT$ e ritorno che gira in senso orario (quindi il segmento passante per $c$ all'andata e l'arco destro al ritorno). Sia $\gamma$ l'arco percorso al ritorno.
  \begin{center}
    \definecolor{uququq}{rgb}{0.25,0.25,0.25}
\begin{tikzpicture}[line cap=round,line join=round,>=triangle 45,x=1.0cm,y=1.0cm]
\draw[->,color=black] (-3.14,0) -- (3.52,0);
\foreach \x in {-3,-2,-1,1,2,3}
\draw[shift={(\x,0)},color=black] (0pt,2pt) -- (0pt,-2pt);
\draw[->,color=black] (0,-3.24) -- (0,3.34);
\foreach \y in {-3,-2,-1,1,2,3}
\draw[shift={(0,\y)},color=black] (2pt,0pt) -- (-2pt,0pt);
\clip(-3.14,-3.24) rectangle (3.52,3.34);
\draw(0,0) circle (2.49cm);
\draw (0,0)-- (0.8,2.36);
\draw (0.8,-2.36)-- (0.8,2.36);
\begin{scriptsize}
\fill [color=black] (0.8,0) circle (1.5pt);
\draw[color=black] (1.02,0.2) node {$c$};
\fill [color=black] (0,2.36) circle (1.5pt);
\draw[color=black] (-0.22,2.3) node {$T$};
\fill [color=uququq] (0.8,2.36) circle (1.5pt);
\draw[color=uququq] (0.96,2.62) node {$c+iT$};
\fill [color=uququq] (0.8,-2.36) circle (1.5pt);
\draw[color=uququq] (0.96,-2.56) node {$c-iT$};
\end{scriptsize}
\end{tikzpicture}
  \end{center}
  Allora per il teorema dei residui
  \begin{gather*}
    \left|\frac{1}{2\pi i}\int_{c-iT}^{c+iT} \frac{y^s}{s}\diff s\right|=\left|\frac{1}{2\pi i}\int_\gamma \frac{y^s}{s}\diff s \right| \le \\
    \frac{y^c}{2\pi} \int_{\gamma} \frac{\diff|s|}{|s|} \le y^c\frac{\pi R}{2\pi R} \le \frac{y^c}{2}.
  \end{gather*}
  Per $y>1$, basta ripetere le stesse stime ma con i cammini ``di sinistra'' percorsi in senso antiorario, facendo attenzione al polo in $0$: $\displaystyle \underset{s=0}{\text{Res}}\,\frac{y^s}{s}=1$.

  Adesso facciamo $y=1$ (non saremo rigorosi, ma si può sistemare facilmente):
  \begin{gather*}
    \frac{1}{2\pi i}\int_{c-iT}^{c+iT}\frac{\diff s}{s}=\frac{1}{2\pi i}\int_{-T}^T \frac{i\diff t}{c+it}=\frac{1}{2\pi}\int_{-T}^T \frac{c-it}{c^2+t^2}\diff t=\\
    =\frac{1}{2\pi}\int_{-T}^T \frac{c}{c^2+t^2}\diff t=\frac{1}{\pi}\int_0^T \frac{c}{c^2+t^2}\diff t\overset{x=t/c}{=}\frac{1}{\pi}\int_0^{T/c} \frac{\diff x}{x^2+1}= \\
    =\frac{1}{\pi}\left(\int_0^{+\infty}\frac{\diff x}{x^2+1}-\int_{T/c}^{+\infty}\frac{\diff x}{x^2+1}\right) \le \frac{1}{2}+\frac{c}{T}.
  \end{gather*}
\end{proof}

Prendiamo ora $y=\frac{x}{n}$ con $n \in \mathbb{N}$. Si ha
$$\frac{1}{2\pi i}\int_{c-iT}^{c+iT} \frac{\Lambda(n)}{n^s}x^s\frac{\diff s}{s}=\Lambda(n)\cdot \begin{cases}
  1+O\Bigg(\left(\dfrac{x}{n}\right)^c\min\left\{1,\dfrac{1}{T\left|\log\left(\frac{x}{n}\right)\right|}\right\}\Bigg) & \mbox{se }n < x \\
  \dfrac{1}{2}+O\left(\dfrac{c}{T}\right) & \mbox{se }n=x \text{ (e }n=p^a\text{)} \\
  O\Bigg(\left(\dfrac{x}{n}\right)^c\min\left\{1,\dfrac{1}{T\left|\log\left(\frac{x}{n}\right)\right|}\right\}\Bigg) & \mbox{se }n>x.
\end{cases}$$
Di conseguenza, per $c>1$ abbiamo
\begin{gather*}
  \sum_{n \le x} \Lambda(n)+\frac{\Lambda(x)}{2}=\frac{1}{2\pi i}\int_{c-iT}^{c+iT} \left(\sum_{n=1}^{+\infty}\frac{\Lambda(n)}{n^s}\right)\frac{x^s}{s}\diff s+\\
  +O\left(\sum_{n=1,n\not=x}^{+\infty} \frac{\Lambda(n)}{n^c}x^c\min\left\{1,\frac{1}{T\left|\log\left(\frac{x}{n}\right)\right|}\right\}+\frac{c\Lambda(x)}{T}\right).
\end{gather*}
Vogliamo dare una stima del resto, facendo un po' di casi. Prendiamo anche $c=1+\dfrac{1}{\log{x}}$, per avere $x^c=ex \ll x$. Resterà così.
\begin{enumerate}
  \item Partiamo dal più semplice: $\dfrac{c\Lambda(x)}{T} \ll \dfrac{\log{x}}{T}$.
  \item Ora facciamo la somma per $n \le \frac{3}{4}x$ o $n \ge \frac{5}{4}x$, per cui $\left|\log\left(\frac{x}{n}\right)\right| \gg 1$. Dobbiamo dunque stimare
  $$\left(\sum_{n \le \frac{3}{4}x}+\sum_{n \ge \frac{5}{4}x}\right)\frac{\Lambda(n)}{n^c}\cdot\frac{x^c}{T} \ll \frac{x}{T}\sum_{n=1}^{+\infty} \frac{\Lambda(n)}{n^c} \ll \frac{x\log{x}}{T},$$
  dove l'ultimo passaggio segue scrivendo la somma come $-\frac{\zeta'}{\zeta}(c)$ e usando che $\frac{\zeta'}{\zeta}(\sigma) \ll \frac{1}{\sigma-1}$.
  \item Per $\frac{3}{4}x<n<x$, sia $\frac{3}{4}x<x_1<x$ la massima tra le potenze di un primo in quell'intervallo. Allora
  \begin{gather*}
    \log\left(\frac{x}{x_1}\right)=-\log\left(\frac{x_1}{x}\right)=-\log\left(1-\frac{x-x_1}{x}\right)=\\
    =\frac{x-x_1}{x}+\frac{1}{2}\left(\frac{x-x_1}{x}\right)^2+\dots>\frac{x-x_1}{x},
  \end{gather*}
  dunque la stima del termine $n=x_1$ diventa
  $$\implies \Lambda(x_1)\left(\frac{x}{x_1}\right)^c\frac{x}{T(x-x_1)} \ll \frac{x\log{x}}{T(x-x_1)}.$$
  E $\frac{3}{4}x<n<x_1$? Abbiamo
  \begin{gather*}
    \log\left(\frac{x}{n}\right) \ge \log\left(\frac{x_1}{n}\right)=-\log\left(\frac{n}{x_1}\right)=\\
    =-\log\left(1-\frac{x_1-n}{x_1}\right)
    \ge \frac{x_1-n}{x_1}=\frac{\nu}{x_1}
  \end{gather*}
  e troviamo
  $$\sum_{1 \le \nu <x_1} \frac{\Lambda(x_1-\nu)x^cx_1}{(x_1-\nu)^cT\nu} \ll \frac{x(\log{x})^2}{T}.$$
  Il caso $x<n<\frac{5}{4}x$ è analogo, prendendo $x_2$ la minima potenza di primo.
\end{enumerate}

Ora, definendo $\langle x\rangle$ come la distanza di $x$ dalla potenza di primo più vicina, mettendo assieme abbiamo trovato che
$$\psi_0(x)=\frac{1}{2\pi i}\int_{c-iT}^{c+iT} -\frac{\zeta'}{\zeta}(s)\frac{x^s}{s}\diff s+O\left(\frac{x\log^2{x}}{T}+\frac{x\log{x}}{T\langle x\rangle}\right).$$


\subsection{La trasformata di Mellin e alcune conseguenze}
\begin{defn}
  Sia $f:\mathbb{R}^+ \longrightarrow \mathbb{R}$ t.c. $\displaystyle \int_0^{+\infty} |f(x)|x^{\sigma}\frac{\diff x}{x}<+\infty$ per $\sigma \in \mathbb{R}$ fissato. Si dice \textit{trasformata di Mellin} di $f$ la seguente:
  $$\wideparen{f}(s)=\int_0^{+\infty} f(x)x^s\frac{\diff x}{x}\text{ con }s=\sigma+it.$$
\end{defn}

\begin{ex}
  Se $f(x)=e^{-x}$ e $\sigma>0$, allora $\wideparen{f}(s)=\Gamma(s)$.
\end{ex}

\begin{oss} \label{melfou}
  Sia $x=e^{-2\pi u} \implies \frac{\diff x}{x}=-2\pi\diff u$. Allora
  \begin{gather*}
    \wideparen{f}(\sigma+it)=2\pi\int_{-\infty}^{+\infty} f(e^{-2\pi u})e^{-2\pi u\sigma-2\pi itu}\diff u= \\
    =2\pi \int_{-\infty}^{+\infty} \phi_{\sigma}(u)e^{-2\pi itu}\diff u=2\pi\hat{\phi}_{\sigma}(t).
  \end{gather*}
\end{oss}

\begin{prop} \label{mellinv}
  Se $f$ è di classe $C^1$ e $\displaystyle \int_0^{+\infty} |f(x)|x^{\sigma}\frac{\diff x}{x}<+\infty$, si ha
  $$f(x)=\frac{1}{2\pi i} \int_{\sigma-i\infty}^{\sigma+i\infty} \wideparen{f}(s)x^{-s}\diff s.$$
\end{prop}

\begin{proof}
  Dall'osservazione \ref{melfou} abbiamo
  \begin{gather*}
    f(e^{-2\pi u})e^{-2\pi\sigma u}=\varphi_{\sigma}(u)=\widehat{\widehat{\varphi}}_{\sigma}(-u)=\frac{1}{2\pi}\int_{-\infty}^{+\infty} \wideparen{f}(\sigma+it)e^{2\pi iut}\diff t \implies \\
    \implies f(e^{-2\pi u})=\frac{1}{2\pi}\int_{-\infty}^{+\infty} \wideparen{f}(\sigma+it)e^{2\pi(\sigma+it)u}\diff t=\frac{1}{2\pi i}\int_{\sigma-i\infty}^{\sigma+i\infty} \wideparen{f}(s)e^{2\pi su}\diff s.
  \end{gather*}
  Basta porre $x=e^{-2\pi u}$.
\end{proof}

\begin{oss}
  Se $f$ è $C^1$ a tratti con limite destro e sinistro finiti nei punti di discontinuità, la proposizione \ref{mellinv} continua a valere purché si applichi a $\tilde{f}$, che coincide con $f$ a parte nei punti di discontinuità dove è uguale alla media dei due limiti, e interpretando l'integrale come limite. In tutti i casi si prende $\sigma>0$.
\end{oss}

\begin{oss}
  Data $g(s)$ olomorfa in $\sigma_1 < \sigma < \sigma_2$ che sia anche continua in $\sigma_1 \le \sigma \le \sigma_2$ e t.c. $g(\sigma+it) \overset{|t| \longrightarrow +\infty}{\longrightarrow} 0$ per ogni $\sigma_1 \le \sigma \le \sigma_2$, si può dimostrare che $\displaystyle f(x)=\frac{1}{2\pi i}\int_{\sigma-i\infty}^{\sigma+i\infty} g(s)x^{-s}\diff s$ non dipende da $\sigma$ e si ha $g(s)=\wideparen{f}(s)$.
\end{oss}

\begin{ex}
  Vediamo la trasformata di Mellin di alcune funzioni, poi applichiamo la formula di inversione scrivendo però $y=1/x$. Nel secondo e terzo esempio, l'integrale della trasformata si fa iterando per parti.
  \begin{enumerate}
    \item
    \begin{gather*}
      f(x)=\begin{cases}
        1 &\mbox{se } 0<x<1 \\
        1/2 &\mbox{se } x=1 \\
        0 &\mbox{se } x>1
    \end{cases} \implies \\
    \wideparen{f}(s)=\frac{1}{s}, \quad \frac{1}{2\pi i}\lim_{T \longrightarrow +\infty} \int_{\sigma-iT}^{\sigma+iT} \frac{y^s}{s}\diff s=\begin{cases}
      1 &\mbox{se } y>1 \\
      1/2 &\mbox{se } y=1 \\
      0 &\mbox{se } 0<y<1.
    \end{cases}
  \end{gather*}
  \item
  \begin{gather*}
    f(x)=\begin{cases}
      \dfrac{(1-x)^k}{k!} &\mbox{se } 0<x \le 1 \\
      0 &\mbox{se } x \ge 1
      \end{cases} \implies \\
      \wideparen{f}(s)=\frac{1}{s(s+1)\dots(s+k)}, \\ \frac{1}{2\pi i} \int_{\sigma-i\infty}^{\sigma+i\infty} \frac{y^s}{s(s+1)\dots(s+k)}\diff s=\begin{cases}
        \dfrac{1}{k!}\left(1-\dfrac{1}{y}\right)^k &\mbox{se } y \ge 1 \\
        0 &\mbox{se } 0<y \le 1.
      \end{cases}
    \end{gather*}
    \item
    \begin{gather*}
      f(x)=\begin{cases}
        \dfrac{(-\log{x})^k}{k!} &\mbox{se } 0<x \le 1 \\
        0 &\mbox{se } x \ge 1
      \end{cases} \implies \\
      \wideparen{f}(s)=\frac{1}{s^{k+1}}, \quad \frac{1}{2\pi i} \int_{\sigma-i\infty}^{\sigma+i\infty} \frac{y^s}{s^{k+1}}\diff s=\begin{cases}
        \dfrac{(\log{y})^k}{k!} &\mbox{se } y \ge 1 \\
        0 &\mbox{se } 0<y \le 1.
      \end{cases}
    \end{gather*}
  \end{enumerate}
\end{ex}

Nell'ultimo caso, prendendo $k=1$ e $y=x/n$, per $\sigma>1$ si ha
$$\sum_{n \le x} \Lambda(n)\log\left(\frac{x}{n}\right)=\frac{1}{2\pi i}\int_{\sigma-i\infty}^{\sigma+i\infty} \left(\sum_{n=1}^{+\infty} \frac{\Lambda(n)}{n^s}\right)\frac{x^s}{s^2}\diff s=\frac{1}{2\pi i}\int_{\sigma-i\infty}^{\sigma+i\infty} -\frac{\zeta'}{\zeta}(s)\frac{x^s}{s^2}\diff s.$$

\begin{oss} \label{psiuno}
  Sommando per parti otteniamo
  \begin{gather*}
    \sum_{n \le x} \Lambda(n)\log\left(\frac{x}{n}\right)=\sum_{n \le x} \Lambda(n) \cdot 0+\int_2^x \sum_{n \le u} \Lambda(u)\frac{\diff u}{u}= \\
    =\int_2^x \frac{\psi(u)}{u}\diff u=:\psi_1(u).
  \end{gather*}
\end{oss}

\begin{oss}
  Posto $\displaystyle \psi_l(x)=\int_2^x \frac{\psi_{l-1}(u)}{u}\diff u$ per $l \ge 2$, si ha
  $$\psi_k(x)=\frac{1}{k!}\sum_{n \le x} \Lambda(n)\Bigg(\log\left(\frac{x}{n}\right)\Bigg)^k.$$
\end{oss}

\begin{prop}
  Vale la seguente formula esplicita:
  $$\psi_1(x)=x-\sum_{\rho} \frac{x^{\rho}}{\rho^2}-\frac{\zeta'}{\zeta}(0)\log{x}-\left(\frac{\zeta'}{\zeta}\right)'(0)-\frac{1}{4}\sum_{n=1}^{+\infty} \frac{x^{-2n}}{n^2}.$$
\end{prop}

\begin{proof}
  Basta combinare l'osservazione \ref{psiuno} e la formula vista subito prima per poi applicare il teorema dei residui.
\end{proof}

\begin{cor}
  $\psi_1(x)=x+O\big(x\exp(-c\sqrt{\log{x}})\big)$.
\end{cor}

\begin{proof}
  Basta applicare de la Vallée-Poussin come si è fatto per $\psi_0$.
\end{proof}

Se valesse RH si otterrebbe $x+O(\sqrt{x})$, invece con QRH $x+O(x^{\theta})$.

Mostriamo adesso che $\psi_1(x)=x+o(x)$. Dalla formula esplicita si ha
$$\left|\sum_{\rho} \frac{x^{\rho}}{\rho^2}\right| \le \sum_{\rho} \frac{x^{\beta}}{|\rho|^2} \implies \frac{\psi_1(x)-x}{x} \ll \sum_{\rho} \frac{x^{\beta-1}}{|\rho|^2}.$$
Poiché $\beta<1$, si ha
$$\lim_{x \longrightarrow +\infty} \sum_{\rho} \frac{x^{\beta-1}}{|\rho|^2}=\sum_{\rho} \frac{1}{|\rho|^2} \lim_{x \longrightarrow +\infty} x^{\beta-1}=0.$$

\begin{oss} \label{xpiuox}
  Per $x \ge 2$ e $h \le x$ abbiamo
  \begin{gather*}
    \int_x^{x+h} \frac{\psi(u)}{u}\diff u \ge \frac{h}{x+h}\psi(x), \quad \int_{x-h}^x \frac{\psi(u)}{u}\diff u \le \frac{h}{x-h}\psi(x) \implies \\
    \implies (x-h)\frac{\psi_1(x)-\psi_1(x-h)}{h} \le \psi(x) \le (x+h)\frac{\psi_1(x+h)-\psi_1(x)}{h}.
  \end{gather*}
\end{oss}

\begin{prop}
  Si ha $\psi(x)=x+o(x)$.
\end{prop}

\begin{proof}
  \begin{gather*}
    \psi_1(y) \sim y \implies (x-h)\frac{\psi_1(x)-\psi_1(x-h)}{h} \sim (x-h)\frac{x-x+h}{h}\sim x-h, \\
    (x+h)\frac{\psi_1(x+h)-\psi_1(x)}{h} \sim x+h.
  \end{gather*}
  Basta prendere $h=o(x)$ (con $h=1$ costante si va sul sicuro) e dall'osservazione \ref{xpiuox} si ha
  $$x \sim f(x) \le \psi(x) \le g(x) \sim x \implies \psi(x) \sim x.$$
\end{proof}

Non dimostreremo il seguente risultato, che si trova nel capitolo 3 di \cite{T}.

\begin{thm}
  (Landau) Se si hanno $\phi(t), \theta(t)$ t.c. $\phi$ è monotona non decrescente e tende a $+\infty$, $\theta$ è monotona non crescente e $0 < \theta(t) \le 1$, inoltre $\dfrac{\phi(t)}{\theta(t)}=o(e^{\phi(t)})$ e per $1-\theta(t) \le \sigma \le 2$ si ha $\zeta(s) \ll e^{\phi(t)}$, allora esiste $c_0$ t.c.
  $$\beta<1-\frac{\theta(2t+1)}{c_0\phi(2t+1)}.$$
  In particolare, prendendo $\phi(t)=\log{t}$ e $\theta(t)=1/2$, si ottiene $\beta<1-\dfrac{1}{2c_0\log(2t+1)}$.
\end{thm}

Littlewood: $\theta(t)=\frac{(\log\log{t})^2}{\log{t}}, \phi(t)=A\log\log{t} \implies \beta<1-\frac{C\log\log{T}}{\log{T}}$.

Vinogradov: $\theta(t)=\frac{A}{(\log{t})^{2/3-2\epsilon}}, \phi(t)=(\log{t})^{\epsilon}, \zeta(s) \ll \exp\big((\log{t})^{\epsilon}\big) \implies$ \\
$\implies \beta<1-\frac{C}{(\log{t})^{2/3-\epsilon}}$.

Adesso altre cose che non vedremo nel dettaglio.

\begin{enumerate}
  \item $\displaystyle \sum_{N<n \le 2N} e^{2\pi i\alpha n} \ll \frac{1}{\|\alpha\|}$ con $0<\alpha<1$ e $\|\alpha\|=\min\big\{|\alpha-n|, n \in \mathbb{N}\cup\{0\}\big\}$;
  \item $\displaystyle \sum_{N<n \le 2N} \Lambda(n)e^{2\pi i\alpha n} \ll \left(\frac{N}{\sqrt{q}}+\sqrt{Nq}+N^{3/4}\right)\log^4{N}$ dove $\left|\alpha-\frac{a}{q}\right| \le \frac{1}{q^2}$.
  Con questo risultato si riesce a dimostrare il teorema di Vinogradov (1930): $2N+1=p_1+p_2+p_3$ (è la versione per i dispari della congettura di Goldbach);
  \item $\displaystyle \sum_{N<n \le 2N} n^{-it}$ (Vinogradov, 1958). Questo porta a delle stime per $\zeta(\sigma+it)$ quando $1-\sigma \ll \frac{1}{(\log{t})^{2/3-\epsilon}}$;
  \item c'è una stima per $\displaystyle \sum_{N<n \le 2N} \frac{\Lambda(n)}{n^{it}}$? Non davvero, però vale il seguente risultato dovuto a Turán: se $\displaystyle \sum_{N<n \le 2N} \frac{\Lambda(n)}{n^{it}} \ll \frac{N}{t^b}$ per $N \ge t^a$, allora $\beta<1-\frac{b^2}{a^3}$ (correggere i coefficienti di $a$ e $b$ se e quando lo ridice).
\end{enumerate}


\newpage

\section{Titolo da decidere}

\subsection{I caratteri e il teorema di Dirichlet}
Da scrivere.


\newpage

\begin{thebibliography}{widest entry}
  \bibitem[D]{D} Davenport Multiplicative Number Theory (da riportare meglio)
  \bibitem[T]{T} Titchmarsh The Theory of the Riemann Zeta Function (da riportare meglio)
  \bibitem[JBG]{JBG} J. B. Garnett: \textbf{Bounded Analytic Functions (Revised First Edition)}. Springer, New York, 2007
  \bibitem[NN]{NN} R. Narasimhan, Y. Nievergelt: \textbf{Complex analysis in one variable (2nd edition)}. Springer, New York, 2001
\end{thebibliography}


\section*{Ringraziamenti}
\addcontentsline{toc}{section}{Ringraziamenti}
Da scrivere alla fine del corso. \\
Quelli sopra nella bibliografia sono degli esempi lasciati per ricordarsi qual è il modo giusto di scriverli, da togliere dopo aver inserito la bibliografia giusta per queste dispense.


\end{document}
